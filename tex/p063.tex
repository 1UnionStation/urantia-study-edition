\upaper{63}{The First Human Family}
\author{Life Carrier}
\vs p063 0:1 Urantia was registered as an inhabited world when the first two human beings --- the twins --- were 11 years old, and before they had become the parents of the first\hyp{}born of the second generation of actual human beings. And the archangel message from Salvington, on this occasion of formal planetary recognition, closed with these words:
\vs p063 0:2 “Man\hyp{}mind has appeared on 606 of Satania, and these parents of the new race shall be called \bibemph{Andon} and \bibemph{Fonta.} And all archangels pray that these creatures may speedily be endowed with the personal indwelling of the gift of the spirit of the Universal Father.”
\vs p063 0:3 \pc Andon is the Nebadon name which signifies “the first Fatherlike creature to exhibit human perfection hunger.” Fonta signifies “the first Sonlike creature to exhibit human perfection hunger.” Andon and Fonta never knew these names until they were bestowed upon them at the time of fusion with their Thought Adjusters. Throughout their mortal sojourn on Urantia they called each other Sonta\hyp{}an and Sonta\hyp{}en, Sonta\hyp{}an meaning “loved by mother,” Sonta\hyp{}en signifying “loved by father.” They gave themselves these names, and the meanings are significant of their mutual regard and affection.
\usection{1.\bibnobreakspace Andon and Fonta}
\vs p063 1:1 In many respects, Andon and Fonta were the most remarkable pair of human beings that have ever lived on the face of the earth. This wonderful pair, the actual parents of all mankind, were in every way superior to many of their immediate descendants, and they were radically different from all of their ancestors, both immediate and remote.
\vs p063 1:2 The parents of this first human couple were apparently little different from the average of their tribe, though they were among its more intelligent members, that group which first learned to throw stones and to use clubs in fighting. They also made use of sharp spicules of stone, flint, and bone.
\vs p063 1:3 While still living with his parents, Andon had fastened a sharp piece of flint on the end of a club, using animal tendons for this purpose, and on no less than a dozen occasions he made good use of such a weapon in saving both his own life and that of his equally adventurous and inquisitive sister, who unfailingly accompanied him on all of his tours of exploration.
\vs p063 1:4 The decision of Andon and Fonta to flee from the Primates tribes implies a quality of mind far above the baser intelligence which characterized so many of their later descendants who stooped to mate with their retarded cousins of the simian tribes. But their vague feeling of being something more than mere animals was due to the possession of personality and was augmented by the indwelling presence of the Thought Adjusters.
\usection{2.\bibnobreakspace The Flight of the Twins}
\vs p063 2:1 After Andon and Fonta had decided to flee northward, they succumbed to their fears for a time, especially the fear of displeasing their father and immediate family. They envisaged being set upon by hostile relatives and thus recognized the possibility of meeting death at the hands of their already jealous tribesmen. As youngsters, the twins had spent most of their time in each other’s company and for this reason had never been overly popular with their animal cousins of the Primates tribe. Nor had they improved their standing in the tribe by building a separate, and a very superior, tree home.
\vs p063 2:2 And it was in this new home among the treetops, one night after they had been awakened by a violent storm, and as they held each other in fearful and fond embrace, that they finally and fully made up their minds to flee from the tribal habitat and the home treetops.
\vs p063 2:3 They had already prepared a crude treetop retreat some half\hyp{}day’s journey to the north. This was their secret and safe hiding place for the first day away from the home forests. Notwithstanding that the twins shared the Primates’ deathly fear of being on the ground at nighttime, they sallied forth shortly before nightfall on their northern trek. While it required unusual courage for them to undertake this night journey, even with a full moon, they correctly concluded that they were less likely to be missed and pursued by their tribesmen and relatives. And they safely made their previously prepared rendezvous shortly after midnight.
\vs p063 2:4 On their northward journey they discovered an exposed flint deposit and, finding many stones suitably shaped for various uses, gathered up a supply for the future. In attempting to chip these flints so that they would be better adapted for certain purposes, Andon discovered their sparking quality and conceived the idea of building fire. But the notion did not take firm hold of him at the time as the climate was still salubrious and there was little need of fire.
\vs p063 2:5 But the autumn sun was getting lower in the sky, and as they journeyed northward, the nights grew cooler and cooler. Already they had been forced to make use of animal skins for warmth. Before they had been away from home one moon, Andon signified to his mate that he thought he could make fire with the flint. They tried for two months to utilize the flint spark for kindling a fire but only met with failure. Each day this couple would strike the flints and endeavour to ignite the wood. Finally, one evening about the time of the setting of the sun, the secret of the technique was unravelled when it occurred to Fonta to climb a near\hyp{}by tree to secure an abandoned bird’s nest. The nest was dry and highly inflammable and consequently flared right up into a full blaze the moment the spark fell upon it. They were so surprised and startled at their success that they almost lost the fire, but they saved it by the addition of suitable fuel, and then began the first search for firewood by the parents of all mankind.
\vs p063 2:6 This was one of the most joyous moments in their short but eventful lives. All night long they sat up watching their fire burn, vaguely realizing that they had made a discovery which would make it possible for them to defy climate and thus forever to be independent of their animal relatives of the southern lands. After three days’ rest and enjoyment of the fire, they journeyed on.
\vs p063 2:7 The Primates ancestors of Andon had often replenished fire which had been kindled by lightning, but never before had the creatures of earth possessed a method of starting fire at will. But it was a long time before the twins learned that dry moss and other materials would kindle fire just as well as birds’ nests.
\usection{3.\bibnobreakspace Andon’s Family}
\vs p063 3:1 It was almost two years from the night of the twins’ departure from home before their first child was born. They named him Sontad; and Sontad was the first creature to be born on Urantia who was wrapped in protective coverings at the time of birth. The human race had begun, and with this new evolution there appeared the instinct properly to care for the increasingly enfeebled infants which would characterize the progressive development of mind of the intellectual order as contrasted with the more purely animal type.
\vs p063 3:2 Andon and Fonta had 19 children in all, and they lived to enjoy the association of almost 50 grandchildren and half a dozen great\hyp{}grandchildren. The family was domiciled in four adjoining rock shelters, or semicaves, three of which were interconnected by hallways which had been excavated in the soft limestone with flint tools devised by Andon’s children.
\vs p063 3:3 These early Andonites evinced a very marked clannish spirit; they hunted in groups and never strayed very far from the homesite. They seemed to realize that they were an isolated and unique group of living beings and should therefore avoid becoming separated. This feeling of intimate kinship was undoubtedly due to the enhanced mind ministry of the adjutant spirits.
\vs p063 3:4 \pc Andon and Fonta laboured incessantly for the nurture and uplift of the clan. They lived to the age of 42, when both were killed at the time of an earthquake by the falling of an overhanging rock. 5 of their children and 11 grandchildren perished with them, and almost a score of their descendants suffered serious injuries.
\vs p063 3:5 Upon the death of his parents, Sontad, despite a seriously injured foot, immediately assumed the leadership of the clan and was ably assisted by his wife, his eldest sister. Their first task was to roll up stones to effectively entomb their dead parents, brothers, sisters, and children. Undue significance should not attach to this act of burial. Their ideas of survival after death were very vague and indefinite, being largely derived from their fantastic and variegated dream life.
\vs p063 3:6 \pc This family of Andon and Fonta held together until the 20\ts{th} generation, when combined food competition and social friction brought about the beginning of dispersion.
\usection{4.\bibnobreakspace The Andonic Clans}
\vs p063 4:1 Primitive man --- the Andonites --- had black eyes and a swarthy complexion, something of a cross between yellow and red. Melanin is a colouring substance which is found in the skins of all human beings. It is the original Andonic skin pigment. In general appearance and skin colour these early Andonites more nearly resembled the present\hyp{}day Eskimo than any other type of living human beings. They were the first creatures to use the skins of animals as a protection against cold; they had little more hair on their bodies than present\hyp{}day humans.
\vs p063 4:2 The tribal life of the animal ancestors of these early men had foreshadowed the beginnings of numerous social conventions, and with the expanding emotions and augmented brain powers of these beings, there was an immediate development in social organization and a new division of clan labour. They were exceedingly imitative, but the play instinct was only slightly developed, and the sense of humour was almost entirely absent. Primitive man smiled occasionally, but he never indulged in hearty laughter. Humour was the legacy of the later Adamic race. These early human beings were not so sensitive to pain nor so reactive to unpleasant situations as were many of the later evolving mortals. Childbirth was not a painful or distressing ordeal to Fonta and her immediate progeny.
\vs p063 4:3 \pc They were a wonderful tribe. The males would fight heroically for the safety of their mates and their offspring; the females were affectionately devoted to their children. But their patriotism was wholly limited to the immediate clan. They were very loyal to their families; they would die without question in defence of their children, but they were not able to grasp the idea of trying to make the world a better place for their grandchildren. Altruism was as yet unborn in the human heart, notwithstanding that all of the emotions essential to the birth of religion were already present in these Urantia aborigines.
\vs p063 4:4 These early men possessed a touching affection for their comrades and certainly had a real, although crude, idea of friendship. It was a common sight in later times, during their constantly recurring battles with the inferior tribes, to see one of these primitive men valiantly fighting with one hand while he struggled on, trying to protect and save an injured fellow warrior. Many of the most noble and highly human traits of subsequent evolutionary development were touchingly foreshadowed in these primitive peoples.
\vs p063 4:5 \pc The original Andonic clan maintained an unbroken line of leadership until the 27\ts{th} generation, when, no male offspring appearing among Sontad’s direct descendants, two rival would\hyp{}be rulers of the clan fell to fighting for supremacy.
\vs p063 4:6 Before the extensive dispersion of the Andonic clans a well\hyp{}developed language had evolved from their early efforts to intercommunicate. This language continued to grow, and almost daily additions were made to it because of the new inventions and adaptations to environment which were developed by these active, restless, and curious people. And this language became the word of Urantia, the tongue of the early human family, until the later appearance of the coloured races.
\vs p063 4:7 \pc As time passed, the Andonic clans grew in number, and the contact of the expanding families developed friction and misunderstandings. Only two things came to occupy the minds of these peoples: hunting to obtain food and fighting to avenge themselves against some real or supposed injustice or insult at the hands of the neighbouring tribes.
\vs p063 4:8 Family feuds increased, tribal wars broke out, and serious losses were sustained among the very best elements of the more able and advanced groups. Some of these losses were irreparable; some of the most valuable strains of ability and intelligence were forever lost to the world. This early race and its primitive civilization were threatened with extinction by this incessant warfare of the clans.
\vs p063 4:9 It is impossible to induce such primitive beings long to live together in peace. Man is the descendant of fighting animals, and when closely associated, uncultured people irritate and offend each other. The Life Carriers know this tendency among evolutionary creatures and accordingly make provision for the eventual separation of developing human beings into at least 3, and more often 6, distinct and separate races.
\usection{5.\bibnobreakspace Dispersion of the Andonites}
\vs p063 5:1 The early Andon races did not penetrate very far into Asia, and they did not at first enter Africa. The geography of those times pointed them north, and farther and farther north these people journeyed until they were hindered by the slowly advancing ice of the third glacier.
\vs p063 5:2 Before this extensive ice sheet reached France and the British Isles, the descendants of Andon and Fonta had pushed on westward over Europe and had established more than 1,000 separate settlements along the great rivers leading to the then warm waters of the North Sea.
\vs p063 5:3 These Andonic tribes were the early river dwellers of France; they lived along the river Somme for tens of thousands of years. The Somme is the one river unchanged by the glaciers, running down to the sea in those days much as it does today. And that explains why so much evidence of the Andonic descendants is found along the course of this river valley.
\vs p063 5:4 These aborigines of Urantia were not tree dwellers, though in emergencies they still betook themselves to the treetops. They regularly dwelt under the shelter of overhanging cliffs along the rivers and in hillside grottoes which afforded a good view of the approaches and sheltered them from the elements. They could thus enjoy the comfort of their fires without being too much inconvenienced by the smoke. They were not really cave dwellers either, though in subsequent times the later ice sheets came farther south and drove their descendants to the caves. They preferred to camp near the edge of a forest and beside a stream.
\vs p063 5:5 They very early became remarkably clever in disguising their partially sheltered abodes and showed great skill in constructing stone sleeping chambers, dome\hyp{}shaped stone huts, into which they crawled at night. The entrance to such a hut was closed by rolling a stone in front of it, a large stone which had been placed inside for this purpose before the roof stones were finally put in place.
\vs p063 5:6 The Andonites were fearless and successful hunters and, with the exception of wild berries and certain fruits of the trees, lived exclusively on flesh. As Andon had invented the stone ax, so his descendants early discovered and made effective use of the throwing stick and the harpoon. At last a tool\hyp{}creating mind was functioning in conjunction with an implement\hyp{}using hand, and these early humans became highly skillful in the fashioning of flint tools. They travelled far and wide in search of flint, much as present\hyp{}day humans journey to the ends of the earth in quest of gold, platinum, and diamonds.
\vs p063 5:7 And in many other ways these Andon tribes manifested a degree of intelligence which their retrogressing descendants did not attain in 500,000 years, though they did again and again rediscover various methods of kindling fire.
\usection{6.\bibnobreakspace Onagar --- The First Truth Teacher}
\vs p063 6:1 As the Andonic dispersion extended, the cultural and spiritual status of the clans retrogressed for nearly 10,000 years until the days of Onagar, who assumed the leadership of these tribes, brought peace among them, and for the first time, led all of them in the worship of the “Breath Giver to men and animals.”
\vs p063 6:2 \pc Andon’s philosophy had been most confused; he had barely escaped becoming a fire worshipper because of the great comfort derived from his accidental discovery of fire. Reason, however, directed him from his own discovery to the sun as a superior and more awe\hyp{}inspiring source of heat and light, but it was too remote, and so he failed to become a sun worshipper.
\vs p063 6:3 The Andonites early developed a fear of the elements --- thunder, lightning, rain, snow, hail, and ice. But hunger was the constantly recurring urge of these early days, and since they largely subsisted on animals, they eventually evolved a form of animal worship. To Andon, the larger food animals were symbols of creative might and sustaining power. From time to time it became the custom to designate various of these larger animals as objects of worship. During the vogue of a particular animal, crude outlines of it would be drawn on the walls of the caves, and later on, as continued progress was made in the arts, such an animal god was engraved on various ornaments.
\vs p063 6:4 Very early the Andonic peoples formed the habit of refraining from eating the flesh of the animal of tribal veneration. Presently, in order more suitably to impress the minds of their youths, they evolved a ceremony of reverence which was carried out about the body of one of these venerated animals; and still later on, this primitive performance developed into the more elaborate sacrificial ceremonies of their descendants. And this is the origin of sacrifices as a part of worship. This idea was elaborated by Moses in the Hebrew ritual and was preserved, in principle, by the Apostle Paul as the doctrine of atonement for sin by “the shedding of blood.”
\vs p063 6:5 That food was the all\hyp{}important thing in the lives of these primitive human beings is shown by the prayer taught these simple folks by Onagar, their great teacher. And this prayer was:
\vs p063 6:6 “O Breath of Life, give us this day our daily food, deliver us from the curse of the ice, save us from our forest enemies, and with mercy receive us into the Great Beyond.”
\vs p063 6:7 \pc Onagar maintained headquarters on the northern shores of the ancient Mediterranean in the region of the present Caspian Sea at a settlement called Oban, the tarrying place on the westward turning of the travel trail leading up northward from the Mesopotamian southland. From Oban he sent out teachers to the remote settlements to spread his new doctrines of one Deity and his concept of the hereafter, which he called the Great Beyond. These emissaries of Onagar were the world’s first missionaries; they were also the first human beings to cook meat, the first regularly to use fire in the preparation of food. They cooked flesh on the ends of sticks and also on hot stones; later on they roasted large pieces in the fire, but their descendants almost entirely reverted to the use of raw flesh.
\vs p063 6:8 Onagar was born 983,323 years ago (from A.D.\,1934), and he lived to be 69 years of age. The record of the achievements of this master mind and spiritual leader of the pre\hyp{}Planetary Prince days is a thrilling recital of the organization of these primitive peoples into a real society. He instituted an efficient tribal government, the like of which was not attained by succeeding generations in many millenniums. Never again, until the arrival of the Planetary Prince, was there such a high spiritual civilization on earth. These simple people had a real though primitive religion, but it was subsequently lost to their deteriorating descendants.
\vs p063 6:9 Although both Andon and Fonta had received Thought Adjusters, as had many of their descendants, it was not until the days of Onagar that the Adjusters and guardian seraphim came in great numbers to Urantia. This was, indeed, the golden age of primitive man.
\usection{7.\bibnobreakspace The Survival of Andon and Fonta}
\vs p063 7:1 Andon and Fonta, the splendid founders of the human race, received recognition at the time of the adjudication of Urantia upon the arrival of the Planetary Prince, and in due time they emerged from the regime of the mansion worlds with citizenship status on Jerusem. Although they have never been permitted to return to Urantia, they are cognizant of the history of the race they founded. They grieved over the Caligastia betrayal, sorrowed because of the Adamic failure, but rejoiced exceedingly when announcement was received that Michael had selected their world as the theatre for his final bestowal.
\vs p063 7:2 On Jerusem both Andon and Fonta were fused with their Thought Adjusters, as also were several of their children, including Sontad, but the majority of even their immediate descendants only achieved Spirit fusion.
\vs p063 7:3 Andon and Fonta, shortly after their arrival on Jerusem, received permission from the System Sovereign to return to the first mansion world to serve with the morontia personalities who welcome the pilgrims of time from Urantia to the heavenly spheres. And they have been assigned indefinitely to this service. They sought to send greetings to Urantia in connection with these revelations, but this request was wisely denied them.
\vs p063 7:4 \pc And this is the recital of the most heroic and fascinating chapter in all the history of Urantia, the story of the evolution, life struggles, death, and eternal survival of the unique parents of all mankind.
\vsetoff
\vs p063 7:5 [Presented by a Life Carrier resident on Urantia.]
