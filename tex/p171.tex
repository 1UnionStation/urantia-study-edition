\upaper{171}{On the Way to Jerusalem}
\author{Midwayer Commission}
\vs p171 0:1 The day after the memorable sermon on “The Kingdom of Heaven,” Jesus announced that on the following day he and the apostles would depart for the Passover at Jerusalem, visiting numerous cities in southern Perea on the way.
\vs p171 0:2 The address on the kingdom and the announcement that he was going to the Passover set all his followers to thinking that he was going up to Jerusalem to inaugurate the temporal kingdom of Jewish supremacy. No matter what Jesus said about the nonmaterial character of the kingdom, he could not wholly remove from the minds of his Jewish hearers the idea that the Messiah was to establish some kind of nationalistic government with headquarters at Jerusalem.
\vs p171 0:3 What Jesus said in his Sabbath sermon only tended to confuse the majority of his followers; very few were enlightened by the Master’s discourse. The leaders understood something of his teachings regarding the inner kingdom, \textcolour{ubdarkred}{“the kingdom of heaven within you,”} but they also knew that he had spoken about another and future kingdom, and it was this kingdom they believed he was now going up to Jerusalem to establish. When they were disappointed in this expectation, when he was rejected by the Jews, and later on, when Jerusalem was literally destroyed, they still clung to this hope, sincerely believing that the Master would soon return to the world in great power and majestic glory to establish the promised kingdom.
\vs p171 0:4 \pc It was on this Sunday afternoon that Salome the mother of James and John Zebedee came to Jesus with her two apostle sons and, in the manner of approaching an Oriental potentate, sought to have Jesus promise in advance to grant whatever request she might make. But the Master would not promise; instead, he asked her, \textcolour{ubdarkred}{“What do you want me to do for you?”} Then answered Salome: “Master, now that you are going up to Jerusalem to establish the kingdom, I would ask you in advance to promise me that these my sons shall have honour with you, the one to sit on your right hand and the other to sit on your left hand in your kingdom.”
\vs p171 0:5 When Jesus heard Salome’s request, he said: \textcolour{ubdarkred}{“Woman, you know not what you ask.”} And then, looking straight into the eyes of the two honour\hyp{}seeking apostles, he said: \textcolour{ubdarkred}{“Because I have long known and loved you; because I have even lived in your mother’s house; because Andrew has assigned you to be with me at all times; therefore do you permit your mother to come to me secretly, making this unseemly request. But let me ask you: Are you able to drink the cup I am about to drink?”} And without a moment for thought, James and John answered, “Yes, Master, we are able.” Said Jesus: \textcolour{ubdarkred}{“I am saddened that you know not why we go up to Jerusalem; I am grieved that you understand not the nature of my kingdom; I am disappointed that you bring your mother to make this request of me; but I know you love me in your hearts; therefore I declare that you shall indeed drink of my cup of bitterness and share in my humiliation, but to sit on my right hand and on my left hand is not mine to give. Such honours are reserved for those who have been designated by my Father.”}
\vs p171 0:6 By this time someone had carried word of this conference to Peter and the other apostles, and they were highly indignant that James and John would seek to be preferred before them, and that they would secretly go with their mother to make such a request. When they fell to arguing among themselves, Jesus called them all together and said: \textcolour{ubdarkred}{“You well understand how the rulers of the gentiles lord it over their subjects, and how those who are great exercise authority. But it shall not be so in the kingdom of heaven. Whosoever would be great among you, let him first become your servant. He who would be first in the kingdom, let him become your minister. I declare to you that the Son of Man came not to be ministered to but to minister; and I now go up to Jerusalem to lay down my life in the doing of the Father’s will and in the service of my brethren.”} When the apostles heard these words, they withdrew by themselves to pray. That evening, in response to the labours of Peter, James and John made suitable apologies to the ten and were restored to the good graces of their brethren.
\vs p171 0:7 In asking for places on the right hand and on the left hand of Jesus at Jerusalem, the sons of Zebedee little realized that in less than one month their beloved teacher would be hanging on a Roman cross with a dying thief on one side and another transgressor on the other side. And their mother, who was present at the crucifixion, well remembered the foolish request she had made of Jesus at Pella regarding the honours she so unwisely sought for her apostle sons.
\usection{1.\bibnobreakspace The Departure from Pella}
\vs p171 1:1 On the forenoon of Monday, March 13, Jesus and his 12 apostles took final leave of the Pella encampment, starting south on their tour of the cities of southern Perea, where Abner’s associates were at work. They spent more than two weeks visiting among the 70 and then went directly to Jerusalem for the Passover.
\vs p171 1:2 When the Master left Pella, the disciples encamped with the apostles, about 1,000 in number, followed after him. About one half of this group left him at the Jordan ford on the road to Jericho when they learned he was going over to Heshbon, and after he had preached the sermon on “Counting the Cost.” They went on up to Jerusalem, while the other half followed him for two weeks, visiting the towns in southern Perea.
\vs p171 1:3 In a general way, most of Jesus’ immediate followers understood that the camp at Pella had been abandoned, but they really thought this indicated that their Master at last intended to go to Jerusalem and lay claim to David’s throne. A large majority of his followers never were able to grasp any other concept of the kingdom of heaven; no matter what he taught them, they would not give up this Jewish idea of the kingdom.
\vs p171 1:4 Acting on the instructions of the Apostle Andrew, David Zebedee closed the visitors’ camp at Pella on Wednesday, March 15. At this time almost 4,000 visitors were in residence, and this does not include the 1,000 and more persons who sojourned with the apostles at what was known as the teachers’ camp, and who went south with Jesus and the 12. Much as David disliked to do it, he sold the entire equipment to numerous buyers and proceeded with the funds to Jerusalem, subsequently turning the money over to Judas Iscariot.
\vs p171 1:5 \pc David was present in Jerusalem during the tragic last week, taking his mother back with him to Bethsaida after the crucifixion. While awaiting Jesus and the apostles, David stopped with Lazarus at Bethany and became tremendously agitated by the manner in which the Pharisees had begun to persecute and harass him since his resurrection. Andrew had directed David to discontinue the messenger service; and this was construed by all as an indication of the early establishment of the kingdom at Jerusalem. David found himself without a job, and he had about decided to become the self\hyp{}appointed defender of Lazarus when presently the object of his indignant solicitude fled in haste to Philadelphia. Accordingly, sometime after the resurrection and also after the death of his mother, David betook himself to Philadelphia, having first assisted Martha and Mary in disposing of their real estate; and there, in association with Abner and Lazarus, he spent the remainder of his life, becoming the financial overseer of all those large interests of the kingdom which had their centre at Philadelphia during the lifetime of Abner.
\vs p171 1:6 Within a short time after the destruction of Jerusalem, Antioch became the headquarters of \bibemph{Pauline Christianity,} while Philadelphia remained the centre of the \bibemph{Abnerian kingdom of heaven.} From Antioch the Pauline version of the teachings of Jesus and about Jesus spread to all the Western world; from Philadelphia the missionaries of the Abnerian version of the kingdom of heaven spread throughout Mesopotamia and Arabia until the later times when these uncompromising emissaries of the teachings of Jesus were overwhelmed by the sudden rise of Islam.
\usection{2.\bibnobreakspace On Counting the Cost}
\vs p171 2:1 When Jesus and the company of almost 1,000 followers arrived at the Bethany ford of the Jordan sometimes called Bethabara, his disciples began to realize that he was not going directly to Jerusalem. While they hesitated and debated among themselves, Jesus climbed upon a huge stone and delivered that discourse which has become known as “Counting the Cost.” The Master said:
\vs p171 2:2 \pc \textcolour{ubdarkred}{“You who would follow after me from this time on, must be willing to pay the price of wholehearted dedication to the doing of my Father’s will. If you would be my disciples, you must be willing to forsake father, mother, wife, children, brothers, and sisters. If any one of you would now be my disciple, you must be willing to give up even your life just as the Son of Man is about to offer up his life for the completion of the mission of doing the Father’s will on earth and in the flesh.}
\vs p171 2:3 \textcolour{ubdarkred}{“If you are not willing to pay the full price, you can hardly be my disciple. Before you go further, you should each sit down and count the cost of being my disciple. Which one of you would undertake to build a watchtower on your lands without first sitting down to count up the cost to see whether you had money enough to complete it? If you fail thus to reckon the cost, after you have laid the foundation, you may discover that you are unable to finish that which you have begun, and therefore will all your neighbours mock you, saying, ‘Behold, this man began to build but was unable to finish his work.’ Again, what king, when he prepares to make war upon another king, does not first sit down and take counsel as to whether he will be able, with 10,000 men, to meet him who comes against him with 20,000? If the king cannot afford to meet his enemy because he is unprepared, he sends an embassy to this other king, even when he is yet a great way off, asking for terms of peace.}
\vs p171 2:4 \textcolour{ubdarkred}{“Now, then, must each of you sit down and count the cost of being my disciple. From now on you will not be able to follow after us, listening to the teaching and beholding the works; you will be required to face bitter persecutions and to bear witness for this gospel in the face of crushing disappointment. If you are unwilling to renounce all that you are and to dedicate all that you have, then are you unworthy to be my disciple. If you have already conquered yourself within your own heart, you need have no fear of that outward victory which you must presently gain when the Son of Man is rejected by the chief priests and the Sadducees and is given into the hands of mocking unbelievers.}
\vs p171 2:5 \textcolour{ubdarkred}{“Now should you examine yourself to find out your motive for being my disciple. If you seek honour and glory, if you are worldly minded, you are like the salt when it has lost its savour. And when that which is valued for its saltiness has lost its savour, wherewith shall it be seasoned? Such a condiment is useless; it is fit only to be cast out among the refuse. Now have I warned you to turn back to your homes in peace if you are not willing to drink with me the cup which is being prepared. Again and again have I told you that my kingdom is not of this world, but you will not believe me. He who has ears to hear let him hear what I say.”}
\vs p171 2:6 \pc Immediately after speaking these words, Jesus, leading the 12, started off on the way to Heshbon, followed by about 500. After a brief delay the other half of the multitude went on up to Jerusalem. His apostles, together with the leading disciples, thought much about these words, but still they clung to the belief that, after this brief period of adversity and trial, the kingdom would certainly be set up somewhat in accordance with their long\hyp{}cherished hopes.
\usection{3.\bibnobreakspace The Perean Tour}
\vs p171 3:1 For more than two weeks Jesus and the 12, followed by a crowd of several hundred disciples, journeyed about in southern Perea, visiting all of the towns wherein the 70 laboured. Many gentiles lived in this region, and since few were going up to the Passover feast at Jerusalem, the messengers of the kingdom went right on with their work of teaching and preaching.
\vs p171 3:2 Jesus met Abner at Heshbon, and Andrew directed that the labours of the 70 should not be interrupted by the Passover feast; Jesus advised that the messengers should go forward with their work in complete disregard of what was about to happen at Jerusalem. He also counselled Abner to permit the women’s corps, at least such as desired, to go to Jerusalem for the Passover. And this was the last time Abner ever saw Jesus in the flesh. His farewell to Abner was: \textcolour{ubdarkred}{“My son, I know you will be true to the kingdom, and I pray the Father to grant you wisdom that you may love and understand your brethren.”}
\vs p171 3:3 As they travelled from city to city, large numbers of their followers deserted to go on to Jerusalem so that, by the time Jesus started for the Passover, the number of those who followed along with him day by day had dwindled to less than 200.
\vs p171 3:4 The apostles understood that Jesus was going to Jerusalem for the Passover. They knew that the Sanhedrin had broadcast a message to all Israel that he had been condemned to die and directing that anyone knowing his whereabouts should inform the Sanhedrin; and yet, despite all this, they were not so alarmed as they had been when he had announced to them in Philadelphia that he was going to Bethany to see Lazarus. This change of attitude from that of intense fear to a state of hushed expectancy was mostly because of Lazarus’s resurrection. They had reached the conclusion that Jesus might, in an emergency, assert his divine power and put to shame his enemies. This hope, coupled with their more profound and mature faith in the spiritual supremacy of their Master, accounted for the outward courage displayed by his immediate followers, who now made ready to follow him into Jerusalem in the very face of the open declaration of the Sanhedrin that he must die.
\vs p171 3:5 The majority of the apostles and many of his inner disciples did not believe it possible for Jesus to die; they, believing that he was “the resurrection and the life,” regarded him as immortal and already triumphant over death.
\usection{4.\bibnobreakspace Teaching at Livias}
\vs p171 4:1 On Wednesday evening, March 29, Jesus and his followers encamped at Livias on their way to Jerusalem, after having completed their tour of the cities of southern Perea. It was during this night at Livias that Simon Zelotes and Simon Peter, having conspired to have delivered into their hands at this place more than 100 swords, received and distributed these arms to all who would accept them and wear them concealed beneath their cloaks. Simon Peter was still wearing his sword on the night of the Master’s betrayal in the garden.
\vs p171 4:2 Early on Thursday morning before the others were awake, Jesus called Andrew and said: \textcolour{ubdarkred}{“Awaken your brethren! I have something to say to them.”} Jesus knew about the swords and which of his apostles had received and were wearing these weapons, but he never disclosed to them that he knew such things. When Andrew had aroused his associates, and they had assembled off by themselves, Jesus said: \textcolour{ubdarkred}{“My children, you have been with me a long while, and I have taught you much that is needful for this time, but I would now warn you not to put your trust in the uncertainties of the flesh nor in the frailties of man’s defence against the trials and testing which lie ahead of us. I have called you apart here by yourselves that I may once more plainly tell you that we are going up to Jerusalem, where you know the Son of Man has already been condemned to death. Again am I telling you that the Son of Man will be delivered into the hands of the chief priests and the religious rulers; that they will condemn him and then deliver him into the hands of the gentiles. And so will they mock the Son of Man, even spit upon him and scourge him, and they will deliver him up to death. And when they kill the Son of Man, be not dismayed, for I declare that on the third day he shall rise. Take heed to yourselves and remember that I have forewarned you.”}
\vs p171 4:3 Again were the apostles amazed, stunned; but they could not bring themselves to regard his words as literal; they could not comprehend that the Master meant just what he said. They were so blinded by their persistent belief in the temporal kingdom on earth, with headquarters at Jerusalem, that they simply could not --- would not --- permit themselves to accept Jesus’ words as literal. They pondered all that day as to what the Master could mean by such strange pronouncements. But none of them dared to ask him a question concerning these statements. Not until after his death did these bewildered apostles wake up to the realization that the Master had spoken to them plainly and directly in anticipation of his crucifixion.
\vs p171 4:4 \pc It was here at Livias, just after breakfast, that certain friendly Pharisees came to Jesus and said: “Flee in haste from these parts, for Herod, just as he sought John, now seeks to kill you. He fears an uprising of the people and has decided to kill you. We bring you this warning that you may escape.”
\vs p171 4:5 And this was partly true. The resurrection of Lazarus frightened and alarmed Herod, and knowing that the Sanhedrin had dared to condemn Jesus, even in advance of a trial, Herod made up his mind either to kill Jesus or to drive him out of his domains. He really desired to do the latter since he so feared him that he hoped he would not be compelled to execute him.
\vs p171 4:6 When Jesus heard what the Pharisees had to say, he replied: \textcolour{ubdarkred}{“I well know about Herod and his fear of this gospel of the kingdom. But, mistake not, he would much prefer that the Son of Man go up to Jerusalem to suffer and die at the hands of the chief priests; he is not anxious, having stained his hands with the blood of John, to become responsible for the death of the Son of Man. Go you and tell that fox that the Son of Man preaches in Perea today, tomorrow goes into Judea, and after a few days, will be perfected in his mission on earth and prepared to ascend to the Father.”}
\vs p171 4:7 Then turning to his apostles, Jesus said: \textcolour{ubdarkred}{“From olden times the prophets have perished in Jerusalem, and it is only befitting that the Son of Man should go up to the city of the Father’s house to be offered up as the price of human bigotry and as the result of religious prejudice and spiritual blindness. O Jerusalem, Jerusalem, which kills the prophets and stones the teachers of truth! How often would I have gathered your children together even as a hen gathers her own brood under her wings, but you would not let me do it! Behold, your house is about to be left to you desolate! You will many times desire to see me, but you shall not. You will then seek but not find me.”} And when he had spoken, he turned to those around him and said: \textcolour{ubdarkred}{“Nevertheless, let us go up to Jerusalem to attend the Passover and do that which becomes us in fulfilling the will of the Father in heaven.”}
\vs p171 4:8 \pc It was a confused and bewildered group of believers who this day followed Jesus into Jericho. The apostles could discern only the certain note of final triumph in Jesus’ declarations regarding the kingdom; they just could not bring themselves to that place where they were willing to grasp the warnings of the impending setback. When Jesus spoke of \textcolour{ubdarkred}{“rising on the third day,”} they seized upon this statement as signifying a sure triumph of the kingdom immediately following an unpleasant preliminary skirmish with the Jewish religious leaders. The “third day” was a common Jewish expression signifying “presently” or “soon thereafter.” When Jesus spoke of \textcolour{ubdarkred}{“rising,”} they thought he referred to the “rising of the kingdom.”
\vs p171 4:9 Jesus had been accepted by these believers as the Messiah, and the Jews knew little or nothing about a suffering Messiah. They did not understand that Jesus was to accomplish many things by his death which could never have been achieved by his life. While it was the resurrection of Lazarus that nerved the apostles to enter Jerusalem, it was the memory of the transfiguration that sustained the Master at this trying period of his bestowal.
\usection{5.\bibnobreakspace The Blind Man at Jericho}
\vs p171 5:1 Late on the afternoon of Thursday, March 30, Jesus and his apostles, at the head of a band of about 200 followers, approached the walls of Jericho. As they came near the gate of the city, they encountered a throng of beggars, among them one Bartimeus, an elderly man who had been blind from his youth. This blind beggar had heard much about Jesus and knew all about his healing of the blind Josiah at Jerusalem. He had not known of Jesus’ last visit to Jericho until he had gone on to Bethany. Bartimeus had resolved that he would never again allow Jesus to visit Jericho without appealing to him for the restoration of his sight.
\vs p171 5:2 News of Jesus’ approach had been heralded throughout Jericho, and hundreds of the inhabitants flocked forth to meet him. When this great crowd came back escorting the Master into the city, Bartimeus, hearing the heavy tramping of the multitude, knew that something unusual was happening, and so he asked those standing near him what was going on. And one of the beggars replied, “Jesus of Nazareth is passing by.” When Bartimeus heard that Jesus was near, he lifted up his voice and began to cry aloud, “Jesus, Jesus, have mercy upon me!” And as he continued to cry louder and louder, some of those near to Jesus went over and rebuked him, requesting him to hold his peace; but it was of no avail; he cried only the more and the louder.
\vs p171 5:3 When Jesus heard the blind man crying out, he stood still. And when he saw him, he said to his friends, \textcolour{ubdarkred}{“Bring the man to me.”} And then they went over to Bartimeus, saying: “Be of good cheer; come with us, for the Master calls for you.” When Bartimeus heard these words, he threw aside his cloak, springing forward toward the centre of the road, while those near by guided him to Jesus. Addressing Bartimeus, Jesus said: \textcolour{ubdarkred}{“What do you want me to do for you?”} Then answered the blind man, “I would have my sight restored.” And when Jesus heard this request and saw his faith, he said: \textcolour{ubdarkred}{“You shall receive your sight; go your way; your faith has made you whole.”} Immediately he received his sight, and he remained near Jesus, glorifying God, until the Master started on the next day for Jerusalem, and then he went before the multitude declaring to all how his sight had been restored in Jericho.\tunemarkup{private}{\begin{figure}[H]\centering\includegraphics[scale=\tunemarkup{pgkobomini}{0.56}\tunemarkup{pgkoboaurahd}{0.65}\tunemarkup{pghanlin}{0.56}\tunemarkup{pgnexus7}{0.58}\tunemarkup{pgkindledx}{0.55}]{../urantia-pictures/Harold_Copping_The_Healing_of_the_Blind_Bartimaeus_525.jpg}\caption{The Healing of Blind Bartimaeus by Harold~Copping}\end{figure}}
\usection{6.\bibnobreakspace The Visit to Zaccheus}
\vs p171 6:1 When the Master’s procession entered Jericho, it was nearing sundown, and he was minded to abide there for the night. As Jesus passed by the customs house, Zaccheus the chief publican, or tax collector, happened to be present, and he much desired to see Jesus. This chief publican was very rich and had heard much about this prophet of Galilee. He had resolved that he would see what sort of a man Jesus was the next time he chanced to visit Jericho; accordingly, Zaccheus sought to press through the crowd, but it was too great, and being short of stature, he could not see over their heads. And so the chief publican followed on with the crowd until they came near the centre of the city and not far from where he lived. When he saw that he would be unable to penetrate the crowd, and thinking that Jesus might be going right on through the city without stopping, he ran on ahead and climbed up into a sycamore tree whose spreading branches overhung the roadway. He knew that in this way he could obtain a good view of the Master as he passed by. And he was not disappointed, for, as Jesus passed by, he stopped and, looking up at Zaccheus, said: \textcolour{ubdarkred}{“Make haste, Zaccheus, and come down, for tonight I must abide at your house.”} And when Zaccheus heard these astonishing words, he almost fell out of the tree in his haste to get down, and going up to Jesus, he expressed great joy that the Master should be willing to stop at his house.
\vs p171 6:2 They went at once to the home of Zaccheus, and those who lived in Jericho were much surprised that Jesus would consent to abide with the chief publican. Even while the Master and his apostles lingered with Zaccheus before the door of his house, one of the Jericho Pharisees, standing near by, said: “You see how this man has gone to lodge with a sinner, an apostate son of Abraham who is an extortioner and a robber of his own people.” And when Jesus heard this, he looked down at Zaccheus and smiled. Then Zaccheus stood upon a stool and said: “Men of Jericho, hear me! I may be a publican and a sinner, but the great Teacher has come to abide in my house; and before he goes in, I tell you that I am going to bestow one half of all my goods upon the poor, and beginning tomorrow, if I have wrongfully exacted aught from any man, I will restore fourfold. I am going to seek salvation with all my heart and learn to do righteousness in the sight of God.”
\vs p171 6:3 When Zaccheus had ceased speaking, Jesus said: \textcolour{ubdarkred}{“Today has salvation come to this home, and you have become indeed a son of Abraham.”} And turning to the crowd assembled about them, Jesus said: \textcolour{ubdarkred}{“And marvel not at what I say nor take offence at what we do, for I have all along declared that the Son of Man has come to seek and to save that which is lost.”}
\vs p171 6:4 They lodged with Zaccheus for the night. On the morrow they arose and made their way up the “road of robbers” to Bethany on their way to the Passover at Jerusalem.
\usection{7.\bibnobreakspace “As Jesus Passed By”}
\vs p171 7:1 Jesus spread good cheer everywhere he went. He was full of grace and truth. His associates never ceased to wonder at the gracious words that proceeded out of his mouth. You can cultivate gracefulness, but graciousness is the aroma of friendliness which emanates from a love\hyp{}saturated soul.
\vs p171 7:2 Goodness always compels respect, but when it is devoid of grace, it often repels affection. Goodness is universally attractive only when it is gracious. Goodness is effective only when it is attractive.
\vs p171 7:3 Jesus really understood men; therefore could he manifest genuine sympathy and show sincere compassion. But he seldom indulged in pity. While his compassion was boundless, his sympathy was practical, personal, and constructive. Never did his familiarity with suffering breed indifference, and he was able to minister to distressed souls without increasing their self\hyp{}pity.
\vs p171 7:4 Jesus could help men so much because he loved them so sincerely. He truly loved each man, each woman, and each child. He could be such a true friend because of his remarkable insight --- he knew so fully what was in the heart and in the mind of man. He was an interested and keen observer. He was an expert in the comprehension of human need, clever in detecting human longings.
\vs p171 7:5 Jesus was never in a hurry. He had time to comfort his fellow men “as he passed by.” And he always made his friends feel at ease. He was a charming listener. He never engaged in the meddlesome probing of the souls of his associates. As he comforted hungry minds and ministered to thirsty souls, the recipients of his mercy did not so much feel that they were confessing \bibemph{to} him as that they were conferring \bibemph{with} him. They had unbounded confidence in him because they saw he had so much faith in them.
\vs p171 7:6 He never seemed to be curious about people, and he never manifested a desire to direct, manage, or follow them up. He inspired profound self\hyp{}confidence and robust courage in all who enjoyed his association. When he smiled on a man, that mortal experienced increased capacity for solving his manifold problems.
\vs p171 7:7 Jesus loved men so much and so wisely that he never hesitated to be severe with them when the occasion demanded such discipline. He frequently set out to help a person by asking for help. In this way he elicited interest, appealed to the better things in human nature.
\vs p171 7:8 The Master could discern saving faith in the gross superstition of the woman who sought healing by touching the hem of his garment. He was always ready and willing to stop a sermon or detain a multitude while he ministered to the needs of a single person, even to a little child. Great things happened not only because people had faith in Jesus, but also because Jesus had so much faith in them.
\vs p171 7:9 Most of the really important things which Jesus said or did seemed to happen casually, “as he passed by.” There was so little of the professional, the well\hyp{}planned, or the premeditated in the Master’s earthly ministry. He dispensed health and scattered happiness naturally and gracefully as he journeyed through life. It was literally true, “He went about doing good.”
\vs p171 7:10 And it behoves the Master’s followers in all ages to learn to minister as “they pass by” --- to do unselfish good as they go about their daily duties.
\usection{8.\bibnobreakspace Parable of the Pounds}
\vs p171 8:1 They did not start from Jericho until near noon since they sat up late the night before while Jesus taught Zaccheus and his family the gospel of the kingdom. About halfway up the ascending road to Bethany the party paused for lunch while the multitude passed on to Jerusalem, not knowing that Jesus and the apostles were going to abide that night on the Mount of Olives.
\vs p171 8:2 The parable of the pounds, unlike the parable of the talents, which was intended for all the disciples, was spoken more exclusively to the apostles and was largely based on the experience of Archelaus and his futile attempt to gain the rule of the kingdom of Judea. This is one of the few parables of the Master to be founded on an actual historic character. It was not strange that they should have had Archelaus in mind inasmuch as the house of Zaccheus in Jericho was very near the ornate palace of Archelaus, and his aqueduct ran along the road by which they had departed from Jericho.
\vs p171 8:3 \pc Said Jesus: \textcolour{ubdarkred}{“You think that the Son of Man goes up to Jerusalem to receive a kingdom, but I declare that you are doomed to disappointment. Do you not remember about a certain prince who went into a far country to receive for himself a kingdom, but even before he could return, the citizens of his province, who in their hearts had already rejected him, sent an embassy after him, saying, ‘We will not have this man to reign over us’? As this king was rejected in the temporal rule, so is the Son of Man to be rejected in the spiritual rule. Again I declare that my kingdom is not of this world; but if the Son of Man had been accorded the spiritual rule of his people, he would have accepted such a kingdom of men’s souls and would have reigned over such a dominion of human hearts. Notwithstanding that they reject my spiritual rule over them, I will return again to receive from others such a kingdom of spirit as is now denied me. You will see the Son of Man rejected now, but in another age that which the children of Abraham now reject will be received and exalted.}
\vs p171 8:4 \textcolour{ubdarkred}{“And now, as the rejected nobleman of this parable, I would call before me my 12 servants, special stewards, and giving into each of your hands the sum of one pound, I would admonish each to heed well my instructions that you trade diligently with your trust fund while I am away that you may have wherewith to justify your stewardship when I return, when a reckoning shall be required of you.}
\vs p171 8:5 \textcolour{ubdarkred}{“And even if this rejected Son should not return, another Son will be sent to receive this kingdom, and this Son will then send for all of you to receive your report of stewardship and to be made glad by your gains.}
\vs p171 8:6 \textcolour{ubdarkred}{“And when these stewards were subsequently called together for an accounting, the first came forward, saying, ‘Lord, with your pound I have made ten pounds more.’ And his master said to him: ‘Well done; you are a good servant; because you have proved faithful in this matter, I will give you authority over ten cities.’ And the second came, saying, ‘Your pound left with me, Lord, has made five pounds.’ And the master said, ‘I will accordingly make you ruler over five cities.’ And so on down through the others until the last of the servants, on being called to account, reported: ‘Lord, behold, here is your pound, which I have kept safely done up in this napkin. And this I did because I feared you; I believed that you were unreasonable, seeing that you take up where you have not laid down, and that you seek to reap where you have not sown.’ Then said his lord: ‘You negligent and unfaithful servant, I will judge you out of your own mouth. You knew that I reap where I have apparently not sown; therefore you knew this reckoning would be required of you. Knowing this, you should have at least given my money to the banker that at my coming I might have had it with proper interest.’}
\vs p171 8:7 \textcolour{ubdarkred}{“And then said this ruler to those who stood by: ‘Take the money from this slothful servant and give it to him who has ten pounds.’ And when they reminded the master that such a one already had ten pounds, he said: ‘To every one who has shall be given more, but from him who has not, even that which he has shall be taken away from him.’”}
\vs p171 8:8 \pc And then the apostles sought to know the difference between the meaning of this parable and that of the former parable of the talents, but Jesus would only say, in answer to their many questions: \textcolour{ubdarkred}{“Ponder well these words in your hearts while each of you finds out their true meaning.”}
\vs p171 8:9 It was Nathaniel who so well taught the meaning of these two parables in the after years, summing up his teachings in these conclusions:
\vs p171 8:10 \ublistelem{1.}\bibnobreakspace Ability is the practical measure of life’s opportunities. You will never be held responsible for the accomplishment of that which is beyond your abilities.
\vs p171 8:11 \ublistelem{2.}\bibnobreakspace Faithfulness is the unerring measure of human trustworthiness. He who is faithful in little things is also likely to exhibit faithfulness in everything consistent with his endowments.
\vs p171 8:12 \ublistelem{3.}\bibnobreakspace The Master grants the lesser reward for lesser faithfulness when there is like opportunity.
\vs p171 8:13 \ublistelem{4.}\bibnobreakspace He grants a like reward for like faithfulness when there is lesser opportunity.
\vs p171 8:14 \pc When they had finished their lunch, and after the multitude of followers had gone on toward Jerusalem, Jesus, standing there before the apostles in the shade of an overhanging rock by the roadside, with cheerful dignity and a gracious majesty pointed his finger westward, saying: \textcolour{ubdarkred}{“Come, my brethren, let us go on into Jerusalem, there to receive that which awaits us; thus shall we fulfil the will of the heavenly Father in all things.”}
\vs p171 8:15 And so Jesus and his apostles resumed this, the Master’s last journey to Jerusalem in the likeness of the flesh of mortal man.
\quizlink
