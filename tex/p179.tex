\upaper{179}{The Last Supper}
\author{Midwayer Commission}
\vs p179 0:1 During the afternoon of this Thursday, when Philip reminded the Master about the approaching Passover and inquired concerning his plans for its celebration, he had in mind the Passover supper which was due to be eaten on the evening of the next day, Friday. It was the custom to begin the preparations for the celebration of the Passover not later than noon of the preceding day. And since the Jews reckoned the day as beginning at sunset, this meant that Saturday’s Passover supper would be eaten on Friday night, sometime before the midnight hour.
\vs p179 0:2 The apostles were, therefore, entirely at a loss to understand the Master’s announcement that they would celebrate the Passover one day early. They thought, at least some of them did, that he knew he would be placed under arrest before the time of the Passover supper on Friday night and was therefore calling them together for a special supper on this Thursday evening. Others thought that this was merely a special occasion which was to precede the regular Passover celebration.
\vs p179 0:3 The apostles knew that Jesus had celebrated other Passovers without the lamb; they knew that he did not personally participate in any sacrificial service of the Jewish system. He had many times partaken of the paschal lamb as a guest, but always, when he was the host, no lamb was served. It would not have been a great surprise to the apostles to have seen the lamb omitted even on Passover night, and since this supper was given one day earlier, they thought nothing of its absence.
\vs p179 0:4 After receiving the greetings of welcome extended by the father and mother of John Mark, the apostles went immediately to the upper chamber while Jesus lingered behind to talk with the Mark family.
\vs p179 0:5 It had been understood beforehand that the Master was to celebrate this occasion alone with his 12 apostles; therefore no servants were provided to wait upon them.
\usection{1.\bibnobreakspace The Desire for Preference}
\vs p179 1:1 When the apostles had been shown upstairs by John Mark, they beheld a large and commodious chamber, which was completely furnished for the supper, and observed that the bread, wine, water, and herbs were all in readiness on one end of the table. Except for the end on which rested the bread and wine, this long table was surrounded by 13 reclining couches, just such as would be provided for the celebration of the Passover in a well\hyp{}to\hyp{}do Jewish household.
\vs p179 1:2 As the 12 entered this upper chamber, they noticed, just inside the door, the pitchers of water, the basins, and towels for laving their dusty feet; and since no servant had been provided to render this service, the apostles began to look at one another as soon as John Mark had left them, and each began to think within himself, Who shall wash our feet? And each likewise thought that it would not be he who would thus seem to act as the servant of the others.
\vs p179 1:3 As they stood there, debating in their hearts, they surveyed the seating arrangement of the table, taking note of the higher divan of the host with one couch on the right and 11 arranged around the table on up to opposite this second seat of honour on the host’s right.
\vs p179 1:4 They expected the Master to arrive any moment, but they were in a quandary as to whether they should seat themselves or await his coming and depend on him to assign them their places. While they hesitated, Judas stepped over to the seat of honour, at the left of the host, and signified that he intended there to recline as the preferred guest. This act of Judas immediately stirred up a heated dispute among the other apostles. Judas had no sooner seized the seat of honour than John Zebedee laid claim to the next preferred seat, the one on the right of the host. Simon Peter was so enraged at this assumption of choice positions by Judas and John that, as the other angry apostles looked on, he marched clear around the table and took his place on the lowest couch, the end of the seating order and just opposite to that chosen by John Zebedee. Since others had seized the high seats, Peter thought to choose the lowest, and he did this, not merely in protest against the unseemly pride of his brethren, but with the hope that Jesus, when he should come and see him in the place of least honour, would call him up to a higher one, thus displacing one who had presumed to honour himself.
\vs p179 1:5 With the highest and the lowest positions thus occupied, the rest of the apostles chose places, some near Judas and some near Peter, until all were located. They were seated about the U\hyp{}shaped table on these reclining divans in the following order: on the right of the Master, John; on the left, Judas, Simon Zelotes, Matthew, James Zebedee, Andrew, the Alpheus twins, Philip, Nathaniel, Thomas, and Simon Peter.
\vs p179 1:6 \pc They are gathered together to celebrate, at least in spirit, an institution which antedated even Moses and referred to the times when their fathers were slaves in Egypt. This supper is their last rendezvous with Jesus, and even in such a solemn setting, under the leadership of Judas the apostles are led once more to give way to their old predilection for honour, preference, and personal exaltation.
\vs p179 1:7 \pc They were still engaged in voicing angry recriminations when the Master appeared in the doorway, where he hesitated a moment as a look of disappointment slowly crept over his face. Without comment he went to his place, and he did not disturb their seating arrangement.
\vs p179 1:8 They were now ready to begin the supper, except that their feet were still unwashed, and they were in anything but a pleasant frame of mind. When the Master arrived, they were still engaged in making uncomplimentary remarks about one another, to say nothing of the thoughts of some who had sufficient emotional control to refrain from publicly expressing their feelings.
\usection{2.\bibnobreakspace Beginning the Supper}
\vs p179 2:1 For a few moments after the Master had gone to his place, not a word was spoken. Jesus looked them all over and, relieving the tension with a smile, said: \textcolour{ubdarkred}{“I have greatly desired to eat this Passover with you. I wanted to eat with you once more before I suffered, and realizing that my hour has come, I arranged to have this supper with you tonight, for, as concerns the morrow, we are all in the hands of the Father, whose will I have come to execute. I shall not again eat with you until you sit down with me in the kingdom which my Father will give me when I have finished that for which he sent me into this world.”}
\vs p179 2:2 After the wine and the water had been mixed, they brought the cup to Jesus, who, when he had received it from the hand of Thaddeus, held it while he offered thanks. And when he had finished offering thanks, he said: \textcolour{ubdarkred}{“Take this cup and divide it among yourselves and, when you partake of it, realize that I shall not again drink with you the fruit of the vine since this is our last supper. When we sit down again in this manner, it will be in the kingdom to come.”}
\vs p179 2:3 Jesus began thus to talk to his apostles because he knew that his hour had come. He understood that the time had come when he was to return to the Father, and that his work on earth was almost finished. The Master knew he had revealed the Father’s love on earth and had shown forth his mercy to mankind, and that he had completed that for which he came into the world, even to the receiving of all power and authority in heaven and on earth. Likewise, he knew Judas Iscariot had fully made up his mind to deliver him that night into the hands of his enemies. He fully realized that this traitorous betrayal was the work of Judas, but that it also pleased Lucifer, Satan, and Caligastia the prince of darkness. But he feared none of those who sought his spiritual overthrow any more than he feared those who sought to accomplish his physical death. The Master had but one anxiety, and that was for the safety and salvation of his chosen followers. And so, with the full knowledge that the Father had put all things under his authority, the Master now prepared to enact the parable of brotherly love.
\usection{3.\bibnobreakspace Washing the Apostles’ Feet}
\vs p179 3:1 After drinking the first cup of the Passover, it was the Jewish custom for the host to arise from the table and wash his hands. Later on in the meal and after the second cup, all of the guests likewise rose up and washed their hands. Since the apostles knew that their Master never observed these rites of ceremonial hand washing, they were very curious to know what he intended to do when, after they had partaken of this first cup, he arose from the table and silently made his way over to near the door, where the water pitchers, basins, and towels had been placed. And their curiosity grew into astonishment as they saw the Master remove his outer garment, gird himself with a towel, and begin to pour water into one of the foot basins. Imagine the amazement of these 12 men, who had so recently refused to wash one another’s feet, and who had engaged in such unseemly disputes about positions of honour at the table, when they saw him make his way around the unoccupied end of the table to the lowest seat of the feast, where Simon Peter reclined, and, kneeling down in the attitude of a servant, make ready to wash Simon’s feet. As the Master knelt, all 12 arose as one man to their feet; even the traitorous Judas so far forgot his infamy for a moment as to arise with his fellow apostles in this expression of surprise, respect, and utter amazement.
\vs p179 3:2 There stood Simon Peter, looking down into the upturned face of his Master. Jesus said nothing; it was not necessary that he should speak. His attitude plainly revealed that he was minded to wash Simon Peter’s feet. Notwithstanding his frailties of the flesh, Peter loved the Master. This Galilean fisherman was the first human being wholeheartedly to believe in the divinity of Jesus \bibemph{and} to make full and public confession of that belief. And Peter had never since really doubted the divine nature of the Master. Since Peter so revered and honoured Jesus in his heart, it was not strange that his soul resented the thought of Jesus’ kneeling there before him in the attitude of a menial servant and proposing to wash his feet as would a slave. When Peter presently collected his wits sufficiently to address the Master, he spoke the heart feelings of all his fellow apostles.
\vs p179 3:3 After a few moments of this great embarrassment, Peter said, “Master, do you really mean to wash my feet?” And then, looking up into Peter’s face, Jesus said: \textcolour{ubdarkred}{“You may not fully understand what I am about to do, but hereafter you will know the meaning of all these things.”} Then Simon Peter, drawing a long breath, said, “Master, you shall never wash my feet!” And each of the apostles nodded their approval of Peter’s firm declaration of refusal to allow Jesus thus to humble himself before them.
\vs p179 3:4 The dramatic appeal of this unusual scene at first touched the heart of even Judas Iscariot; but when his vainglorious intellect passed judgment upon the spectacle, he concluded that this gesture of humility was just one more episode which conclusively proved that Jesus would never qualify as Israel’s deliverer, and that he had made no mistake in the decision to desert the Master’s cause.\tunemarkup{private}{\begin{figure}[H]\centering\includegraphics[width=\columnwidth]{images/Jesus-washing-feet.jpg}\caption{The Master Teacher by Del~Parson}\end{figure}}
\vs p179 3:5 As they all stood there in breathless amazement, Jesus said: \textcolour{ubdarkred}{“Peter, I declare that, if I do not wash your feet, you will have no part with me in that which I am about to perform.”} When Peter heard this declaration, coupled with the fact that Jesus continued kneeling there at his feet, he made one of those decisions of blind acquiescence in compliance with the wish of one whom he respected and loved. As it began to dawn on Simon Peter that there was attached to this proposed enactment of service some signification that determined one’s future connection with the Master’s work, he not only became reconciled to the thought of allowing Jesus to wash his feet but, in his characteristic and impetuous manner, said: “Then, Master, wash not my feet only but also my hands and my head.”
\vs p179 3:6 As the Master made ready to begin washing Peter’s feet, he said: \textcolour{ubdarkred}{“He who is already clean needs only to have his feet washed. You who sit with me tonight are clean --- but not all. But the dust of your feet should have been washed away before you sat down at meat with me. And besides, I would perform this service for you as a parable to illustrate the meaning of a new commandment which I will presently give you.”}
\vs p179 3:7 In like manner the Master went around the table, in silence, washing the feet of his 12 apostles, not even passing by Judas. When Jesus had finished washing the feet of the 12, he donned his cloak, returned to his place as host, and after looking over his bewildered apostles, said:
\vs p179 3:8 \pc \textcolour{ubdarkred}{“Do you really understand what I have done to you? You call me Master, and you say well, for so I am. If, then, the Master has washed your feet, why was it that you were unwilling to wash one another’s feet? What lesson should you learn from this parable in which the Master so willingly does that service which his brethren were unwilling to do for one another? Verily, verily, I say to you: A servant is not greater than his master; neither is one who is sent greater than he who sends him. You have seen the way of service in my life among you, and blessed are you who will have the gracious courage so to serve. But why are you so slow to learn that the secret of greatness in the spiritual kingdom is not like the methods of power in the material world?}
\vs p179 3:9 \textcolour{ubdarkred}{“When I came into this chamber tonight, you were not content proudly to refuse to wash one another’s feet, but you must also fall to disputing among yourselves as to who should have the places of honour at my table. Such honours the Pharisees and the children of this world seek, but it should not be so among the ambassadors of the heavenly kingdom. Do you not know that there can be no place of preferment at my table? Do you not understand that I love each of you as I do the others? Do you not know that the place nearest me, as men regard such honours, can mean nothing concerning your standing in the kingdom of heaven? You know that the kings of the gentiles have lordship over their subjects, while those who exercise this authority are sometimes called benefactors. But it shall not be so in the kingdom of heaven. He who would be great among you, let him become as the younger; while he who would be chief, let him become as one who serves. Who is the greater, he who sits at meat, or he who serves? Is it not commonly regarded that he who sits at meat is the greater? But you will observe that I am among you as one who serves. If you are willing to become fellow servants with me in doing the Father’s will, in the kingdom to come you shall sit with me in power, still doing the Father’s will in future glory.”}
\vs p179 3:10 When Jesus had finished speaking, the Alpheus twins brought on the bread and wine, with the bitter herbs and the paste of dried fruits, for the next course of the Last Supper.
\usection{4.\bibnobreakspace Last Words to the Betrayer}
\vs p179 4:1 For some minutes the apostles ate in silence, but under the influence of the Master’s cheerful demeanour they were soon drawn into conversation, and ere long the meal was proceeding as if nothing out of the ordinary had occurred to interfere with the good cheer and social accord of this extraordinary occasion. After some time had elapsed, in about the middle of this second course of the meal, Jesus, looking them over, said: \textcolour{ubdarkred}{“I have told you how much I desired to have this supper with you, and knowing how the evil forces of darkness have conspired to bring about the death of the Son of Man, I determined to eat this supper with you in this secret chamber and a day in advance of the Passover since I will not be with you by this time tomorrow night. I have repeatedly told you that I must return to the Father. Now has my hour come, but it was not required that one of you should betray me into the hands of my enemies.”}
\vs p179 4:2 When the 12 heard this, having already been robbed of much of their self\hyp{}assertiveness and self\hyp{}confidence by the parable of the feet washing and the Master’s subsequent discourse, they began to look at one another while in disconcerted tones they hesitatingly inquired, “Is it I?” And when they had all so inquired, Jesus said: \textcolour{ubdarkred}{“While it is necessary that I go to the Father, it was not required that one of you should become a traitor to fulfil the Father’s will. This is the coming to fruit of the concealed evil in the heart of one who failed to love the truth with his whole soul. How deceitful is the intellectual pride that precedes the spiritual downfall! My friend of many years, who even now eats my bread, will be willing to betray me, even as he now dips his hand with me in the dish.”}
\vs p179 4:3 And when Jesus had thus spoken, they all began again to ask, “Is it I?” And as Judas, sitting on the left of his Master, again asked, “Is it I?” Jesus, dipping the bread in the dish of herbs, handed it to Judas, saying, \textcolour{ubdarkred}{“You have said.”} But the others did not hear Jesus speak to Judas. John, who reclined on Jesus’ right hand, leaned over and asked the Master: “Who is it? We should know who it is that has proved untrue to his trust.” Jesus answered: \textcolour{ubdarkred}{“Already have I told you, even he to whom I gave the sop.”} But it was so natural for the host to give a sop to the one who sat next to him on the left that none of them took notice of this, even though the Master had so plainly spoken. But Judas was painfully conscious of the meaning of the Master’s words associated with his act, and he became fearful lest his brethren were likewise now aware that he was the betrayer.
\vs p179 4:4 Peter was highly excited by what had been said, and leaning forward over the table, he addressed John, “Ask him who it is, or if he has told you, tell me who is the betrayer.”
\vs p179 4:5 Jesus brought their whisperings to an end by saying: \textcolour{ubdarkred}{“I sorrow that this evil should have come to pass and hoped even up to this hour that the power of truth might triumph over the deceptions of evil, but such victories are not won without the faith of the sincere love of truth. I would not have told you these things at this, our last supper, but I desire to warn you of these sorrows and so prepare you for what is now upon us. I have told you of this because I desire that you should recall, after I have gone, that I knew about all these evil plottings, and that I forewarned you of my betrayal. And I do all this only that you may be strengthened for the temptations and trials which are just ahead.”}
\vs p179 4:6 When Jesus had thus spoken, leaning over toward Judas, he said: \textcolour{ubdarkred}{“What you have decided to do, do quickly.”} And when Judas heard these words, he arose from the table and hastily left the room, going out into the night to do what he had set his mind to accomplish. When the other apostles saw Judas hasten off after Jesus had spoken to him, they thought he had gone to procure something additional for the supper or to do some other errand for the Master since they supposed he still carried the bag.
\vs p179 4:7 \pc Jesus now knew that nothing could be done to keep Judas from turning traitor. He started with 12 --- now he had 11. He chose 6 of these apostles, and though Judas was among those nominated by his first\hyp{}chosen apostles, still the Master accepted him and had, up to this very hour, done everything possible to sanctify and save him, even as he had wrought for the peace and salvation of the others.
\vs p179 4:8 This supper, with its tender episodes and softening touches, was Jesus’ last appeal to the deserting Judas, but it was of no avail. Warning, even when administered in the most tactful manner and conveyed in the most kindly spirit, as a rule, only intensifies hatred and fires the evil determination to carry out to the full one’s own selfish projects, when love is once really dead.
\usection{5.\bibnobreakspace Establishing the Remembrance Supper}
\vs p179 5:1 As they brought Jesus the third cup of wine, the “cup of blessing,” he arose from the couch and, taking the cup in his hands, blessed it, saying: \textcolour{ubdarkred}{“Take this cup, all of you, and drink of it. This shall be the cup of my remembrance. This is the cup of the blessing of a new dispensation of grace and truth. This shall be to you the emblem of the bestowal and ministry of the divine Spirit of Truth. And I will not again drink this cup with you until I drink in new form with you in the Father’s eternal kingdom.”}
\vs p179 5:2 The apostles all sensed that something out of the ordinary was transpiring as they drank of this cup of blessing in profound reverence and perfect silence. The old Passover commemorated the emergence of their fathers from a state of racial slavery into individual freedom; now the Master was instituting a new remembrance supper as a symbol of the new dispensation wherein the enslaved individual emerges from the bondage of ceremonialism and selfishness into the spiritual joy of the brotherhood and fellowship of the liberated faith sons of the living God.
\vs p179 5:3 When they had finished drinking this new cup of remembrance, the Master took up the bread and, after giving thanks, broke it in pieces and, directing them to pass it around, said: \textcolour{ubdarkred}{“Take this bread of remembrance and eat it. I have told you that I am the bread of life. And this bread of life is the united life of the Father and the Son in one gift. The word of the Father, as revealed in the Son, is indeed the bread of life.”} When they had partaken of the bread of remembrance, the symbol of the living word of truth incarnated in the likeness of mortal flesh, they all sat down.
\vs p179 5:4 \pc In instituting this remembrance supper, the Master, as was always his habit, resorted to parables and symbols. He employed symbols because he wanted to teach certain great spiritual truths in such a manner as to make it difficult for his successors to attach precise interpretations and definite meanings to his words. In this way he sought to prevent successive generations from crystallizing his teaching and binding down his spiritual meanings by the dead chains of tradition and dogma. In the establishment of the only ceremony or sacrament associated with his whole life mission, Jesus took great pains to \bibemph{suggest} his meanings rather than to commit himself to \bibemph{precise definitions.} He did not wish to destroy the individual’s concept of divine communion by establishing a precise form; neither did he desire to limit the believer’s spiritual imagination by formally cramping it. He rather sought to set man’s reborn soul free upon the joyous wings of a new and living spiritual liberty.
\vs p179 5:5 Notwithstanding the Master’s effort thus to establish this new sacrament of the remembrance, those who followed after him in the intervening centuries saw to it that his express desire was effectively thwarted in that his simple spiritual symbolism of that last night in the flesh has been reduced to precise interpretations and subjected to the almost mathematical precision of a set formula. Of all Jesus’ teachings none have become more tradition\hyp{}standardized.
\vs p179 5:6 This supper of remembrance, when it is partaken of by those who are Son\hyp{}believing and God\hyp{}knowing, does not need to have associated with its symbolism any of man’s puerile misinterpretations regarding the meaning of the divine presence, for upon all such occasions the Master is \bibemph{really present.} The remembrance supper is the believer’s symbolic rendezvous with Michael. When you become thus spirit\hyp{}conscious, the Son is actually present, and his spirit fraternizes with the indwelling fragment of his Father.
\vs p179 5:7 \pc After they had engaged in meditation for a few moments, Jesus continued speaking: \textcolour{ubdarkred}{“When you do these things, recall the life I have lived on earth among you and rejoice that I am to continue to live on earth with you and to serve through you. As individuals, contend not among yourselves as to who shall be greatest. Be you all as brethren. And when the kingdom grows to embrace large groups of believers, likewise should you refrain from contending for greatness or seeking preferment between such groups.”}
\vs p179 5:8 And this mighty occasion took place in the upper chamber of a friend. There was nothing of sacred form or of ceremonial consecration about either the supper or the building. The remembrance supper was established without ecclesiastical sanction.
\vs p179 5:9 When Jesus had thus established the supper of the remembrance, he said to the 11: “And as often as you do this, do it in remembrance of me. And when you do remember me, first look back upon my life in the flesh, recall that I was once with you, and then, by faith, discern that you shall all sometime sup with me in the Father’s eternal kingdom. This is the new Passover which I leave with you, even the memory of my bestowal life, the word of eternal truth; and of my love for you, the outpouring of my Spirit of Truth upon all flesh.”\fnc{\ldots{}he said to the \bibtextul{twelve:} “And as often as you do this\ldots{} \bibexpl{There were only eleven apostles still present for the establishment of the remembrance supper because Judas had left earlier; so the “twelve” of the 1955 text was incorrect, and was changed to “apostles” to make this sentence consistent with the rest of the narrative. However, if the manuscript had read “apostles” it could not have become “twelve” in the course of text preparation, therefore a different solution was required. The committee adopted “eleven” as the resolution of this problem based on the proposition that the manuscript contained numerals at this point --- as written documents commonly do --- thus “11.” At some point prior to formatting for printing, the last digit was changed either by accident or through the common typographical error of seeing what you expect to see rather than what is on the page. When the number was formatted for printing, the “12” which was so similar to “11” became “twelve” which is completely dissimilar to “eleven.” [Note that there are several other examples of errors in the 1955 text that apparently had a similar origin: see \bibref[37:8.3]{p037 8:3}, \bibref[41:4.4]{p041 4:4} and \bibref[43:1.6]{p043 1:6}; the several time statements that are formatted incorrectly --- \bibref[134:3.3.1-3]{p0134 3:1} and \bibref[177:4.1]{p0177 4:1} also lend weight to the idea that numbers were written as numerals in the manuscript (as is common practice), and were formatted to words later in the process of text preparation.]Also:First printing: \ldots{}and then, by faith, discern that you shall all \bibtextul{some time} sup with me\ldots{}Changed to: \ldots{}and then, by faith, discern that you shall all \bibtextul{sometime} sup with me \ldots{} --- See note for \bibref[60:3.20]{p060 3:2}.}}
\vs p179 5:10 And they ended this celebration of the old but bloodless Passover in connection with the inauguration of the new supper of the remembrance, by singing, all together, the 118\ts{th} Psalm.
\quizlink
