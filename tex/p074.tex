\upaper{74}{Adam and Eve}
\author{Solonia}
\vs p074 0:1 Adam and Eve arrived on Urantia, from the year A.D.\,1934, 37,848 years ago. It was in midseason when the Garden was in the height of bloom that they arrived. At high noon and unannounced, the two seraphic transports, accompanied by the Jerusem personnel entrusted with the transportation of the biologic uplifters to Urantia, settled slowly to the surface of the revolving planet in the vicinity of the temple of the Universal Father. All the work of rematerializing the bodies of Adam and Eve was carried on within the precincts of this newly created shrine. And from the time of their arrival ten days passed before they were re\hyp{}created in dual human form for presentation as the world’s new rulers. They regained consciousness simultaneously. The Material Sons and Daughters always serve together. It is the essence of their service at all times and in all places never to be separated. They are designed to work in pairs; seldom do they function alone.
\usection{1.\bibnobreakspace Adam and Eve on Jerusem}
\vs p074 1:1 The Planetary Adam and Eve of Urantia were members of the senior corps of Material Sons on Jerusem, being jointly number 14,311. They belonged to the third physical series and were a little more than 2.4\,m in height.
\vs p074 1:2 At the time Adam was chosen to come to Urantia, he was employed, with his mate, in the trial\hyp{}and\hyp{}testing physical laboratories of Jerusem. For more than 15,000 years they had been directors of the division of experimental energy as applied to the modification of living forms. Long before this they had been teachers in the citizenship schools for new arrivals on Jerusem. And all this should be borne in mind in connection with the narration of their subsequent conduct on Urantia.
\vs p074 1:3 When the proclamation was issued calling for volunteers for the mission of Adamic adventure on Urantia, the entire senior corps of Material Sons and Daughters volunteered. The Melchizedek examiners, with the approval of Lanaforge and the Most Highs of Edentia, finally selected the Adam and Eve who subsequently came to function as the biologic uplifters of Urantia.
\vs p074 1:4 Adam and Eve had remained loyal to Michael during the Lucifer rebellion; nevertheless, the pair were called before the System Sovereign and his entire cabinet for examination and instruction. The details of Urantia affairs were fully presented; they were exhaustively instructed as to the plans to be pursued in accepting the responsibilities of rulership on such a strife\hyp{}torn world. They were put under joint oaths of allegiance to the Most Highs of Edentia and to Michael of Salvington. And they were duly advised to regard themselves as subject to the Urantia corps of Melchizedek receivers until that governing body should see fit to relinquish rule on the world of their assignment.
\vs p074 1:5 \pc This Jerusem pair left behind them on the capital of Satania and elsewhere, 100 offspring --- 50 sons and 50 daughters --- magnificent creatures who had escaped the pitfalls of progression, and who were all in commission as faithful stewards of universe trust at the time of their parents’ departure for Urantia. And they were all present in the beautiful temple of the Material Sons attendant upon the farewell exercises associated with the last ceremonies of the bestowal acceptance. These children accompanied their parents to the dematerialization headquarters of their order and were the last to bid them farewell and divine speed as they fell asleep in the personality lapse of consciousness which precedes the preparation for seraphic transport. The children spent some time together at the family rendezvous rejoicing that their parents were soon to become the visible heads, in reality the sole rulers, of planet 606 in the system of Satania.
\vs p074 1:6 And thus did Adam and Eve leave Jerusem amidst the acclaim and well\hyp{}wishing of its citizens. They went forth to their new responsibilities adequately equipped and fully instructed concerning every duty and danger to be encountered on Urantia.
\usection{2.\bibnobreakspace Arrival of Adam and Eve}
\vs p074 2:1 Adam and Eve fell asleep on Jerusem, and when they awakened in the Father’s temple on Urantia in the presence of the mighty throng assembled to welcome them, they were face to face with two beings of whom they had heard much, Van and his faithful associate Amadon. These two heroes of the Caligastia secession were the first to welcome them in their new garden home.
\vs p074 2:2 The tongue of Eden was an Andonic dialect as spoken by Amadon. Van and Amadon had markedly improved this language by creating a new alphabet of 24 letters, and they had hoped to see it become the tongue of Urantia as the Edenic culture would spread throughout the world. Adam and Eve had fully mastered this human dialect before they departed from Jerusem so that this son of Andon heard the exalted ruler of his world address him in his own tongue.
\vs p074 2:3 And on that day there was great excitement and joy throughout Eden as the runners went in great haste to the rendezvous of the carrier pigeons assembled from near and far, shouting: “Let loose the birds; let them carry the word that the promised Son has come.” Hundreds of believer settlements had faithfully, year after year, kept up the supply of these home\hyp{}reared pigeons for just such an occasion.
\vs p074 2:4 \pc As the news of Adam’s arrival spread abroad, thousands of the near\hyp{}by tribesmen accepted the teachings of Van and Amadon, while for months and months pilgrims continued to pour into Eden to welcome Adam and Eve and to do homage to their unseen Father.
\vs p074 2:5 \pc Soon after their awakening, Adam and Eve were escorted to the formal reception on the great mound to the north of the temple. This natural hill had been enlarged and made ready for the installation of the world’s new rulers. Here, at noon, the Urantia reception committee welcomed this Son and Daughter of the system of Satania. Amadon was chairman of this committee, which consisted of 12 members embracing a representative of each of the six Sangik races; the acting chief of the midwayers; Annan, a loyal daughter and spokesman for the Nodites; Noah, the son of the architect and builder of the Garden and executive of his deceased father’s plans; and the two resident Life Carriers.
\vs p074 2:6 The next act was the delivery of the charge of planetary custody to Adam and Eve by the senior Melchizedek, chief of the council of receivership on Urantia. The Material Son and Daughter took the oath of allegiance to the Most Highs of Norlatiadek and to Michael of Nebadon and were proclaimed rulers of Urantia by Van, who thereby relinquished the titular authority which for over 150,000 years he had held by virtue of the action of the Melchizedek receivers.
\vs p074 2:7 And Adam and Eve were invested with kingly robes on this occasion, the time of their formal induction into world rulership. Not all of the arts of Dalamatia had been lost to the world; weaving was still practised in the days of Eden.
\vs p074 2:8 Then was heard the archangels’ proclamation, and the broadcast voice of Gabriel decreed the second judgment roll call of Urantia and the resurrection of the sleeping survivors of the second dispensation of grace and mercy on 606 of Satania. The dispensation of the Prince has passed; the age of Adam, the third planetary epoch, opens amidst scenes of simple grandeur; and the new rulers of Urantia start their reign under seemingly favourable conditions, notwithstanding the world\hyp{}wide confusion occasioned by lack of the co\hyp{}operation of their predecessor in authority on the planet.\fnc{The dispensation of the Prince has \bibtextul{passed,} the age of Adam,\ldots{} \bibexpl{The initial clause is a complete sentence; a semicolon is the correct way of linking the two parts of the larger sentence.}}
\usection{3.\bibnobreakspace Adam and Eve Learn about the Planet}
\vs p074 3:1 And now, after their formal installation, Adam and Eve became painfully aware of their planetary isolation. Silent were the familiar broadcasts, and absent were all the circuits of extraplanetary communication. Their Jerusem fellows had gone to worlds running along smoothly with a well\hyp{}established Planetary Prince and an experienced staff ready to receive them and competent to co\hyp{}operate with them during their early experience on such worlds. But on Urantia rebellion had changed everything. Here the Planetary Prince was very much present, and though shorn of most of his power to work evil, he was still able to make the task of Adam and Eve difficult and to some extent hazardous. It was a serious and disillusioned Son and Daughter of Jerusem who walked that night through the Garden under the shining of the full moon, discussing plans for the next day.
\vs p074 3:2 Thus ended the first day of Adam and Eve on isolated Urantia, the confused planet of the Caligastia betrayal; and they walked and talked far into the night, their first night on earth --- and it was so lonely.
\vs p074 3:3 \pc Adam’s second day on earth was spent in session with the planetary receivers and the advisory council. From the Melchizedeks, and their associates, Adam and Eve learned more about the details of the Caligastia rebellion and the result of that upheaval upon the world’s progress. And it was, on the whole, a disheartening story, this long recital of the mismanagement of world affairs. They learned all the facts regarding the utter collapse of the Caligastia scheme for accelerating the process of social evolution. They also arrived at a full realization of the folly of attempting to achieve planetary advancement independently of the divine plan of progression. And thus ended a sad but enlightening day --- their second on Urantia.
\vs p074 3:4 \pc The third day was devoted to an inspection of the Garden. From the large passenger birds --- the fandors --- Adam and Eve looked down upon the vast stretches of the Garden while being carried through the air over this, the most beautiful spot on earth. This day of inspection ended with an enormous banquet in honour of all who had laboured to create this garden of Edenic beauty and grandeur. And again, late into the night of their third day, the Son and his mate walked in the Garden and talked about the immensity of their problems.
\vs p074 3:5 \pc On the fourth day Adam and Eve addressed the Garden assembly. From the inaugural mount they spoke to the people concerning their plans for the rehabilitation of the world and outlined the methods whereby they would seek to redeem the social culture of Urantia from the low levels to which it had fallen as a result of sin and rebellion. This was a great day, and it closed with a feast for the council of men and women who had been selected to assume responsibilities in the new administration of world affairs. Take note! women as well as men were in this group, and that was the first time such a thing had occurred on earth since the days of Dalamatia. It was an astounding innovation to behold Eve, a woman, sharing the honours and responsibilities of world affairs with a man. And thus ended the fourth day on earth.
\vs p074 3:6 \pc The fifth day was occupied with the organization of the temporary government, the administration which was to function until the Melchizedek receivers should leave Urantia.
\vs p074 3:7 \pc The sixth day was devoted to an inspection of the numerous types of men and animals. Along the walls eastward in Eden, Adam and Eve were escorted all day, viewing the animal life of the planet and arriving at a better understanding as to what must be done to bring order out of the confusion of a world inhabited by such a variety of living creatures.
\vs p074 3:8 It greatly surprised those who accompanied Adam on this trip to observe how fully he understood the nature and function of the thousands upon thousands of animals shown him. The instant he glanced at an animal, he would indicate its nature and behaviour. Adam could give names descriptive of the origin, nature, and function of all material creatures on sight. Those who conducted him on this tour of inspection did not know that the world’s new ruler was one of the most expert anatomists of all Satania; and Eve was equally proficient. Adam amazed his associates by describing hosts of living things too small to be seen by human eyes.
\vs p074 3:9 When the sixth day of their sojourn on earth was over, Adam and Eve rested for the first time in their new home in “the east of Eden.” The first six days of the Urantia adventure had been very busy, and they looked forward with great pleasure to an entire day of freedom from all activities.
\vs p074 3:10 But circumstances dictated otherwise. The experience of the day just past in which Adam had so intelligently and so exhaustively discussed the animal life of Urantia, together with his masterly inaugural address and his charming manner, had so won the hearts and overcome the intellects of the Garden dwellers that they were not only wholeheartedly disposed to accept the newly arrived Son and Daughter of Jerusem as rulers, but the majority were about ready to fall down and worship them as gods.
\usection{4.\bibnobreakspace The First Upheaval}
\vs p074 4:1 That night, the night following the sixth day, while Adam and Eve slumbered, strange things were transpiring in the vicinity of the Father’s temple in the central sector of Eden. There, under the rays of the mellow moon, hundreds of enthusiastic and excited men and women listened for hours to the impassioned pleas of their leaders. They meant well, but they simply could not understand the simplicity of the fraternal and democratic manner of their new rulers. And long before daybreak the new and temporary administrators of world affairs reached a virtually unanimous conclusion that Adam and his mate were altogether too modest and unassuming. They decided that Divinity had descended to earth in bodily form, that Adam and Eve were in reality gods or else so near such an estate as to be worthy of reverent worship.
\vs p074 4:2 The amazing events of the first 6 days of Adam and Eve on earth were entirely too much for the unprepared minds of even the world’s best men; their heads were in a whirl; they were swept along with the proposal to bring the noble pair up to the Father’s temple at high noon in order that everyone might bow down in respectful worship and prostrate themselves in humble submission. And the Garden dwellers were really sincere in all of this.
\vs p074 4:3 Van protested. Amadon was absent, being in charge of the guard of honour which had remained behind with Adam and Eve overnight. But Van’s protest was swept aside. He was told that he was likewise too modest, too unassuming; that he was not far from a god himself, else how had he lived so long on earth, and how had he brought about such a great event as the advent of Adam? And as the excited Edenites were about to seize him and carry him up to the mount for adoration, Van made his way out through the throng and, being able to communicate with the midwayers, sent their leader in great haste to Adam.
\vs p074 4:4 It was near the dawn of their seventh day on earth that Adam and Eve heard the startling news of the proposal of these well\hyp{}meaning but misguided mortals; and then, even while the passenger birds were swiftly winging to bring them to the temple, the midwayers, being able to do such things, transported Adam and Eve to the Father’s temple. It was early on the morning of this seventh day and from the mount of their so recent reception that Adam held forth in explanation of the orders of divine sonship and made clear to these earth minds that only the Father and those whom he designates may be worshipped. Adam made it plain that he would accept any honour and receive all respect, but worship never!
\vs p074 4:5 It was a momentous day, and just before noon, about the time of the arrival of the seraphic messenger bearing the Jerusem acknowledgement of the installation of the world’s rulers, Adam and Eve, moving apart from the throng, pointed to the Father’s temple and said: “Go you now to the material emblem of the Father’s invisible presence and bow down in worship of him who made us all and who keeps us living. And let this act be the sincere pledge that you never will again be tempted to worship anyone but God.” They all did as Adam directed. The Material Son and Daughter stood alone on the mount with bowed heads while the people prostrated themselves about the temple.
\vs p074 4:6 \pc And this was the origin of the Sabbath\hyp{}day tradition. Always in Eden the seventh day was devoted to the noontide assembly at the temple; long it was the custom to devote this day to self\hyp{}culture. The forenoon was devoted to physical improvement, the noontime to spiritual worship, the afternoon to mind culture, while the evening was spent in social rejoicing. This was never the law in Eden, but it was the custom as long as the Adamic administration held sway on earth.
\usection{5.\bibnobreakspace Adam’s Administration}
\vs p074 5:1 For almost seven years after Adam’s arrival the Melchizedek receivers remained on duty, but the time finally came when they turned the administration of world affairs over to Adam and returned to Jerusem.
\vs p074 5:2 The farewell of the receivers occupied the whole of a day, and during the evening the individual Melchizedeks gave Adam and Eve their parting advice and best wishes. Adam had several times requested his advisers to remain on earth with him, but always were these petitions denied. The time had come when the Material Sons must assume full responsibility for the conduct of world affairs. And so, at midnight, the seraphic transports of Satania left the planet with 14 beings for Jerusem, the translation of Van and Amadon occurring simultaneously with the departure of the 12 Melchizedeks.
\vs p074 5:3 \pc All went fairly well for a time on Urantia, and it appeared that Adam would, eventually, be able to develop some plan for promoting the gradual extension of the Edenic civilization. Pursuant to the advice of the Melchizedeks, he began to foster the arts of manufacture with the idea of developing trade relations with the outside world. When Eden was disrupted, there were over 100 primitive manufacturing plants in operation, and extensive trade relations with the near\hyp{}by tribes had been established.
\vs p074 5:4 For ages Adam and Eve had been instructed in the technique of improving a world in readiness for their specialized contributions to the advancement of evolutionary civilization; but now they were face to face with pressing problems, such as the establishment of law and order in a world of savages, barbarians, and semicivilized human beings. Aside from the cream of the earth’s population, assembled in the Garden, only a few groups, here and there, were at all ready for the reception of the Adamic culture.
\vs p074 5:5 Adam made a heroic and determined effort to establish a world government, but he met with stubborn resistance at every turn. Adam had already put in operation a system of group control throughout Eden and had federated all of these companies into the Edenic league. But trouble, serious trouble, ensued when he went outside the Garden and sought to apply these ideas to the outlying tribes. The moment Adam’s associates began to work outside the Garden, they met the direct and well\hyp{}planned resistance of Caligastia and Daligastia. The fallen Prince had been deposed as world ruler, but he had not been removed from the planet. He was still present on earth and able, at least to some extent, to resist all of Adam’s plans for the rehabilitation of human society. Adam tried to warn the races against Caligastia, but the task was made very difficult because his archenemy was invisible to the eyes of mortals.
\vs p074 5:6 Even among the Edenites there were those confused minds that leaned toward the Caligastia teaching of unbridled personal liberty; and they caused Adam no end of trouble; always were they upsetting the best\hyp{}laid plans for orderly progression and substantial development. He was finally compelled to withdraw his program for immediate socialization; he fell back on Van’s method of organization, dividing the Edenites into companies of 100 with captains over each and with lieutenants in charge of groups of 10.
\vs p074 5:7 Adam and Eve had come to institute representative government in the place of monarchial, but they found no government worthy of the name on the face of the whole earth. For the time being Adam abandoned all effort to establish representative government, and before the collapse of the Edenic regime he succeeded in establishing almost 100 outlying trade and social centres where strong individuals ruled in his name. Most of these centres had been organized aforetime by Van and Amadon.
\vs p074 5:8 The sending of ambassadors from one tribe to another dates from the times of Adam. This was a great forward step in the evolution of government.
\usection{6.\bibnobreakspace Home Life of Adam and Eve}
\vs p074 6:1 The Adamic family grounds embraced a little over 13 km\ts{2}. Immediately surrounding this homesite, provision had been made for the care of more than 300,000 of the pure\hyp{}line offspring. But only the first unit of the projected buildings was ever constructed. Before the size of the Adamic family outgrew these early provisions, the whole Edenic plan had been disrupted and the Garden vacated.
\vs p074 6:2 \pc Adamson was the first\hyp{}born of the violet race of Urantia, being followed by his sister and Eveson, the second son of Adam and Eve. Eve was the mother of five children before the Melchizedeks left --- three sons and two daughters. The next two were twins. She bore 63 children, 32 daughters and 31 sons, before the default. When Adam and Eve left the Garden, their family consisted of 4 generations numbering 1,647 pure\hyp{}line descendants. They had 42 children after leaving the Garden besides the 2 offspring of joint parentage with the mortal stock of earth. And this does not include the Adamic parentage to the Nodite and evolutionary races.
\vs p074 6:3 The Adamic children did not take milk from animals when they ceased to nurse the mother’s breast at one year of age. Eve had access to the milk of a great variety of nuts and to the juices of many fruits, and knowing full well the chemistry and energy of these foods, she suitably combined them for the nourishment of her children until the appearance of teeth.
\vs p074 6:4 While cooking was universally employed outside of the immediate Adamic sector of Eden, there was no cooking in Adam’s household. They found their foods --- fruits, nuts, and cereals --- ready prepared as they ripened. They ate once a day, shortly after noontime. Adam and Eve also imbibed “light and energy” direct from certain space emanations in conjunction with the ministry of the tree of life.
\vs p074 6:5 \pc The bodies of Adam and Eve gave forth a shimmer of light, but they always wore clothing in conformity with the custom of their associates. Though wearing very little during the day, at eventide they donned night wraps. The origin of the traditional halo encircling the heads of supposed pious and holy men dates back to the days of Adam and Eve. Since the light emanations of their bodies were so largely obscured by clothing, only the radiating glow from their heads was discernible. The descendants of Adamson always thus portrayed their concept of individuals believed to be extraordinary in spiritual development.
\vs p074 6:6 Adam and Eve could communicate with each other and with their immediate children over a distance of about 80\,km. This thought exchange was effected by means of the delicate gas chambers located in close proximity to their brain structures. By this mechanism they could send and receive thought oscillations. But this power was instantly suspended upon the mind’s surrender to the discord and disruption of evil.
\vs p074 6:7 \pc The Adamic children attended their own schools until they were 16, the younger being taught by the elder. The little folks changed activities every 30 minutes, the older every hour. And it was certainly a new sight on Urantia to observe these children of Adam and Eve at play, joyous and exhilarating activity just for the sheer fun of it. The play and humour of the present\hyp{}day races are largely derived from the Adamic stock. The Adamites all had a great appreciation of music as well as a keen sense of humour.
\vs p074 6:8 The average age of betrothal was 18, and these youths then entered upon a two years’ course of instruction in preparation for the assumption of marital responsibilities. At 20 they were eligible for marriage; and after marriage they began their lifework or entered upon special preparation therefor.
\vs p074 6:9 The practice of some subsequent nations of permitting the royal families, supposedly descended from the gods, to marry brother to sister, dates from the traditions of the Adamic offspring --- mating, as they must needs, with one another. The marriage ceremonies of the first and second generations of the Garden were always performed by Adam and Eve.
\usection{7.\bibnobreakspace Life in the Garden}
\vs p074 7:1 The children of Adam, except for four years’ attendance at the western schools, lived and worked in the “east of Eden.” They were trained intellectually until they were 16 in accordance with the methods of the Jerusem schools. From 16 to 20 they were taught in the Urantia schools at the other end of the Garden, serving there also as teachers in the lower grades.
\vs p074 7:2 The entire purpose of the western school system of the Garden was \bibemph{socialization.} The forenoon periods of recess were devoted to practical horticulture and agriculture, the afternoon periods to competitive play. The evenings were employed in social intercourse and the cultivation of personal friendships. Religious and sexual training were regarded as the province of the home, the duty of parents.
\vs p074 7:3 The teaching in these schools included instruction regarding:
\vs p074 7:4 \ublistelem{1.}\bibnobreakspace Health and the care of the body.
\vs p074 7:5 \ublistelem{2.}\bibnobreakspace The golden rule, the standard of social intercourse.
\vs p074 7:6 \ublistelem{3.}\bibnobreakspace The relation of individual rights to group rights and community obligations.
\vs p074 7:7 \ublistelem{4.}\bibnobreakspace History and culture of the various earth races.
\vs p074 7:8 \ublistelem{5.}\bibnobreakspace Methods of advancing and improving world trade.
\vs p074 7:9 \ublistelem{6.}\bibnobreakspace Co\hyp{}ordination of conflicting duties and emotions.
\vs p074 7:10 \ublistelem{7.}\bibnobreakspace The cultivation of play, humour, and competitive substitutes for physical fighting.
\vs p074 7:11 \pc The schools, in fact every activity of the Garden, were always open to visitors. Unarmed observers were freely admitted to Eden for short visits. To sojourn in the Garden a Urantian had to be “adopted.” He received instructions in the plan and purpose of the Adamic bestowal, signified his intention to adhere to this mission, and then made declaration of loyalty to the social rule of Adam and the spiritual sovereignty of the Universal Father.
\vs p074 7:12 \pc The laws of the Garden were based on the older codes of Dalamatia and were promulgated under seven heads:
\vs p074 7:13 \ublistelem{1.}\bibnobreakspace The laws of health and sanitation.
\vs p074 7:14 \ublistelem{2.}\bibnobreakspace The social regulations of the Garden.
\vs p074 7:15 \ublistelem{3.}\bibnobreakspace The code of trade and commerce.
\vs p074 7:16 \ublistelem{4.}\bibnobreakspace The laws of fair play and competition.
\vs p074 7:17 \ublistelem{5.}\bibnobreakspace The laws of home life.
\vs p074 7:18 \ublistelem{6.}\bibnobreakspace The civil codes of the golden rule.
\vs p074 7:19 \ublistelem{7.}\bibnobreakspace The seven commands of supreme moral rule.
\vs p074 7:20 \pc The moral law of Eden was little different from the seven commandments of Dalamatia. But the Adamites taught many additional reasons for these commands; for instance, regarding the injunction against murder, the indwelling of the Thought Adjuster was presented as an additional reason for not destroying human life. They taught that “whoso sheds man’s blood by man shall his blood be shed, for in the image of God made he man.”
\vs p074 7:21 The public worship hour of Eden was noon; sunset was the hour of family worship. Adam did his best to discourage the use of set prayers, teaching that effective prayer must be wholly individual, that it must be the “desire of the soul”; but the Edenites continued to use the prayers and forms handed down from the times of Dalamatia. Adam also endeavoured to substitute the offerings of the fruit of the land for the blood sacrifices in the religious ceremonies but had made little progress before the disruption of the Garden.
\vs p074 7:22 \pc Adam endeavoured to teach the races sex equality. The way Eve worked by the side of her husband made a profound impression upon all dwellers in the Garden. Adam definitely taught them that the woman, equally with the man, contributes those life factors which unite to form a new being. Theretofore, mankind had presumed that all procreation resided in the “loins of the father.” They had looked upon the mother as being merely a provision for nurturing the unborn and nursing the newborn.
\vs p074 7:23 Adam taught his contemporaries all they could comprehend, but that was not very much, comparatively speaking. Nevertheless, the more intelligent of the races of earth looked forward eagerly to the time when they would be permitted to intermarry with the superior children of the violet race. And what a different world Urantia would have become if this great plan of uplifting the races had been carried out! Even as it was, tremendous gains resulted from the small amount of the blood of this imported race which the evolutionary peoples incidentally secured.
\vs p074 7:24 And thus did Adam work for the welfare and uplift of the world of his sojourn. But it was a difficult task to lead these mixed and mongrel peoples in the better way.
\usection{8.\bibnobreakspace The Legend of Creation}
\vs p074 8:1 The story of the creation of Urantia in six days was based on the tradition that Adam and Eve had spent just six days in their initial survey of the Garden. This circumstance lent almost sacred sanction to the time period of the week, which had been originally introduced by the Dalamatians. Adam’s spending six days inspecting the Garden and formulating preliminary plans for organization was not prearranged; it was worked out from day to day. The choosing of the seventh day for worship was wholly incidental to the facts herewith narrated.
\vs p074 8:2 The legend of the making of the world in six days was an afterthought, in fact, more than 30,000 years afterwards. One feature of the narrative, the sudden appearance of the sun and moon, may have taken origin in the traditions of the onetime sudden emergence of the world from a dense space cloud of minute matter which had long obscured both sun and moon.
\vs p074 8:3 The story of creating Eve out of Adam’s rib is a confused condensation of the Adamic arrival and the celestial surgery connected with the interchange of living substances associated with the coming of the corporeal staff of the Planetary Prince more than 450,000 years previously.
\vs p074 8:4 \pc The majority of the world’s peoples have been influenced by the tradition that Adam and Eve had physical forms created for them upon their arrival on Urantia. The belief in man’s having been created from clay was well\hyp{}nigh universal in the Eastern Hemisphere; this tradition can be traced from the Philippine Islands around the world to Africa. And many groups accepted this story of man’s clay origin by some form of special creation in the place of the earlier beliefs in progressive creation --- evolution.
\vs p074 8:5 Away from the influences of Dalamatia and Eden, mankind tended toward the belief in the gradual ascent of the human race. The fact of evolution is not a modern discovery; the ancients understood the slow and evolutionary character of human progress. The early Greeks had clear ideas of this despite their proximity to Mesopotamia. Although the various races of earth became sadly mixed up in their notions of evolution, nevertheless, many of the primitive tribes believed and taught that they were the descendants of various animals. Primitive peoples made a practice of selecting for their “totems” the animals of their supposed ancestry. Certain North American Indian tribes believed they originated from beavers and coyotes. Certain African tribes teach that they are descended from the hyena, a Malay tribe from the lemur, a New Guinea group from the parrot.
\vs p074 8:6 The Babylonians, because of immediate contact with the remnants of the civilization of the Adamites, enlarged and embellished the story of man’s creation; they taught that he had descended directly from the gods. They held to an aristocratic origin for the race which was incompatible with even the doctrine of creation out of clay.
\vs p074 8:7 \pc The Old Testament account of creation dates from long after the time of Moses; he never taught the Hebrews such a distorted story. But he did present a simple and condensed narrative of creation to the Israelites, hoping thereby to augment his appeal to worship the Creator, the Universal Father, whom he called the Lord God of Israel.
\vs p074 8:8 In his early teachings, Moses very wisely did not attempt to go back of Adam’s time, and since Moses was the supreme teacher of the Hebrews, the stories of Adam became intimately associated with those of creation. That the earlier traditions recognized pre\hyp{}Adamic civilization is clearly shown by the fact that later editors, intending to eradicate all reference to human affairs before Adam’s time, neglected to remove the telltale reference to Cain’s emigration to the “land of Nod,” where he took himself a wife.
\vs p074 8:9 The Hebrews had no written language in general usage for a long time after they reached Palestine. They learned the use of an alphabet from the neighbouring Philistines, who were political refugees from the higher civilization of Crete. The Hebrews did little writing until about 900\,B.C., and having no written language until such a late date, they had several different stories of creation in circulation, but after the Babylonian captivity they inclined more toward accepting a modified Mesopotamian version.
\vs p074 8:10 Jewish tradition became crystallized about Moses, and because he endeavoured to trace the lineage of Abraham back to Adam, the Jews assumed that Adam was the first of all mankind. Yahweh was the creator, and since Adam was supposed to be the first man, he must have made the world just prior to making Adam. And then the tradition of Adam’s six days got woven into the story, with the result that almost 1,000 years after Moses’ sojourn on earth the tradition of creation in six days was written out and subsequently credited to him.
\vs p074 8:11 When the Jewish priests returned to Jerusalem, they had already completed the writing of their narrative of the beginning of things. Soon they made claims that this recital was a recently discovered story of creation written by Moses. But the contemporary Hebrews of around 500\,B.C. did not consider these writings to be divine revelations; they looked upon them much as later peoples regard mythological narratives.
\vs p074 8:12 This spurious document, reputed to be the teachings of Moses, was brought to the attention of Ptolemy, the Greek king of Egypt, who had it translated into Greek by a commission of 70 scholars for his new library at Alexandria. And so this account found its place among those writings which subsequently became a part of the later collections of the “sacred scriptures” of the Hebrew and Christian religions. And through identification with these theological systems, such concepts for a long time profoundly influenced the philosophy of many Occidental peoples.
\vs p074 8:13 The Christian teachers perpetuated the belief in the fiat creation of the human race, and all this led directly to the formation of the hypothesis of a onetime golden age of utopian bliss and the theory of the fall of man or superman which accounted for the nonutopian condition of society. These outlooks on life and man’s place in the universe were at best discouraging since they were predicated upon a belief in retrogression rather than progression, as well as implying a vengeful Deity, who had vented wrath upon the human race in retribution for the errors of certain onetime planetary administrators.
\vs p074 8:14 \pc The “golden age” is a myth, but Eden was a fact, and the Garden civilization was actually overthrown. Adam and Eve carried on in the Garden for 117 years when, through the impatience of Eve and the errors of judgment of Adam, they presumed to turn aside from the ordained way, speedily bringing disaster upon themselves and ruinous retardation upon the developmental progression of all Urantia.
\vsetoff
\vs p074 8:15 [Narrated by Solonia, the seraphic “voice in the Garden.”]
\quizlink
