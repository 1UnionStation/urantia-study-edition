\upaper{164}{At the Feast of Dedication}
\author{Midwayer Commission}
\vs p164 0:1 As the camp at Pella was being established, Jesus, taking with him Nathaniel and Thomas, secretly went up to Jerusalem to attend the feast of the dedication. Not until they passed over the Jordan at the Bethany ford, did the two apostles become aware that their Master was going on to Jerusalem. When they perceived that he really intended to be present at the feast of dedication, they remonstrated with him most earnestly, and using every sort of argument, they sought to dissuade him. But their efforts were of no avail; Jesus was determined to visit Jerusalem. To all their entreaties and to all their warnings emphasizing the folly and danger of placing himself in the hands of the Sanhedrin, he would reply only, \textcolour{ubdarkred}{“I would give these teachers in Israel another opportunity to see the light, before my hour comes.”}
\vs p164 0:2 On they went toward Jerusalem, the two apostles continuing to express their feelings of fear and to voice their doubts about the wisdom of such an apparently presumptuous undertaking. They reached Jericho about 16:30 and prepared to lodge there for the night.
\usection{1.\bibnobreakspace Story of the Good Samaritan}
\vs p164 1:1 That evening a considerable company gathered about Jesus and the two apostles to ask questions, many of which the apostles answered, while others the Master discussed. In the course of the evening a certain lawyer, seeking to entangle Jesus in a compromising disputation, said: “Teacher, I would like to ask you just what I should do to inherit eternal life?” Jesus answered, \textcolour{ubdarkred}{“What is written in the law and the prophets; how do you read the Scriptures?”} The lawyer, knowing the teachings of both Jesus and the Pharisees, answered: “To love the Lord God with all your heart, soul, mind, and strength, and your neighbour as yourself.” Then said Jesus: \textcolour{ubdarkred}{“You have answered right; this, if you really do, will lead to life everlasting.”}
\vs p164 1:2 But the lawyer was not wholly sincere in asking this question, and desiring to justify himself while also hoping to embarrass Jesus, he ventured to ask still another question. Drawing a little closer to the Master, he said, “But, Teacher, I should like you to tell me just who is my neighbour?” The lawyer asked this question hoping to entrap Jesus into making some statement that would contravene the Jewish law which defined one’s neighbour as “the children of one’s people.” The Jews looked upon all others as “gentile dogs.” This lawyer was somewhat familiar with Jesus’ teachings and therefore well knew that the Master thought differently; thus he hoped to lead him into saying something which could be construed as an attack upon the sacred law.
\vs p164 1:3 But Jesus discerned the lawyer’s motive, and instead of falling into the trap, he proceeded to tell his hearers a story, a story which would be fully appreciated by any Jericho audience. Said Jesus: \textcolour{ubdarkred}{“A certain man was going down from Jerusalem to Jericho, and he fell into the hands of cruel brigands, who robbed him, stripped him and beat him, and departing, left him half dead. Very soon, by chance, a certain priest was going down that way, and when he came upon the wounded man, seeing his sorry plight, he passed by on the other side of the road. And in like manner a Levite also, when he came along and saw the man, passed by on the other side. Now, about this time, a certain Samaritan, as he journeyed down to Jericho, came across this wounded man; and when he saw how he had been robbed and beaten, he was moved with compassion, and going over to him, he bound up his wounds, pouring on oil and wine, and setting the man upon his own beast, brought him here to the inn and took care of him. And on the morrow he took out some money and, giving it to the host, said: ‘Take good care of my friend, and if the expense is more, when I come back again, I will repay you.’ Now let me ask you: Which of these three turned out to be the neighbour of him who fell among the robbers?”} And when the lawyer perceived that he had fallen into his own snare, he answered, “He who showed mercy on him.” And Jesus said, \textcolour{ubdarkred}{“Go and do likewise.”}
\vs p164 1:4 The lawyer answered, “He who showed mercy,” that he might refrain from even speaking that odious word, Samaritan. The lawyer was forced to give the very answer to the question, “Who is my neighbour?” which Jesus wished given, and which, if Jesus had so stated, would have directly involved him in the charge of heresy. Jesus not only confounded the dishonest lawyer, but he told his hearers a story which was at the same time a beautiful admonition to all his followers and a stunning rebuke to all Jews regarding their attitude toward the Samaritans. And this story has continued to promote brotherly love among all who have subsequently believed the gospel of Jesus.
\usection{2.\bibnobreakspace At Jerusalem}
\vs p164 2:1 Jesus had attended the feast of tabernacles that he might proclaim the gospel to the pilgrims from all parts of the empire; he now went up to the feast of the dedication for just one purpose: to give the Sanhedrin and the Jewish leaders another chance to see the light. The principal event of these few days in Jerusalem occurred on Friday night at the home of Nicodemus. Here were gathered together some 25 Jewish leaders who believed Jesus’ teaching. Among this group were 14 men who were then, or had recently been, members of the Sanhedrin. This meeting was attended by Eber, Matadormus, and Joseph of Arimathea.
\vs p164 2:2 On this occasion Jesus’ hearers were all learned men, and both they and his two apostles were amazed at the breadth and depth of the remarks which the Master made to this distinguished group. Not since the times when he had taught in Alexandria, Rome, and in the islands of the Mediterranean, had he exhibited such learning and shown such a grasp of the affairs of men, both secular and religious.
\vs p164 2:3 When this little meeting broke up, all went away mystified by the Master’s personality, charmed by his gracious manner, and in love with the man. They had sought to advise Jesus concerning his desire to win the remaining members of the Sanhedrin. The Master listened attentively, but silently, to all their proposals. He well knew none of their plans would work. He surmised that the majority of the Jewish leaders never would accept the gospel of the kingdom; nevertheless, he gave them all this one more chance to choose. But when he went forth that night, with Nathaniel and Thomas, to lodge on the Mount of Olives, he had not yet decided upon the method he would pursue in bringing his work once more to the notice of the Sanhedrin.
\vs p164 2:4 That night Nathaniel and Thomas slept little; they were too much amazed by what they had heard at Nicodemus’s house. They thought much over the final remark of Jesus regarding the offer of the former and present members of the Sanhedrin to go with him before the 70. The Master said: \textcolour{ubdarkred}{“No, my brethren, it would be to no purpose. You would multiply the wrath to be visited upon your own heads, but you would not in the least mitigate the hatred which they bear me. Go, each of you, about the Father’s business as the spirit leads you while I once more bring the kingdom to their notice in the manner which my Father may direct.”}
\usection{3.\bibnobreakspace Healing the Blind Beggar}
\vs p164 3:1 The next morning the three went over to Martha’s home at Bethany for breakfast and then went immediately into Jerusalem. This Sabbath morning, as Jesus and his two apostles drew near the temple, they encountered a well\hyp{}known beggar, a man who had been born blind, sitting at his usual place. Although these mendicants did not solicit or receive alms on the Sabbath day, they were permitted thus to sit in their usual places. Jesus paused and looked upon the beggar. As he gazed upon this man who had been born blind, the idea came into his mind as to how he would once more bring his mission on earth to the notice of the Sanhedrin and the other Jewish leaders and religious teachers.
\vs p164 3:2 As the Master stood there before the blind man, engrossed in deep thought, Nathaniel, pondering the possible cause of this man’s blindness, asked: “Master, who did sin, this man or his parents, that he should be born blind?”
\vs p164 3:3 \pc The rabbis taught that all such cases of blindness from birth were caused by sin. Not only were children conceived and born in sin, but a child could be born blind as a punishment for some specific sin committed by its father. They even taught that a child itself might sin before it was born into the world. They also taught that such defects could be caused by some sin or other indulgence of the mother while carrying the child.
\vs p164 3:4 There was, throughout all these regions, a lingering belief in reincarnation. The older Jewish teachers, together with Plato, Philo, and many of the Essenes, tolerated the theory that men may reap in one incarnation what they have sown in a previous existence; thus in one life they were believed to be expiating the sins committed in preceding lives. The Master found it difficult to make men believe that their souls had not had previous existences.
\vs p164 3:5 However, inconsistent as it seems, while such blindness was supposed to be the result of sin, the Jews held that it was meritorious in a high degree to give alms to these blind beggars. It was the custom of these blind men constantly to chant to the passers\hyp{}by, “O tenderhearted, gain merit by assisting the blind.”
\vs p164 3:6 \pc Jesus entered into the discussion of this case with Nathaniel and Thomas, not only because he had already decided to use this blind man as the means of that day bringing his mission once more prominently to the notice of the Jewish leaders, but also because he always encouraged his apostles to seek for the true causes of all phenomena, natural or spiritual. He had often warned them to avoid the common tendency to assign spiritual causes to commonplace physical events.
\vs p164 3:7 Jesus decided to use this beggar in his plans for that day’s work, but before doing anything for the blind man, Josiah by name, he proceeded to answer Nathaniel’s question. Said the Master: \textcolour{ubdarkred}{“Neither did this man sin nor his parents that the works of God might be manifest in him. This blindness has come upon him in the natural course of events, but we must now do the works of Him who sent me, while it is still day, for the night will certainly come when it will be impossible to do the work we are about to perform. When I am in the world, I am the light of the world, but in only a little while I will not be with you.”}
\vs p164 3:8 When Jesus had spoken, he said to Nathaniel and Thomas: \textcolour{ubdarkred}{“Let us create the sight of this blind man on this Sabbath day that the scribes and Pharisees may have the full occasion which they seek for accusing the Son of Man.”} Then, stooping over, he spat on the ground and mixed the clay with the spittle, and speaking of all this so that the blind man could hear, he went up to Josiah and put the clay over his sightless eyes, saying: \textcolour{ubdarkred}{“Go, my son, wash away this clay in the pool of Siloam, and immediately you shall receive your sight.”} And when Josiah had so washed in the pool of Siloam, he returned to his friends and family, seeing.\tunemarkup{private}{\begin{figure}[H]\centering\includegraphics[scale=\tunemarkup{pgkoboaurahd}{0.65}\tunemarkup{pghanlin}{0.43}\tunemarkup{pgnexus7}{0.63}\tunemarkup{pgkindledx}{0.48}]{../urantia-pictures/At_The_Pool_Of_Siloam.jpg}\caption{At The Pool Of Siloam by Harold~Copping}\end{figure}}
\vs p164 3:9 Having always been a beggar, he knew nothing else; so, when the first excitement of the creation of his sight had passed, he returned to his usual place of alms\hyp{}seeking. His friends, neighbours, and all who had known him aforetime, when they observed that he could see, all said, “Is this not Josiah the blind beggar?” Some said it was he, while others said, “No, it is one like him, but this man can see.” But when they asked the man himself, he answered, “I am he.”
\vs p164 3:10 When they began to inquire of him how he was able to see, he answered them: “A man called Jesus came by this way, and when talking about me with his friends, he made clay with spittle, anointed my eyes, and directed that I should go and wash in the pool of Siloam. I did what this man told me, and immediately I received my sight. And that is only a few hours ago. I do not yet know the meaning of much that I see.” And when the people who began to gather about him asked where they could find the strange man who had healed him, Josiah could answer only that he did not know.
\vs p164 3:11 \pc This is one of the strangest of all the Master’s miracles. This man did not ask for healing. He did not know that the Jesus who had directed him to wash at Siloam, and who had promised him vision, was the prophet of Galilee who had preached in Jerusalem during the feast of tabernacles. This man had little faith that he would receive his sight, but the people of that day had great faith in the efficacy of the spittle of a great or holy man; and from Jesus’ conversation with Nathaniel and Thomas, Josiah had concluded that his would\hyp{}be benefactor was a great man, a learned teacher or a holy prophet; accordingly he did as Jesus directed him.
\vs p164 3:12 Jesus made use of the clay and the spittle and directed him to wash in the symbolic pool of Siloam for three reasons:
\vs p164 3:13 \ublistelem{1.}\bibnobreakspace This was not a miracle response to the individual’s faith. This was a wonder which Jesus chose to perform for a purpose of his own, but which he so arranged that this man might derive lasting benefit therefrom.
\vs p164 3:14 \ublistelem{2.}\bibnobreakspace As the blind man had not asked for healing, and since the faith he had was slight, these material acts were suggested for the purpose of encouraging him. He did believe in the superstition of the efficacy of spittle, and he knew the pool of Siloam was a semisacred place. But he would hardly have gone there had it not been necessary to wash away the clay of his anointing. There was just enough ceremony about the transaction to induce him to act.
\vs p164 3:15 \ublistelem{3.}\bibnobreakspace But Jesus had a third reason for resorting to these material means in connection with this unique transaction: This was a miracle wrought purely in obedience to his own choosing, and thereby he desired to teach his followers of that day and all subsequent ages to refrain from despising or neglecting material means in the healing of the sick. He wanted to teach them that they must cease to regard miracles as the only method of curing human diseases.
\vs p164 3:16 \pc Jesus gave this man his sight by miraculous working, on this Sabbath morning and in Jerusalem near the temple, for the prime purpose of making this act an open challenge to the Sanhedrin and all the Jewish teachers and religious leaders. This was his way of proclaiming an open break with the Pharisees. He was always positive in everything he did. And it was for the purpose of bringing these matters before the Sanhedrin that Jesus brought his two apostles to this man early in the afternoon of this Sabbath day and deliberately provoked those discussions which compelled the Pharisees to take notice of the miracle.
\usection{4.\bibnobreakspace Josiah Before the Sanhedrin}
\vs p164 4:1 By midafternoon the healing of Josiah had raised such a discussion around the temple that the leaders of the Sanhedrin decided to convene the council in its usual temple meeting place. And they did this in violation of a standing rule which forbade the meeting of the Sanhedrin on the Sabbath day. Jesus knew that Sabbath breaking would be one of the chief charges to be brought against him when the final test came, and he desired to be brought before the Sanhedrin for adjudication of the charge of having healed a blind man on the Sabbath day, when the very session of the high Jewish court sitting in judgment on him for this act of mercy would be deliberating on these matters on the Sabbath day and in direct violation of their own self\hyp{}imposed laws.
\vs p164 4:2 But they did not call Jesus before them; they feared to. Instead, they sent forthwith for Josiah. After some preliminary questioning, the spokesman for the Sanhedrin (about 50 members being present) directed Josiah to tell them what had happened to him. Since his healing that morning Josiah had learned from Thomas, Nathaniel, and others that the Pharisees were angry about his healing on the Sabbath, and that they were likely to make trouble for all concerned; but Josiah did not yet perceive that Jesus was he who was called the Deliverer. So, when the Pharisees questioned him, he said: “This man came along, put clay upon my eyes, told me to go wash in Siloam, and I do now see.”
\vs p164 4:3 One of the older Pharisees, after making a lengthy speech, said: “This man cannot be from God because you can see that he does not observe the Sabbath. He violates the law, first, in making the clay, then, in sending this beggar to wash in Siloam on the Sabbath day. Such a man cannot be a teacher sent from God.”
\vs p164 4:4 Then one of the younger men who secretly believed in Jesus, said: “If this man is not sent by God, how can he do these things? We know that one who is a common sinner cannot perform such miracles. We all know this beggar and that he was born blind; now he sees. Will you still say that this prophet does all these wonders by the power of the prince of devils?” And for every Pharisee who dared to accuse and denounce Jesus one would arise to ask entangling and embarrassing questions, so that a serious division arose among them. The presiding officer saw whither they were drifting, and in order to allay the discussion, he prepared further to question the man himself. Turning to Josiah, he said: “What do you have to say about this man, this Jesus, whom you claim opened your eyes?” And Josiah answered, “I think he is a prophet.”
\vs p164 4:5 The leaders were greatly troubled and, knowing not what else to do, decided to send for Josiah’s parents to learn whether he had actually been born blind. They were loath to believe that the beggar had been healed.
\vs p164 4:6 It was well known about Jerusalem, not only that Jesus was denied entrance into all synagogues, but that all who believed in his teaching were likewise cast out of the synagogue, excommunicated from the congregation of Israel; and this meant denial of all rights and privileges of every sort throughout all Jewry except the right to buy the necessaries of life.
\vs p164 4:7 When, therefore, Josiah’s parents, poor and fear\hyp{}burdened souls, appeared before the august Sanhedrin, they were afraid to speak freely. Said the spokesman of the court: “Is this your son? and do we understand aright that he was born blind? If this is true, how is it that he can now see?” And then Josiah’s father, seconded by his mother, answered: “We know that this is our son, and that he was born blind, but how it is that he has come to see, or who it was that opened his eyes, we know not. Ask him; he is of age; let him speak for himself.”
\vs p164 4:8 They now called Josiah up before them a second time. They were not getting along well with their scheme of holding a formal trial, and some were beginning to feel strange about doing this on the Sabbath; accordingly, when they recalled Josiah, they attempted to ensnare him by a different mode of attack. The officer of the court spoke to the former blind man, saying: “Why do you not give God the glory for this? why do you not tell us the whole truth about what happened? We all know that this man is a sinner. Why do you refuse to discern the truth? You know that both you and this man stand convicted of Sabbath breaking. Will you not atone for your sin by acknowledging God as your healer, if you still claim that your eyes have this day been opened?”
\vs p164 4:9 But Josiah was neither dumb nor lacking in humour; so he replied to the officer of the court: “Whether this man is a sinner, I know not; but one thing I do know --- that, whereas I was blind, now I see.” And since they could not entrap Josiah, they sought further to question him, asking: “Just how did he open your eyes? what did he actually do to you? what did he say to you? did he ask you to believe in him?”
\vs p164 4:10 Josiah replied, somewhat impatiently: “I have told you exactly how it all happened, and if you did not believe my testimony, why would you hear it again? Would you by any chance also become his disciples?” When Josiah had thus spoken, the Sanhedrin broke up in confusion, almost violence, for the leaders rushed upon Josiah, angrily exclaiming: “You may talk about being this man’s disciple, but we are disciples of Moses, and we are the teachers of the laws of God. We know that God spoke through Moses, but as for this man Jesus, we know not whence he is.”
\vs p164 4:11 Then Josiah, standing upon a stool, shouted abroad to all who could hear, saying: “Hearken, you who claim to be the teachers of all Israel, while I declare to you that herein is a great marvel since you confess that you know not whence this man is, and yet you know of a certainty, from the testimony which you have heard, that he opened my eyes. We all know that God does not perform such works for the ungodly; that God would do such a thing only at the request of a true worshipper --- for one who is holy and righteous. You know that not since the beginning of the world have you ever heard of the opening of the eyes of one who was born blind. Look, then, all of you, upon me and realize what has been done this day in Jerusalem! I tell you, if this man were not from God, he could not do this.” And as the Sanhedrists departed in anger and confusion, they shouted to him: “You were altogether born in sin, and do you now presume to teach us? Maybe you were not really born blind, and even if your eyes were opened on the Sabbath day, this was done by the power of the prince of devils.” And they went at once to the synagogue to cast out Josiah.
\vs p164 4:12 Josiah entered this trial with meagre ideas about Jesus and the nature of his healing. Most of the daring testimony which he so cleverly and courageously bore before this supreme tribunal of all Israel developed in his mind as the trial proceeded along such unfair and unjust lines.
\usection{5.\bibnobreakspace Teaching in Solomon’s Porch}
\vs p164 5:1 All of the time this Sabbath\hyp{}breaking session of the Sanhedrin was in progress in one of the temple chambers, Jesus was walking about near at hand, teaching the people in Solomon’s Porch, hoping that he would be summoned before the Sanhedrin where he could tell them the good news of the liberty and joy of divine sonship in the kingdom of God. But they were afraid to send for him. They were always disconcerted by these sudden and public appearances of Jesus in Jerusalem. The very occasion they had so ardently sought, Jesus now gave them, but they feared to bring him before the Sanhedrin even as a witness, and even more they feared to arrest him.
\vs p164 5:2 This was midwinter in Jerusalem, and the people sought the partial shelter of Solomon’s Porch; and as Jesus lingered, the crowds asked him many questions, and he taught them for more than two hours. Some of the Jewish teachers sought to entrap him by publicly asking him: “How long will you hold us in suspense? If you are the Messiah, why do you not plainly tell us?” Said Jesus: \textcolour{ubdarkred}{“I have told you about myself and my Father many times, but you will not believe me. Can you not see that the works I do in my Father’s name bear witness for me? But many of you believe not because you belong not to my fold. The teacher of truth attracts only those who hunger for the truth and who thirst for righteousness. My sheep hear my voice and I know them and they follow me. And to all who follow my teaching I give eternal life; they shall never perish, and no one shall snatch them out of my hand. My Father, who has given me these children, is greater than all, so that no one is able to pluck them out of my Father’s hand. The Father and I are one.”} Some of the unbelieving Jews rushed over to where they were still building the temple to pick up stones to cast at Jesus, but the believers restrained them.
\vs p164 5:3 Jesus continued his teaching: \textcolour{ubdarkred}{“Many loving works have I shown you from the Father, so that now would I inquire for which one of these good works do you think to stone me?”} And then answered one of the Pharisees: “For no good work would we stone you but for blasphemy, inasmuch as you, being a man, dare to make yourself equal with God.” And Jesus answered: \textcolour{ubdarkred}{“You charge the Son of Man with blasphemy because you refused to believe me when I declared to you that I was sent by God. If I do not the works of God, believe me not, but if I do the works of God, even though you believe not in me, I should think you would believe the works. But that you may be certain of what I proclaim, let me again assert that the Father is in me and I in the Father, and that, as the Father dwells in me, so will I dwell in every one who believes this gospel.”} And when the people heard these words, many of them rushed out to lay hands upon the stones to cast at him, but he passed out through the temple precincts; and meeting Nathaniel and Thomas, who had been in attendance upon the session of the Sanhedrin, he waited with them near the temple until Josiah came from the council chamber.
\vs p164 5:4 Jesus and the two apostles did not go in search of Josiah at his home until they heard he had been cast out of the synagogue. When they came to his house, Thomas called him out in the yard, and Jesus, speaking to him, said: \textcolour{ubdarkred}{“Josiah, do you believe in the Son of God?”} And Josiah answered, “Tell me who he is that I may believe in him.” And Jesus said: \textcolour{ubdarkred}{“You have both seen and heard him, and it is he who now speaks to you.”} And Josiah said, “Lord, I believe,” and falling down, he worshipped.
\vs p164 5:5 When Josiah learned that he had been cast out of the synagogue, he was at first greatly downcast, but he was much encouraged when Jesus directed that he should immediately prepare to go with them to the camp at Pella. This simple\hyp{}minded man of Jerusalem had indeed been cast out of a Jewish synagogue, but behold the Creator of a universe leading him forth to become associated with the spiritual nobility of that day and generation.
\vs p164 5:6 And now Jesus left Jerusalem, not again to return until near the time when he prepared to leave this world. With the two apostles and Josiah the Master went back to Pella. And Josiah proved to be one of the recipients of the Master’s miraculous ministry who turned out fruitfully, for he became a lifelong preacher of the gospel of the kingdom.
\quizlink
