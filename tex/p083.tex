\upaper{83}{The Marriage Institution}
\uminitoc{Marriage as a Societal Institution}
\uminitoc{Courtship and Betrothal}
\uminitoc{Purchase and Dowry}
\uminitoc{The Wedding Ceremony}
\uminitoc{Plural Marriages}
\uminitoc{True Monogamy --- Pair Marriage}
\uminitoc{The Dissolution of Wedlock}
\uminitoc{The Idealization of Marriage}
\author{Chief of Seraphim}
\vs p083 0:1 This is the recital of the early beginnings of the institution of marriage. It has progressed steadily from the loose and promiscuous matings of the herd through many variations and adaptations, even to the appearance of those marriage standards which eventually culminated in the realization of pair matings, the union of one man and one woman to establish a home of the highest social order.
\vs p083 0:2 Marriage has been many times in jeopardy, and the marriage mores have drawn heavily on both property and religion for support; but the real influence which forever safeguards marriage and the resultant family is the simple and innate biologic fact that men and women positively will not live without each other, be they the most primitive savages or the most cultured mortals.
\vs p083 0:3 It is because of the sex urge that selfish man is lured into making something better than an animal out of himself. The self\hyp{}regarding and self\hyp{}gratifying sex relationship entails the certain consequences of self\hyp{}denial and ensures the assumption of altruistic duties and numerous race\hyp{}benefiting home responsibilities. Herein has sex been the unrecognized and unsuspected civilizer of the savage; for this same sex impulse automatically and unerringly \bibemph{compels man to think} and eventually \bibemph{leads him to love.}
\usection{Marriage as a Societal Institution}
\vs p083 1:1 Marriage is society’s mechanism designed to regulate and control those many human relations which arise out of the physical fact of bisexuality. As such an institution, marriage functions in two directions:
\vs p083 1:2 \ublistelem{1.}\bibnobreakspace In the regulation of personal sex relations.
\vs p083 1:3 \ublistelem{2.}\bibnobreakspace In the regulation of descent, inheritance, succession, and social order, this being its older and original function.
\vs p083 1:4 \pc The family, which grows out of marriage, is itself a stabilizer of the marriage institution together with the property mores. Other potent factors in marriage stability are pride, vanity, chivalry, duty, and religious convictions. But while marriages may be approved or disapproved on high, they are hardly made in heaven. The human family is a distinctly human institution, an evolutionary development. Marriage is an institution of society, not a department of the church. True, religion should mightily influence it but should not undertake exclusively to control and regulate it.
\vs p083 1:5 Primitive marriage was primarily industrial; and even in modern times it is often a social or business affair. Through the influence of the mixture of the Andite stock and as a result of the mores of advancing civilization, marriage is slowly becoming mutual, romantic, parental, poetical, affectionate, ethical, and even idealistic. Selection and so\hyp{}called romantic love, however, were at a minimum in primitive mating. During early times husband and wife were not much together; they did not even eat together very often. But among the ancients, personal affection was not strongly linked to sex attraction; they became fond of one another largely because of living and working together.
\usection{Courtship and Betrothal}
\vs p083 2:1 Primitive marriages were always planned by the parents of the boy and girl. The transition stage between this custom and the times of free choosing was occupied by the marriage broker or professional matchmaker. These matchmakers were at first the barbers; later, the priests. Marriage was originally a group affair; then a family matter; only recently has it become an individual adventure.
\vs p083 2:2 Coercion, not attraction, was the approach to primitive marriage. In early times woman had no sex aloofness, only sex inferiority as inculcated by the mores. As raiding preceded trading, so marriage by capture preceded marriage by contract. Some women would connive at capture in order to escape the domination of the older men of their tribe; they preferred to fall into the hands of men of their own age from another tribe. This pseudo elopement was the transition stage between capture by force and subsequent courtship by charming.
\vs p083 2:3 An early type of wedding ceremony was the mimic flight, a sort of elopement rehearsal which was once a common practice. Later, mock capture became a part of the regular wedding ceremony. A modern girl’s pretensions to resist “capture,” to be reticent toward marriage, are all relics of olden customs. The carrying of the bride over the threshold is reminiscent of a number of ancient practices, among others, of the days of wife stealing.
\vs p083 2:4 Woman was long denied full freedom of self\hyp{}disposal in marriage, but the more intelligent women have always been able to circumvent this restriction by the clever exercise of their wits. Man has usually taken the lead in courtship, but not always. Woman sometimes formally, as well as covertly, initiates marriage. And as civilization has progressed, women have had an increasing part in all phases of courtship and marriage.
\vs p083 2:5 Increasing love, romance, and personal selection in premarital courtship are an Andite contribution to the world races. The relations between the sexes are evolving favourably; many advancing peoples are gradually substituting somewhat idealized concepts of sex attraction for those older motives of utility and ownership. Sex impulse and feelings of affection are beginning to displace cold calculation in the choosing of life partners.
\vs p083 2:6 The betrothal was originally equivalent to marriage; and among early peoples sex relations were conventional during the engagement. In recent times, religion has established a sex taboo on the period between betrothal and marriage.
\usection{Purchase and Dowry}
\vs p083 3:1 The ancients mistrusted love and promises; they thought that abiding unions must be guaranteed by some tangible security, property. For this reason, the purchase price of a wife was regarded as a forfeit or deposit which the husband was doomed to lose in case of divorce or desertion. Once the purchase price of a bride had been paid, many tribes permitted the husband’s brand to be burned upon her. Africans still buy their wives. A love wife, or a white man’s wife, they compare to a cat because she costs nothing.
\vs p083 3:2 The bride shows were occasions for dressing up and decorating daughters for public exhibition with the idea of their bringing higher prices as wives. But they were not sold as animals --- among the later tribes such a wife was not transferable. Neither was her purchase always just a cold\hyp{}blooded money transaction; service was equivalent to cash in the purchase of a wife. If an otherwise desirable man could not pay for his wife, he could be adopted as a son by the girl’s father and then could marry. And if a poor man sought a wife and could not meet the price demanded by a grasping father, the elders would often bring pressure to bear upon the father which would result in a modification of his demands, or else there might be an elopement.
\vs p083 3:3 As civilization progressed, fathers did not like to appear to sell their daughters, and so, while continuing to accept the bride purchase price, they initiated the custom of giving the pair valuable presents which about equalled the purchase money. And upon the later discontinuance of payment for the bride, these presents became the bride’s dowry.
\vs p083 3:4 The idea of a dowry was to convey the impression of the bride’s independence, to suggest far removal from the times of slave wives and property companions. A man could not divorce a dowered wife without paying back the dowry in full. Among some tribes a mutual deposit was made with the parents of both bride and groom to be forfeited in case either deserted the other, in reality a marriage bond. During the period of transition from purchase to dowry, if the wife were purchased, the children belonged to the father; if not, they belonged to the wife’s family.
\usection{The Wedding Ceremony}
\vs p083 4:1 The wedding ceremony grew out of the fact that marriage was originally a community affair, not just the culmination of a decision of two individuals. Mating was of group concern as well as a personal function.
\vs p083 4:2 \pc Magic, ritual, and ceremony surrounded the entire life of the ancients, and marriage was no exception. As civilization advanced, as marriage became more seriously regarded, the wedding ceremony became increasingly pretentious. Early marriage was a factor in property interests, even as it is today, and therefore required a legal ceremony, while the social status of subsequent children demanded the widest possible publicity. Primitive man had no records; therefore must the marriage ceremony be witnessed by many persons.
\vs p083 4:3 At first the wedding ceremony was more on the order of a betrothal and consisted only in public notification of intention of living together; later it consisted in formal eating together. Among some tribes the parents simply took their daughter to the husband; in other cases the only ceremony was the formal exchange of presents, after which the bride’s father would present her to the groom. Among many Levantine peoples it was the custom to dispense with all formality, marriage being consummated by sex relations. The red man was the first to develop the more elaborate celebration of weddings.
\vs p083 4:4 \pc Childlessness was greatly dreaded, and since barrenness was attributed to spirit machinations, efforts to ensure fecundity also led to the association of marriage with certain magical or religious ceremonials. And in this effort to ensure a happy and fertile marriage, many charms were employed; even the astrologers were consulted to ascertain the birth stars of the contracting parties. At one time the human sacrifice was a regular feature of all weddings among well\hyp{}to\hyp{}do people.
\vs p083 4:5 Lucky days were sought out, Thursday being most favourably regarded, and weddings celebrated at the full of the moon were thought to be exceptionally fortunate. It was the custom of many Near Eastern peoples to throw grain upon the newlyweds; this was a magical rite which was supposed to ensure fecundity. Certain Oriental peoples used rice for this purpose.
\vs p083 4:6 Fire and water were always considered the best means of resisting ghosts and evil spirits; hence altar fires and lighted candles, as well as the baptismal sprinkling of holy water, were usually in evidence at weddings. For a long time it was customary to set a false wedding day and then suddenly postpone the event so as to put the ghosts and spirits off the track.
\vs p083 4:7 The teasing of newlyweds and the pranks played upon honeymooners are all relics of those far\hyp{}distant days when it was thought best to appear miserable and ill at ease in the sight of the spirits so as to avoid arousing their envy. The wearing of the bridal veil is a relic of the times when it was considered necessary to disguise the bride so that ghosts might not recognize her and also to hide her beauty from the gaze of the otherwise jealous and envious spirits. The bride’s feet must never touch the ground just prior to the ceremony. Even in the XX century it is still the custom under the Christian mores to stretch carpets from the carriage landing to the church altar.
\vs p083 4:8 One of the most ancient forms of the wedding ceremony was to have a priest bless the wedding bed to ensure the fertility of the union; this was done long before any formal wedding ritual was established. During this period in the evolution of the marriage mores the wedding guests were expected to file through the bedchamber at night, thus constituting legal witness to the consummation of marriage.
\vs p083 4:9 The luck element, that in spite of all premarital tests certain marriages turned out bad, led primitive man to seek insurance protection against marriage failure; led him to go in quest of priests and magic. And this movement culminated directly in modern church weddings. But for a long time marriage was generally recognized as consisting in the decisions of the contracting parents --- later of the pair --- while for the last 500 years church and state have assumed jurisdiction and now presume to make pronouncements of marriage.
\usection{Plural Marriages}
\vs p083 5:1 In the early history of marriage the unmarried women belonged to the men of the tribe. Later on, a woman had only one husband at a time. This practice of \bibemph{one\hyp{}man\hyp{}at\hyp{}a\hyp{}time} was the first step away from the promiscuity of the herd. While a woman was allowed but one man, her husband could sever such temporary relationships at will. But these loosely regulated associations were the first step toward living pairwise in distinction to living herdwise. In this stage of marriage development children usually belonged to the mother.
\vs p083 5:2 The next step in mating evolution was the \bibemph{group marriage.} This communal phase of marriage had to intervene in the unfolding of family life because the marriage mores were not yet strong enough to make pair associations permanent. The brother and sister marriages belonged to this group; five brothers of one family would marry five sisters of another. All over the world the looser forms of communal marriage gradually evolved into various types of group marriage. And these group associations were largely regulated by the totem mores. Family life slowly and surely developed because sex and marriage regulation favoured the survival of the tribe itself by ensuring the survival of larger numbers of children.
\vs p083 5:3 Group marriages gradually gave way before the emerging practices of polygamy --- polygyny and polyandry --- among the more advanced tribes. But polyandry was never general, being usually limited to queens and rich women; furthermore, it was customarily a family affair, one wife for several brothers. Caste and economic restrictions sometimes made it necessary for several men to content themselves with one wife. Even then, the woman would marry only one, the others being loosely tolerated as “uncles” of the joint progeny.
\vs p083 5:4 The Jewish custom requiring that a man consort with his deceased brother’s widow for the purpose of “raising up seed for his brother,” was the custom of more than half the ancient world. This was a relic of the time when marriage was a family affair rather than an individual association.
\vs p083 5:5 The institution of polygyny recognized, at various times, four sorts of wives:
\vs p083 5:6 \ublistelem{1.}\bibnobreakspace The ceremonial or legal wives.
\vs p083 5:7 \ublistelem{2.}\bibnobreakspace Wives of affection and permission.
\vs p083 5:8 \ublistelem{3.}\bibnobreakspace Concubines, contractual wives.
\vs p083 5:9 \ublistelem{4.}\bibnobreakspace Slave wives.
\vs p083 5:10 \pc True polygyny, where all the wives are of equal status and all the children equal, has been very rare. Usually, even with plural marriages, the home was dominated by the head wife, the status companion. She alone had the ritual wedding ceremony, and only the children of such a purchased or dowered spouse could inherit unless by special arrangement with the status wife.
\vs p083 5:11 The status wife was not necessarily the love wife; in early times she usually was not. The love wife, or sweetheart, did not appear until the races were considerably advanced, more particularly after the blending of the evolutionary tribes with the Nodites and Adamites.
\vs p083 5:12 The taboo wife --- one wife of legal status --- created the concubine mores. Under these mores a man might have only one wife, but he could maintain sex relations with any number of concubines. Concubinage was the stepping stone to monogamy, the first move away from frank polygyny. The concubines of the Jews, Romans, and Chinese were very frequently the handmaidens of the wife. Later on, as among the Jews, the legal wife was looked upon as the mother of all children born to the husband.
\vs p083 5:13 The olden taboos on sex relations with a pregnant or nursing wife tended greatly to foster polygyny. Primitive women aged very early because of frequent childbearing coupled with hard work. (Such overburdened wives only managed to exist by virtue of the fact that they were put in isolation one week out of each month when they were not heavy with child.) Such a wife often grew tired of bearing children and would request her husband to take a second and younger wife, one able to help with both childbearing and the domestic work. The new wives were therefore usually hailed with delight by the older spouses; there existed nothing on the order of sex jealousy.
\vs p083 5:14 The number of wives was only limited by the ability of the man to provide for them. Wealthy and able men wanted large numbers of children, and since the infant mortality was very high, it required an assembly of wives to recruit a large family. Many of these plural wives were mere labourers, slave wives.
\vs p083 5:15 Human customs evolve, but very slowly. The purpose of a harem was to build up a strong and numerous body of blood kin for the support of the throne. A certain chief was once convinced that he should not have a harem, that he should be contented with one wife; so he promptly dismissed his harem. The dissatisfied wives went to their homes, and their offended relatives swept down on the chief in wrath and did away with him then and there.
\usection{True Monogamy --- Pair Marriage}
\vs p083 6:1 Monogamy is monopoly; it is good for those who attain this desirable state, but it tends to work a biologic hardship on those who are not so fortunate. But quite regardless of the effect on the individual, monogamy is decidedly best for the children.
\vs p083 6:2 The earliest monogamy was due to force of circumstances, poverty. Monogamy is cultural and societal, artificial and unnatural, that is, unnatural to evolutionary man. It was wholly natural to the purer Nodites and Adamites and has been of great cultural value to all advanced races.
\vs p083 6:3 The Chaldean tribes recognized the right of a wife to impose a premarital pledge upon her spouse not to take a second wife or concubine; both the Greeks and the Romans favoured monogamous marriage. Ancestor worship has always fostered monogamy, as has the Christian error of regarding marriage as a sacrament. Even the elevation of the standard of living has consistently militated against plural wives. By the time of Michael’s advent on Urantia practically all of the civilized world had attained the level of theoretical monogamy. But this passive monogamy did not mean that mankind had become habituated to the practice of real pair marriage.
\vs p083 6:4 \pc While pursuing the monogamic goal of the ideal pair marriage, which is, after all, something of a monopolistic sex association, society must not overlook the unenviable situation of those unfortunate men and women who fail to find a place in this new and improved social order, even when having done their best to co\hyp{}operate with, and enter into, its requirements. Failure to gain mates in the social arena of competition may be due to insurmountable difficulties or multitudinous restrictions which the current mores have imposed. Truly, monogamy is ideal for those who are in, but it must inevitably work great hardship on those who are left out in the cold of solitary existence.
\vs p083 6:5 Always have the unfortunate few had to suffer that the majority might advance under the developing mores of evolving civilization; but always should the favoured majority look with kindness and consideration on their less fortunate fellows who must pay the price of failure to attain membership in the ranks of those ideal sex partnerships which afford the satisfaction of all biologic urges under the sanction of the highest mores of advancing social evolution.
\vs p083 6:6 \pc Monogamy always has been, now is, and forever will be the idealistic goal of human sex evolution. This ideal of true pair marriage entails self\hyp{}denial, and therefore does it so often fail just because one or both of the contracting parties are deficient in that acme of all human virtues, rugged self\hyp{}control.
\vs p083 6:7 Monogamy is the yardstick which measures the advance of social civilization as distinguished from purely biologic evolution. Monogamy is not necessarily biologic or natural, but it is indispensable to the immediate maintenance and further development of social civilization. It contributes to a delicacy of sentiment, a refinement of moral character, and a spiritual growth which are utterly impossible in polygamy. A woman never can become an ideal mother when she is all the while compelled to engage in rivalry for her husband’s affections.
\vs p083 6:8 Pair marriage favours and fosters that intimate understanding and effective co\hyp{}operation which is best for parental happiness, child welfare, and social efficiency. Marriage, which began in crude coercion, is gradually evolving into a magnificent institution of self\hyp{}culture, self\hyp{}control, self\hyp{}expression, and self\hyp{}perpetuation.
\usection{The Dissolution of Wedlock}
\vs p083 7:1 In the early evolution of the marital mores, marriage was a loose union which could be terminated at will, and the children always followed the mother; the mother\hyp{}child bond is instinctive and has functioned regardless of the developmental stage of the mores.
\vs p083 7:2 Among primitive peoples only about one half the marriages proved satisfactory. The most frequent cause for separation was barrenness, which was always blamed on the wife; and childless wives were believed to become snakes in the spirit world. Under the more primitive mores, divorce was had at the option of the man alone, and these standards have persisted to the XX century among some peoples.
\vs p083 7:3 As the mores evolved, certain tribes developed two forms of marriage: the ordinary, which permitted divorce, and the priest marriage, which did not allow for separation. The inauguration of wife purchase and wife dowry, by introducing a property penalty for marriage failure, did much to lessen separation. And, indeed, many modern unions are stabilized by this ancient property factor.
\vs p083 7:4 The social pressure of community standing and property privileges has always been potent in the maintenance of the marriage taboos and mores. Down through the ages marriage has made steady progress and stands on advanced ground in the modern world, notwithstanding that it is threateningly assailed by widespread dissatisfaction among those peoples where individual choice --- a new liberty --- figures most largely. While these upheavals of adjustment appear among the more progressive races as a result of suddenly accelerated social evolution, among the less advanced peoples marriage continues to thrive and slowly improve under the guidance of the older mores.
\vs p083 7:5 The new and sudden substitution of the more ideal but extremely individualistic love motive in marriage for the older and long\hyp{}established property motive, has unavoidably caused the marriage institution to become temporarily unstable. Man’s marriage motives have always far transcended actual marriage morals, and in the XIX and XX centuries the Occidental ideal of marriage has suddenly far outrun the self\hyp{}centred and but partially controlled sex impulses of the races. The presence of large numbers of unmarried persons in any society indicates the temporary breakdown or the transition of the mores.
\vs p083 7:6 The real test of marriage, all down through the ages, has been that continuous intimacy which is inescapable in all family life. Two pampered and spoiled youths, educated to expect every indulgence and full gratification of vanity and ego, can hardly hope to make a great success of marriage and home building --- a lifelong partnership of self\hyp{}effacement, compromise, devotion, and unselfish dedication to child culture.
\vs p083 7:7 The high degree of imagination and fantastic romance entering into courtship is largely responsible for the increasing divorce tendencies among modern Occidental peoples, all of which is further complicated by woman’s greater personal freedom and increased economic liberty. Easy divorce, when the result of lack of self\hyp{}control or failure of normal personality adjustment, only leads directly back to those crude societal stages from which man has emerged so recently and as the result of so much personal anguish and racial suffering.
\vs p083 7:8 But just so long as society fails to properly educate children and youths, so long as the social order fails to provide adequate premarital training, and so long as unwise and immature youthful idealism is to be the arbiter of the entrance upon marriage, just so long will divorce remain prevalent. And in so far as the social group falls short of providing marriage preparation for youths, to that extent must divorce function as the social safety valve which prevents still worse situations during the ages of the rapid growth of the evolving mores.
\vs p083 7:9 \pc The ancients seem to have regarded marriage just about as seriously as some present\hyp{}day people do. And it does not appear that many of the hasty and unsuccessful marriages of modern times are much of an improvement over the ancient practices of qualifying young men and women for mating. The great inconsistency of modern society is to exalt love and to idealize marriage while disapproving of the fullest examination of both.
\usection{The Idealization of Marriage}
\vs p083 8:1 Marriage which culminates in the home is indeed man’s most exalted institution, but it is essentially human; it should never have been called a sacrament. The Sethite priests made marriage a religious ritual; but for thousands of years after Eden, mating continued as a purely social and civil institution.
\vs p083 8:2 The likening of human associations to divine associations is most unfortunate. The union of husband and wife in the marriage\hyp{}home relationship is a material function of the mortals of the evolutionary worlds. True, indeed, much spiritual progress may accrue consequent upon the sincere human efforts of husband and wife to progress, but this does not mean that marriage is necessarily sacred. Spiritual progress is attendant upon sincere application to other avenues of human endeavour.
\vs p083 8:3 Neither can marriage be truly compared to the relation of the Adjuster to man nor to the fraternity of Christ Michael and his human brethren. At scarcely any point are such relationships comparable to the association of husband and wife. And it is most unfortunate that the human misconception of these relationships has produced so much confusion as to the status of marriage.
\vs p083 8:4 It is also unfortunate that certain groups of mortals have conceived of marriage as being consummated by divine action. Such beliefs lead directly to the concept of the indissolubility of the marital state regardless of the circumstances or wishes of the contracting parties. But the very fact of marriage dissolution itself indicates that Deity is not a conjoining party to such unions. If God has once joined any two things or persons together, they will remain thus joined until such a time as the divine will decrees their separation. But, regarding marriage, which is a human institution, who shall presume to sit in judgment, to say which marriages are unions that might be approved by the universe supervisors in contrast with those which are purely human in nature and origin?
\vs p083 8:5 Nevertheless, there is an ideal of marriage on the spheres on high. On the capital of each local system the Material Sons and Daughters of God do portray the height of the ideals of the union of man and woman in the bonds of marriage and for the purpose of procreating and rearing offspring. After all, the ideal mortal marriage is \bibemph{humanly} sacred.
\vs p083 8:6 \pc Marriage always has been and still is man’s supreme dream of temporal ideality. Though this beautiful dream is seldom realized in its entirety, it endures as a glorious ideal, ever luring progressing mankind on to greater strivings for human happiness. But young men and women should be taught something of the realities of marriage before they are plunged into the exacting demands of the interassociations of family life; youthful idealization should be tempered with some degree of premarital disillusionment.
\vs p083 8:7 The youthful idealization of marriage\tunemarkup{pgkoboaurahd}{\linebreak} should not, however, be discouraged; such dreams are the visualization of the future goal of family life. This attitude is both stimulating and helpful providing it does not produce an insensitivity to the realization of the practical and commonplace requirements of marriage and subsequent family life.
\vs p083 8:8 The ideals of marriage have made great progress in recent times; among some peoples woman enjoys practically equal rights with her consort. In concept, at least, the family is becoming a loyal partnership for rearing offspring, accompanied by sexual fidelity. But even this newer version of marriage need not presume to swing so far to the extreme as to confer mutual monopoly of all personality and individuality. Marriage is not just an individualistic ideal; it is the evolving social partnership of a man and a woman, existing and functioning under the current mores, restricted by the taboos, and enforced by the laws and regulations of society.
\vs p083 8:9 XX century marriages stand high in comparison with those of past ages, notwithstanding that the home institution is now undergoing a serious testing because of the problems so suddenly thrust upon the social organization by the precipitate augmentation of woman’s liberties, rights so long denied her in the tardy evolution of the mores of past generations.
\vsetoff
\vs p083 8:10 [Presented by the Chief of Seraphim stationed on Urantia.]
\quizlink
