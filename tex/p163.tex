\upaper{163}{Ordination of the Seventy at Magadan}
\uminitoc{Ordination of the Seventy}
\uminitoc{The Rich Young Man and Others}
\uminitoc{The Discussion about Wealth}
\uminitoc{Farewell to the Seventy}
\uminitoc{Moving the Camp to Pella}
\uminitoc{The Return of the Seventy}
\uminitoc{Preparation for the Last Mission}
\author{Midwayer Commission}
\vs p163 0:1 A few days after the return of Jesus and the twelve to Magadan from Jerusalem, Abner and a group of some 50 disciples arrived from Bethlehem. At this time there were also assembled at Magadan Camp the evangelistic corps, the women’s corps, and about 150 other true and tried disciples from all parts of Palestine. After devoting a few days to visiting and the reorganization of the camp, Jesus and the twelve began a course of intensive training for this special group of believers, and from this well\hyp{}trained and experienced aggregation of disciples the Master subsequently chose the 70 teachers and sent them forth to proclaim the gospel of the kingdom. This regular instruction began on Friday, November 4, and continued until Sabbath, November 19.
\vs p163 0:2 Jesus gave a talk to this company each morning. Peter taught methods of public preaching; Nathaniel instructed them in the art of teaching; Thomas explained how to answer questions; while Matthew directed the organization of their group finances. The other apostles also participated in this training in accordance with their special experience and natural talents.
\usection{Ordination of the Seventy}
\vs p163 1:1 The 70 were ordained by Jesus on Sabbath afternoon, November 19, at the Magadan Camp, and Abner was placed at the head of these gospel preachers and teachers. This corps of 70 consisted of Abner and 10 of the former apostles of John, 51 of the earlier evangelists, and 8 other disciples who had distinguished themselves in the service of the kingdom.
\vs p163 1:2 About 14:00 on this Sabbath afternoon, between showers of rain, a company of believers, augmented by the arrival of David and the majority of his messenger corps and numbering over 400, assembled on the shore of the lake of Galilee to witness the ordination of the 70.
\vs p163 1:3 Before Jesus laid his hands upon the heads of the 70 to set them apart as gospel messengers, addressing them, he said: \textcolour{ubdarkred}{“The harvest is indeed plenteous, but the labourers are few; therefore I exhort all of you to pray that the Lord of the harvest will send still other labourers into his harvest. I am about to set you apart as messengers of the kingdom; I am about to send you to Jew and gentile as lambs among wolves. As you go your ways, two and two, I instruct you to carry neither purse nor extra clothing, for you go forth on this first mission for only a short season. Salute no man by the way, attend only to your work. Whenever you go to stay at a home, first say: Peace be to this household. If those who love peace live therein, you shall abide there; if not, then shall you depart. And having selected this home, remain there for your stay in that city, eating and drinking whatever is set before you. And you do this because the labourer is worthy of his sustenance. Move not from house to house because a better lodging may be offered. Remember, as you go forth proclaiming peace on earth and good will among men, you must contend with bitter and self\hyp{}deceived enemies; therefore be as wise as serpents while you are also as harmless as doves.}
\vs p163 1:4 \textcolour{ubdarkred}{“And everywhere you go, preach, saying, ‘The kingdom of heaven is at hand,’ and minister to all who may be sick in either mind or body. Freely you have received of the good things of the kingdom; freely give. If the people of any city receive you, they shall find an abundant entrance into the Father’s kingdom; but if the people of any city refuse to receive this gospel, still shall you proclaim your message as you depart from that unbelieving community, saying, even as you leave, to those who reject your teaching: ‘Notwithstanding you reject the truth, it remains that the kingdom of God has come near you.’ He who hears you hears me. And he who hears me hears Him who sent me. He who rejects your gospel message rejects me. And he who rejects me rejects Him who sent me.”}
\vs p163 1:5 When Jesus had thus spoken to the 70, he began with Abner and, as they knelt in a circle about him, laid his hands upon the head of every man.
\vs p163 1:6 Early the next morning Abner sent the 70 messengers into all the cities of Galilee, Samaria, and Judea. And these 35 couples went forth preaching and teaching for about 6 weeks, all of them returning to the new camp near Pella, in Perea, on Friday, December 30.
\usection{The Rich Young Man and Others}
\vs p163 2:1 Over 50 disciples who sought ordination and appointment to membership in the 70 were rejected by the committee appointed by Jesus to select these candidates. This committee consisted of Andrew, Abner, and the acting head of the evangelistic corps. In all cases where this committee of three were not unanimous in agreement, they brought the candidate to Jesus, and while the Master never rejected a single person who craved ordination as a gospel messenger, there were more than a dozen who, when they had talked with Jesus, no more desired to become gospel messengers.
\vs p163 2:2 \pc One earnest disciple came to Jesus, saying: “Master, I would be one of your new apostles, but my father is very old and near death; could I be permitted to return home to bury him?” To this man Jesus said: \textcolour{ubdarkred}{“My son, the foxes have holes, and the birds of heaven have nests, but the Son of Man has nowhere to lay his head. You are a faithful disciple, and you can remain such while you return home to minister to your loved ones, but not so with my gospel messengers. They have forsaken all to follow me and proclaim the kingdom. If you would be an ordained teacher, you must let others bury the dead while you go forth to publish the good news.”} And this man went away in great disappointment.
\vs p163 2:3 Another disciple came to the Master and said: “I would become an ordained messenger, but I would like to go to my home for a short while to comfort my family.” And Jesus replied: \textcolour{ubdarkred}{“If you would be ordained, you must be willing to forsake all. The gospel messengers cannot have divided affections. No man, having put his hand to the plough, if he turns back, is worthy to become a messenger of the kingdom.”}
\vs p163 2:4 \pc Then Andrew brought to Jesus a certain rich young man who was a devout believer, and who desired to receive ordination. This young man, Matadormus, was a member of the Jerusalem Sanhedrin; he had heard Jesus teach and had been subsequently instructed in the gospel of the kingdom by Peter and the other apostles. Jesus talked with Matadormus concerning the requirements of ordination and requested that he defer decision until after he had thought more fully about the matter. Early the next morning, as Jesus was going for a walk, this young man accosted him and said: “Master, I would know from you the assurances of eternal life. Seeing that I have observed all the commandments from my youth, I would like to know what more I must do to gain eternal life?” In answer to this question Jesus said: \textcolour{ubdarkred}{“If you keep all the commandments --- do not commit adultery, do not kill, do not steal, do not bear false witness, do not defraud, honour your parents --- you do well, but salvation is the reward of faith, not merely of works. Do you believe this gospel of the kingdom?”} And Matadormus answered: “Yes, Master, I do believe everything you and your apostles have taught me.” And Jesus said, \textcolour{ubdarkred}{“Then are you indeed my disciple and a child of the kingdom.”}
\vs p163 2:5 Then said the young man: “But, Master, I am not content to be your disciple; I would be one of your new messengers.” When Jesus heard this, he looked down upon him with a great love and said: \textcolour{ubdarkred}{“I will have you to be one of my messengers if you are willing to pay the price, if you will supply the one thing which you lack.”} Matadormus replied: “Master, I will do anything if I may be allowed to follow you.” Jesus, kissing the kneeling young man on the forehead, said: \textcolour{ubdarkred}{“If you would be my messenger, go and sell all that you have and, when you have bestowed the proceeds upon the poor or upon your brethren, come and follow me, and you shall have treasure in the kingdom of heaven.”}\tunemarkup{private}{\begin{figure}[H]\centering\includegraphics[width=\columnwidth]{images/Rich.jpg}\caption{Christ and the Rich Young Ruler by Heinrich~Hofmann}\end{figure}}
\vs p163 2:6 When Matadormus heard this, his countenance fell. He arose and went away sorrowful, for he had great possessions. This wealthy young Pharisee had been raised to believe that wealth was the token of God’s favour. Jesus knew that he was not free from the love of himself and his riches. The Master wanted to deliver him from the \bibemph{love} of wealth, not necessarily from the wealth. While the disciples of Jesus did not part with all their worldly goods, the apostles and the 70 did. Matadormus desired to be one of the 70 new messengers, and that was the reason for Jesus’ requiring him to part with all of his temporal possessions.
\vs p163 2:7 \pc Almost every human being has some one thing which is held on to as a pet evil, and which the entrance into the kingdom of heaven requires as a part of the price of admission. If Matadormus had parted with his wealth, it probably would have been put right back into his hands for administration as treasurer of the 70. For later on, after the establishment of the church at Jerusalem, he did obey the Master’s injunction, although it was then too late to enjoy membership in the 70, and he became the treasurer of the Jerusalem church, of which James the Lord’s brother in the flesh was the head.
\vs p163 2:8 Thus always it was and forever will be: Men must arrive at their own decisions. There is a certain range of the freedom of choice which mortals may exercise. The forces of the spiritual world will not coerce man; they allow him to go the way of his own choosing.
\vs p163 2:9 Jesus foresaw that Matadormus, with his riches, could not possibly become an ordained associate of men who had forsaken all for the gospel; at the same time, he saw that, without his riches, he would become the ultimate leader of all of them. But, like Jesus’ own brethren, he never became great in the kingdom because he deprived himself of that intimate and personal association with the Master which might have been his experience had he been willing to do at this time the very thing which Jesus asked, and which, several years subsequently, he actually did.
\vs p163 2:10 Riches have nothing directly to do with entrance into the kingdom of heaven, but the \bibemph{love of wealth does.} The spiritual loyalties of the kingdom are incompatible with servility to materialistic mammon. Man may not share his supreme loyalty to a spiritual ideal with a material devotion.
\vs p163 2:11 Jesus never taught that it was wrong to have wealth. He required only the twelve and the 70 to dedicate all of their worldly possessions to the common cause. Even then, he provided for the profitable liquidation of their property, as in the case of the Apostle Matthew. Jesus many times advised his well\hyp{}to\hyp{}do disciples as he taught the rich man of Rome. The Master regarded the wise investment of excess earnings as a legitimate form of insurance against future and unavoidable adversity. When the apostolic treasury was overflowing, Judas put funds on deposit to be used subsequently when they might suffer greatly from a diminution of income. This Judas did after consultation with Andrew. Jesus never personally had anything to do with the apostolic finances except in the disbursement of alms. But there was one economic abuse which he many times condemned, and that was the unfair exploitation of the weak, unlearned, and less fortunate of men by their strong, keen, and more intelligent fellows. Jesus declared that such inhuman treatment of men, women, and children was incompatible with the ideals of the brotherhood of the kingdom of heaven.
\usection{The Discussion about Wealth}
\vs p163 3:1 By the time Jesus had finished talking with Matadormus, Peter and a number of the apostles had gathered about him, and as the rich young man was departing, Jesus turned around to face the apostles and said: \textcolour{ubdarkred}{“You see how difficult it is for those who have riches to enter fully into the kingdom of God! Spiritual worship cannot be shared with material devotions; no man can serve two masters. You have a saying that it is ‘easier for a camel to go through the eye of a needle than for the heathen to inherit eternal life.’ And I declare that it is as easy for this camel to go through the needle’s eye as for these self\hyp{}satisfied rich ones to enter the kingdom of heaven.”}
\vs p163 3:2 When Peter and the apostles heard these words, they were astonished exceedingly, so much so that Peter said: “Who then, Lord, can be saved? Shall all who have riches be kept out of the kingdom?” And Jesus replied: \textcolour{ubdarkred}{“No, Peter, but all who put their trust in riches shall hardly enter into the spiritual life that leads to eternal progress. But even then, much which is impossible to man is not beyond the reach of the Father in heaven; rather should we recognize that with God all things are possible.”}
\vs p163 3:3 As they went off by themselves, Jesus was grieved that Matadormus did not remain with them, for he greatly loved him. And when they had walked down by the lake, they sat there beside the water, and Peter, speaking for the twelve (who were all present by this time), said: “We are troubled by your words to the rich young man. Shall we require those who would follow you to give up all their worldly goods?” And Jesus said: \textcolour{ubdarkred}{“No, Peter, only those who would become apostles, and who desire to live with me as you do and as one family. But the Father requires that the affections of his children be pure and undivided. Whatever thing or person comes between you and the love of the truths of the kingdom, must be surrendered. If one’s wealth does not invade the precincts of the soul, it is of no consequence in the spiritual life of those who would enter the kingdom.”}
\vs p163 3:4 And then said Peter, “But, Master, we have left everything to follow you, what then shall we have?” And Jesus spoke to all of the twelve: \textcolour{ubdarkred}{“Verily, verily, I say to you, there is no man who has left wealth, home, wife, brethren, parents, or children for my sake and for the sake of the kingdom of heaven who shall not receive manifold more in this world, perhaps with some persecutions, and in the world to come eternal life. But many who are first shall be last, while the last shall often be first. The Father deals with his creatures in accordance with their needs and in obedience to his just laws of merciful and loving consideration for the welfare of a universe.}
\vs p163 3:5 \textcolour{ubdarkred}{“The kingdom of heaven is like a householder who was a large employer of men, and who went out early in the morning to hire labourers to work in his vineyard. When he had agreed with the labourers to pay them a denarius a day, he sent them into the vineyard. Then he went out about 9:00, and seeing others standing in the market place idle, he said to them: ‘Go you also to work in my vineyard, and whatsoever is right I will pay you.’ And they went at once to work. Again he went out about 12:00 and about 15:00 and did likewise. And going to the market place about 17:00, he found still others standing idle, and he inquired of them, ‘Why do you stand here idle all the day?’ And the men answered, ‘Because nobody has hired us.’ Then said the householder: ‘Go you also to work in my vineyard, and whatever is right I will pay you.’}
\vs p163 3:6 \textcolour{ubdarkred}{“When evening came, this owner of the vineyard said to his steward: ‘Call the labourers and pay them their wages, beginning with the last hired and ending with the first.’ When those who were hired about 17:00 came, they received a denarius each, and so it was with each of the other labourers. When the men who were hired at the beginning of the day saw how the later comers were paid, they expected to receive more than the amount agreed upon. But like the others every man received only a denarius. And when each had received his pay, they complained to the householder, saying: ‘These men who were hired last worked only one hour, and yet you have paid them the same as us who have borne the burden of the day in the scorching sun.’}
\vs p163 3:7 \textcolour{ubdarkred}{“Then answered the householder: ‘My friends, I do you no wrong. Did not each of you agree to work for a denarius a day? Take now that which is yours and go your way, for it is my desire to give to those who came last as much as I have given to you. Is it not lawful for me to do what I will with my own? or do you begrudge my generosity because I desire to be good and to show mercy?’”}
\usection{Farewell to the Seventy}
\vs p163 4:1 It was a stirring time about the Magadan Camp the day the 70 went forth on their first mission. Early that morning, in his last talk with the 70, Jesus placed emphasis on the following:
\vs p163 4:2 \ublistelem{1.}\bibnobreakspace The gospel of the kingdom must be proclaimed to all the world, to gentile as well as to Jew.
\vs p163 4:3 \ublistelem{2.}\bibnobreakspace While ministering to the sick, refrain from teaching the expectation of miracles.
\vs p163 4:4 \ublistelem{3.}\bibnobreakspace Proclaim a spiritual brotherhood of the sons of God, not an outward kingdom of worldly power and material glory.
\vs p163 4:5 \ublistelem{4.}\bibnobreakspace Avoid loss of time through overmuch social visiting and other trivialities which might detract from wholehearted devotion to preaching the gospel.
\vs p163 4:6 \ublistelem{5.}\bibnobreakspace If the first house to be selected for a headquarters proves to be a worthy home, abide there throughout the sojourn in that city.
\vs p163 4:7 \ublistelem{6.}\bibnobreakspace Make clear to all faithful believers that the time for an open break with the religious leaders of the Jews at Jerusalem has now come.
\vs p163 4:8 \ublistelem{7.}\bibnobreakspace Teach that man’s whole duty is summed up in this one commandment: Love the Lord your God with all your mind and soul and your neighbour as yourself. (This they were to teach as man’s whole duty in place of the 613 rules of living expounded by the Pharisees.)
\vs p163 4:9 \pc When Jesus had talked thus to the 70 in the presence of all the apostles and disciples, Simon Peter took them off by themselves and preached to them their ordination sermon, which was an elaboration of the Master’s charge given at the time he laid his hands upon them and set them apart as messengers of the kingdom. Peter exhorted the 70 to cherish in their experience the following virtues:
\vs p163 4:10 \ublistelem{1.}\bibnobreakspace \bibemph{Consecrated devotion.} To pray always for more labourers to be sent forth into the gospel harvest. He explained that, when one so prays, he will the more likely say, “Here am I; send me.” He admonished them to neglect not their daily worship.
\vs p163 4:11 \ublistelem{2.}\bibnobreakspace \bibemph{True courage.} He warned them that they would encounter hostility and be certain to meet with persecution. Peter told them their mission was no undertaking for cowards and advised those who were afraid to step out before they started. But none withdrew.
\vs p163 4:12 \ublistelem{3.}\bibnobreakspace \bibemph{Faith and trust.} They must go forth on this short mission wholly unprovided for; they must trust the Father for food and shelter and all other things needful.
\vs p163 4:13 \ublistelem{4.}\bibnobreakspace \bibemph{Zeal and initiative.} They must be possessed with zeal and intelligent enthusiasm; they must attend strictly to their Master’s business. Oriental salutation was a lengthy and elaborate ceremony; therefore had they been instructed to \textcolour{ubdarkred}{“salute no man by the way,”} which was a common method of exhorting one to go about his business without the waste of time. It had nothing to do with the matter of friendly greeting.
\vs p163 4:14 \ublistelem{5.}\bibnobreakspace \bibemph{Kindness and courtesy.} The Master had instructed them to avoid unnecessary waste of time in social ceremonies, but he enjoined courtesy toward all with whom they should come in contact. They were to show every kindness to those who might entertain them in their homes. They were strictly warned against leaving a modest home to be entertained in a more comfortable or influential one.
\vs p163 4:15 \ublistelem{6.}\bibnobreakspace \bibemph{Ministry to the sick.} The 70 were charged by Peter to search out the sick in mind and body and to do everything in their power to bring about the alleviation or cure of their maladies.
\vs p163 4:16 \pc And when they had been thus charged and instructed, they started out, two and two, on their mission in Galilee, Samaria, and Judea.
\vs p163 4:17 Although the Jews had a peculiar regard for the number 70, sometimes considering the nations of heathendom as being 70 in number, and although these 70 messengers were to go with the gospel to all peoples, still as far as we can discern, it was only coincidental that this group happened to number just 70. Certain it was that Jesus would have accepted no less than half a dozen others, but they were unwilling to pay the price of forsaking wealth and families.
\usection{Moving the Camp to Pella}
\vs p163 5:1 Jesus and the twelve now prepared to establish their last headquarters in Perea, near Pella, where the Master was baptized in the Jordan. The last ten days of November were spent in council at Magadan, and on Tuesday, December 6, the entire company of almost 300 started out at daybreak with all their effects to lodge that night near Pella by the river. This was the same site, by the spring, that John the Baptist had occupied with his camp several years before.
\vs p163 5:2 After the breaking up of the Magadan Camp, David Zebedee returned to Bethsaida and began immediately to curtail the messenger service. The kingdom was taking on a new phase. Daily, pilgrims arrived from all parts of Palestine and even from remote regions of the Roman Empire. Believers occasionally came from Mesopotamia and from the lands east of the Tigris. Accordingly, on Sunday, December 18, David, with the help of his messenger corps, loaded on to the pack animals the camp equipage, then stored in his father’s house, with which he had formerly conducted the camp of Bethsaida by the lake. Bidding farewell to Bethsaida for the time being, he proceeded down the lake shore and along the Jordan to a point about 2.4\,km north of the apostolic camp; and in less than a week he was prepared to offer hospitality to almost 1500 pilgrim visitors. The apostolic camp could accommodate about 500. This was the rainy season in Palestine, and these accommodations were required to take care of the ever\hyp{}increasing number of inquirers, mostly earnest, who came into Perea to see Jesus and to hear his teaching.
\vs p163 5:3 David did all this on his own initiative, though he had taken counsel with Philip and Matthew at Magadan. He employed the larger part of his former messenger corps as his helpers in conducting this camp; he now used less than 20 men on regular messenger duty. Near the end of December and before the return of the 70, almost 800 visitors were gathered about the Master, and they found lodging in David’s camp.
\usection{The Return of the Seventy}
\vs p163 6:1 On Friday, December 30, while Jesus was away in the near\hyp{}by hills with Peter, James, and John, the 70 messengers were arriving by couples, accompanied by numerous believers, at the Pella headquarters. All 70 were assembled at the teaching site about 17:00 when Jesus returned to the camp. The evening meal was delayed for more than an hour while these enthusiasts for the gospel of the kingdom related their experiences. David’s messengers had brought much of this news to the apostles during previous weeks, but it was truly inspiring to hear these newly ordained teachers of the gospel personally tell how their message had been received by hungry Jews and gentiles. At last Jesus was able to see men going out to spread the good news without his personal presence. The Master now knew that he could leave this world without seriously hindering the progress of the kingdom.
\vs p163 6:2 When the 70 related how “even the devils were subject” to them, they referred to the wonderful cures they had wrought in the cases of victims of nervous disorders. Nevertheless, there had been a few cases of real spirit possession relieved by these ministers, and referring to these, Jesus said: \textcolour{ubdarkred}{“It is not strange that these disobedient minor spirits should be subject to you, seeing that I beheld Satan falling as lightning from heaven. But rejoice not so much over this, for I declare to you that, as soon as I return to my Father, we will send forth our spirits into the very minds of men so that no more can these few lost spirits enter the minds of unfortunate mortals. I rejoice with you that you have power with men, but be not lifted up because of this experience but the rather rejoice that your names are written on the rolls of heaven, and that you are thus to go forward in an endless career of spiritual conquest.”}
\vs p163 6:3 And it was at this time, just before partaking of the evening meal, that Jesus experienced one of those rare moments of emotional ecstasy which his followers had occasionally witnessed. He said: \textcolour{ubdarkred}{“I thank you, my Father, Lord of heaven and earth, that, while this wonderful gospel was hidden from the wise and self\hyp{}righteous, the spirit has revealed these spiritual glories to these children of the kingdom. Yes, my Father, it must have been pleasing in your sight to do this, and I rejoice to know that the good news will spread to all the world even after I shall have returned to you and the work which you have given me to perform. I am mightily moved as I realize you are about to deliver all authority into my hands, that only you really know who I am, and that only I really know you, and those to whom I have revealed you. And when I have finished this revelation to my brethren in the flesh, I will continue the revelation to your creatures on high.”}
\vs p163 6:4 When Jesus had thus spoken to the Father, he turned aside to speak to his apostles and ministers: \textcolour{ubdarkred}{“Blessed are the eyes which see and the ears which hear these things. Let me say to you that many prophets and many of the great men of the past ages have desired to behold what you now see, but it was not granted them. And many generations of the children of light yet to come will, when they hear of these things, envy you who have heard and seen them.”}
\vs p163 6:5 Then, speaking to all the disciples, he said: \textcolour{ubdarkred}{“You have heard how many cities and villages have received the good news of the kingdom, and how my ministers and teachers have been received by both the Jew and the gentile. And blessed indeed are these communities which have elected to believe the gospel of the kingdom. But woe upon the light\hyp{}rejecting inhabitants of Chorazin, Bethsaida\hyp{}Julias, and Capernaum, the cities which did not well receive these messengers. I declare that, if the mighty works done in these places had been done in Tyre and Sidon, the people of these so\hyp{}called heathen cities would have long since repented in sackcloth and ashes. It shall indeed be more tolerable for Tyre and Sidon in the day of judgment.”}
\vs p163 6:6 \pc The next day being the Sabbath, Jesus went apart with the 70 and said to them: \textcolour{ubdarkred}{“I did indeed rejoice with you when you came back bearing the good tidings of the reception of the gospel of the kingdom by so many people scattered throughout Galilee, Samaria, and Judea. But why were you so surprisingly elated? Did you not expect that your message would manifest power in its delivery? Did you go forth with so little faith in this gospel that you come back in surprise at its effectiveness? And now, while I would not quench your spirit of rejoicing, I would sternly warn you against the subtleties of pride, spiritual pride. If you could understand the downfall of Lucifer, the iniquitous one, you would solemnly shun all forms of spiritual pride.}\tunemarkup{private}{\begin{figure}[H]\centering\includegraphics[width=\columnwidth]{images/Come.jpg}\caption{Come Unto Me by Harold~Copping}\end{figure}}
\vs p163 6:7 \textcolour{ubdarkred}{“You have entered upon this great work of teaching mortal man that he is a son of God. I have shown you the way; go forth to do your duty and be not weary in well doing. To you and to all who shall follow in your steps down through the ages, let me say: I always stand near, and my invitation\hyp{}call is, and ever shall be, Come to me all you who labour and are heavy laden, and I will give you rest. Take my yoke upon you and learn of me, for I am true and loyal, and you shall find spiritual rest for your souls.”}
\vs p163 6:8 \pc And they found the Master’s words to be true when they put his promises to the test. And since that day countless thousands also have tested and proved the surety of these same promises.
\usection{Preparation for the Last Mission}
\vs p163 7:1 The next few days were busy times in the Pella camp; preparations for the Perean mission were being completed. Jesus and his associates were about to enter upon their last mission, the three months’ tour of all Perea, which terminated only upon the Master’s entering Jerusalem for his final labours on earth. Throughout this period the headquarters of Jesus and the twelve apostles was maintained here at the Pella camp.
\vs p163 7:2 It was no longer necessary for Jesus to go abroad to teach the people. They now came to him in increasing numbers each week and from all parts, not only from Palestine but from the whole Roman world and from the Near East. Although the Master participated with the 70 in the tour of Perea, he spent much of his time at the Pella camp, teaching the multitude and instructing the twelve. Throughout this three months’ period at least ten of the apostles remained with Jesus.
\vs p163 7:3 The women’s corps also prepared to go out, two and two, with the 70 to labour in the larger cities of Perea. This original group of twelve women had recently trained a larger corps of 50 women in the work of home visitation and in the art of ministering to the sick and the afflicted. Perpetua, Simon Peter’s wife, became a member of this new division of the women’s corps and was entrusted with the leadership of the enlarged women’s work under Abner. After Pentecost she remained with her illustrious husband, accompanying him on all of his missionary tours; and on the day Peter was crucified in Rome, she was fed to the wild beasts in the arena. This new women’s corps also had as members the wives of Philip and Matthew and the mother of James and John.
\vs p163 7:4 The work of the kingdom now prepared to enter upon its terminal phase under the personal leadership of Jesus. And this present phase was one of spiritual depth in contrast with the miracle\hyp{}minded and wonder\hyp{}seeking multitudes who followed after the Master during the former days of popularity in Galilee. However, there were still any number of his followers who were material\hyp{}minded, and who failed to grasp the truth that the kingdom of heaven is the spiritual brotherhood of man founded on the eternal fact of the universal fatherhood of God.
\quizlink
