\upaper{146}{First Preaching Tour of Galilee}
\author{Midwayer Commission}
\vs p146 0:1 The first public preaching tour of Galilee began on Sunday, January 18, A.D.\,28, and continued for about two months, ending with the return to Capernaum on March 17. On this tour Jesus and the 12 apostles, assisted by the former apostles of John, preached the gospel and baptized believers in Rimmon, Jotapata, Ramah, Zebulun, Iron, Gischala, Chorazin, Madon, Cana, Nain, and Endor. In these cities they tarried and taught, while in many other smaller towns they proclaimed the gospel of the kingdom as they passed through.
\vs p146 0:2 This was the first time Jesus permitted his associates to preach without restraint. On this tour he cautioned them on only three occasions; he admonished them to remain away from Nazareth and to be discreet when passing through Capernaum and Tiberias. It was a source of great satisfaction to the apostles at last to feel they were at liberty to preach and teach without restriction, and they threw themselves into the work of preaching the gospel, ministering to the sick, and baptizing believers, with great earnestness and joy.
\usection{1.\bibnobreakspace Preaching at Rimmon}
\vs p146 1:1 The small city of Rimmon had once been dedicated to the worship of a Babylonian god of the air, Ramman. Many of the earlier Babylonian and later Zoroastrian teachings were still embraced in the beliefs of the Rimmonites; therefore did Jesus and the 24 devote much of their time to the task of making plain the difference between these older beliefs and the new gospel of the kingdom. Peter here preached one of the great sermons of his early career on “Aaron and the Golden Calf.”
\vs p146 1:2 Although many of the citizens of Rimmon became believers in Jesus’ teachings, they made great trouble for their brethren in later years. It is difficult to convert nature worshippers to the full fellowship of the adoration of a spiritual ideal during the short space of a single lifetime.
\vs p146 1:3 \pc Many of the better of the Babylonian and Persian ideas of light and darkness, good and evil, time and eternity, were later incorporated in the doctrines of so\hyp{}called Christianity, and their inclusion rendered the Christian teachings more immediately acceptable to the peoples of the Near East. In like manner, the inclusion of many of Plato’s theories of the ideal spirit or invisible patterns of all things visible and material, as later adapted by Philo to the Hebrew theology, made Paul’s Christian teachings more easy of acceptance by the western Greeks.
\vs p146 1:4 \pc It was at Rimmon that Todan first heard the gospel of the kingdom, and he later carried this message into Mesopotamia and far beyond. He was among the first to preach the good news to those who dwelt beyond the Euphrates.
\usection{2.\bibnobreakspace At Jotapata}
\vs p146 2:1 While the common people of Jotapata heard Jesus and his apostles gladly and many accepted the gospel of the kingdom, it was the discourse of Jesus to the 24 on the second evening of their sojourn in this small town that distinguishes the Jotapata mission. Nathaniel was confused in his mind about the Master’s teachings concerning prayer, thanksgiving, and worship, and in response to his question Jesus spoke at great length in further explanation of his teaching. Summarized in modern phraseology, this discourse may be presented as emphasizing the following points:
\vs p146 2:2 \ublistelem{1.}\bibnobreakspace The conscious and persistent regard for iniquity in the heart of man gradually destroys the prayer connection of the human soul with the spirit circuits of communication between man and his Maker. Naturally God hears the petition of his child, but when the human heart deliberately and persistently harbours the concepts of iniquity, there gradually ensues the loss of personal communion between the earth child and his heavenly Father.
\vs p146 2:3 \ublistelem{2.}\bibnobreakspace That prayer which is inconsistent with the known and established laws of God is an abomination to the Paradise Deities. If man will not listen to the Gods as they speak to their creation in the laws of spirit, mind, and matter, the very act of such deliberate and conscious disdain by the creature turns the ears of spirit personalities away from hearing the personal petitions of such lawless and disobedient mortals. Jesus quoted to his apostles from the Prophet Zechariah: “But they refused to hearken and pulled away the shoulder and stopped their ears that they should not hear. Yes, they made their hearts adamant like a stone, lest they should hear my law and the words which I sent by my spirit through the prophets; therefore did the results of their evil thinking come as a great wrath upon their guilty heads. And so it came to pass that they cried for mercy, but there was no ear open to hear.” And then Jesus quoted the proverb of the wise man who said: “He who turns away his ear from hearing the divine law, even his prayer shall be an abomination.”
\vs p146 2:4 \ublistelem{3.}\bibnobreakspace By opening the human end of the channel of the God\hyp{}man communication, mortals make immediately available the ever\hyp{}flowing stream of divine ministry to the creatures of the worlds. When man hears God’s spirit speak within the human heart, inherent in such an experience is the fact that God simultaneously hears that man’s prayer. Even the forgiveness of sin operates in this same unerring fashion. The Father in heaven has forgiven you even before you have thought to ask him, but such forgiveness is not available in your personal religious experience until such a time as you forgive your fellow men. God’s forgiveness in \bibemph{fact} is not conditioned upon your forgiving your fellows, but in \bibemph{experience} it is exactly so conditioned. And this fact of the synchrony of divine and human forgiveness was thus recognized and linked together in the prayer which Jesus taught the apostles.
\vs p146 2:5 \ublistelem{4.}\bibnobreakspace There is a basic law of justice in the universe which mercy is powerless to circumvent. The unselfish glories of Paradise are not possible of reception by a thoroughly selfish creature of the realms of time and space. Even the infinite love of God cannot force the salvation of eternal survival upon any mortal creature who does not choose to survive. Mercy has great latitude of bestowal, but, after all, there are mandates of justice which even love combined with mercy cannot effectively abrogate. Again Jesus quoted from the Hebrew scriptures: \textcolour{ubdarkred}{“I have called and you refused to hear; I stretched out my hand, but no man regarded. You have set at naught all my counsel, and you have rejected my reproof, and because of this rebellious attitude it becomes inevitable that you shall call upon me and fail to receive an answer. Having rejected the way of life, you may seek me diligently in your times of suffering, but you will not find me.”}
\vs p146 2:6 \ublistelem{5.}\bibnobreakspace They who would receive mercy must show mercy; judge not that you be not judged. With the spirit with which you judge others you also shall be judged. Mercy does not wholly abrogate universe fairness. In the end it will prove true: “Whoso stops his ears to the cry of the poor, he also shall some day cry for help, and no one will hear him.” The sincerity of any prayer is the assurance of its being heard; the spiritual wisdom and universe consistency of any petition is the determiner of the time, manner, and degree of the answer. A wise father does not \bibemph{literally} answer the foolish prayers of his ignorant and inexperienced children, albeit the children may derive much pleasure and real soul satisfaction from the making of such absurd petitions.
\vs p146 2:7 \ublistelem{6.}\bibnobreakspace When you have become wholly dedicated to the doing of the will of the Father in heaven, the answer to all your petitions will be forthcoming because your prayers will be in full accordance with the Father’s will, and the Father’s will is ever manifest throughout his vast universe. What the true son desires and the infinite Father wills IS. Such a prayer cannot remain unanswered, and no other sort of petition can possibly be fully answered.
\vs p146 2:8 \ublistelem{7.}\bibnobreakspace The cry of the righteous is the faith act of the child of God which opens the door of the Father’s storehouse of goodness, truth, and mercy, and these good gifts have long been in waiting for the son’s approach and personal appropriation. Prayer does not change the divine attitude toward man, but it does change man’s attitude toward the changeless Father. The \bibemph{motive} of the prayer gives it right of way to the divine ear, not the social, economic, or outward religious status of the one who prays.
\vs p146 2:9 \ublistelem{8.}\bibnobreakspace Prayer may not be employed to avoid the delays of time or to transcend the handicaps of space. Prayer is not designed as a technique for aggrandizing self or for gaining unfair advantage over one’s fellows. A thoroughly selfish soul cannot pray in the true sense of the word. Said Jesus: \textcolour{ubdarkred}{“Let your supreme delight be in the character of God, and he shall surely give you the sincere desires of your heart.” “Commit your way to the Lord; trust in him, and he will act.” “For the Lord hears the cry of the needy, and he will regard the prayer of the destitute.”}
\vs p146 2:10 \ublistelem{9.}\bibnobreakspace \textcolour{ubdarkred}{“I have come forth from the Father; if, therefore, you are ever in doubt as to what you would ask of the Father, ask in my name, and I will present your petition in accordance with your real needs and desires and in accordance with my Father’s will.”} Guard against the great danger of becoming self\hyp{}centred in your prayers. Avoid praying much for yourself; pray more for the spiritual progress of your brethren. Avoid materialistic praying; pray in the spirit and for the abundance of the gifts of the spirit.
\vs p146 2:11 \ublistelem{10.}\bibnobreakspace When you pray for the sick and afflicted, do not expect that your petitions will take the place of loving and intelligent ministry to the necessities of these afflicted ones. Pray for the welfare of your families, friends, and fellows, but especially pray for those who curse you, and make loving petitions for those who persecute you. \textcolour{ubdarkred}{“But when to pray, I will not say. Only the spirit that dwells within you may move you to the utterance of those petitions which are expressive of your inner relationship with the Father of spirits.”}
\vs p146 2:12 \ublistelem{11.}\bibnobreakspace Many resort to prayer only when in trouble. Such a practice is thoughtless and misleading. True, you do well to pray when harassed, but you should also be mindful to speak as a son to your Father even when all goes well with your soul. Let your real petitions always be in secret. Do not let men hear your personal prayers. Prayers of thanksgiving are appropriate for groups of worshippers, but the prayer of the soul is a personal matter. There is but one form of prayer which is appropriate for all God’s children, and that is: \textcolour{ubdarkred}{“Nevertheless, your will be done.”}
\vs p146 2:13 \ublistelem{12.}\bibnobreakspace All believers in this gospel should pray sincerely for the extension of the kingdom of heaven. Of all the prayers of the Hebrew scriptures he commented most approvingly on the petition of the Psalmist: “Create in me a clean heart, O God, and renew a right spirit within me. Purge me from secret sins and keep back your servant from presumptuous transgression.” Jesus commented at great length on the relation of prayer to careless and offending speech, quoting: \textcolour{ubdarkred}{“Set a watch, O Lord, before my mouth; keep the door of my lips.” “The human tongue,”} said Jesus, \textcolour{ubdarkred}{“is a member which few men can tame, but the spirit within can transform this unruly member into a kindly voice of tolerance and an inspiring minister of mercy.”}
\vs p146 2:14 \ublistelem{13.}\bibnobreakspace Jesus taught that the prayer for divine guidance over the pathway of earthly life was next in importance to the petition for a knowledge of the Father’s will. In reality this means a prayer for divine wisdom. Jesus never taught that human knowledge and special skill could be gained by prayer. But he did teach that prayer is a factor in the enlargement of one’s capacity to receive the presence of the divine spirit. When Jesus taught his associates to pray in the spirit and in truth, he explained that he referred to praying sincerely and in accordance with one’s enlightenment, to praying wholeheartedly and intelligently, earnestly and steadfastly.
\vs p146 2:15 \ublistelem{14.}\bibnobreakspace Jesus warned his followers against thinking that their prayers would be rendered more efficacious by ornate repetitions, eloquent phraseology, fasting, penance, or sacrifices. But he did exhort his believers to employ prayer as a means of leading up through thanksgiving to true worship. Jesus deplored that so little of the spirit of thanksgiving was to be found in the prayers and worship of his followers. He quoted from the Scriptures on this occasion, saying: \textcolour{ubdarkred}{“It is a good thing to give thanks to the Lord and to sing praises to the name of the Most High, to acknowledge his loving\hyp{}kindness every morning and his faithfulness every night, for God has made me glad through his work. In everything I will give thanks according to the will of God.”}
\vs p146 2:16 \ublistelem{15.}\bibnobreakspace And then Jesus said: \textcolour{ubdarkred}{“Be not constantly overanxious about your common needs. Be not apprehensive concerning the problems of your earthly existence, but in all these things by prayer and supplication, with the spirit of sincere thanksgiving, let your needs be spread out before your Father who is in heaven.”} Then he quoted from the Scriptures: \textcolour{ubdarkred}{“I will praise the name of God with a song and will magnify him with thanksgiving. And this will please the Lord better than the sacrifice of an ox or bullock with horns and hoofs.”}
\vs p146 2:17 \ublistelem{16.}\bibnobreakspace Jesus taught his followers that, when they had made their prayers to the Father, they should remain for a time in silent receptivity to afford the indwelling spirit the better opportunity to speak to the listening soul. The spirit of the Father speaks best to man when the human mind is in an attitude of true worship. We worship God by the aid of the Father’s indwelling spirit and by the illumination of the human mind through the ministry of truth. Worship, taught Jesus, makes one increasingly like the being who is worshipped. Worship is a transforming experience whereby the finite gradually approaches and ultimately attains the presence of the Infinite.
\vs p146 2:18 \pc And many other truths did Jesus tell his apostles about man’s communion with God, but not many of them could fully encompass his teaching.
\usection{3.\bibnobreakspace The Stop at Ramah}
\vs p146 3:1 At Ramah Jesus had the memorable discussion with the aged Greek philosopher who taught that science and philosophy were sufficient to satisfy the needs of human experience. Jesus listened with patience and sympathy to this Greek teacher, allowing the truth of many things he said but pointing out that, when he was through, he had failed in his discussion of human existence to explain \textcolour{ubdarkred}{“whence, why, and whither,”} and added: \textcolour{ubdarkred}{“Where you leave off, we begin. Religion is a revelation to man’s soul dealing with spiritual realities which the mind alone could never discover or fully fathom. Intellectual strivings may reveal the facts of life, but the gospel of the kingdom unfolds the \bibemph{truths} of being. You have discussed the material shadows of truth; will you now listen while I tell you about the eternal and spiritual realities which cast these transient time shadows of the material facts of mortal existence?”} For more than an hour Jesus taught this Greek the saving truths of the gospel of the kingdom. The old philosopher was susceptible to the Master’s mode of approach, and being sincerely honest of heart, he quickly believed this gospel of salvation.
\vs p146 3:2 The apostles were a bit disconcerted by the open manner of Jesus’ assent to many of the Greek’s propositions, but Jesus afterwards privately said to them: \textcolour{ubdarkred}{“My children, marvel not that I was tolerant of the Greek’s philosophy. True and genuine inward certainty does not in the least fear outward analysis, nor does truth resent honest criticism. You should never forget that intolerance is the mask covering up the entertainment of secret doubts as to the trueness of one’s belief. No man is at any time disturbed by his neighbour’s attitude when he has perfect confidence in the truth of that which he wholeheartedly believes. Courage is the confidence of thoroughgoing honesty about those things which one professes to believe. Sincere men are unafraid of the critical examination of their true convictions and noble ideals.”}
\vs p146 3:3 \pc On the second evening at Ramah, Thomas asked Jesus this question: “Master, how can a new believer in your teaching really know, really be certain, about the truth of this gospel of the kingdom?”
\vs p146 3:4 And Jesus said to Thomas: \textcolour{ubdarkred}{“Your assurance that you have entered into the kingdom family of the Father, and that you will eternally survive with the children of the kingdom, is wholly a matter of personal experience --- faith in the word of truth. Spiritual assurance is the equivalent of your personal religious experience in the eternal realities of divine truth and is otherwise equal to your intelligent understanding of truth realities plus your spiritual faith and minus your honest doubts.}
\vs p146 3:5 \textcolour{ubdarkred}{“The Son is naturally endowed with the life of the Father. Having been endowed with the living spirit of the Father, you are therefore sons of God. You survive your life in the material world of the flesh because you are identified with the Father’s living spirit, the gift of eternal life. Many, indeed, had this life before I came forth from the Father, and many more have received this spirit because they believed my word; but I declare that, when I return to the Father, he will send his spirit into the hearts of all men.}
\vs p146 3:6 \textcolour{ubdarkred}{“While you cannot observe the divine spirit at work in your minds, there is a practical method of discovering the degree to which you have yielded the control of your soul powers to the teaching and guidance of this indwelling spirit of the heavenly Father, and that is the degree of your love for your fellow men. This spirit of the Father partakes of the love of the Father, and as it dominates man, it unfailingly leads in the directions of divine worship and loving regard for one’s fellows. At first you believe that you are sons of God because my teaching has made you more conscious of the inner leadings of our Father’s indwelling presence; but presently the Spirit of Truth shall be poured out upon all flesh, and it will live among men and teach all men, even as I now live among you and speak to you the words of truth. And this Spirit of Truth, speaking for the spiritual endowments of your souls, will help you to know that you are the sons of God. It will unfailingly bear witness with the Father’s indwelling presence, your spirit, then dwelling in all men as it now dwells in some, telling you that you are in reality the sons of God.}
\vs p146 3:7 \textcolour{ubdarkred}{“Every earth child who follows the leading of this spirit shall eventually know the will of God, and he who surrenders to the will of my Father shall abide forever. The way from the earth life to the eternal estate has not been made plain to you, but there is a way, there always has been, and I have come to make that way new and living. He who enters the kingdom has eternal life already --- he shall never perish. But much of this you will the better understand when I shall have returned to the Father and you are able to view your present experiences in retrospect.”}
\vs p146 3:8 And all who heard these blessed words were greatly cheered. The Jewish teachings had been confused and uncertain regarding the survival of the righteous, and it was refreshing and inspiring for Jesus’ followers to hear these very definite and positive words of assurance about the eternal survival of all true believers.
\vs p146 3:9 \pc The apostles continued to preach and baptize believers, while they kept up the practice of visiting from house to house, comforting the downcast and ministering to the sick and afflicted. The apostolic organization was expanded in that each of Jesus’ apostles now had one of John’s as an associate; Abner was the associate of Andrew; and this plan prevailed until they went down to Jerusalem for the next Passover.
\vs p146 3:10 \pc The special instruction given by Jesus during their stay at Zebulun had chiefly to do with further discussions of the mutual obligations of the kingdom and embraced teaching designed to make clear the differences between personal religious experience and the amities of social religious obligations. This was one of the few times the Master ever discussed the social aspects of religion. Throughout his entire earth life Jesus gave his followers very little instruction regarding the socialization of religion.
\vs p146 3:11 In Zebulun the people were of a mixed race, hardly Jew or gentile, and few of them really believed in Jesus, notwithstanding they had heard of the healing of the sick at Capernaum.
\usection{4.\bibnobreakspace The Gospel at Iron}
\vs p146 4:1 At Iron, as in many of even the smaller cities of Galilee and Judea, there was a synagogue, and during the earlier times of Jesus’ ministry it was his custom to speak in these synagogues on the Sabbath day. Sometimes he would speak at the morning service, and Peter or one of the other apostles would preach at the afternoon hour. Jesus and the apostles would also often teach and preach at the weekday evening assemblies at the synagogue. Although the religious leaders at Jerusalem became increasingly antagonistic toward Jesus, they exercised no direct control over the synagogues outside of that city. It was not until later in Jesus’ public ministry that they were able to create such a widespread sentiment against him as to bring about the almost universal closing of the synagogues to his teaching. At this time all the synagogues of Galilee and Judea were open to him.\fnc{\ldots{}teach and preach at the \bibtextul{week-day} evening assemblies\ldots{} \bibexpl{The closed form weekday, unlike week-end/week end, is the one found in both Webster’s and OED; further, as noted for \bibref[142:8.4]{p0142 8:4}, it was decided that “weekday” and “weekend” should have the same format (as they do in modern usage).}}
\vs p146 4:2 Iron was the site of extensive mineral mines for those days, and since Jesus had never shared the life of the miner, he spent most of his time, while sojourning at Iron, in the mines. While the apostles visited the homes and preached in the public places, Jesus worked in the mines with these underground labourers. The fame of Jesus as a healer had spread even to this remote village, and many sick and afflicted sought help at his hands, and many were greatly benefited by his healing ministry. But in none of these cases did the Master perform a so\hyp{}called miracle of healing save in that of the leper.
\vs p146 4:3 \pc Late on the afternoon of the third day at Iron, as Jesus was returning from the mines, he chanced to pass through a narrow side street on his way to his lodging place. As he drew near the squalid hovel of a certain leprous man, the afflicted one, having heard of his fame as a healer, made bold to accost him as he passed his door, saying as he knelt before him: “Lord, if only you would, you could make me clean. I have heard the message of your teachers, and I would enter the kingdom if I could be made clean.” And the leper spoke in this way because among the Jews lepers were forbidden even to attend the synagogue or otherwise engage in public worship. This man really believed that he could not be received into the coming kingdom unless he could find a cure for his leprosy. And when Jesus saw him in his affliction and heard his words of clinging faith, his human heart was touched, and the divine mind was moved with compassion. As Jesus looked upon him, the man fell upon his face and worshipped. Then the Master stretched forth his hand and, touching him, said: \textcolour{ubdarkred}{“I will --- be clean.”} And immediately he was healed; the leprosy no longer afflicted him.
\vs p146 4:4 When Jesus had lifted the man upon his feet, he charged him: \textcolour{ubdarkred}{“See that you tell no man about your healing but rather go quietly about your business, showing yourself to the priest and offering those sacrifices commanded by Moses in testimony of your cleansing.”} But this man did not do as Jesus had instructed him. Instead, he began to publish abroad throughout the town that Jesus had cured his leprosy, and since he was known to all the village, the people could plainly see that he had been cleansed of his disease. He did not go to the priests as Jesus had admonished him. As a result of his spreading abroad the news that Jesus had healed him, the Master was so thronged by the sick that he was forced to rise early the next day and leave the village. Although Jesus did not again enter the town, he remained two days in the outskirts near the mines, continuing to instruct the believing miners further regarding the gospel of the kingdom.
\vs p146 4:5 This cleansing of the leper was the first so\hyp{}called miracle which Jesus had intentionally and deliberately performed up to this time. And this was a case of real leprosy.
\vs p146 4:6 \pc From Iron they went to Gischala, spending two days proclaiming the gospel, and then departed for Chorazin, where they spent almost a week preaching the good news; but they were unable to win many believers for the kingdom in Chorazin. In no place where Jesus had taught had he met with such a general rejection of his message. The sojourn at Chorazin was very depressing to most of the apostles, and Andrew and Abner had much difficulty in upholding the courage of their associates. And so, passing quietly through Capernaum, they went on to the village of Madon, where they fared little better. There prevailed in the minds of most of the apostles the idea that their failure to meet with success in these towns so recently visited was due to Jesus’ insistence that they refrain, in their teaching and preaching, from referring to him as a healer. How they wished he would cleanse another leper or in some other manner so manifest his power as to attract the attention of the people! But the Master was unmoved by their earnest urging.
\usection{5.\bibnobreakspace Back in Cana}
\vs p146 5:1 The apostolic party was greatly cheered when Jesus announced, \textcolour{ubdarkred}{“Tomorrow we go to Cana.”} They knew they would have a sympathetic hearing at Cana, for Jesus was well known there. They were doing well with their work of bringing people into the kingdom when, on the third day, there arrived in Cana a certain prominent citizen of Capernaum, Titus, who was a partial believer, and whose son was critically ill. He heard that Jesus was at Cana; so he hastened over to see him. The believers at Capernaum thought Jesus could heal any sickness.
\vs p146 5:2 When this nobleman had located Jesus in Cana, he besought him to hurry over to Capernaum and heal his afflicted son. While the apostles stood by in breathless expectancy, Jesus, looking at the father of the sick boy, said: \textcolour{ubdarkred}{“How long shall I bear with you? The power of God is in your midst, but except you see signs and behold wonders, you refuse to believe.”} But the nobleman pleaded with Jesus, saying: “My Lord, I do believe, but come ere my child perishes, for when I left him he was even then at the point of death.” And when Jesus had bowed his head a moment in silent meditation, he suddenly spoke, \textcolour{ubdarkred}{“Return to your home; your son will live.”} Titus believed the word of Jesus and hastened back to Capernaum. And as he was returning, his servants came out to meet him, saying, “Rejoice, for your son is improved --- he lives.” Then Titus inquired of them at what hour the boy began to mend, and when the servants answered “yesterday about the seventh hour the fever left him,” the father recalled that it was about that hour when Jesus had said, “Your son will live.” And Titus henceforth believed with a whole heart, and all his family also believed. This son became a mighty minister of the kingdom and later yielded up his life with those who suffered in Rome. Though the entire household of Titus, their friends, and even the apostles regarded this episode as a miracle, it was not. At least this was not a miracle of curing physical disease. It was merely a case of preknowledge concerning the course of natural law, just such knowledge as Jesus frequently resorted to subsequent to his baptism.
\vs p146 5:3 Again was Jesus compelled to hasten away from Cana because of the undue attention attracted by the second episode of this sort to attend his ministry in this village. The townspeople remembered the water and the wine, and now that he was supposed to have healed the nobleman’s son at so great a distance, they came to him, not only bringing the sick and afflicted but also sending messengers requesting that he heal sufferers at a distance. And when Jesus saw that the whole countryside was aroused, he said, \textcolour{ubdarkred}{“Let us go to Nain.”}
\usection{6.\bibnobreakspace Nain and the Widow’s Son}
\vs p146 6:1 These people believed in signs; they were a wonder\hyp{}seeking generation. By this time the people of central and southern Galilee had become miracle minded regarding Jesus and his personal ministry. Scores, hundreds, of honest persons suffering from purely nervous disorders and afflicted with emotional disturbances came into Jesus’ presence and then returned home to their friends announcing that Jesus had healed them. And such cases of mental healing these ignorant and simple\hyp{}minded people regarded as physical healing, miraculous cures.
\vs p146 6:2 \pc When Jesus sought to leave Cana and go to Nain, a great multitude of believers and many curious people followed after him. They were bent on beholding miracles and wonders, and they were not to be disappointed. As Jesus and his apostles drew near the gate of the city, they met a funeral procession on its way to the near\hyp{}by cemetery, carrying the only son of a widowed mother of Nain. This woman was much respected, and half of the village followed the bearers of the bier of this supposedly dead boy. When the funeral procession had come up to Jesus and his followers, the widow and her friends recognized the Master and besought him to bring the son back to life. Their miracle expectancy was aroused to such a high pitch they thought Jesus could cure any human disease, and why could not such a healer even raise the dead? Jesus, while being thus importuned, stepped forward and, raising the covering of the bier, examined the boy. Discovering that the young man was not really dead, he perceived the tragedy which his presence could avert; so, turning to the mother, he said: \textcolour{ubdarkred}{“Weep not. Your son is not dead; he sleeps. He will be restored to you.”} And then, taking the young man by the hand, he said, \textcolour{ubdarkred}{“Awake and arise.”} And the youth who was supposed to be dead presently sat up and began to speak, and Jesus sent them back to their homes.
\vs p146 6:3 Jesus endeavoured to calm the multitude and vainly tried to explain that the lad was not really dead, that he had not brought him back from the grave, but it was useless. The multitude which followed him, and the whole village of Nain, were aroused to the highest pitch of emotional frenzy. Fear seized many, panic others, while still others fell to praying and wailing over their sins. And it was not until long after nightfall that the clamouring multitude could be dispersed. And, of course, notwithstanding Jesus’ statement that the boy was not dead, everyone insisted that a miracle had been wrought, even the dead raised. Although Jesus told them the boy was merely in a deep sleep, they explained that that was the manner of his speaking and called attention to the fact that he always in great modesty tried to hide his miracles.
\vs p146 6:4 So the word went abroad throughout Galilee and into Judea that Jesus had raised the widow’s son from the dead, and many who heard this report believed it. Never was Jesus able to make even all his apostles fully understand that the widow’s son was not really dead when he bade him awake and arise. But he did impress them sufficiently to keep it out of all subsequent records except that of Luke, who recorded it as the episode had been related to him. And again was Jesus so besieged as a physician that he departed early the next day for Endor.
\usection{7.\bibnobreakspace At Endor}
\vs p146 7:1 At Endor Jesus escaped for a few days from the clamouring multitudes in quest of physical healing. During their sojourn at this place the Master recounted for the instruction of the apostles the story of King Saul and the witch of Endor. Jesus plainly told his apostles that the stray and rebellious midwayers who had oftentimes impersonated the supposed spirits of the dead would soon be brought under control so that they could no more do these strange things. He told his followers that, after he returned to the Father, and after they had poured out their spirit upon all flesh, no more could such semispirit beings --- so\hyp{}called unclean spirits --- possess the feeble\hyp{} and evil\hyp{}minded among mortals.
\vs p146 7:2 Jesus further explained to his apostles that the spirits of departed human beings do not come back to the world of their origin to communicate with their living fellows. Only after the passing of a dispensational age would it be possible for the advancing spirit of mortal man to return to earth and then only in exceptional cases and as a part of the spiritual administration of the planet.
\vs p146 7:3 When they had rested two days, Jesus said to his apostles: \textcolour{ubdarkred}{“On the morrow let us return to Capernaum to tarry and teach while the countryside quiets down. At home they will have by this time partly recovered from this sort of excitement.”}
