\upaper{196}{The Faith of Jesus}
\author{Midwayer Commission}
\vs p196 0:1 Jesus enjoyed a sublime and wholehearted faith in God. He experienced the ordinary ups and downs of mortal existence, but he never religiously doubted the certainty of God’s watchcare and guidance. His faith was the outgrowth of the insight born of the activity of the divine presence, his indwelling Adjuster. His faith was neither traditional nor merely intellectual; it was wholly personal and purely spiritual.
\vs p196 0:2 The human Jesus saw God as being holy, just, and great, as well as being true, beautiful, and good. All these attributes of divinity he focused in his mind as the “will of the Father in heaven.” Jesus’ God was at one and the same time “The Holy One of Israel” and “The living and loving Father in heaven.” The concept of God as a Father was not original with Jesus, but he exalted and elevated the idea into a sublime experience by achieving a new revelation of God and by proclaiming that every mortal creature is a child of this Father of love, a son of God.
\vs p196 0:3 Jesus did not cling to faith in God as would a struggling soul at war with the universe and at death grips with a hostile and sinful world; he did not resort to faith merely as a consolation in the midst of difficulties or as a comfort in threatened despair; faith was not just an illusory compensation for the unpleasant realities and the sorrows of living. In the very face of all the natural difficulties and the temporal contradictions of mortal existence, he experienced the tranquillity of supreme and unquestioned trust in God and felt the tremendous thrill of living, by faith, in the very presence of the heavenly Father. And this triumphant faith was a living experience of actual spirit attainment. Jesus’ great contribution to the values of human experience was not that he revealed so many new ideas about the Father in heaven, but rather that he so magnificently and humanly demonstrated a new and higher type of \bibemph{living faith in God.} Never on all the worlds of this universe, in the life of any one mortal, did God ever become such a \bibemph{living reality} as in the human experience of Jesus of Nazareth.
\vs p196 0:4 In the Master’s life on Urantia, this and all other worlds of the local creation discover a new and higher type of religion, religion based on personal spiritual relations with the Universal Father and wholly validated by the supreme authority of genuine personal experience. This living faith of Jesus was more than an intellectual reflection, and it was not a mystic meditation.
\vs p196 0:5 Theology may fix, formulate, define, and dogmatize faith, but in the human life of Jesus faith was personal, living, original, spontaneous, and purely spiritual. This faith was not reverence for tradition nor a mere intellectual belief which he held as a sacred creed, but rather a sublime experience and a profound conviction which \bibemph{securely held him.} His faith was so real and all\hyp{}encompassing that it absolutely swept away any spiritual doubts and effectively destroyed every conflicting desire. Nothing was able to tear him away from the spiritual anchorage of this fervent, sublime, and undaunted faith. Even in the face of apparent defeat or in the throes of disappointment and threatening despair, he calmly stood in the divine presence free from fear and fully conscious of spiritual invincibility. Jesus enjoyed the invigorating assurance of the possession of unflinching faith, and in each of life’s trying situations he unfailingly exhibited an unquestioning loyalty to the Father’s will. And this superb faith was undaunted even by the cruel and crushing threat of an ignominious death.
\vs p196 0:6 In a religious genius, strong spiritual faith so many times leads directly to disastrous fanaticism, to exaggeration of the religious ego, but it was not so with Jesus. He was not unfavourably affected in his practical life by his extraordinary faith and spirit attainment because this spiritual exaltation was a wholly unconscious and spontaneous soul expression of his personal experience with God.
\vs p196 0:7 The all\hyp{}consuming and indomitable spiritual faith of Jesus never became fanatical, for it never attempted to run away with his well\hyp{}balanced intellectual judgments concerning the proportional values of practical and commonplace social, economic, and moral life situations. The Son of Man was a splendidly unified human personality; he was a perfectly endowed divine being; he was also magnificently co\hyp{}ordinated as a combined human and divine being functioning on earth as a single personality. Always did the Master co\hyp{}ordinate the faith of the soul with the wisdom\hyp{}appraisals of seasoned experience. Personal faith, spiritual hope, and moral devotion were always correlated in a matchless religious unity of harmonious association with the keen realization of the reality and sacredness of all human loyalties --- personal honour, family love, religious obligation, social duty, and economic necessity.
\vs p196 0:8 The faith of Jesus visualized all spirit values as being found in the kingdom of God; therefore he said, \textcolour{ubdarkred}{“Seek first the kingdom of heaven.”} Jesus saw in the advanced and ideal fellowship of the kingdom the achievement and fulfilment of the “will of God.” The very heart of the prayer which he taught his disciples was, “Your kingdom come; your will be done.” Having thus conceived of the kingdom as comprising the will of God, he devoted himself to the cause of its realization with amazing self\hyp{}forgetfulness and unbounded enthusiasm. But in all his intense mission and throughout his extraordinary life there never appeared the fury of the fanatic nor the superficial frothiness of the religious egotist.
\vs p196 0:9 The Master’s entire life was consistently conditioned by this living faith, this sublime religious experience. This spiritual attitude wholly dominated his thinking and feeling, his believing and praying, his teaching and preaching. This personal faith of a son in the certainty and security of the guidance and protection of the heavenly Father imparted to his unique life a profound endowment of spiritual reality. And yet, despite this very deep consciousness of close relationship with divinity, this Galilean, God’s Galilean, when addressed as Good Teacher, instantly replied, \textcolour{ubdarkred}{“Why do you call me good?”} When we stand confronted by such splendid self\hyp{}forgetfulness, we begin to understand how the Universal Father found it possible so fully to manifest himself to him and reveal himself through him to the mortals of the realms.
\vs p196 0:10 Jesus brought to God, as a man of the realm, the greatest of all offerings: the consecration and dedication of his own will to the majestic service of doing the divine will. Jesus always and consistently interpreted religion wholly in terms of the Father’s will. When you study the career of the Master, as concerns prayer or any other feature of the religious life, look not so much for what he taught as for what he did. Jesus never prayed as a religious duty. To him prayer was a sincere expression of spiritual attitude, a declaration of soul loyalty, a recital of personal devotion, an expression of thanksgiving, an avoidance of emotional tension, a prevention of conflict, an exaltation of intellection, an ennoblement of desire, a vindication of moral decision, an enrichment of thought, an invigoration of higher inclinations, a consecration of impulse, a clarification of viewpoint, a declaration of faith, a transcendental surrender of will, a sublime assertion of confidence, a revelation of courage, the proclamation of discovery, a confession of supreme devotion, the validation of consecration, a technique for the adjustment of difficulties, and the mighty mobilization of the combined soul powers to withstand all human tendencies toward selfishness, evil, and sin. He lived just such a life of prayerful consecration to the doing of his Father’s will and ended his life triumphantly with just such a prayer. The secret of his unparalleled religious life was this consciousness of the presence of God; and he attained it by intelligent prayer and sincere worship --- unbroken communion with God --- and not by leadings, voices, visions, or extraordinary religious practices.
\vs p196 0:11 In the earthly life of Jesus, religion was a living experience, a direct and personal movement from spiritual reverence to practical righteousness. The faith of Jesus bore the transcendent fruits of the divine spirit. His faith was not immature and credulous like that of a child, but in many ways it did resemble the unsuspecting trust of the child mind. Jesus trusted God much as the child trusts a parent. He had a profound confidence in the universe --- just such a trust as the child has in its parental environment. Jesus’ wholehearted faith in the fundamental goodness of the universe very much resembled the child’s trust in the security of its earthly surroundings. He depended on the heavenly Father as a child leans upon its earthly parent, and his fervent faith never for one moment doubted the certainty of the heavenly Father’s overcare. He was not disturbed seriously by fears, doubts, and scepticism. Unbelief did not inhibit the free and original expression of his life. He combined the stalwart and intelligent courage of a full\hyp{}grown man with the sincere and trusting optimism of a believing child. His faith grew to such heights of trust that it was devoid of fear.
\vs p196 0:12 The faith of Jesus attained the purity of a child’s trust. His faith was so absolute and undoubting that it responded to the charm of the contact of fellow beings and to the wonders of the universe. His sense of dependence on the divine was so complete and so confident that it yielded the joy and the assurance of absolute personal security. There was no hesitating pretense in his religious experience. In this giant intellect of the full\hyp{}grown man the faith of the child reigned supreme in all matters relating to the religious consciousness. It is not strange that he once said, \textcolour{ubdarkred}{“Except you become as a little child, you shall not enter the kingdom.”} Notwithstanding that Jesus’ faith was \bibemph{childlike,} it was in no sense \bibemph{childish.}
\vs p196 0:13 Jesus does not require his disciples to believe in him but rather to believe \bibemph{with} him, believe in the reality of the love of God and in full confidence accept the security of the assurance of sonship with the heavenly Father. The Master desires that all his followers should fully share his transcendent faith. Jesus most touchingly challenged his followers, not only to believe \bibemph{what} he believed, but also to believe \bibemph{as} he believed. This is the full significance of his one supreme requirement, \textcolour{ubdarkred}{“Follow me.”}
\vs p196 0:14 Jesus’ earthly life was devoted to one great purpose --- doing the Father’s will, living the human life religiously and by faith. The faith of Jesus was trusting, like that of a child, but it was wholly free from presumption. He made robust and manly decisions, courageously faced manifold disappointments, resolutely surmounted extraordinary difficulties, and unflinchingly confronted the stern requirements of duty. It required a strong will and an unfailing confidence to believe what Jesus believed and \bibemph{as} he believed.
\usection{1.\bibnobreakspace Jesus --- The Man}
\vs p196 1:1 Jesus’ devotion to the Father’s will and the service of man was even more than mortal decision and human determination; it was a wholehearted consecration of himself to such an unreserved bestowal of love. No matter how great the fact of the sovereignty of Michael, you must not take the human Jesus away from men. The Master has ascended on high as a man, as well as God; he belongs to men; men belong to him. How unfortunate that religion itself should be so misinterpreted as to take the human Jesus away from struggling mortals! Let not the discussions of the humanity or the divinity of the Christ obscure the saving truth that Jesus of Nazareth was a religious man who, by faith, achieved the knowing and the doing of the will of God; he was the most truly religious man who has ever lived on Urantia.
\vs p196 1:2 The time is ripe to witness the figurative resurrection of the human Jesus from his burial tomb amidst the theological traditions and the religious dogmas of 19 centuries. Jesus of Nazareth must not be longer sacrificed to even the splendid concept of the glorified Christ. What a transcendent service if, through this revelation, the Son of Man should be recovered from the tomb of traditional theology and be presented as the living Jesus to the church that bears his name, and to all other religions! Surely the Christian fellowship of believers will not hesitate to make such adjustments of faith and of practices of living as will enable it to “follow after” the Master in the demonstration of his real life of religious devotion to the doing of his Father’s will and of consecration to the unselfish service of man. Do professed Christians fear the exposure of a self\hyp{}sufficient and unconsecrated fellowship of social respectability and selfish economic maladjustment? Does institutional Christianity fear the possible jeopardy, or even the overthrow, of traditional ecclesiastical authority if the Jesus of Galilee is reinstated in the minds and souls of mortal men as the ideal of personal religious living? Indeed, the social readjustments, the economic transformations, the moral rejuvenations, and the religious revisions of Christian civilization would be drastic and revolutionary if the living religion of Jesus should suddenly supplant the theologic religion about Jesus.\tunemarkup{private}{\begin{figure}[H]\centering\includegraphics[scale=\tunemarkup{pgkoboaurahd}{0.53}\tunemarkup{pghanlin}{0.47}\tunemarkup{pgnexus7}{0.47}\tunemarkup{pgkindledx}{0.36}]{../urantia-pictures/Jon_Detoy_Follow_Me_525.jpg}\caption{Follow me by Jon~DeToy}\end{figure}}
\vs p196 1:3 \pc To “follow Jesus” means to personally share his religious faith and to enter into the spirit of the Master’s life of unselfish service for man. One of the most important things in human living is to find out what Jesus believed, to discover his ideals, and to strive for the achievement of his exalted life purpose. Of all human knowledge, that which is of greatest value is to know the religious life of Jesus and how he lived it.
\vs p196 1:4 The common people heard Jesus gladly, and they will again respond to the presentation of his sincere human life of consecrated religious motivation if such truths shall again be proclaimed to the world. The people heard him gladly because he was one of them, an unpretentious layman; the world’s greatest religious teacher was indeed a layman.
\vs p196 1:5 It should not be the aim of kingdom believers literally to imitate the outward life of Jesus in the flesh but rather to share his faith; to trust God as he trusted God and to believe in men as he believed in men. Jesus never argued about either the fatherhood of God or the brotherhood of men; he was a living illustration of the one and a profound demonstration of the other.
\vs p196 1:6 Just as men must progress from the consciousness of the human to the realization of the divine, so did Jesus ascend from the nature of man to the consciousness of the nature of God. And the Master made this great ascent from the human to the divine by the conjoint achievement of the faith of his mortal intellect and the acts of his indwelling Adjuster. The fact\hyp{}realization of the attainment of totality of divinity (all the while fully conscious of the reality of humanity) was attended by seven stages of faith consciousness of progressive divinization. These stages of progressive self\hyp{}realization were marked off by the following extraordinary events in the Master’s bestowal experience:
\vs p196 1:7 \ublistelem{1.}\bibnobreakspace The arrival of the Thought Adjuster.
\vs p196 1:8 \ublistelem{2.}\bibnobreakspace The messenger of Immanuel who appeared to him at Jerusalem when he was about 12 years old.
\vs p196 1:9 \ublistelem{3.}\bibnobreakspace The manifestations attendant upon his baptism.
\vs p196 1:10 \ublistelem{4.}\bibnobreakspace The experiences on the Mount of Transfiguration.
\vs p196 1:11 \ublistelem{5.}\bibnobreakspace The morontia resurrection.
\vs p196 1:12 \ublistelem{6.}\bibnobreakspace The spirit ascension.
\vs p196 1:13 \ublistelem{7.}\bibnobreakspace The final embrace of the Paradise Father, conferring unlimited sovereignty of his universe.
\usection{2.\bibnobreakspace The Religion of Jesus}
\vs p196 2:1 Some day a reformation in the Christian church may strike deep enough to get back to the unadulterated religious teachings of Jesus, the author and finisher of our faith. You may \bibemph{preach} a religion \bibemph{about} Jesus, but, perforce, you must \bibemph{live} the religion \bibemph{of} Jesus. In the enthusiasm of Pentecost, Peter unintentionally inaugurated a new religion, the religion of the risen and glorified Christ. The Apostle Paul later on transformed this new gospel into Christianity, a religion embodying his own theologic views and portraying his own \bibemph{personal experience} with the Jesus of the Damascus road. The gospel of the kingdom is founded on the personal religious experience of the Jesus of Galilee; Christianity is founded almost exclusively on the personal religious experience of the Apostle Paul. Almost the whole of the New Testament is devoted, not to the portrayal of the significant and inspiring religious life of Jesus, but to a discussion of Paul’s religious experience and to a portrayal of his personal religious convictions. The only notable exceptions to this statement, aside from certain parts of Matthew, Mark, and Luke, are the Book of Hebrews and the Epistle of James. Even Peter, in his writing, only once reverted to the personal religious life of his Master. The New Testament is a superb Christian document, but it is only meagrely Jesusonian.
\vs p196 2:2 Jesus’ life in the flesh portrays a transcendent religious growth from the early ideas of primitive awe and human reverence up through years of personal spiritual communion until he finally arrived at that advanced and exalted status of the consciousness of his oneness with the Father. And thus, in one short life, did Jesus traverse that experience of religious spiritual progression which man begins on earth and ordinarily achieves only at the conclusion of his long sojourn in the spirit training schools of the successive levels of the pre\hyp{}Paradise career. Jesus progressed from a purely human consciousness of the faith certainties of personal religious experience to the sublime spiritual heights of the positive realization of his divine nature and to the consciousness of his close association with the Universal Father in the management of a universe. He progressed from the humble status of mortal dependence which prompted him spontaneously to say to the one who called him Good Teacher, \textcolour{ubdarkred}{“Why do you call me good? None is good but God,”} to that sublime consciousness of achieved divinity which led him to exclaim, \textcolour{ubdarkred}{“Which one of you convicts me of sin?”} And this progressing ascent from the human to the divine was an exclusively mortal achievement. And when he had thus attained divinity, he was still the same human Jesus, the Son of Man as well as the Son of God.
\vs p196 2:3 Mark, Matthew, and Luke retain something of the picture of the human Jesus as he engaged in the superb struggle to ascertain the divine will and to do that will. John presents a picture of the triumphant Jesus as he walked on earth in the full consciousness of divinity. The great mistake that has been made by those who have studied the Master’s life is that some have conceived of him as entirely human, while others have thought of him as only divine. Throughout his entire experience he was truly both human and divine, even as he yet is.
\vs p196 2:4 But the greatest mistake was made in that, while the human Jesus was recognized as \bibemph{having} a religion, the divine Jesus (Christ) almost overnight became a religion. Paul’s Christianity made sure of the adoration of the divine Christ, but it almost wholly lost sight of the struggling and valiant human Jesus of Galilee, who, by the valour of his personal religious faith and the heroism of his indwelling Adjuster, ascended from the lowly levels of humanity to become one with divinity, thus becoming the new and living way whereby all mortals may so ascend from humanity to divinity. Mortals in all stages of spirituality and on all worlds may find in the personal life of Jesus that which will strengthen and inspire them as they progress from the lowest spirit levels up to the highest divine values, from the beginning to the end of all personal religious experience.
\vs p196 2:5 At the time of the writing of the New Testament, the authors not only most profoundly believed in the divinity of the risen Christ, but they also devotedly and sincerely believed in his immediate return to earth to consummate the heavenly kingdom. This strong faith in the Lord’s immediate return had much to do with the tendency to omit from the record those references which portrayed the purely human experiences and attributes of the Master. The whole Christian movement tended away from the human picture of Jesus of Nazareth toward the exaltation of the risen Christ, the glorified and soon\hyp{}returning Lord Jesus Christ.
\vs p196 2:6 \pc Jesus founded the religion of personal experience in doing the will of God and serving the human brotherhood; Paul founded a religion in which the glorified Jesus became the object of worship and the brotherhood consisted of fellow believers in the divine Christ. In the bestowal of Jesus these two concepts were potential in his divine\hyp{}human life, and it is indeed a pity that his followers failed to create a unified religion which might have given proper recognition to both the human and the divine natures of the Master as they were inseparably bound up in his earth life and so gloriously set forth in the original gospel of the kingdom.
\vs p196 2:7 You would be neither shocked nor disturbed by some of Jesus’ strong pronouncements if you would only remember that he was the world’s most wholehearted and devoted religionist. He was a wholly consecrated mortal, unreservedly dedicated to doing his Father’s will. Many of his apparently hard sayings were more of a personal confession of faith and a pledge of devotion than commands to his followers. And it was this very singleness of purpose and unselfish devotion that enabled him to effect such extraordinary progress in the conquest of the human mind in one short life. Many of his declarations should be considered as a confession of what he demanded of himself rather than what he required of all his followers. In his devotion to the cause of the kingdom, Jesus burned all bridges behind him; he sacrificed all hindrances to the doing of his Father’s will.
\vs p196 2:8 Jesus blessed the poor because they were usually sincere and pious; he condemned the rich because they were usually wanton and irreligious. He would equally condemn the irreligious pauper and commend the consecrated and worshipful man of wealth.
\vs p196 2:9 Jesus led men to feel at home in the world; he delivered them from the slavery of taboo and taught them that the world was not fundamentally evil. He did not long to escape from his earthly life; he mastered a technique of acceptably doing the Father’s will while in the flesh. He attained an idealistic religious life in the very midst of a realistic world. Jesus did not share Paul’s pessimistic view of humankind. The Master looked upon men as the sons of God and foresaw a magnificent and eternal future for those who chose survival. He was not a moral sceptic; he viewed man positively, not negatively. He saw most men as weak rather than wicked, more distraught than depraved. But no matter what their status, they were all God’s children and his brethren.
\vs p196 2:10 He taught men to place a high value upon themselves in time and in eternity. Because of this high estimate which Jesus placed upon men, he was willing to spend himself in the unremitting service of humankind. And it was this infinite worth of the finite that made the golden rule a vital factor in his religion. What mortal can fail to be uplifted by the extraordinary faith Jesus has in him?
\vs p196 2:11 Jesus offered no rules for social advancement; his was a religious mission, and religion is an exclusively individual experience. The ultimate goal of society’s most advanced achievement can never hope to transcend Jesus’ brotherhood of men based on the recognition of the fatherhood of God. The ideal of all social attainment can be realized only in the coming of this divine kingdom.
\usection{3.\bibnobreakspace The Supremacy of Religion}
\vs p196 3:1 Personal, spiritual religious experience is an efficient solvent for most mortal difficulties; it is an effective sorter, evaluator, and adjuster of all human problems. Religion does not remove or destroy human troubles, but it does dissolve, absorb, illuminate, and transcend them. True religion unifies the personality for effective adjustment to all mortal requirements. Religious faith --- the positive leading of the indwelling divine presence --- unfailingly enables the God\hyp{}knowing man to bridge that gulf existing between the intellectual logic which recognizes the Universal First Cause as \bibemph{It} and those positive affirmations of the soul which aver this First Cause is \bibemph{He,} the heavenly Father of Jesus’ gospel, the personal God of human salvation.
\vs p196 3:2 There are just three elements in universal reality: fact, idea, and relation. The religious consciousness identifies these realities as science, philosophy, and truth. Philosophy would be inclined to view these activities as reason, wisdom, and faith --- physical reality, intellectual reality, and spiritual reality. We are in the habit of designating these realities as thing, meaning, and value.
\vs p196 3:3 The progressive comprehension of reality is the equivalent of approaching God. The finding of God, the consciousness of identity with reality, is the equivalent of the experiencing of self\hyp{}completion --- self\hyp{}entirety, self\hyp{}totality. The experiencing of total reality is the full realization of God, the finality of the God\hyp{}knowing experience.
\vs p196 3:4 The full summation of human life is the knowledge that man is educated by fact, ennobled by wisdom, and saved --- justified --- by religious faith.
\vs p196 3:5 Physical certainty consists in the logic of science; moral certainty, in the wisdom of philosophy; spiritual certainty, in the truth of genuine religious experience.
\vs p196 3:6 The mind of man can attain high levels of spiritual insight and corresponding spheres of divinity of values because it is not wholly material. There is a spirit nucleus in the mind of man --- the Adjuster of the divine presence. There are three separate evidences of this spirit indwelling of the human mind:
\vs p196 3:7 \ublistelem{1.}\bibnobreakspace Humanitarian fellowship --- love. The purely animal mind may be gregarious for self\hyp{}protection, but only the spirit\hyp{}indwelt intellect is unselfishly altruistic and unconditionally loving.
\vs p196 3:8 \ublistelem{2.}\bibnobreakspace Interpretation of the universe --- wisdom. Only the spirit\hyp{}indwelt mind can comprehend that the universe is friendly to the individual.
\vs p196 3:9 \ublistelem{3.}\bibnobreakspace Spiritual evaluation of life --- worship. Only the spirit\hyp{}indwelt man can realize the divine presence and seek to attain a fuller experience in and with this foretaste of divinity.
\vs p196 3:10 \pc The human mind does not create real values; human experience does not yield universe insight. Concerning insight, the recognition of moral values and the discernment of spiritual meanings, all that the human mind can do is to discover, recognize, interpret, and \bibemph{choose.}
\vs p196 3:11 The moral values of the universe become intellectual possessions by the exercise of the three basic judgments, or choices, of the mortal mind:
\vs p196 3:12 \ublistelem{1.}\bibnobreakspace Self\hyp{}judgment --- moral choice.
\vs p196 3:13 \ublistelem{2.}\bibnobreakspace Social\hyp{}judgment --- ethical choice.
\vs p196 3:14 \ublistelem{3.}\bibnobreakspace God\hyp{}judgment --- religious choice.
\vs p196 3:15 \pc Thus it appears that all human progress is effected by a technique of conjoint \bibemph{revelational evolution.}
\vs p196 3:16 Unless a divine lover lived in man, he could not unselfishly and spiritually love. Unless an interpreter lived in the mind, man could not truly realize the unity of the universe. Unless an evaluator dwelt with man, he could not possibly appraise moral values and recognize spiritual meanings. And this lover hails from the very source of infinite love; this interpreter is a part of Universal Unity; this evaluator is the child of the Centre and Source of all absolute values of divine and eternal reality.
\vs p196 3:17 Moral evaluation with a religious meaning --- spiritual insight --- connotes the individual’s choice between good and evil, truth and error, material and spiritual, human and divine, time and eternity. Human survival is in great measure dependent on consecrating the human will to the choosing of those values selected by this spirit\hyp{}value sorter --- the indwelling interpreter and unifier. Personal religious experience consists in two phases: discovery in the human mind and revelation by the indwelling divine spirit. Through oversophistication or as a result of the irreligious conduct of professed religionists, a man, or even a generation of men, may elect to suspend their efforts to discover the God who indwells them; they may fail to progress in and attain the divine revelation. But such attitudes of spiritual nonprogression cannot long persist because of the presence and influence of the indwelling Thought Adjusters.
\vs p196 3:18 This profound experience of the reality of the divine indwelling forever transcends the crude materialistic technique of the physical sciences. You cannot put spiritual joy under a microscope; you cannot weigh love in a balance; you cannot measure moral values; neither can you estimate the quality of spiritual worship.
\vs p196 3:19 The Hebrews had a religion of moral sublimity; the Greeks evolved a religion of beauty; Paul and his conferees founded a religion of faith, hope, and charity. Jesus revealed and exemplified a religion of love: security in the Father’s love, with joy and satisfaction consequent upon sharing this love in the service of the human brotherhood.
\vs p196 3:20 Every time man makes a reflective moral choice, he immediately experiences a new divine invasion of his soul. Moral choosing constitutes religion as the motive of inner response to outer conditions. But such a real religion is not a purely subjective experience. It signifies the whole of the subjectivity of the individual engaged in a meaningful and intelligent response to total objectivity --- the universe and its Maker.
\vs p196 3:21 The exquisite and transcendent experience of loving and being loved is not just a psychic illusion because it is so purely subjective. The one truly divine and objective reality that is associated with mortal beings, the Thought Adjuster, functions to human observation apparently as an exclusively subjective phenomenon. Man’s contact with the highest objective reality, God, is only through the purely subjective experience of knowing him, of worshipping him, of realizing sonship with him.
\vs p196 3:22 True religious worship is not a futile monologue of self\hyp{}deception. Worship is a personal communion with that which is divinely real, with that which is the very source of reality. Man aspires by worship to be better and thereby eventually attains the \bibemph{best.}
\vs p196 3:23 The idealization and attempted service of truth, beauty, and goodness is not a substitute for genuine religious experience --- spiritual reality. Psychology and idealism are not the equivalent of religious reality. The projections of the human intellect may indeed originate false gods --- gods in man’s image --- but the true God\hyp{}consciousness does not have such an origin. The God\hyp{}consciousness is resident in the indwelling spirit. Many of the religious systems of man come from the formulations of the human intellect, but the God\hyp{}consciousness is not necessarily a part of these grotesque systems of religious slavery.
\vs p196 3:24 God is not the mere invention of man’s idealism; he is the very source of all such superanimal insights and values. God is not a hypothesis formulated to unify the human concepts of truth, beauty, and goodness; he is the personality of love from whom all of these universe manifestations are derived. The truth, beauty, and goodness of man’s world are unified by the increasing spirituality of the experience of mortals ascending toward Paradise realities. The unity of truth, beauty, and goodness can only be realized in the spiritual experience of the God\hyp{}knowing personality.
\vs p196 3:25 Morality is the essential pre\hyp{}existent soil of personal God\hyp{}consciousness, the personal realization of the Adjuster’s inner presence, but such morality is not the source of religious experience and the resultant spiritual insight. The moral nature is superanimal but subspiritual. Morality is equivalent to the recognition of duty, the realization of the existence of right and wrong. The moral zone intervenes between the animal and the human types of mind as morontia functions between the material and the spiritual spheres of personality attainment.
\vs p196 3:26 The evolutionary mind is able to discover law, morals, and ethics; but the bestowed spirit, the indwelling Adjuster, reveals to the evolving human mind the lawgiver, the Father\hyp{}source of all that is true, beautiful, and good; and such an illuminated man has a religion and is spiritually equipped to begin the long and adventurous search for God.
\vs p196 3:27 Morality is not necessarily spiritual; it may be wholly and purely human, albeit real religion enhances all moral values, makes them more meaningful. Morality without religion fails to reveal ultimate goodness, and it also fails to provide for the survival of even its own moral values. Religion provides for the enhancement, glorification, and assured survival of everything morality recognizes and approves.
\vs p196 3:28 Religion stands above science, art, philosophy, ethics, and morals, but not independent of them. They are all indissolubly interrelated in human experience, personal and social. Religion is man’s supreme experience in the mortal nature, but finite language makes it forever impossible for theology ever adequately to depict real religious experience.
\vs p196 3:29 \pc Religious insight possesses the power of turning defeat into higher desires and new determinations. Love is the highest motivation which man may utilize in his universe ascent. But love, divested of truth, beauty, and goodness, is only a sentiment, a philosophic distortion, a psychic illusion, a spiritual deception. Love must always be redefined on successive levels of morontia and spirit progression.
\vs p196 3:30 \pc Art results from man’s attempt to escape from the lack of beauty in his material environment; it is a gesture toward the morontia level. Science is man’s effort to solve the apparent riddles of the material universe. Philosophy is man’s attempt at the unification of human experience. Religion is man’s supreme gesture, his magnificent reach for final reality, his determination to find God and to be like him.
\vs p196 3:31 \pc In the realm of religious experience, spiritual possibility is potential reality. Man’s forward spiritual urge is not a psychic illusion. All of man’s universe romancing may not be fact, but much, very much, is truth.
\vs p196 3:32 Some men’s lives are too great and noble to descend to the low level of being merely successful. The animal must adapt itself to the environment, but the religious man transcends his environment and in this way escapes the limitations of the present material world through this insight of divine love. This concept of love generates in the soul of man that superanimal effort to find truth, beauty, and goodness; and when he does find them, he is glorified in their embrace; he is consumed with the desire to live them, to do righteousness.
\vs p196 3:33 Be not discouraged; human evolution is still in progress, and the revelation of God to the world, in and through Jesus, shall not fail.
\vs p196 3:34 The great challenge to modern man is to achieve better communication with the divine Monitor that dwells within the human mind. Man’s greatest adventure in the flesh consists in the well\hyp{}balanced and sane effort to advance the borders of self\hyp{}consciousness out through the dim realms of embryonic soul\hyp{}consciousness in a wholehearted effort to reach the borderland of spirit\hyp{}consciousness --- contact with the divine presence. Such an experience constitutes God\hyp{}consciousness, an experience mightily confirmative of the pre\hyp{}existent truth of the religious experience of knowing God. Such spirit\hyp{}consciousness is the equivalent of the knowledge of the actuality of sonship with God. Otherwise, the assurance of sonship is the experience of faith.
\vs p196 3:35 And God\hyp{}consciousness is equivalent to the integration of the self with the universe, and on its highest levels of spiritual reality. Only the spirit content of any value is imperishable. Even that which is true, beautiful, and good may not perish in human experience. If man does not choose to survive, then does the surviving Adjuster conserve those realities born of love and nurtured in service. And all these things are a part of the Universal Father. The Father is living love, and this life of the Father is in his Sons. And the spirit of the Father is in his Sons’ sons --- mortal men. When all is said and done, the Father idea is still the highest human concept of God.\fnc{And the spirit of the Father is in his \bibtextul{Son’s} sons--- mortal men. \bibexpl{Sons’ does appear to be correct in light of the prior sentence which provides the context --- “\ldots{}this life of the Father is in his Sons.”}}
\quizlink
