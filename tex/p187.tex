\upaper{187}{The Crucifixion}
\uminitoc{On the Way to Golgotha}
\uminitoc{The Crucifixion}
\uminitoc{Those Who Saw the Crucifixion}
\uminitoc{The Thief on the Cross}
\uminitoc{Last Hour on the Cross}
\uminitoc{After the Crucifixion}
\author{Midwayer Commission}
\vs p187 0:1 After the two brigands had been made ready, the soldiers, under the direction of a centurion, started for the scene of the crucifixion. The centurion in charge of these twelve soldiers was the same captain who had led forth the Roman soldiers the previous night to arrest Jesus in Gethsemane. It was the Roman custom to assign four soldiers for each person to be crucified. The two brigands were properly scourged before they were taken out to be crucified, but Jesus was given no further physical punishment; the captain undoubtedly thought he had already been sufficiently scourged, even before his condemnation.
\vs p187 0:2 The two thieves crucified with Jesus were associates of Barabbas and would later have been put to death with their leader if he had not been released as the Passover pardon of Pilate. Jesus was thus crucified in the place of Barabbas.
\vs p187 0:3 What Jesus is now about to do, submit to death on the cross, he does of his own free will. In foretelling this experience, he said: \textcolour{ubdarkred}{“The Father loves and sustains me because I am willing to lay down my life. But I will take it up again. No one takes my life away from me --- I lay it down of myself. I have authority to lay it down, and I have authority to take it up. I have received such a commandment from my Father.”}
\vs p187 0:4 It was just before 9:00 this morning when the soldiers led Jesus from the praetorium on the way to Golgotha. They were followed by many who secretly sympathized with Jesus, but most of this group of 200 or more were either his enemies or curious idlers who merely desired to enjoy the shock of witnessing the crucifixions. Only a few of the Jewish leaders went out to see Jesus die on the cross. Knowing that he had been turned over to the Roman soldiers by Pilate, and that he was condemned to die, they busied themselves with their meeting in the temple, whereat they discussed what should be done with his followers.
\usection{On the Way to Golgotha}
\vs p187 1:1 Before leaving the courtyard of the praetorium, the soldiers placed the crossbeam on Jesus’ shoulders. It was the custom to compel the condemned man to carry the crossbeam to the site of the crucifixion. Such a condemned man did not carry the whole cross, only this shorter timber. The longer and upright pieces of timber for the three crosses had already been transported to Golgotha and, by the time of the arrival of the soldiers and their prisoners, had been firmly implanted in the ground.
\vs p187 1:2 According to custom the captain led the procession, carrying small white boards on which had been written with charcoal the names of the criminals and the nature of the crimes for which they had been condemned. For the two thieves the centurion had notices which gave their names, underneath which was written the one word, “Brigand.” It was the custom, after the victim had been nailed to the crossbeam and hoisted to his place on the upright timber, to nail this notice to the top of the cross, just above the head of the criminal, that all witnesses might know for what crime the condemned man was being crucified. The legend which the centurion carried to put on the cross of Jesus had been written by Pilate himself in Latin, Greek, and Aramaic, and it read: “Jesus of Nazareth --- the King of the Jews.”
\vs p187 1:3 Some of the Jewish authorities who were yet present when Pilate wrote this legend made vigorous protest against calling Jesus the “king of the Jews.” But Pilate reminded them that such an accusation was part of the charge which led to his condemnation. When the Jews saw they could not prevail upon Pilate to change his mind, they pleaded that at least it be modified to read, “He said, ‘I am the king of the Jews.’” But Pilate was adamant; he would not alter the writing. To all further supplication he only replied, “What I have written, I have written.”
\vs p187 1:4 Ordinarily, it was the custom to journey to Golgotha by the longest road in order that a large number of persons might view the condemned criminal, but on this day they went by the most direct route to the Damascus gate, which led out of the city to the north, and following this road, they soon arrived at Golgotha, the official crucifixion site of Jerusalem. Beyond Golgotha were the villas of the wealthy, and on the other side of the road were the tombs of many well\hyp{}to\hyp{}do Jews.
\vs p187 1:5 \pc Crucifixion was not a Jewish mode of punishment. Both the Greeks and the Romans learned this method of execution from the Phoenicians. Even Herod, with all his cruelty, did not resort to crucifixion. The Romans never crucified a Roman citizen; only slaves and subject peoples were subjected to this dishonourable mode of death. During the siege of Jerusalem, just 40 years after the crucifixion of Jesus, all of Golgotha was covered by thousands upon thousands of crosses upon which, from day to day, there perished the flower of the Jewish race. A terrible harvest, indeed, of the seed\hyp{}sowing of this day.
\vs p187 1:6 \pc As the death procession passed along the narrow streets of Jerusalem, many of the tenderhearted Jewish women who had heard Jesus’ words of good cheer and compassion, and who knew of his life of loving ministry, could not refrain from weeping when they saw him being led forth to such an ignoble death. As he passed by, many of these women bewailed and lamented. And when some of them even dared to follow along by his side, the Master turned his head toward them and said: \textcolour{ubdarkred}{“Daughters of Jerusalem, weep not for me, but rather weep for yourselves and for your children. My work is about done --- soon I go to my Father --- but the times of terrible trouble for Jerusalem are just beginning. Behold, the days are coming in which you shall say: Blessed are the barren and those whose breasts have never suckled their young. In those days will you pray the rocks of the hills to fall on you in order that you may be delivered from the terrors of your troubles.”}
\vs p187 1:7 These women of Jerusalem were indeed courageous to manifest sympathy for Jesus, for it was strictly against the law to show friendly feelings for one who was being led forth to crucifixion. It was permitted the rabble to jeer, mock, and ridicule the condemned, but it was not allowed that any sympathy should be expressed. Though Jesus appreciated the manifestation of sympathy in this dark hour when his friends were in hiding, he did not want these kindhearted women to incur the displeasure of the authorities by daring to show compassion in his behalf. Even at such a time as this Jesus thought little about himself, only of the terrible days of tragedy ahead for Jerusalem and the whole Jewish nation.
\vs p187 1:8 As the Master trudged along on the way to the crucifixion, he was very weary; he was nearly exhausted. He had had neither food nor water since the Last Supper at the home of Elijah Mark; neither had he been permitted to enjoy one moment of sleep. In addition, there had been one hearing right after another up to the hour of his condemnation, not to mention the abusive scourgings with their accompanying physical suffering and loss of blood. Superimposed upon all this was his extreme mental anguish, his acute spiritual tension, and a terrible feeling of human loneliness.
\vs p187 1:9 Shortly after passing through the gate on the way out of the city, as Jesus staggered on bearing the crossbeam, his physical strength momentarily gave way, and he fell beneath the weight of his heavy burden. The soldiers shouted at him and kicked him, but he could not arise. When the captain saw this, knowing what Jesus had already endured, he commanded the soldiers to desist. Then he ordered a passerby, one Simon from Cyrene, to take the crossbeam from Jesus’ shoulders and compelled him to carry it the rest of the way to Golgotha.
\vs p187 1:10 \pc This man Simon had come all the way from Cyrene, in northern Africa, to attend the Passover. He was stopping with other Cyrenians just outside the city walls and was on his way to the temple services in the city when the Roman captain commanded him to carry Jesus’ crossbeam. Simon lingered all through the hours of the Master’s death on the cross, talking with many of his friends and with his enemies. After the resurrection and before leaving Jerusalem, he became a valiant believer in the gospel of the kingdom, and when he returned home, he led his family into the heavenly kingdom. His two sons, Alexander and Rufus, became very effective teachers of the new gospel in Africa. But Simon never knew that Jesus, whose burden he bore, and the Jewish tutor who once befriended his injured son, were the same person.
\vs p187 1:11 \pc It was shortly after 9:00 when this procession of death arrived at Golgotha, and the Roman soldiers set themselves about the task of nailing the two brigands and the Son of Man to their respective crosses.
\usection{The Crucifixion}
\vs p187 2:1 The soldiers first bound the Master’s arms with cords to the crossbeam, and then they nailed his hands to the wood. When they had hoisted this crossbeam up on the post, and after they had nailed it securely to the upright timber of the cross, they bound and nailed his feet to the wood, using one long nail to penetrate both feet. The upright timber had a large peg, inserted at the proper height, which served as a sort of saddle for supporting the body weight. The cross was not high, the Master’s feet being only about 1\,m from the ground. He was therefore able to hear all that was said of him in derision and could plainly see the expression on the faces of all those who so thoughtlessly mocked him. And also could those present easily hear all that Jesus said during these hours of lingering torture and slow death.
\vs p187 2:2 It was the custom to remove all clothes from those who were to be crucified, but since the Jews greatly objected to the public exposure of the naked human form, the Romans always provided a suitable loin cloth for all persons crucified at Jerusalem. Accordingly, after Jesus’ clothes had been removed, he was thus garbed before he was put upon the cross.
\vs p187 2:3 Crucifixion was resorted to in order to provide a cruel and lingering punishment, the victim sometimes not dying for several days. There was considerable sentiment against crucifixion in Jerusalem, and there existed a society of Jewish women who always sent a representative to crucifixions for the purpose of offering drugged wine to the victim in order to lessen his suffering. But when Jesus tasted this narcotized wine, as thirsty as he was, he refused to drink it. The Master chose to retain his human consciousness until the very end. He desired to meet death, even in this cruel and inhuman form, and conquer it by voluntary submission to the full human experience.
\vs p187 2:4 Before Jesus was put on his cross, the two brigands had already been placed on their crosses, all the while cursing and spitting upon their executioners. Jesus’ only words, as they nailed him to the crossbeam, were, \textcolour{ubdarkred}{“Father, forgive them, for they know not what they do.”} He could not have so mercifully and lovingly interceded for his executioners if such thoughts of affectionate devotion had not been the mainspring of all his life of unselfish service. The ideas, motives, and longings of a lifetime are openly revealed in a crisis.
\vs p187 2:5 After the Master was hoisted on the cross, the captain nailed the title up above his head, and it read in three languages, “Jesus of Nazareth --- the King of the Jews.” The Jews were infuriated by this believed insult. But Pilate was chafed by their disrespectful manner; he felt he had been intimidated and humiliated, and he took this method of obtaining petty revenge. He could have written “Jesus, a rebel.” But he well knew how these Jerusalem Jews detested the very name of Nazareth, and he was determined thus to humiliate them. He knew that they would also be cut to the very quick by seeing this executed Galilean called “The King of the Jews.”
\vs p187 2:6 Many of the Jewish leaders, when they learned how Pilate had sought to deride them by placing this inscription on the cross of Jesus, hastened out to Golgotha, but they dared not attempt to remove it since the Roman soldiers were standing on guard. Not being able to remove the title, these leaders mingled with the crowd and did their utmost to incite derision and ridicule, lest any give serious regard to the inscription.
\vs p187 2:7 The Apostle John, with Mary the mother of Jesus, Ruth, and Jude, arrived on the scene just after Jesus had been hoisted to his position on the cross, and just as the captain was nailing the title above the Master’s head. John was the only one of the 11 apostles to witness the crucifixion, and even he was not present all of the time since he ran into Jerusalem to bring back his mother and her friends soon after he had brought Jesus’ mother to the scene.
\vs p187 2:8 As Jesus saw his mother, with John and his brother and sister, he smiled but said nothing. Meanwhile the four soldiers assigned to the Master’s crucifixion, as was the custom, had divided his clothes among them, one taking the sandals, one the turban, one the girdle, and the fourth the cloak. This left the tunic, or seamless vestment reaching down to near the knees, to be cut up into four pieces, but when the soldiers saw what an unusual garment it was, they decided to cast lots for it. Jesus looked down on them while they divided his garments, and the thoughtless crowd jeered at him.
\vs p187 2:9 \pc It was well that the Roman soldiers took possession of the Master’s clothing. Otherwise, if his followers had gained possession of these garments, they would have been tempted to resort to superstitious relic worship. The Master desired that his followers should have nothing material to associate with his life on earth. He wanted to leave mankind only the memory of a human life dedicated to the high spiritual ideal of being consecrated to doing the Father’s will.
\usection{Those Who Saw the Crucifixion}
\vs p187 3:1 At about 9:30 this Friday morning, Jesus was hung upon the cross. Before 11:00, upward of 1,000 persons had assembled to witness this spectacle of the crucifixion of the Son of Man. Throughout these dreadful hours the unseen hosts of a universe stood in silence while they gazed upon this extraordinary phenomenon of the Creator as he was dying the death of the creature, even the most ignoble death of a condemned criminal.
\vs p187 3:2 Standing near the cross at one time or another during the crucifixion were Mary, Ruth, Jude, John, Salome (John’s mother), and a group of earnest women believers including Mary the wife of Clopas and sister of Jesus’ mother, Mary Magdalene, and Rebecca, onetime of Sepphoris. These and other friends of Jesus held their peace while they witnessed his great patience and fortitude and gazed upon his intense sufferings.
\vs p187 3:3 Many who passed by wagged their heads and, railing at him, said: “You who would destroy the temple and build it again in three days, save yourself. If you are the Son of God, why do you not come down from your cross?” In like manner some of the rulers of the Jews mocked him, saying, “He saved others, but himself he cannot save.” Others said, “If you are the king of the Jews, come down from the cross, and we will believe in you.” And later on they mocked him the more, saying: “He trusted in God to deliver him. He even claimed to be the Son of God --- look at him now --- crucified between two thieves.” Even the two thieves also railed at him and cast reproach upon him.
\vs p187 3:4 Inasmuch as Jesus would make no reply to their taunts, and since it was nearing noontime of this special preparation day, by 11:30 most of the jesting and jeering crowd had gone its way; less than 50 persons remained on the scene. The soldiers now prepared to eat lunch and drink their cheap, sour wine as they settled down for the long deathwatch. As they partook of their wine, they derisively offered a toast to Jesus, saying, “Hail and good fortune! to the king of the Jews.” And they were astonished at the Master’s tolerant regard of their ridicule and mocking.
\vs p187 3:5 When Jesus saw them eat and drink, he looked down upon them and said, \textcolour{ubdarkred}{“I thirst.”} When the captain of the guard heard Jesus say, “I thirst,” he took some of the wine from his bottle and, putting the saturated sponge stopper upon the end of a javelin, raised it to Jesus so that he could moisten his parched lips.
\vs p187 3:6 Jesus had purposed to live without resort to his supernatural power, and he likewise elected to die as an ordinary mortal upon the cross. He had lived as a man, and he would die as a man --- doing the Father’s will.
\usection{The Thief on the Cross}
\vs p187 4:1 One of the brigands railed at Jesus, saying, “If you are the Son of God, why do you not save yourself and us?” But when he had reproached Jesus, the other thief, who had many times heard the Master teach, said: “Do you have no fear even of God? Do you not see that we are suffering justly for our deeds, but that this man suffers unjustly? Better that we should seek forgiveness for our sins and salvation for our souls.” When Jesus heard the thief say this, he turned his face toward him and smiled approvingly. When the malefactor saw the face of Jesus turned toward him, he mustered up his courage, fanned the flickering flame of his faith, and said, “Lord, remember me when you come into your kingdom.” And then Jesus said, \textcolour{ubdarkred}{“Verily, verily, I say to you today, you shall sometime be with me in Paradise.”}
\vs p187 4:2 The Master had time amidst the pangs of mortal death to listen to the faith confession of the believing brigand. When this thief reached out for salvation, he found deliverance. Many times before this he had been constrained to believe in Jesus, but only in these last hours of consciousness did he turn with a whole heart toward the Master’s teaching. When he saw the manner in which Jesus faced death upon the cross, this thief could no longer resist the conviction that this Son of Man was indeed the Son of God.
\vs p187 4:3 \pc During this episode of the conversion and reception of the thief into the kingdom by Jesus, the Apostle John was absent, having gone into the city to bring his mother and her friends to the scene of the crucifixion. Luke subsequently heard this story from the converted Roman captain of the guard.
\vs p187 4:4 The Apostle John told about the crucifixion as he remembered the event \bibfrac{2}{3}\ts{rds} of a century after its occurrence. The other records were based upon the recital of the Roman centurion on duty who, because of what he saw and heard, subsequently believed in Jesus and entered into the full fellowship of the kingdom of heaven on earth.
\vs p187 4:5 \pc This young man, the penitent brigand, had been led into a life of violence and wrongdoing by those who extolled such a career of robbery as an effective patriotic protest against political oppression and social injustice. And this sort of teaching, plus the urge for adventure, led many otherwise well\hyp{}meaning youths to enlist in these daring expeditions of robbery. This young man had looked upon Barabbas as a hero. Now he saw that he had been mistaken. Here on the cross beside him he saw a really great man, a true hero. Here was a hero who fired his zeal and inspired his highest ideas of moral self\hyp{}respect and quickened all his ideals of courage, manhood, and bravery. In beholding Jesus, there sprang up in his heart an overwhelming sense of love, loyalty, and genuine greatness.
\vs p187 4:6 And if any other person among the jeering crowd had experienced the birth of faith within his soul and had appealed to the mercy of Jesus, he would have been received with the same loving consideration that was displayed toward the believing brigand.
\vs p187 4:7 \pc Just after the repentant thief heard the Master’s promise that they should sometime meet in Paradise, John returned from the city, bringing with him his mother and a company of almost a dozen women believers. John took up his position near Mary the mother of Jesus, supporting her. Her son Jude stood on the other side. As Jesus looked down upon this scene, it was noontide, and he said to his mother, \textcolour{ubdarkred}{“Woman, behold your son!”} And speaking to John, he said, \textcolour{ubdarkred}{“My son, behold your mother!”} And then he addressed them both, saying, \textcolour{ubdarkred}{“I desire that you depart from this place.”} And so John and Jude led Mary away from Golgotha. John took the mother of Jesus to the place where he tarried in Jerusalem and then hastened back to the scene of the crucifixion. After the Passover Mary returned to Bethsaida, where she lived at John’s home for the rest of her natural life. Mary did not live quite one year after the death of Jesus.
\vs p187 4:8 After Mary left, the other women withdrew for a short distance and remained in attendance upon Jesus until he expired on the cross, and they were yet standing by when the body of the Master was taken down for burial.
\usection{Last Hour on the Cross}
\vs p187 5:1 Although it was early in the season for such a phenomenon, shortly after 12:00 the sky darkened by reason of the fine sand in the air. The people of Jerusalem knew that this meant the coming of one of those hot\hyp{}wind sandstorms from the Arabian Desert. Before 13:00 the sky was so dark the sun was hid, and the remainder of the crowd hastened back to the city. When the Master gave up his life shortly after this hour, less than 30 people were present, only the 13 Roman soldiers and a group of about 15 believers. These believers were all women except two, Jude, Jesus’ brother, and John Zebedee, who returned to the scene just before the Master expired.
\vs p187 5:2 Shortly after 13:00, amidst the increasing darkness of the fierce sandstorm, Jesus began to fail in human consciousness. His last words of mercy, forgiveness, and admonition had been spoken. His last wish --- concerning the care of his mother --- had been expressed. During this hour of approaching death the human mind of Jesus resorted to the repetition of many passages in the Hebrew scriptures, particularly the Psalms. The last conscious thought of the human Jesus was concerned with the repetition in his mind of a portion of the Book of Psalms now known as the 20\ts{th}, 21\ts{st}, and 22\ts{nd} Psalms. While his lips would often move, he was too weak to utter the words as these passages, which he so well knew by heart, would pass through his mind. Only a few times did those standing by catch some utterance, such as, \textcolour{ubdarkred}{“I know the Lord will save his anointed,” “Your hand shall find out all my enemies,”} and \textcolour{ubdarkred}{“My God, my God, why have you forsaken me?”} Jesus did not for one moment entertain the slightest doubt that he had lived in accordance with the Father’s will; and he never doubted that he was now laying down his life in the flesh in accordance with his Father’s will. He did not feel that the Father had forsaken him; he was merely reciting in his vanishing consciousness many Scriptures, among them this 22\ts{nd} Psalm, which begins with “My God, my God, why have you forsaken me?” And this happened to be one of the three passages which were spoken with sufficient clearness to be heard by those standing by.
\vs p187 5:3 \pc The last request which the mortal Jesus made of his fellows was about 13:30 when, a second time, he said, \textcolour{ubdarkred}{“I thirst,”} and the same captain of the guard again moistened his lips with the same sponge wet in the sour wine, in those days commonly called vinegar.
\vs p187 5:4 \pc The sandstorm grew in intensity and the heavens increasingly darkened. Still the soldiers and the small group of believers stood by. The soldiers crouched near the cross, huddled together to protect themselves from the cutting sand. The mother of John and others watched from a distance where they were somewhat sheltered by an overhanging rock. When the Master finally breathed his last, there were present at the foot of his cross John Zebedee, his brother Jude, his sister Ruth, Mary Magdalene, and Rebecca, onetime of Sepphoris.
\vs p187 5:5 It was just before 15:00 when Jesus, with a loud voice, cried out, \textcolour{ubdarkred}{“It is finished! Father, into your hands I commend my spirit.”} And when he had thus spoken, he bowed his head and gave up the life struggle. When the Roman centurion saw how Jesus died, he smote his breast and said: “This was indeed a righteous man; truly he must have been a Son of God.” And from that hour he began to believe in Jesus.
\vs p187 5:6 \pc Jesus died royally --- as he had lived. He freely admitted his kingship and remained master of the situation throughout the tragic day. He went willingly to his ignominious death, after he had provided for the safety of his chosen apostles. He wisely restrained Peter’s trouble\hyp{}making violence and provided that John might be near him right up to the end of his mortal existence. He revealed his true nature to the murderous Sanhedrin and reminded Pilate of the source of his sovereign authority as a Son of God. He started out to Golgotha bearing his own crossbeam and finished up his loving bestowal by handing over his spirit of mortal acquirement to the Paradise Father. After such a life --- and at such a death --- the Master could truly say, \textcolour{ubdarkred}{“It is finished.”}
\vs p187 5:7 \pc Because this was the preparation day for both the Passover and the Sabbath, the Jews did not want these bodies to be exposed on Golgotha. Therefore they went before Pilate asking that the legs of these three men be broken, that they be dispatched, so that they could be taken down from their crosses and cast into the criminal burial pits before sundown. When Pilate heard this request, he forthwith sent three soldiers to break the legs and dispatch Jesus and the two brigands.
\vs p187 5:8 When these soldiers arrived at Golgotha, they did accordingly to the two thieves, but they found Jesus already dead, much to their surprise. However, in order to make sure of his death, one of the soldiers pierced his left side with his spear. Though it was common for the victims of crucifixion to linger alive upon the cross for even two or three days, the overwhelming emotional agony and the acute spiritual anguish of Jesus brought an end to his mortal life in the flesh in a little less than 5½ hours.
\usection{After the Crucifixion}
\vs p187 6:1 In the midst of the darkness of the sandstorm, about 15:30, David Zebedee sent out the last of the messengers carrying the news of the Master’s death. The last of his runners he dispatched to the home of Martha and Mary in Bethany, where he supposed the mother of Jesus stopped with the rest of her family.
\vs p187 6:2 After the death of the Master, John sent the women, in charge of Jude, to the home of Elijah Mark, where they tarried over the Sabbath day. John himself, being well known by this time to the Roman centurion, remained at Golgotha until Joseph and Nicodemus arrived on the scene with an order from Pilate authorizing them to take possession of the body of Jesus.
\vs p187 6:3 Thus ended a day of tragedy and sorrow for a vast universe whose myriads of intelligences had shuddered at the shocking spectacle of the crucifixion of the human incarnation of their beloved Sovereign; they were stunned by this exhibition of mortal callousness and human perversity.
\quizlink
