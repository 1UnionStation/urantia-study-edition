\upaper{1}{The Universal Father}
\uminitoc{The Father’s Name}
\uminitoc{The Reality of God}
\uminitoc{God is a Universal Spirit}
\uminitoc{The Mystery of God}
\uminitoc{Personality of the Universal Father}
\uminitoc{Personality in the Universe}
\uminitoc{Spiritual Value of the Personality Concept}
\author{Divine Counsellor}
\vs p001 0:1 The Universal Father is the God of all creation, the First Source and Centre of all things and beings. First think of God as a creator, then as a controller, and lastly as an infinite upholder. The truth about the Universal Father had begun to dawn upon mankind when the prophet said: “You, God, are alone; there is none beside you. You have created the heaven and the heaven of heavens, with all their hosts; you preserve and control them. By the Sons of God were the universes made. The Creator covers himself with light as with a garment and stretches out the heavens as a curtain.”\fnst{\textbf{The Creator \ldots\ curtain}, cf. Psalms~104:1--2: ``Bless the \textsc{Lord}, O my soul. O \textsc{Lord} my God, thou art very great; thou art clothed with honour and majesty. Who coverest \bibemph{thyself} with light as \bibemph{with} a garment: who stretchest out the heavens like a curtain.''} Only the concept of the Universal Father --- one God in the place of many gods --- enabled mortal man to comprehend the Father as divine creator and infinite controller.
\vs p001 0:2 The myriads of planetary systems were all made to be eventually inhabited by many different types of intelligent creatures, beings who could know God, receive the divine affection, and love him in return. The universe of universes is the work of God and the dwelling place of his diverse creatures. “God created the heavens and formed the earth; he established the universe and created this world not in vain; he formed it to be inhabited.”\fnst{\textbf{“God created \ldots\ inhabited.”}, cf. Isaiah 45:18: ``For thus saith the \textsc{Lord} that created the heavens; God himself that formed the earth and made it; he hath established it, he created it not in vain, he formed it to be inhabited: I \bibemph{am} the \textsc{Lord}; and \bibemph{there is} none else.''}
\vs p001 0:3 The enlightened worlds all recognize and worship the Universal Father, the eternal maker and infinite upholder of all creation. The will creatures of universe upon universe have embarked upon the long, long Paradise journey, the fascinating struggle of the eternal adventure of attaining God the Father. The transcendent goal of the children of time is to find the eternal God, to comprehend the divine nature, to recognize the Universal Father. God\hyp{}knowing creatures have only one supreme ambition, just one consuming desire, and that is to become, as they are in their spheres, like him as he is in his Paradise perfection of personality and in his universal sphere of righteous supremacy. From the Universal Father who inhabits eternity there has gone forth the supreme mandate, “Be you perfect, even as I am perfect.”\fnst{\textbf{“Be you perfect, even as I am perfect.”}, cf. Matthew~5:48: ``Be ye therefore perfect, even as your Father which is in heaven is perfect.''} In love and mercy the messengers of Paradise have carried this divine exhortation down through the ages and out through the universes, even to such lowly animal\hyp{}origin creatures as the human races of Urantia.
\vs p001 0:4 This magnificent and universal injunction to strive for the attainment of the perfection of divinity is the first duty, and should be the highest ambition, of all the struggling creature creation of the God of perfection. This possibility of the attainment of divine perfection is the final and certain destiny of all man’s eternal spiritual progress.
\vs p001 0:5 Urantia mortals can hardly hope to be perfect in the infinite sense, but it is entirely possible for human beings, starting out as they do on this planet, to attain the supernal and divine goal which the infinite God has set for mortal man; and when they do achieve this destiny, they will, in all that pertains to self\hyp{}realization and mind attainment, be just as replete in their sphere of divine perfection as God himself is in his sphere of infinity and eternity. Such perfection may not be universal in the material sense, unlimited in intellectual grasp, or final in spiritual experience, but it is final and complete in all finite aspects of divinity of will, perfection of personality motivation, and God\hyp{}consciousness.
\vs p001 0:6 This is the true meaning of that divine command, “Be you perfect, even as I am perfect,” which ever urges mortal man onward and beckons him inward in that long and fascinating struggle for the attainment of higher and higher levels of spiritual values and true universe meanings. This sublime search for the God of universes is the supreme adventure of the inhabitants of all the worlds of time and space.
\usection{The Father’s Name}
\vs p001 1:1 Of all the names by which God the Father is known throughout the universes, those which designate him as the First Source and the Universe Centre are most often encountered. The First Father is known by various names in different universes and in different sectors of the same universe. The names which the creature assigns to the Creator are much dependent on the creature’s concept of the Creator. The First Source and Universe Centre has never revealed himself by name, only by nature. If we believe that we are the children of this Creator, it is only natural that we should eventually call him Father. But this is the name of our own choosing, and it grows out of the recognition of our personal relationship with the First Source and Centre.
\vs p001 1:2 The Universal Father never imposes any form of arbitrary recognition, formal worship, or slavish service upon the intelligent will creatures of the universes. The evolutionary inhabitants of the worlds of time and space must of themselves --- in their own hearts --- recognize, love, and voluntarily worship him. The Creator refuses to coerce or compel the submission of the spiritual free wills of his material creatures. The affectionate dedication of the human will to the doing of the Father’s will is man’s choicest gift to God; in fact, such a consecration of creature will constitutes man’s only possible gift of true value to the Paradise Father. In God, man lives, moves, and has his being; there is nothing which man can give to God except this choosing to abide by the Father’s will, and such decisions, effected by the intelligent will creatures of the universes, constitute the reality of that true worship which is so satisfying to the love\hyp{}dominated nature of the Creator Father.
\vs p001 1:3 When you have once become truly God\hyp{}conscious, after you really discover the majestic Creator and begin to experience the realization of the indwelling presence of the divine controller, then, in accordance with your enlightenment and in accordance with the manner and method by which the divine Sons reveal God, you will find a name for the Universal Father which will be adequately expressive of your concept of the First Great Source and Centre. And so, on different worlds and in various universes, the Creator becomes known by numerous appellations, in spirit of relationship all meaning the same but, in words and symbols, each name standing for the degree, the depth, of his enthronement in the hearts of his creatures of any given realm.
\vs p001 1:4 \pc Near the centre of the universe of universes, the Universal Father is generally known by names which may be regarded as meaning the First Source. Farther out in the universes of space, the terms employed to designate the Universal Father more often mean the Universal Centre. Still farther out in the starry creation, he is known, as on the headquarters world of your local universe, as the First Creative Source and Divine Centre. In one near\hyp{}by constellation God is called the Father of Universes. In another, the Infinite Upholder, and to the east, the Divine Controller. He has also been designated the Father of Lights, the Gift of Life, and the All\hyp{}powerful One.
\vs p001 1:5 On those worlds where a Paradise Son has lived a bestowal life, God is generally known by some name indicative of personal relationship, tender affection, and fatherly devotion. On your constellation headquarters God is referred to as the Universal Father, and on different planets in your local system of inhabited worlds he is variously known as the Father of Fathers, the Paradise Father, the Havona Father, and the Spirit Father. Those who know God through the revelations of the bestowals of the Paradise Sons, eventually yield to the sentimental appeal of the touching relationship of the creature\hyp{}Creator association and refer to God as “our Father.”
\vs p001 1:6 On a planet of sex creatures, in a world where the impulses of parental emotion are inherent in the hearts of its intelligent beings, the term Father becomes a very expressive and appropriate name for the eternal God. He is best known, most universally acknowledged, on your planet, Urantia, by the name \bibemph{God.} The name he is given is of little importance; the significant thing is that you should know him and aspire to be like him. Your prophets of old truly called him “the everlasting God”\fnst{\textbf{“the everlasting God”}, cf. Isaiah~40:28: \textheb{עוֹלָם אֱלֹהֵי} $=$ ``God of eternity''.} and referred to him as the one who “inhabits eternity.”\fnst{\textbf{one who “inhabits eternity.”}, cf. Isaiah~57:15: \textheb{עַד שֹׁכֵן} $=$ ``dwelling eternally'' or ``inhabiting eternity''.}
\usection{The Reality of God}
\vs p001 2:1 God is primal reality in the spirit world; God is the source of truth in the mind spheres; God overshadows all throughout the material realms. To all created intelligences God is a personality, and to the universe of universes he is the First Source and Centre of eternal reality. God is neither manlike nor machinelike. The First Father is universal spirit, eternal truth, infinite reality, and father personality.
\vs p001 2:2 \pc The eternal God is infinitely more than reality idealized or the universe personalized. God is not simply the supreme desire of man, the mortal quest objectified. Neither is God merely a concept, the power\hyp{}potential of righteousness. The Universal Father is not a synonym for nature, neither is he natural law personified. God is a transcendent reality, not merely man’s traditional concept of supreme values. God is not a psychological focalization of spiritual meanings, neither is he “the noblest work of man.”\fnst{\textbf{“the noblest work of man.”}, The words ``An honest God is the noblest work of man.'' belong to an American lawyer Robert Green Ingersoll (1833--1899).} God may be any or all of these concepts in the minds of men, but he is more. He is a saving person and a loving Father to all who enjoy spiritual peace on earth, and who crave to experience personality survival in death.
\vs p001 2:3 \pc The actuality of the existence of God is demonstrated in human experience by the indwelling of the divine presence, the spirit Monitor sent from Paradise to live in the mortal mind of man and there to assist in evolving the immortal soul of eternal survival. The presence of this divine Adjuster in the human mind is disclosed by three experiential phenomena:
\vs p001 2:4 \ublistelem{1.}\bibnobreakspace The intellectual capacity for knowing God --- God\hyp{}consciousness.
\vs p001 2:5 \ublistelem{2.}\bibnobreakspace The spiritual urge to find God --- God\hyp{}seeking.
\vs p001 2:6 \ublistelem{3.}\bibnobreakspace The personality craving to be like God --- the wholehearted desire to do the Father’s will.
\vs p001 2:7 \pc The existence of God can never be proved by scientific experiment or by the pure reason of logical deduction. God can be realized only in the realms of human experience; nevertheless, the true concept of the reality of God is reasonable to logic, plausible to philosophy, essential to religion, and indispensable to any hope of personality survival.
\vs p001 2:8 Those who know God have experienced the fact of his presence; such God\hyp{}knowing mortals hold in their personal experience the only positive proof of the existence of the living God which one human being can offer to another. The existence of God is utterly beyond all possibility of demonstration except for the contact between the God\hyp{}consciousness of the human mind and the God\hyp{}presence of the Thought Adjuster that indwells the mortal intellect and is bestowed upon man as the free gift of the Universal Father.
\vs p001 2:9 \pc In theory you may think of God as the Creator, and he is the personal creator of Paradise and the central universe of perfection, but the universes of time and space are all created and organized by the Paradise corps of the Creator Sons. The Universal Father is not the personal creator of the local universe of Nebadon; the universe in which you live is the creation of his Son Michael. Though the Father does not personally create the evolutionary universes, he does control them in many of their universal relationships and in certain of their manifestations of physical, mindal, and spiritual energies. God the Father is the personal creator of the Paradise universe and, in association with the Eternal Son, the creator of all other personal universe Creators.
\vs p001 2:10 \pc As a physical controller in the material universe of universes, the First Source and Centre functions in the patterns of the eternal Isle of Paradise, and through this absolute gravity centre the eternal God exercises cosmic overcontrol of the physical level equally in the central universe and throughout the universe of universes. As mind, God functions in the Deity of the Infinite Spirit; as spirit, God is manifest in the person of the Eternal Son and in the persons of the divine children of the Eternal Son. This interrelation of the First Source and Centre with the co\hyp{}ordinate Persons and Absolutes of Paradise does not in the least preclude the \bibemph{direct} personal action of the Universal Father throughout all creation and on all levels thereof. Through the presence of his fragmentized spirit the Creator Father maintains immediate contact with his creature children and his created universes.
\usection{God is a Universal Spirit}
\vs p001 3:1 “God is spirit.”\fnst{\textbf{“God is spirit.”}, cf. John~4:24: ``God \bibemph{is} a Spirit: and they that worship him must worship \bibemph{him} in spirit and in truth.''} He is a universal spiritual presence. The Universal Father is an infinite spiritual reality; he is “the sovereign, eternal, immortal, invisible, and only true God.”\fnst{\textbf{“the sovereign, \ldots\ only true God.”}, cf. 1~Timothy~1:17: ``Now unto the King eternal, immortal, invisible, the only wise God, \bibemph{be} honour and glory for ever and ever. Amen.''} Even though you are “the offspring of God,” you ought not to think that the Father is like yourselves in form and physique because you are said to be created “in his image” --- indwelt by Mystery Monitors dispatched from the central abode of his eternal presence. Spirit beings are real, notwithstanding they are invisible to human eyes; even though they have not flesh and blood.
\vs p001 3:2 Said the seer of old: “Lo, he goes by me, and I see him not; he passes on also, but I perceive him not.”\fnst{\textbf{“Lo, he goes \ldots\ perceive him not.”}, This is a direct quote of Job~9:11.} We may constantly observe the works of God, we may be highly conscious of the material evidences of his majestic conduct, but rarely may we gaze upon the visible manifestation of his divinity, not even to behold the presence of his delegated spirit of human indwelling.
\vs p001 3:3 The Universal Father is not invisible because he is hiding himself away from the lowly creatures of materialistic handicaps and limited spiritual endowments. The situation rather is: “You cannot see my face, for no mortal can see me and live.”\fnst{\textbf{“You cannot \ldots\ and live.”}, cf. Exodus~33:20: ``And he said, Thou canst not see my face: for there shall no man see me, and live.''} No material man could behold the spirit God and preserve his mortal existence. The glory and the spiritual brilliance of the divine personality presence is impossible of approach by the lower groups of spirit beings or by any order of material personalities. The spiritual luminosity of the Father’s personal presence is a “light which no mortal man can approach; which no material creature has seen or can see.”\fnst{\textbf{“light which \ldots\ can see.”}, cf. 1~Timothy~6:16: ``Who only hath immortality, dwelling in the light which no man can approach unto; whom no man hath seen, nor can see: to whom \bibemph{be} honour and power everlasting. Amen.''} But it is not necessary to see God with the eyes of the flesh in order to discern him by the faith\hyp{}vision of the spiritualized mind.
\vs p001 3:4 \pc The spirit nature of the Universal Father is shared fully with his coexistent self, the Eternal Son of Paradise. Both the Father and the Son in like manner share the universal and eternal spirit fully and unreservedly with their conjoint personality co\hyp{}ordinate, the Infinite Spirit. God’s spirit is, in and of himself, absolute; in the Son it is unqualified, in the Spirit, universal, and in and by all of them, infinite.
\vs p001 3:5 \pc God is a universal spirit; God is the universal person. The supreme personal reality of the finite creation is spirit; the ultimate reality of the personal cosmos is absonite spirit. Only the levels of infinity are absolute, and only on such levels is there finality of oneness between matter, mind, and spirit.
\vs p001 3:6 \pc In the universes God the Father is, in potential, the overcontroller of matter, mind, and spirit. Only by means of his far\hyp{}flung personality circuit does God deal directly with the personalities of his vast creation of will creatures, but he is contactable (outside of Paradise) only in the presences of his fragmented entities, the will of God abroad in the universes. This Paradise spirit that indwells the minds of the mortals of time and there fosters the evolution of the immortal soul of the surviving creature is of the nature and divinity of the Universal Father. But the minds of such evolutionary creatures originate in the local universes and must gain divine perfection by achieving those experiential transformations of spiritual attainment which are the inevitable result of a creature’s choosing to do the will of the Father in heaven.
\vs p001 3:7 \pc In the inner experience of man, mind is joined to matter. Such material\hyp{}linked minds cannot survive mortal death. The technique of survival is embraced in those adjustments of the human will and those transformations in the mortal mind whereby such a God\hyp{}conscious intellect gradually becomes spirit taught and eventually spirit led. This evolution of the human mind from matter association to spirit union results in the transmutation of the potentially spirit phases of the mortal mind into the morontia realities of the immortal soul. Mortal mind subservient to matter is destined to become increasingly material and consequently to suffer eventual personality extinction; mind yielded to spirit is destined to become increasingly spiritual and ultimately to achieve oneness with the surviving and guiding divine spirit and in this way to attain survival and eternity of personality existence.
\vs p001 3:8 I come forth from the Eternal, and I have repeatedly returned to the presence of the Universal Father. I know of the actuality and personality of the First Source and Centre, the Eternal and Universal Father. I know that, while the great God is absolute, eternal, and infinite, he is also good, divine, and gracious. I know the truth of the great declarations: “God is spirit” and “God is love,” and these two attributes are most completely revealed to the universe in the Eternal Son.
\usection{The Mystery of God}
\vs p001 4:1 The infinity of the perfection of God is such that it eternally constitutes him mystery. And the greatest of all the unfathomable mysteries of God is the phenomenon of the divine indwelling of mortal minds. The manner in which the Universal Father sojourns with the creatures of time is the most profound of all universe mysteries; the divine presence in the mind of man is the mystery of mysteries.
\vs p001 4:2 The physical bodies of mortals are “the temples of God.”\fnst{\textbf{“the temples of God.”}, cf. 1~Corinthians~3:16: ``Know ye not that ye are the temple of God, and \bibemph{that} the Spirit of God dwelleth in you?''} Notwithstanding that the Sovereign Creator Sons come near the creatures of their inhabited worlds and “draw all men to themselves”; though they “stand at the door” of consciousness “and knock” and delight to come in to all who will “open the doors of their hearts”; although there does exist this intimate personal communion between the Creator Sons and their mortal creatures, nevertheless, mortal men have something from God himself which actually dwells within them; their bodies are the temples thereof.
\vs p001 4:3 When you are through down here, when your course has been run in temporary form on earth, when your trial trip in the flesh is finished, when the dust that composes the mortal tabernacle “returns to the earth whence it came”; then, it is revealed, the indwelling “Spirit shall return to God who gave it.”\fnst{\textbf{when the dust \ldots\ to God who gave it.”}, cf. Ecclesiastes~12:7: ``Then shall the dust return to the earth as it was: and the spirit shall return unto God who gave it.'' Note, that ``as it was'' is a more accurate rendering of both the Hebrew (MT) \textheb{כְּשֶׁהָיָה} and the Greek (LXX) \textgreek{ὡς ἦν}.} There sojourns within each moral being of this planet a fragment of God, a part and parcel of divinity. It is not yet yours by right of possession, but it is designedly intended to be one with you if you survive the mortal existence.
\vs p001 4:4 \pc We are constantly confronted with this mystery of God; we are nonplussed by the increasing unfolding of the endless panorama of the truth of his infinite goodness, endless mercy, matchless wisdom, and superb character.
\vs p001 4:5 \pc The divine mystery consists in the inherent difference which exists between the finite and the infinite, the temporal and the eternal, the time\hyp{}space creature and the Universal Creator, the material and the spiritual, the imperfection of man and the perfection of Paradise Deity. The God of universal love unfailingly manifests himself to every one of his creatures up to the fullness of that creature’s capacity to spiritually grasp the qualities of divine truth, beauty, and goodness.
\vs p001 4:6 To every spirit being and to every mortal creature in every sphere and on every world of the universe of universes, the Universal Father reveals all of his gracious and divine self that can be discerned or comprehended by such spirit beings and by such mortal creatures. God is no respecter of persons, either spiritual or material. The divine presence which any child of the universe enjoys at any given moment is limited only by the capacity of such a creature to receive and to discern the spirit actualities of the supermaterial world.
\vs p001 4:7 As a reality in human spiritual experience God is not a mystery. But when an attempt is made to make plain the realities of the spirit world to the physical minds of the material order, mystery appears: mysteries so subtle and so profound that only the faith\hyp{}grasp of the God\hyp{}knowing mortal can achieve the philosophic miracle of the recognition of the Infinite by the finite, the discernment of the eternal God by the evolving mortals of the material worlds of time and space.
\usection{Personality of the Universal Father}
\vs p001 5:1 Do not permit the magnitude of God, his infinity, either to obscure or eclipse his personality. “He who planned the ear, shall he not hear? He who formed the eye, shall he not see?”\fnst{\textbf{“He who planned \ldots\ not see?”}, cf. Psalms~94:9: ``He that planted the ear, shall he not hear? he that formed the eye, shall he not see?'' Note, that ``he that planted'' is a more accurate translation of the Hebrew \textheb{הֲנֹטַע} and Greek \textgreek{ὁ φυτεύσας}, than ``he who planned'' given in the text. This is probably intentional to differentiate between God as the First Source and Centre from the Creator Son or even from the ultimate actual agencies (Life Carriers) who directly ``planted'' the ear and all other organs. But if so, then why is \textheb{יֹצֵר} correctly given as ``he who formed'' in the second part of this verse?} The Universal Father is the acme of divine personality; he is the origin and destiny of personality throughout all creation. God is both infinite and personal; he is an infinite personality. The Father is truly a personality, notwithstanding that the infinity of his person places him forever beyond the full comprehension of material and finite beings.
\vs p001 5:2 God is much more than a personality as personality is understood by the human mind; he is even far more than any possible concept of a superpersonality. But it is utterly futile to discuss such incomprehensible concepts of divine personality with the minds of material creatures whose maximum concept of the reality of being consists in the idea and ideal of personality. The material creature’s highest possible concept of the Universal Creator is embraced within the spiritual ideals of the exalted idea of divine personality. Therefore, although you may know that God must be much more than the human conception of personality, you equally well know that the Universal Father cannot possibly be anything less than an eternal, infinite, true, good, and beautiful personality.
\vs p001 5:3 God is not hiding from any of his creatures. He is unapproachable to so many orders of beings only because he “dwells in a light which no material creature can approach.” The immensity and grandeur of the divine personality is beyond the grasp of the unperfected mind of evolutionary mortals. He “measures the waters in the hollow of his hand, measures a universe with the span of his hand. It is he who sits on the circle of the earth, who stretches out the heavens as a curtain and spreads them out as a universe to dwell in.” “Lift up your eyes on high and behold who has created all these things, who brings out their worlds by number and calls them all by their names”; and so it is true that “the invisible things of God are partially understood by the things which are made.”\fnst{\textbf{“the invisible \ldots\ are made.”}, cf. Romans~1:20: ``For the invisible things of him from the creation of the world are clearly seen, being understood by the things that are made, \bibemph{even} his eternal power and Godhead; so that they are without excuse.''} Today, and as you are, you must discern the invisible Maker through his manifold and diverse creation, as well as through the revelation and ministration of his Sons and their numerous subordinates.
\vs p001 5:4 Even though material mortals cannot see the person of God, they should rejoice in the assurance that he is a person; by faith accept the truth which portrays that the Universal Father so loved the world as to provide for the eternal spiritual progression of its lowly inhabitants; that he “delights in his children.” God is lacking in none of those superhuman and divine attributes which constitute a perfect, eternal, loving, and infinite Creator personality.
\vs p001 5:5 \pc In the local creations (excepting the personnel of the superuniverses) God has no personal or residential manifestation aside from the Paradise Creator Sons who are the fathers of the inhabited worlds and the sovereigns of the local universes. If the faith of the creature were perfect, he would assuredly know that when he had seen a Creator Son he had seen the Universal Father; in seeking for the Father, he would not ask nor expect to see other than the Son. Mortal man simply cannot see God until he achieves completed spirit transformation and actually attains Paradise.
\vs p001 5:6 The natures of the Paradise Creator Sons do not encompass all the unqualified potentials of the universal absoluteness of the infinite nature of the First Great Source and Centre, but the Universal Father is in every way \bibemph{divinely} present in the Creator Sons. The Father and his Sons are one. These Paradise Sons of the order of Michael are perfect personalities, even the pattern for all local universe personality from that of the Bright and Morning Star down to the lowest human creature of progressing animal evolution.
\vs p001 5:7 \pc Without God and except for his great and central person, there would be no personality throughout all the vast universe of universes. \bibemph{God is personality.}
\vs p001 5:8 \pc Notwithstanding that God is an eternal power, a majestic presence, a transcendent ideal, and a glorious spirit, though he is all these and infinitely more, nonetheless, he is truly and everlastingly a perfect Creator personality, a person who can “know and be known,” who can “love and be loved,” and one who can befriend us; while you can be known, as other humans have been known, as the friend of God. He is a real spirit and a spiritual reality.
\vs p001 5:9 As we see the Universal Father revealed throughout his universe; as we discern him indwelling his myriads of creatures; as we behold him in the persons of his Sovereign Sons; as we continue to sense his divine presence here and there, near and afar, let us not doubt nor question his personality primacy. Notwithstanding all these far\hyp{}flung distributions, he remains a true person and everlastingly maintains personal connection with the countless hosts of his creatures scattered throughout the universe of universes.
\vs p001 5:10 \pc The idea of the personality of the Universal Father is an enlarged and truer concept of God which has come to mankind chiefly through revelation. Reason, wisdom, and religious experience all infer and imply the personality of God, but they do not altogether validate it. Even the indwelling Thought Adjuster is prepersonal. The truth and maturity of any religion is directly proportional to its concept of the infinite personality of God and to its grasp of the absolute unity of Deity. The idea of a personal Deity becomes, then, the measure of religious maturity after religion has first formulated the concept of the unity of God.
\vs p001 5:11 Primitive religion had many personal gods, and they were fashioned in the image of man. Revelation affirms the validity of the personality concept of God which is merely possible in the scientific postulate of a First Cause and is only provisionally suggested in the philosophic idea of Universal Unity. Only by personality approach can any person begin to comprehend the unity of God. To deny the personality of the First Source and Centre leaves one only the choice of two philosophic dilemmas: materialism or pantheism.
\vs p001 5:12 In the contemplation of Deity, the concept of personality must be divested of the idea of corporeality. A material body is not indispensable to personality in either man or God. The corporeality error is shown in both extremes of human philosophy. In materialism, since man loses his body at death, he ceases to exist as a personality; in pantheism, since God has no body, he is not, therefore, a person. The superhuman type of progressing personality functions in a union of mind and spirit.
\vs p001 5:13 \pc Personality is not simply an attribute of God; it rather stands for the totality of the co\hyp{}ordinated infinite nature and the unified divine will which is exhibited in eternity and universality of perfect expression. Personality, in the supreme sense, is the revelation of God to the universe of universes.
\vs p001 5:14 \pc God, being eternal, universal, absolute, and infinite, does not grow in knowledge nor increase in wisdom. God does not acquire experience, as finite man might conjecture or comprehend, but he does, within the realms of his own eternal personality, enjoy those continuous expansions of self\hyp{}realization which are in certain ways comparable to, and analogous with, the acquirement of new experience by the finite creatures of the evolutionary worlds.
\vs p001 5:15 The absolute perfection of the infinite God would cause him to suffer the awful limitations of unqualified finality of perfectness were it not a fact that the Universal Father directly participates in the personality struggle of every imperfect soul in the wide universe who seeks, by divine aid, to ascend to the spiritually perfect worlds on high. This progressive experience of every spirit being and every mortal creature throughout the universe of universes is a part of the Father’s ever\hyp{}expanding Deity\hyp{}consciousness of the never\hyp{}ending divine circle of ceaseless self\hyp{}realization.
\vs p001 5:16 It is literally true: “In all your afflictions he is afflicted.”\fnst{\textbf{“In all \ldots\ afflicted.”}, cf. Isaiah~63:9: ``In all their affliction he was afflicted, and the angel of his presence saved them: in his love and in his pity he redeemed them; and he bare them, and carried them all the days of old.''} “In all your triumphs he triumphs in and with you.” His prepersonal divine spirit is a real part of you. The Isle of Paradise responds to all the physical metamorphoses of the universe of universes; the Eternal Son includes all the spirit impulses of all creation; the Conjoint Actor encompasses all the mind expression of the expanding cosmos. The Universal Father realizes in the fullness of the divine consciousness all the individual experience of the progressive struggles of the expanding minds and the ascending spirits of every entity, being, and personality of the whole evolutionary creation of time and space. And all this is literally true, for “in Him we all live and move and have our being.”\fnst{\textbf{“in Him we all live and move and have our being.”}, cf. Acts~17:28: ``For in him we live, and move, and have our being; as certain also of your own poets have said, For we are also his offspring.'' Here, by ``certain also of your own poets'' Paul is probably referring to Aratus (315--240\,B.C.) who wrote in \textgreek{Φαινόμενα}~4--5: ``Everywhere everyone is indebted to Zeus. For we are indeed his offspring.''}
\usection{Personality in the Universe}
\vs p001 6:1 Human personality is the time\hyp{}space image\hyp{}shadow cast by the divine Creator personality. And no actuality can ever be adequately comprehended by an examination of its shadow. Shadows should be interpreted in terms of the true substance.
\vs p001 6:2 \pc God is to science a cause, to philosophy an idea, to religion a person, even the loving heavenly Father. God is to the scientist a primal force, to the philosopher a hypothesis of unity, to the religionist a living spiritual experience. Man’s inadequate concept of the personality of the Universal Father can be improved only by man’s spiritual progress in the universe and will become truly adequate only when the pilgrims of time and space finally attain the divine embrace of the living God on Paradise.
\vs p001 6:3 Never lose sight of the antipodal viewpoints of personality as it is conceived by God and man. Man views and comprehends personality, looking from the finite to the infinite; God looks from the infinite to the finite. Man possesses the lowest type of personality; God, the highest, even supreme, ultimate, and absolute. Therefore did the better concepts of the divine personality have patiently to await the appearance of improved ideas of human personality, especially the enhanced revelation of both human and divine personality in the Urantian bestowal life of Michael, the Creator Son.
\vs p001 6:4 \pc The prepersonal divine spirit which indwells the mortal mind carries, in its very presence, the valid proof of its actual existence, but the concept of the divine personality can be grasped only by the spiritual insight of genuine personal religious experience. Any person, human or divine, may be known and comprehended quite apart from the external reactions or the material presence of that person.
\vs p001 6:5 Some degree of moral affinity and spiritual harmony is essential to friendship between two persons; a loving personality can hardly reveal himself to a loveless person. Even to approach the knowing of a divine personality, all of man’s personality endowments must be wholly consecrated to the effort; half\hyp{}hearted, partial devotion will be unavailing.
\vs p001 6:6 The more completely man understands himself and appreciates the personality values of his fellows, the more he will crave to know the Original Personality, and the more earnestly such a God\hyp{}knowing human will strive to become like the Original Personality. You can argue over opinions about God, but experience with him and in him exists above and beyond all human controversy and mere intellectual logic. The God\hyp{}knowing man describes his spiritual experiences, not to convince unbelievers, but for the edification and mutual satisfaction of believers.
\vs p001 6:7 \pc To assume that the universe can be known, that it is intelligible, is to assume that the universe is mind made and personality managed. Man’s mind can only perceive the mind phenomena of other minds, be they human or superhuman. If man’s personality can experience the universe, there is a divine mind and an actual personality somewhere concealed in that universe.
\vs p001 6:8 \pc God is spirit --- spirit personality; man is also a spirit --- potential spirit personality. Jesus of Nazareth attained the full realization of this potential of spirit personality in human experience; therefore his life of achieving the Father’s will becomes man’s most real and ideal revelation of the personality of God. Even though the personality of the Universal Father can be grasped only in actual religious experience, in Jesus’ earth life we are inspired by the perfect demonstration of such a realization and revelation of the personality of God in a truly human experience.
\usection{Spiritual Value of the Personality Concept}
\vs p001 7:1 When Jesus talked about \textcolour{ubdarkred}{“the living God,”} he referred to a personal Deity --- the Father in heaven. The concept of the personality of Deity facilitates fellowship; it favours intelligent worship; it promotes refreshing trustfulness. Interactions can be had between nonpersonal things, but not fellowship. The fellowship relation of father and son, as between God and man, cannot be enjoyed unless both are persons. Only personalities can commune with each other, albeit this personal communion may be greatly facilitated by the presence of just such an impersonal entity as the Thought Adjuster.
\vs p001 7:2 Man does not achieve union with God as a drop of water might find unity with the ocean. Man attains divine union by progressive reciprocal spiritual communion, by personality intercourse with the personal God, by increasingly attaining the divine nature through wholehearted and intelligent conformity to the divine will. Such a sublime relationship can exist only between personalities.
\vs p001 7:3 \pc The concept of truth might possibly be entertained apart from personality, the concept of beauty may exist without personality, but the concept of divine goodness is understandable only in relation to personality. Only a \bibemph{person} can love and be loved. Even beauty and truth would be divorced from survival hope if they were not attributes of a personal God, a loving Father.
\vs p001 7:4 \pc We cannot fully understand how God can be primal, changeless, all\hyp{}powerful, and perfect, and at the same time be surrounded by an ever\hyp{}changing and apparently law\hyp{}limited universe, an evolving universe of relative imperfections. But we can \bibemph{know} such a truth in our own personal experience since we all maintain identity of personality and unity of will in spite of the constant changing of both ourselves and our environment.
\vs p001 7:5 Ultimate universe reality cannot be grasped by mathematics, logic, or philosophy, only by personal experience in progressive conformity to the divine will of a personal God. Neither science, philosophy, nor theology can validate the personality of God. Only the personal experience of the faith sons of the heavenly Father can effect the actual spiritual realization of the personality of God.
\vs p001 7:6 \pc The higher concepts of universe personality imply: identity, self\hyp{}consciousness, self\hyp{}will, and possibility for self\hyp{}revelation. And these characteristics further imply fellowship with other and equal personalities, such as exists in the personality associations of the Paradise Deities. And the absolute unity of these associations is so perfect that divinity becomes known by indivisibility, by oneness. “The Lord God is \bibemph{one.}”\fnst{\textbf{“The Lord God is \bibemph{one.}”}, Deuteronomy~6:4 says: \textheb{אֶחָד יְהוָה אֱלֹהֵינוּ יְהוָה יִשְׂרָאֵל שְׁמַע} ``Hear, O Israel! Yahweh our God, Yahweh \bibemph{is} One.'' This formula is literally quoted in the Greek text of Mark~12:29: \textgreek{Ἄκουε, Ἰσραήλ, κύριος ὁ θεὸς ἡμῶν κύριος εἷς ἐστιν}.} Indivisibility of personality does not interfere with God’s bestowing his spirit to live in the hearts of mortal men. Indivisibility of a human father’s personality does not prevent the reproduction of mortal sons and daughters.
\vs p001 7:7 This concept of indivisibility in association with the concept of unity implies transcendence of both time and space by the Ultimacy of Deity; therefore neither space nor time can be absolute or infinite. The First Source and Centre is that infinity who unqualifiedly transcends all mind, all matter, and all spirit.
\vs p001 7:8 The fact of the Paradise Trinity in no manner violates the truth of the divine unity. The three personalities of Paradise Deity are, in all universe reality reactions and in all creature relations, as one. Neither does the existence of these three eternal persons violate the truth of the indivisibility of Deity. I am fully aware that I have at my command no language adequate to make clear to the mortal mind how these universe problems appear to us. But you should not become discouraged; not all of these things are wholly clear to even the high personalities belonging to my group of Paradise beings. Ever bear in mind that these profound truths pertaining to Deity will increasingly clarify as your minds become progressively spiritualized during the successive epochs of the long mortal ascent to Paradise.
\vsetoff
\vs p001 7:9 [Presented by a Divine Counsellor, a member of a group of celestial personalities assigned by the Ancients of Days on Uversa, the headquarters of the seventh superuniverse, to supervise those portions of this forthcoming revelation which have to do with affairs beyond the borders of the local universe of Nebadon. I am commissioned to sponsor those papers portraying the nature and attributes of God because I represent the highest source of information available for such a purpose on any inhabited world. I have served as a Divine Counsellor in all seven of the superuniverses and have long resided at the Paradise centre of all things. Many times have I enjoyed the supreme pleasure of a sojourn in the immediate personal presence of the Universal Father. I portray the reality and truth of the Father’s nature and attributes with unchallengeable authority; I know whereof I speak.]
\quizlink
\begin{thebibliography}{100}
\bibitem{Knudson1}
Albert C. Knudson.
{<<The Doctrine of God>>.}
{\em New York: Abingdon-Cokesbury Press}, 1930.
\bibitem{Matthews1}
W.R. Matthews, K.C.V.O., D.D., D.Lit.
{<<God in Christian Thought and Experience>>.}
{\em London: Nisbet \&\ Co. Ltd.}, 1930.
\bibitem{Illingworth1}
J.R. Illingworth, M.A.
{<<Personality Human and Divine>>.}
{\em London and New York: The Macmillan Company}, 1894.
\end{thebibliography}
