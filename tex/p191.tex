\upaper{191}{Appearances to the Apostles and Other Leaders}
\uminitoc{The Appearance to Peter}
\uminitoc{First Appearance to the Apostles}
\uminitoc{With the Morontia Creatures}
\uminitoc{The Tenth Appearance (At Philadelphia)}
\uminitoc{Second Appearance to the Apostles}
\uminitoc{The Alexandrian Appearance}
\author{Midwayer Commission}
\vs p191 0:1 Resurrection Sunday was a terrible day in the lives of the apostles; ten of them spent the larger part of the day in the upper chamber behind barred doors. They might have fled from Jerusalem, but they were afraid of being arrested by the agents of the Sanhedrin if they were found abroad. Thomas was brooding over his troubles alone at Bethpage. He would have fared better had he remained with his fellow apostles, and he would have aided them to direct their discussions along more helpful lines.
\vs p191 0:2 All day long John upheld the idea that Jesus had risen from the dead. He recounted no less than five different times when the Master had affirmed he would rise again and at least three times when he alluded to the third day. John’s attitude had considerable influence on them, especially on his brother James and on Nathaniel. John would have influenced them more if he had not been the youngest member of the group.
\vs p191 0:3 Their isolation had much to do with their troubles. John Mark kept them in touch with developments about the temple and informed them as to the many rumours gaining headway in the city, but it did not occur to him to gather up news from the different groups of believers to whom Jesus had already appeared. That was the kind of service which had heretofore been rendered by the messengers of David, but they were all absent on their last assignment as heralds of the resurrection to those groups of believers who dwelt remote from Jerusalem. For the first time in all these years the apostles realized how much they had been dependent on David’s messengers for their daily information regarding the affairs of the kingdom.
\vs p191 0:4 All this day Peter characteristically vacillated emotionally between faith and doubt concerning the Master’s resurrection. Peter could not get away from the sight of the grave cloths resting there in the tomb as if the body of Jesus had just evaporated from within. “But,” reasoned Peter, “if he has risen and can show himself to the women, why does he not show himself to us, his apostles?” Peter would grow sorrowful when he thought that maybe Jesus did not come to them on account of his presence among the apostles, because he had denied him that night in Annas’s courtyard. And then would he cheer himself with the word brought by the women, \textcolour{ubdarkred}{“Go tell my apostles --- and Peter.”} But to derive encouragement from this message implied that he must believe that the women had really seen and heard the risen Master. Thus Peter alternated between faith and doubt throughout the whole day, until a little after 20:00, when he ventured out into the courtyard. Peter thought to remove himself from among the apostles so that he might not prevent Jesus’ coming to them because of his denial of the Master.
\vs p191 0:5 James Zebedee at first advocated that they all go to the tomb; he was strongly in favour of doing something to get to the bottom of the mystery. It was Nathaniel who prevented them from going out in public in response to James’s urging, and he did this by reminding them of Jesus’ warning against unduly jeopardizing their lives at this time. By noontime James had settled down with the others to watchful waiting. He said little; he was tremendously disappointed because Jesus did not appear to them, and he did not know of the Master’s many appearances to other groups and individuals.
\vs p191 0:6 Andrew did much listening this day. He was exceedingly perplexed by the situation and had more than his share of doubts, but he at least enjoyed a certain sense of freedom from responsibility for the guidance of his fellow apostles. He was indeed grateful that the Master had released him from the burdens of leadership before they fell upon these distracting times.
\vs p191 0:7 More than once during the long and weary hours of this tragic day, the only sustaining influence of the group was the frequent contribution of Nathaniel’s characteristic philosophic counsel. He was really the controlling influence among the ten throughout the entire day. Never once did he express himself concerning either belief or disbelief in the Master’s resurrection. But as the day wore on, he became increasingly inclined toward believing that Jesus had fulfilled his promise to rise again.
\vs p191 0:8 Simon Zelotes was too much crushed to participate in the discussions. Most of the time he reclined on a couch in a corner of the room with his face to the wall; he did not speak half a dozen times throughout the whole day. His concept of the kingdom had crashed, and he could not discern that the Master’s resurrection could materially change the situation. His disappointment was very personal and altogether too keen to be recovered from on short notice, even in the face of such a stupendous fact as the resurrection.
\vs p191 0:9 Strange to record, the usually inexpressive Philip did much talking throughout the afternoon of this day. During the forenoon he had little to say, but all afternoon he asked questions of the other apostles. Peter was often annoyed by Philip’s questions, but the others took his inquiries good\hyp{}naturedly. Philip was particularly desirous of knowing, provided Jesus had really risen from the grave, whether his body would bear the physical marks of the crucifixion.
\vs p191 0:10 Matthew was highly confused; he listened to the discussions of his fellows but spent most of the time turning over in his mind the problem of their future finances. Regardless of Jesus’ supposed resurrection, Judas was gone, David had unceremoniously turned the funds over to him, and they were without an authoritative leader. Before Matthew got around to giving serious consideration to their arguments about the resurrection, he had already seen the Master face to face.
\vs p191 0:11 The Alpheus twins took little part in these serious discussions; they were fairly busy with their customary ministrations. One of them expressed the attitude of both when he said, in reply to a question asked by Philip: “We do not understand about the resurrection, but our mother says she talked with the Master, and we believe her.”
\vs p191 0:12 Thomas was in the midst of one of his typical spells of despairing depression. He slept a portion of the day and walked over the hills the rest of the time. He felt the urge to rejoin his fellow apostles, but the desire to be by himself was the stronger.
\vs p191 0:13 The Master put off the first morontia appearance to the apostles for a number of reasons. First, he wanted them to have time, after they heard of his resurrection, to think well over what he had told them about his death and resurrection when he was still with them in the flesh. The Master wanted Peter to wrestle through with some of his peculiar difficulties before he manifested himself to them all. In the second place, he desired that Thomas should be with them at the time of his first appearance. John Mark located Thomas at the home of Simon in Bethpage early this Sunday morning, bringing word to that effect to the apostles about 11:00. Any time during this day Thomas would have gone back to them if Nathaniel or any two of the other apostles had gone for him. He really wanted to return, but having left as he did the evening before, he was too proud to go back of his own accord so soon. By the next day he was so depressed that it required almost a week for him to make up his mind to return. The apostles waited for him, and he waited for his brethren to seek him out and ask him to come back to them. Thomas thus remained away from his associates until the next Saturday evening, when, after darkness had come on, Peter and John went over to Bethpage and brought him back with them. And this is also the reason why they did not go at once to Galilee after Jesus first appeared to them; they would not go without Thomas.
\usection{The Appearance to Peter}
\vs p191 1:1 It was near 20:30 this Sunday evening when Jesus appeared to Simon Peter in the garden of the Mark home. This was his 8\ts{th} morontia manifestation. Peter had lived under a heavy burden of doubt and guilt ever since his denial of the Master. All day Saturday and this Sunday he had fought the fear that, perhaps, he was no longer an apostle. He had shuddered at the fate of Judas and even thought that he, too, had betrayed his Master. All this afternoon he thought that it might be his presence with the apostles that prevented Jesus’ appearing to them, provided, of course, he had really risen from the dead. And it was to Peter, in such a frame of mind and in such a state of soul, that Jesus appeared as the dejected apostle strolled among the flowers and shrubs.
\vs p191 1:2 When Peter thought of the loving look of the Master as he passed by on Annas’s porch, and as he turned over in his mind that wonderful message brought him early that morning by the women who came from the empty tomb, \textcolour{ubdarkred}{“Go tell my apostles --- and Peter”} --- as he contemplated these tokens of mercy, his faith began to surmount his doubts, and he stood still, clenching his fists, while he spoke aloud: “I believe he has risen from the dead; I will go and tell my brethren.” And as he said this, there suddenly appeared in front of him the form of a man, who spoke to him in familiar tones, saying: \textcolour{ubdarkred}{“Peter, the enemy desired to have you, but I would not give you up. I knew it was not from the heart that you disowned me; therefore I forgave you even before you asked; but now must you cease to think about yourself and the troubles of the hour while you prepare to carry the good news of the gospel to those who sit in darkness. No longer should you be concerned with what you may obtain from the kingdom but rather be exercised about what you can give to those who live in dire spiritual poverty. Gird yourself, Simon, for the battle of a new day, the struggle with spiritual darkness and the evil doubtings of the natural minds of men.”}
\vs p191 1:3 Peter and the morontia Jesus walked through the garden and talked of things past, present, and future for almost five minutes. Then the Master vanished from his gaze, saying, \textcolour{ubdarkred}{“Farewell, Peter, until I see you with your brethren.”}
\vs p191 1:4 For a moment, Peter was overcome by the realization that he had talked with the risen Master, and that he could be sure he was still an ambassador of the kingdom. He had just heard the glorified Master exhort him to go on preaching the gospel. And with all this welling up within his heart, he rushed to the upper chamber and into the presence of his fellow apostles, exclaiming in breathless excitement: “I have seen the Master; he was in the garden. I talked with him, and he has forgiven me.”
\vs p191 1:5 Peter’s declaration that he had seen Jesus in the garden made a profound impression upon his fellow apostles, and they were about ready to surrender their doubts when Andrew got up and warned them not to be too much influenced by his brother’s report. Andrew intimated that Peter had seen things which were not real before. Although Andrew did not directly allude to the vision of the night on the Sea of Galilee wherein Peter claimed to have seen the Master coming to them walking on the water, he said enough to betray to all present that he had this incident in mind. Simon Peter was very much hurt by his brother’s insinuations and immediately lapsed into crestfallen silence. The twins felt very sorry for Peter, and they both went over to express their sympathy and to say that they believed him and to reassert that their own mother had also seen the Master.
\usection{First Appearance to the Apostles}
\vs p191 2:1 Shortly after 21:00 that evening, after the departure of Cleopas and Jacob, while the Alpheus twins comforted Peter, and while Nathaniel remonstrated with Andrew, and as the ten apostles were there assembled in the upper chamber with all the doors bolted for fear of arrest, the Master, in morontia form, suddenly appeared in the midst of them, saying: \textcolour{ubdarkred}{“Peace be upon you. Why are you so frightened when I appear, as though you had seen a spirit? Did I not tell you about these things when I was present with you in the flesh? Did I not say to you that the chief priests and the rulers would deliver me up to be killed, that one of your own number would betray me, and that on the third day I would rise? Wherefore all your doubtings and all this discussion about the reports of the women, Cleopas and Jacob, and even Peter? How long will you doubt my words and refuse to believe my promises? And now that you actually see me, will you believe? Even now one of you is absent. When you are gathered together once more, and after all of you know of a certainty that the Son of Man has risen from the grave, go hence into Galilee. Have faith in God; have faith in one another; and so shall you enter into the new service of the kingdom of heaven. I will tarry in Jerusalem with you until you are ready to go into Galilee. My peace I leave with you.”}
\vs p191 2:2 When the morontia Jesus had spoken to them, he vanished in an instant from their sight. And they all fell on their faces, praising God and venerating their vanished Master. This was the Master’s 9\ts{th} morontia appearance.
\usection{With the Morontia Creatures}
\vs p191 3:1 The next day, Monday, was spent wholly with the morontia creatures then present on Urantia. As participants in the Master’s morontia\hyp{}transition experience, there had come to Urantia more than 1,000,000 morontia directors and associates, together with transition mortals of various orders from the seven mansion worlds of Satania. The morontia Jesus sojourned with these splendid intelligences for 40 days. He instructed them and learned from their directors the life of morontia transition as it is traversed by the mortals of the inhabited worlds of Satania as they pass through the system morontia spheres.
\vs p191 3:2 About midnight of this Monday the Master’s morontia form was adjusted for transition to the 2\ts{nd} stage of morontia progression. When he next appeared to his mortal children on earth, it was as a 2\ts{nd}\hyp{}stage morontia being. As the Master progressed in the morontia career, it became, technically, more and more difficult for the morontia intelligences and their transforming associates to visualize the Master to mortal and material eyes.
\vs p191 3:3 Jesus made the transit to the 3\ts{rd} stage of morontia on Friday, April 14; to the 4\ts{th} stage on Monday, the 17\ts{th}; to the 5\ts{th} stage on Saturday, the 22\ts{nd}; to the 6\ts{th} stage on Thursday, the 27\ts{th}; to the 7\ts{th} stage on Tuesday, May 2; to Jerusem citizenship on Sunday, the 7\ts{th}; and he entered the embrace of the Most Highs of Edentia on Sunday, the 14\ts{th}.
\vs p191 3:4 In this manner did Michael of Nebadon complete his service of universe experience since he had already, in connection with his previous bestowals, experienced to the full the life of the ascendant mortals of time and space from the sojourn on the headquarters of the constellation even on to, and through, the service of the headquarters of the superuniverse. And it was by these very morontia experiences that the Creator Son of Nebadon really finished and acceptably terminated his 7\ts{th} and final universe bestowal.
\usection{The Tenth Appearance (At Philadelphia)}
\vs p191 4:1 The 10\ts{th} morontia manifestation of Jesus to mortal recognition occurred a short time after 8:00 on Tuesday, April 11, at Philadelphia, where he showed himself to Abner and Lazarus and some 150 of their associates, including more than 50 of the evangelistic corps of the seventy. This appearance occurred just after the opening of a special meeting in the synagogue which had been called by Abner to discuss the crucifixion of Jesus and the more recent report of the resurrection which had been brought by David’s messenger. Inasmuch as the resurrected Lazarus was now a member of this group of believers, it was not difficult for them to believe the report that Jesus had risen from the dead.
\vs p191 4:2 The meeting in the synagogue was just being opened by Abner and Lazarus, who were standing together in the pulpit, when the entire audience of believers saw the form of the Master appear suddenly. He stepped forward from where he had appeared between Abner and Lazarus, neither of whom had observed him, and saluting the company, said:
\vs p191 4:3 \pc \textcolour{ubdarkred}{“Peace be upon you. You all know that we have one Father in heaven, and that there is but one gospel of the kingdom --- the good news of the gift of eternal life which men receive by faith. As you rejoice in your loyalty to the gospel, pray the Father of truth to shed abroad in your hearts a new and greater love for your brethren. You are to love all men as I have loved you; you are to serve all men as I have served you. With understanding sympathy and brotherly affection, fellowship all your brethren who are dedicated to the proclamation of the good news, whether they be Jew or gentile, Greek or Roman, Persian or Ethiopian. John proclaimed the kingdom in advance; you have preached the gospel in power; the Greeks already teach the good news; and I am soon to send forth the Spirit of Truth into the souls of all these, my brethren, who have so unselfishly dedicated their lives to the enlightenment of their fellows who sit in spiritual darkness. You are all the children of light; therefore stumble not into the misunderstanding entanglements of mortal suspicion and human intolerance. If you are ennobled, by the grace of faith, to love unbelievers, should you not also equally love those who are your fellow believers in the far\hyp{}spreading household of faith? Remember, as you love one another, all men will know that you are my disciples.}
\vs p191 4:4 \textcolour{ubdarkred}{“Go, then, into all the world proclaiming this gospel of the fatherhood of God and the brotherhood of men to all nations and races and ever be wise in your choice of methods for presenting the good news to the different races and tribes of mankind. Freely you have received this gospel of the kingdom, and you will freely give the good news to all nations. Fear not the resistance of evil, for I am with you always, even to the end of the ages. And my peace I leave with you.”}
\vs p191 4:5 \pc When he had said, \textcolour{ubdarkred}{“My peace I leave with you,”} he vanished from their sight. With the exception of one of his appearances in Galilee, where upward of 500 believers saw him at one time, this group in Philadelphia embraced the largest number of mortals who saw him on any single occasion.
\vs p191 4:6 Early the next morning, even while the apostles tarried in Jerusalem awaiting the emotional recovery of Thomas, these believers at Philadelphia went forth proclaiming that Jesus of Nazareth had risen from the dead.
\vs p191 4:7 The next day, Wednesday, Jesus spent without interruption in the society of his morontia associates, and during the midafternoon hours he received visiting morontia delegates from the mansion worlds of every local system of inhabited spheres throughout the constellation of Norlatiadek. And they all rejoiced to know their Creator as one of their own order of universe intelligence.
\usection{Second Appearance to the Apostles}
\vs p191 5:1 Thomas spent a lonesome week alone with himself in the hills around about Olivet. During this time he saw only those at Simon’s house and John Mark. It was about 9:00 on Saturday, April 15, when the two apostles found him and took him back with them to their rendezvous at the Mark home. The next day Thomas listened to the telling of the stories of the Master’s various appearances, but he steadfastly refused to believe. He maintained that Peter had enthused them into thinking they had seen the Master. Nathaniel reasoned with him, but it did no good. There was an emotional stubbornness associated with his customary doubtfulness, and this state of mind, coupled with his chagrin at having run away from them, conspired to create a situation of isolation which even Thomas himself did not fully understand. He had withdrawn from his fellows, he had gone his own way, and now, even when he was back among them, he unconsciously tended to assume an attitude of disagreement. He was slow to surrender; he disliked to give in. Without intending it, he really enjoyed the attention paid him; he derived unconscious satisfaction from the efforts of all his fellows to convince and convert him. He had missed them for a full week, and he obtained considerable pleasure from their persistent attentions.
\vs p191 5:2 They were having their evening meal a little after 18:00, with Peter sitting on one side of Thomas and Nathaniel on the other, when the doubting apostle said: “I will not believe unless I see the Master with my own eyes and put my finger in the mark of the nails.” As they thus sat at supper, and while the doors were securely shut and barred, the morontia Master suddenly appeared inside the curvature of the table and, standing directly in front of Thomas, said:
\vs p191 5:3 “Peace be upon you. For a full week have I tarried that I might appear again when you were all present to hear once more the commission to go into all the world and preach this gospel of the kingdom. Again I tell you: As the Father sent me into the world, so send I you. As I have revealed the Father, so shall you reveal the divine love, not merely with words, but in your daily living. I send you forth, not to love the souls of men, but rather to \bibemph{love men.} You are not merely to proclaim the joys of heaven but also to exhibit in your daily experience these spirit realities of the divine life since you already have eternal life, as the gift of God, through faith. When you have faith, when power from on high, the Spirit of Truth, has come upon you, you will not hide your light here behind closed doors; you will make known the love and the mercy of God to all mankind. Through fear you now flee from the facts of a disagreeable experience, but when you shall have been baptized with the Spirit of Truth, you will bravely and joyously go forth to meet the new experiences of proclaiming the good news of eternal life in the kingdom of God. You may tarry here and in Galilee for a short season while you recover from the shock of the transition from the false security of the authority of traditionalism to the new order of the authority of facts, truth, and faith in the supreme realities of living experience. Your mission to the world is founded on the fact that I lived a God\hyp{}revealing life among you; on the truth that you and all other men are the sons of God; and it shall consist in the life which you will live among men --- the actual and living experience of loving men and serving them, even as I have loved and served you. Let faith reveal your light to the world; let the revelation of truth open the eyes blinded by tradition; let your loving service effectually destroy the prejudice engendered by ignorance. By so drawing close to your fellow men in understanding sympathy and with unselfish devotion, you will lead them into a saving knowledge of the Father’s love. The Jews have extolled goodness; the Greeks have exalted beauty; the Hindus preach devotion; the faraway ascetics teach reverence; the Romans demand loyalty; but I require of my disciples life, even a life of loving service for your brothers in the flesh.”
\vs p191 5:4 When the Master had so spoken, he looked down into the face of Thomas and said: \textcolour{ubdarkred}{“And you, Thomas, who said you would not believe unless you could see me and put your finger in the nail marks of my hands, have now beheld me and heard my words; and though you see no nail marks on my hands, since I am raised in the form that you also shall have when you depart from this world, what will you say to your brethren? You will acknowledge the truth, for already in your heart you had begun to believe even when you so stoutly asserted your unbelief. Your doubts, Thomas, always most stubbornly assert themselves just as they are about to crumble. Thomas, I bid you be not faithless but believing --- and I know you will believe, even with a whole heart.”}
\vs p191 5:5 When Thomas heard these words, he fell on his knees before the morontia Master and exclaimed, “I believe! My Lord and my Master!” Then said Jesus to Thomas: \textcolour{ubdarkred}{“You have believed, Thomas, because you have really seen and heard me. Blessed are those in the ages to come who will believe even though they have not seen with the eye of flesh nor heard with the mortal ear.”}
\vs p191 5:6 And then, as the Master’s form moved over near the head of the table, he addressed them all, saying: \textcolour{ubdarkred}{“And now go all of you to Galilee, where I will presently appear to you.”} After he said this, he vanished from their sight.
\vs p191 5:7 \pc The 11 apostles were now fully convinced that Jesus had risen from the dead, and very early the next morning, before the break of day, they started out for Galilee.
\usection{The Alexandrian Appearance}
\vs p191 6:1 While the 11 apostles were on the way to Galilee, drawing near their journey’s end, on Tuesday evening, April 18, at about 20:30, Jesus appeared to Rodan and some 80 other believers, in Alexandria. This was the Master’s 12\ts{th} appearance in morontia form. Jesus appeared before these Greeks and Jews at the conclusion of the report of David’s messenger regarding the crucifixion. This messenger, being the 5\ts{th} in the Jerusalem\hyp{}Alexandria relay of runners, had arrived in Alexandria late that afternoon, and when he had delivered his message to Rodan, it was decided to call the believers together to receive this tragic word from the messenger himself. At about 20:00, the messenger, Nathan of Busiris, came before this group and told them in detail all that had been told him by the preceding runner. Nathan ended his touching recital with these words: “But David, who sends us this word, reports that the Master, in foretelling his death, declared that he would rise again.” Even as Nathan spoke, the morontia Master appeared there in full view of all. And when Nathan sat down, Jesus said:
\vs p191 6:2 \textcolour{ubdarkred}{“Peace be upon you. That which my Father sent me into the world to establish belongs not to a race, a nation, nor to a special group of teachers or preachers. This gospel of the kingdom belongs to both Jew and gentile, to rich and poor, to free and bond, to male and female, even to the little children. And you are all to proclaim this gospel of love and truth by the lives which you live in the flesh. You shall love one another with a new and startling affection, even as I have loved you. You will serve mankind with a new and amazing devotion, even as I have served you. And when men see you so love them, and when they behold how fervently you serve them, they will perceive that you have become faith\hyp{}fellows of the kingdom of heaven, and they will follow after the Spirit of Truth which they see in your lives, to the finding of eternal salvation.}
\vs p191 6:3 \textcolour{ubdarkred}{“As the Father sent me into this world, even so now send I you. You are all called to carry the good news to those who sit in darkness. This gospel of the kingdom belongs to all who believe it; it shall not be committed to the custody of mere priests. Soon will the Spirit of Truth come upon you, and he shall lead you into all truth. Go you, therefore, into all the world preaching this gospel, and lo, I am with you always, even to the end of the ages.”}
\vs p191 6:4 When the Master had so spoken, he vanished from their sight. All that night these believers remained there together recounting their experiences as kingdom believers and listening to the many words of Rodan and his associates. And they all believed that Jesus had risen from the dead. Imagine the surprise of David’s herald of the resurrection, who arrived the second day after this, when they replied to his announcement, saying: “Yes, we know, for we have seen him. He appeared to us day before yesterday.”
\quizlink
