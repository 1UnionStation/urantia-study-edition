\upaper{128}{Jesus’ Early Manhood}
\author{Midwayer Commission}
\vs p128 0:1 As Jesus of Nazareth entered upon the early years of his adult life, he had lived, and continued to live, a normal and average human life on earth. Jesus came into this world just as other children come; he had nothing to do with selecting his parents. He did choose this particular world as the planet whereon to carry out his seventh and final bestowal, his incarnation in the likeness of mortal flesh, but otherwise he entered the world in a natural manner, growing up as a child of the realm and wrestling with the vicissitudes of his environment just as do other mortals on this and on similar worlds.
\vs p128 0:2 Always be mindful of the twofold purpose of Michael’s bestowal on Urantia:
\vs p128 0:3 \ublistelem{1.}\bibnobreakspace The mastering of the experience of living the full life of a human creature in mortal flesh, the completion of his sovereignty in Nebadon.
\vs p128 0:4 \ublistelem{2.}\bibnobreakspace The revelation of the Universal Father to the mortal dwellers on the worlds of time and space and the more effective leading of these same mortals to a better understanding of the Universal Father.
\vs p128 0:5 \pc All other creature benefits and universe advantages were incidental and secondary to these major purposes of the mortal bestowal.
\usection{1.\bibnobreakspace The Twenty\hyp{}First Year (A.D.\,15)}
\vs p128 1:1 With the attainment of adult years Jesus began in earnest and with full self\hyp{}consciousness the task of completing the experience of mastering the knowledge of the life of his lowest form of intelligent creatures, thereby finally and fully earning the right of unqualified rulership of his self\hyp{}created universe. He entered upon this stupendous task fully realizing his dual nature. But he had already effectively combined these two natures into one --- Jesus of Nazareth.
\vs p128 1:2 Joshua ben Joseph knew full well that he was a man, a mortal man, born of woman. This is shown in the selection of his first title, the \bibemph{Son of Man.} He was truly a partaker of flesh and blood, and even now, as he presides in sovereign authority over the destinies of a universe, he still bears among his numerous well\hyp{}earned titles that of Son of Man. It is literally true that the creative Word --- the Creator Son --- of the Universal Father was “made flesh and dwelt as a man of the realm on Urantia.” He laboured, grew weary, rested, and slept. He hungered and satisfied such cravings with food; he thirsted and quenched his thirst with water. He experienced the full gamut of human feelings and emotions; he was “in all things tested, even as you are,” and he suffered and died.
\vs p128 1:3 He obtained knowledge, gained experience, and combined these into wisdom, just as do other mortals of the realm. Until after his baptism he availed himself of no supernatural power. He employed no agency not a part of his human endowment as a son of Joseph and Mary.
\vs p128 1:4 As to the attributes of his prehuman existence, he emptied himself. Prior to the beginning of his public work his knowledge of men and events was wholly self\hyp{}limited. He was a true man among men.
\vs p128 1:5 \pc It is forever and gloriously true: “We have a high ruler who can be touched with the feeling of our infirmities. We have a Sovereign who was in all points tested and tempted like as we are, yet without sin.” And since he himself has suffered, being tested and tried, he is abundantly able to understand and minister to those who are confused and distressed.
\vs p128 1:6 \pc The Nazareth carpenter now fully understood the work before him, but he chose to live his human life in the channel of its natural flowing. And in some of these matters he is indeed an example to his mortal creatures, even as it is recorded: “Let this mind be in you which was also in Christ Jesus, who, being of the nature of God, thought it not strange to be equal with God. But he made himself to be of little import and, taking upon himself the form of a creature, was born in the likeness of mankind. And being thus fashioned as a man, he humbled himself and became obedient to death, even the death of the cross.”
\vs p128 1:7 He lived his mortal life just as all others of the human family may live theirs, “who in the days of the flesh so frequently offered up prayers and supplications, even with strong feelings and tears, to Him who is able to save from all evil, and his prayers were effective because he believed.” Wherefore it behoved him \bibemph{in every respect} to be made like his brethren that he might become a merciful and understanding sovereign ruler over them.
\vs p128 1:8 Of his human nature he was never in doubt; it was self\hyp{}evident and always present in his consciousness. But of his divine nature there was always room for doubt and conjecture, at least this was true right up to the event of his baptism. The self\hyp{}realization of divinity was a slow and, from the human standpoint, a natural evolutionary revelation. This revelation and self\hyp{}realization of divinity began in Jerusalem when he was not quite 13 years old with the first supernatural occurrence of his human existence; and this experience of effecting the self\hyp{}realization of his divine nature was completed at the time of his second supernatural experience while in the flesh, the episode attendant upon his baptism by John in the Jordan, which event marked the beginning of his public career of ministry and teaching.
\vs p128 1:9 Between these two celestial visitations, one in his 13\ts{th} year and the other at his baptism, there occurred nothing supernatural or superhuman in the life of this incarnated Creator Son. Notwithstanding this, the babe of Bethlehem, the lad, youth, and man of Nazareth, was in reality the incarnated Creator of a universe; but he never once used aught of this power, nor did he utilize the guidance of celestial personalities, aside from that of his guardian seraphim, in the living of his human life up to the day of his baptism by John. And we who thus testify know whereof we speak.
\vs p128 1:10 And yet, throughout all these years of his life in the flesh he was truly divine. He was actually a Creator Son of the Paradise Father. When once he had espoused his public career, subsequent to the technical completion of his purely mortal experience of sovereignty acquirement, he did not hesitate publicly to admit that he was the Son of God. He did not hesitate to declare, \textcolour{ubdarkred}{“I am Alpha and Omega, the beginning and the end, the first and the last.”} He made no protest in later years when he was called Lord of Glory, Ruler of a Universe, the Lord God of all creation, the Holy One of Israel, the Lord of all, our Lord and our God, God with us, having a name above every name and on all worlds, the Omnipotence of a universe, the Universe Mind of this creation, the One in whom are hid all treasures of wisdom and knowledge, the fullness of Him who fills all things, the eternal Word of the eternal God, the One who was before all things and in whom all things consist, the Creator of the heavens and the earth, the Upholder of a universe, the Judge of all the earth, the Giver of life eternal, the True Shepherd, the Deliverer of the worlds, and the Captain of our salvation.
\vs p128 1:11 \pc He never objected to any of these titles as they were applied to him subsequent to the emergence from his purely human life into the later years of his self\hyp{}consciousness of the ministry of divinity in humanity, and for humanity, and to humanity on this world and for all other worlds. Jesus objected to but one title as applied to him: When he was once called Immanuel, he merely replied, \textcolour{ubdarkred}{“Not I, that is my elder brother.”}
\vs p128 1:12 Always, even after his emergence into the larger life on earth, Jesus was submissively subject to the will of the Father in heaven.
\vs p128 1:13 After his baptism he thought nothing of permitting his sincere believers and grateful followers to worship him. Even while he wrestled with poverty and toiled with his hands to provide the necessities of life for his family, his awareness that he was a Son of God was growing; he knew that he was the maker of the heavens and this very earth whereon he was now living out his human existence. And the hosts of celestial beings throughout the great and onlooking universe likewise knew that this man of Nazareth was their beloved Sovereign and Creator\hyp{}father. A profound suspense pervaded the universe of Nebadon throughout these years; all celestial eyes were continuously focused on Urantia --- on Palestine.
\vs p128 1:14 \pc This year Jesus went up to Jerusalem with Joseph to celebrate the Passover. Having taken James to the temple for consecration, he deemed it his duty to take Joseph. Jesus never exhibited any degree of partiality in dealing with his family. He went with Joseph to Jerusalem by the usual Jordan valley route, but he returned to Nazareth by the east Jordan way, which led through Amathus. Going down the Jordan, Jesus narrated Jewish history to Joseph and on the return trip told him about the experiences of the reputed tribes of Ruben, Gad, and Gilead that traditionally had dwelt in these regions east of the river.
\vs p128 1:15 Joseph asked Jesus many leading questions concerning his life mission, but to most of these inquiries Jesus would only reply, \textcolour{ubdarkred}{“My hour has not yet come.”} However, in these intimate discussions many words were dropped which Joseph remembered during the stirring events of subsequent years. Jesus, with Joseph, spent this Passover with his three friends at Bethany, as was his custom when in Jerusalem attending these festival commemorations.
\usection{2.\bibnobreakspace The Twenty\hyp{}Second Year (A.D.\,16)}
\vs p128 2:1 This was one of several years during which Jesus’ brothers and sisters were facing the trials and tribulations peculiar to the problems and readjustments of adolescence. Jesus now had brothers and sisters ranging in ages from 7 to 18, and he was kept busy helping them to adjust themselves to the new awakenings of their intellectual and emotional lives. He had thus to grapple with the problems of adolescence as they became manifest in the lives of his younger brothers and sisters.
\vs p128 2:2 This year Simon graduated from school and began work with Jesus’ old boyhood playmate and ever\hyp{}ready defender, Jacob the stone mason. As a result of several family conferences it was decided that it was unwise for all the boys to take up carpentry. It was thought that by diversifying their trades they would be prepared to take contracts for putting up entire buildings. Again, they had not all kept busy since three of them had been working as full\hyp{}time carpenters.
\vs p128 2:3 Jesus continued this year at house finishing and cabinetwork but spent most of his time at the caravan repair shop. James was beginning to alternate with him in attendance at the shop. The latter part of this year, when carpenter work was slack about Nazareth, Jesus left James in charge of the repair shop and Joseph at the home bench while he went over to Sepphoris to work with a smith. He worked six months with metals and acquired considerable skill at the anvil.
\vs p128 2:4 \pc Before taking up his new employment at Sepphoris, Jesus held one of his periodic family conferences and solemnly installed James, then just past 18 years old, as acting head of the family. He promised his brother hearty support and full co\hyp{}operation and exacted formal promises of obedience to James from each member of the family. From this day James assumed full financial responsibility for the family, Jesus making his weekly payments to his brother. Never again did Jesus take the reins out of James’s hands. While working at Sepphoris he could have walked home every night if necessary, but he purposely remained away, assigning weather and other reasons, but his true motive was to train James and Joseph in the bearing of the family responsibility. He had begun the slow process of weaning his family. Each Sabbath Jesus returned to Nazareth, and sometimes during the week when occasion required, to observe the working of the new plan, to give advice and offer helpful suggestions.
\vs p128 2:5 \pc Living much of the time in Sepphoris for six months afforded Jesus a new opportunity to become better acquainted with the gentile viewpoint of life. He worked with gentiles, lived with gentiles, and in every possible manner did he make a close and painstaking study of their habits of living and of the gentile mind.
\vs p128 2:6 The moral standards of this home city of Herod Antipas were so far below those of even the caravan city of Nazareth that after six months’ sojourn at Sepphoris Jesus was not averse to finding an excuse for returning to Nazareth. The group he worked for were to become engaged on public work in both Sepphoris and the new city of Tiberias, and Jesus was disinclined to have anything to do with any sort of employment under the supervision of Herod Antipas. And there were still other reasons which made it wise, in the opinion of Jesus, for him to go back to Nazareth. When he returned to the repair shop, he did not again assume the personal direction of family affairs. He worked in association with James at the shop and as far as possible permitted him to continue oversight of the home. James’s management of family expenditures and his administration of the home budget were undisturbed.
\vs p128 2:7 It was by just such wise and thoughtful planning that Jesus prepared the way for his eventual withdrawal from active participation in the affairs of his family. When James had had two years’ experience as acting head of the family --- and two full years before he (James) was to be married --- Joseph was placed in charge of the household funds and entrusted with the general management of the home.
\usection{3.\bibnobreakspace The Twenty\hyp{}Third Year (A.D.\,17)}
\vs p128 3:1 This year the financial pressure was slightly relaxed as four were at work. Miriam earned considerable by the sale of milk and butter; Martha had become an expert weaver. The purchase price of the repair shop was over \bibfrac{1}{3}\ts{rd} paid. The situation was such that Jesus stopped work for three weeks to take Simon to Jerusalem for the Passover, and this was the longest period away from daily toil he had enjoyed since the death of his father.
\vs p128 3:2 They journeyed to Jerusalem by way of the Decapolis and through Pella, Gerasa, Philadelphia, Heshbon, and Jericho. They returned to Nazareth by the coast route, touching Lydda, Joppa, Caesarea, thence around Mount Carmel to Ptolemais and Nazareth. This trip fairly well acquainted Jesus with the whole of Palestine north of the Jerusalem district.
\vs p128 3:3 At Philadelphia Jesus and Simon became acquainted with a merchant from Damascus who developed such a great liking for the Nazareth couple that he insisted they stop with him at his Jerusalem headquarters. While Simon gave attendance at the temple, Jesus spent much of his time talking with this well\hyp{}educated and much\hyp{}travelled man of world affairs. This merchant owned over 4,000 caravan camels; he had interests all over the Roman world and was now on his way to Rome. He proposed that Jesus come to Damascus to enter his Oriental import business, but Jesus explained that he did not feel justified in going so far away from his family just then. But on the way back home he thought much about these distant cities and the even more remote countries of the Far West and the Far East, countries he had so frequently heard spoken of by the caravan passengers and conductors.
\vs p128 3:4 Simon greatly enjoyed his visit to Jerusalem. He was duly received into the commonwealth of Israel at the Passover consecration of the new sons of the commandment. While Simon attended the Passover ceremonies, Jesus mingled with the throngs of visitors and engaged in many interesting personal conferences with numerous gentile proselytes.
\vs p128 3:5 Perhaps the most notable of all these contacts was the one with a young Hellenist named Stephen. This young man was on his first visit to Jerusalem and chanced to meet Jesus on Thursday afternoon of Passover week. While they both strolled about viewing the Asmonean palace, Jesus began the casual conversation that resulted in their becoming interested in each other, and which led to a four\hyp{}hour discussion of the way of life and the true God and his worship. Stephen was tremendously impressed with what Jesus said; he never forgot his words.
\vs p128 3:6 And this was the same Stephen who subsequently became a believer in the teachings of Jesus, and whose boldness in preaching this early gospel resulted in his being stoned to death by irate Jews. Some of Stephen’s extraordinary boldness in proclaiming his view of the new gospel was the direct result of this earlier interview with Jesus. But Stephen never even faintly surmised that the Galilean he had talked with some 15 years previously was the very same person whom he later proclaimed the world’s Saviour, and for whom he was so soon to die, thus becoming the first martyr of the newly evolving Christian faith. When Stephen yielded up his life as the price of his attack upon the Jewish temple and its traditional practices, there stood by one named Saul, a citizen of Tarsus. And when Saul saw how this Greek could die for his faith, there were aroused in his heart those emotions which eventually led him to espouse the cause for which Stephen died; later on he became the aggressive and indomitable Paul, the philosopher, if not the sole founder, of the Christian religion.
\vs p128 3:7 \pc On the Sunday after Passover week Simon and Jesus started on their way back to Nazareth. Simon never forgot what Jesus taught him on this trip. He had always loved Jesus, but now he felt that he had begun to know his father\hyp{}brother. They had many heart\hyp{}to\hyp{}heart talks as they journeyed through the country and prepared their meals by the wayside. They arrived home Thursday noon, and Simon kept the family up late that night relating his experiences.
\vs p128 3:8 Mary was much upset by Simon’s report that Jesus spent most of the time when in Jerusalem “visiting with the strangers, especially those from the far countries.” Jesus’ family never could comprehend his great interest in people, his urge to visit with them, to learn about their way of living, and to find out what they were thinking about.
\vs p128 3:9 \pc More and more the Nazareth family became engrossed with their immediate and human problems; not often was mention made of the future mission of Jesus, and very seldom did he himself speak of his future career. His mother rarely thought about his being a child of promise. She was slowly giving up the idea that Jesus was to fulfil any divine mission on earth, yet at times her faith was revived when she paused to recall the Gabriel visitation before the child was born.
\usection{4.\bibnobreakspace The Damascus Episode}
\vs p128 4:1 The last four months of this year Jesus spent in Damascus as the guest of the merchant whom he first met at Philadelphia when on his way to Jerusalem. A representative of this merchant had sought out Jesus when passing through Nazareth and escorted him to Damascus. This part\hyp{}Jewish merchant proposed to devote an extraordinary sum of money to the establishment of a school of religious philosophy at Damascus. He planned to create a centre of learning which would out\hyp{}rival Alexandria. And he proposed that Jesus should immediately begin a long tour of the world’s educational centres preparatory to becoming the head of this new project. This was one of the greatest temptations that Jesus ever faced in the course of his purely human career.
\vs p128 4:2 Presently this merchant brought before Jesus a group of 12 merchants and bankers who agreed to support this newly projected school. Jesus manifested deep interest in the proposed school, helped them plan for its organization, but always expressed the fear that his other and unstated but prior obligations would prevent his accepting the direction of such a pretentious enterprise. His would\hyp{}be benefactor was persistent, and he profitably employed Jesus at his home doing some translating while he, his wife, and their sons and daughters sought to prevail upon Jesus to accept the proffered honour. But he would not consent. He well knew that his mission on earth was not to be supported by institutions of learning; he knew that he must not obligate himself in the least to be directed by the “councils of men,” no matter how well\hyp{}intentioned.
\vs p128 4:3 He who was rejected by the Jerusalem religious leaders, even after he had demonstrated his leadership, was recognized and hailed as a master teacher by the businessmen and bankers of Damascus, and all this when he was an obscure and unknown carpenter of Nazareth.
\vs p128 4:4 He never spoke about this offer to his family, and the end of this year found him back in Nazareth going about his daily duties just as if he had never been tempted by the flattering propositions of his Damascus friends. Neither did these men of Damascus ever associate the later citizen of Capernaum who turned all Jewry upside down with the former carpenter of Nazareth who had dared to refuse the honour which their combined wealth might have procured.
\vs p128 4:5 \pc Jesus most cleverly and intentionally contrived to detach various episodes of his life so that they never became, in the eyes of the world, associated together as the doings of a single individual. Many times in subsequent years he listened to the recital of this very story of the strange Galilean who declined the opportunity of founding a school in Damascus to compete with Alexandria.
\vs p128 4:6 One purpose which Jesus had in mind, when he sought to segregate certain features of his earthly experience, was to prevent the building up of such a versatile and spectacular career as would cause subsequent generations to venerate the teacher in place of obeying the truth which he had lived and taught. Jesus did not want to build up such a human record of achievement as would attract attention from his teaching. Very early he recognized that his followers would be tempted to formulate a religion \bibemph{about} him which might become a competitor of the gospel of the kingdom that he intended to proclaim to the world. Accordingly, he consistently sought to suppress everything during his eventful career which he thought might be made to serve this natural human tendency to exalt the teacher in place of proclaiming his teachings.
\vs p128 4:7 This same motive also explains why he permitted himself to be known by different titles during various epochs of his diversified life on earth. Again, he did not want to bring any undue influence to bear upon his family or others which would lead them to believe in him against their honest convictions. He always refused to take undue or unfair advantage of the human mind. He did not want men to believe in him unless their hearts were responsive to the spiritual realities revealed in his teachings.
\vs p128 4:8 \pc By the end of this year the Nazareth home was running fairly smoothly. The children were growing up, and Mary was becoming accustomed to Jesus’ being away from home. He continued to turn over his earnings to James for the support of the family, retaining only a small portion for his immediate personal expenses.
\vs p128 4:9 As the years passed, it became more difficult to realize that this man was a Son of God on earth. He seemed to become quite like an individual of the realm, just another man among men. And it was ordained by the Father in heaven that the bestowal should unfold in this very way.
\usection{5.\bibnobreakspace The Twenty\hyp{}Fourth Year (A.D.\,18)}
\vs p128 5:1 This was Jesus’ first year of comparative freedom from family responsibility. James was very successful in managing the home with Jesus’ help in counsel and finances.
\vs p128 5:2 \pc The week following the Passover of this year a young man from Alexandria came down to Nazareth to arrange for a meeting, later in the year, between Jesus and a group of Alexandrian Jews at some point on the Palestinian coast. This conference was set for the middle of June, and Jesus went over to Caesarea to meet with five prominent Jews of Alexandria, who besought him to establish himself in their city as a religious teacher, offering as an inducement to begin with, the position of assistant to the chazan in their chief synagogue.
\vs p128 5:3 The spokesmen for this committee explained to Jesus that Alexandria was destined to become the headquarters of Jewish culture for the entire world; that the Hellenistic trend of Jewish affairs had virtually outdistanced the Babylonian school of thought. They reminded Jesus of the ominous rumblings of rebellion in Jerusalem and throughout Palestine and assured him that any uprising of the Palestinian Jews would be equivalent to national suicide, that the iron hand of Rome would crush the rebellion in three months, and that Jerusalem would be destroyed and the temple demolished, that not one stone would be left upon another.
\vs p128 5:4 Jesus listened to all they had to say, thanked them for their confidence, and, in declining to go to Alexandria, in substance said, \textcolour{ubdarkred}{“My hour has not yet come.”} They were nonplussed by his apparent indifference to the honour they had sought to confer upon him. Before taking leave of Jesus, they presented him with a purse in token of the esteem of his Alexandrian friends and in compensation for the time and expense of coming over to Caesarea to confer with them. But he likewise refused the money, saying: \textcolour{ubdarkred}{“The house of Joseph has never received alms, and we cannot eat another’s bread as long as I have strong arms and my brothers can labour.”}
\vs p128 5:5 His friends from Egypt set sail for home, and in subsequent years, when they heard rumours of the Capernaum boatbuilder who was creating such a commotion in Palestine, few of them surmised that he was the babe of Bethlehem grown up and the same strange\hyp{}acting Galilean who had so unceremoniously declined the invitation to become a great teacher in Alexandria.
\vs p128 5:6 \pc Jesus returned to Nazareth. The remainder of this year was the most uneventful six months of his whole career. He enjoyed this temporary respite from the usual program of problems to solve and difficulties to surmount. He communed much with his Father in heaven and made tremendous progress in the mastery of his human mind.
\vs p128 5:7 But human affairs on the worlds of time and space do not run smoothly for long. In December James had a private talk with Jesus, explaining that he was much in love with Esta, a young woman of Nazareth, and that they would sometime like to be married if it could be arranged. He called attention to the fact that Joseph would soon be 18 years old, and that it would be a good experience for him to have a chance to serve as the acting head of the family. Jesus gave consent for James’s marriage two years later, provided he had, during the intervening time, properly trained Joseph to assume direction of the home.
\vs p128 5:8 And now things began to happen --- marriage was in the air. James’s success in gaining Jesus’ assent to his marriage emboldened Miriam to approach her brother\hyp{}father with her plans. Jacob, the younger stone mason, onetime self\hyp{}appointed champion of Jesus, now business associate of James and Joseph, had long sought to gain Miriam’s hand in marriage. After Miriam had laid her plans before Jesus, he directed that Jacob should come to him making formal request for her and promised his blessing for the marriage just as soon as she felt that Martha was competent to assume her duties as eldest daughter.
\vs p128 5:9 \pc When at home, he continued to teach the evening school three times a week, read the Scriptures often in the synagogue on the Sabbath, visited with his mother, taught the children, and in general conducted himself as a worthy and respected citizen of Nazareth in the commonwealth of Israel.
\usection{6.\bibnobreakspace The Twenty\hyp{}Fifth Year (A.D.\,19)}
\vs p128 6:1 This year began with the Nazareth family all in good health and witnessed the finishing of the regular schooling of all the children with the exception of certain work which Martha must do for Ruth.
\vs p128 6:2 \pc Jesus was one of the most robust and refined specimens of manhood to appear on earth since the days of Adam. His physical development was superb. His mind was active, keen, and penetrating --- compared with the average mentality of his contemporaries, it had developed gigantic proportions --- and his spirit was indeed humanly divine.
\vs p128 6:3 \pc The family finances were in the best condition since the disappearance of Joseph’s estate. The final payments had been made on the caravan repair shop; they owed no man and for the first time in years had some funds ahead. This being true, and since he had taken his other brothers to Jerusalem for their first Passover ceremonies, Jesus decided to accompany Jude (who had just graduated from the synagogue school) on his first visit to the temple.
\vs p128 6:4 They went up to Jerusalem and returned by the same route, the Jordan valley, as Jesus feared trouble if he took his young brother through Samaria. Already at Nazareth Jude had got into slight trouble several times because of his hasty disposition, coupled with his strong patriotic sentiments.
\vs p128 6:5 They arrived at Jerusalem in due time and were on their way for a first visit to the temple, the very sight of which had stirred and thrilled Jude to the very depths of his soul, when they chanced to meet Lazarus of Bethany. While Jesus talked with Lazarus and sought to arrange for their joint celebration of the Passover, Jude started up real trouble for them all. Close at hand stood a Roman guard who made some improper remarks regarding a Jewish girl who was passing. Jude flushed with fiery indignation and was not slow in expressing his resentment of such an impropriety directly to and within hearing of the soldier. Now the Roman legionnaires were very sensitive to anything bordering on Jewish disrespect; so the guard promptly placed Jude under arrest. This was too much for the young patriot, and before Jesus could caution him by a warning glance, he had delivered himself of a voluble denunciation of pent\hyp{}up anti\hyp{}Roman feelings, all of which only made a bad matter worse. Jude, with Jesus by his side, was taken at once to the military prison.
\vs p128 6:6 Jesus endeavoured to obtain either an immediate hearing for Jude or else his release in time for the Passover celebration that evening, but he failed in these attempts. Since the next day was a “holy convocation” in Jerusalem, even the Romans would not presume to hear charges against a Jew. Accordingly, Jude remained in confinement until the morning of the second day after his arrest, and Jesus stayed at the prison with him. They were not present in the temple at the ceremony of receiving the sons of the law into the full citizenship of Israel. Jude did not pass through this formal ceremony for several years, until he was next in Jerusalem at a Passover and in connection with his propaganda work in behalf of the Zealots, the patriotic organization to which he belonged and in which he was very active.
\vs p128 6:7 The morning following their second day in prison Jesus appeared before the military magistrate in behalf of Jude. By making apologies for his brother’s youth and by a further explanatory but judicious statement with reference to the provocative nature of the episode which had led up to the arrest of his brother, Jesus so handled the case that the magistrate expressed the opinion that the young Jew might have had some possible excuse for his violent outburst. After warning Jude not to allow himself again to be guilty of such rashness, he said to Jesus in dismissing them: “You had better keep your eye on the lad; he’s liable to make a lot of trouble for all of you.” And the Roman judge spoke the truth. Jude did make considerable trouble for Jesus, and always was the trouble of this same nature --- clashes with the civil authorities because of his thoughtless and unwise patriotic outbursts.
\vs p128 6:8 Jesus and Jude walked over to Bethany for the night, explaining why they had failed to keep their appointment for the Passover supper, and set out for Nazareth the following day. Jesus did not tell the family about his young brother’s arrest at Jerusalem, but he had a long talk with Jude about this episode some three weeks after their return. After this talk with Jesus Jude himself told the family. He never forgot the patience and forbearance his brother\hyp{}father manifested throughout the whole of this trying experience.
\vs p128 6:9 This was the last Passover Jesus attended with any member of his own family. Increasingly the Son of Man was to become separated from close association with his own flesh and blood.
\vs p128 6:10 \pc This year his seasons of deep meditation were often broken into by Ruth and her playmates. And always was Jesus ready to postpone the contemplation of his future work for the world and the universe that he might share in the childish joy and youthful gladness of these youngsters, who never tired of listening to Jesus relate the experiences of his various trips to Jerusalem. They also greatly enjoyed his stories about animals and nature.
\vs p128 6:11 The children were always welcome at the repair shop. Jesus provided sand, blocks, and stones by the side of the shop, and bevies of youngsters flocked there to amuse themselves. When they tired of their play, the more intrepid ones would peek into the shop, and if its keeper were not busy, they would make bold to go in and say, “Uncle Joshua, come out and tell us a big story.” Then they would lead him out by tugging at his hands until he was seated on the favourite rock by the corner of the shop, with the children on the ground in a semicircle before him. And how the little folks did enjoy their Uncle Joshua. They were learning to laugh, and to laugh heartily. It was customary for one or two of the smallest of the children to climb upon his knees and sit there, looking up in wonderment at his expressive features as he told his stories. The children loved Jesus, and Jesus loved the children.
\vs p128 6:12 It was difficult for his friends to comprehend the range of his intellectual activities, how he could so suddenly and so completely swing from the profound discussion of politics, philosophy, or religion to the light\hyp{}hearted and joyous playfulness of these tots of from 5 to 10 years of age. As his own brothers and sisters grew up, as he gained more leisure, and before the grandchildren arrived, he paid a great deal of attention to these little ones. But he did not live on earth long enough to enjoy the grandchildren very much.
\usection{7.\bibnobreakspace The Twenty\hyp{}Sixth Year (A.D.\,20)}
\vs p128 7:1 As this year began, Jesus of Nazareth became strongly conscious that he possessed a wide range of potential power. But he was likewise fully persuaded that this power was not to be employed by his personality as the Son of Man, at least not until his hour should come.
\vs p128 7:2 At this time he thought much but said little about the relation of himself to his Father in heaven. And the conclusion of all this thinking was expressed once in his prayer on the hilltop, when he said: \textcolour{ubdarkred}{“Regardless of who I am and what power I may or may not wield, I always have been, and always will be, subject to the will of my Paradise Father.”} And yet, as this man walked about Nazareth to and from his work, it was literally true --- as concerned a vast universe --- that “in him were hidden all the treasures of wisdom and knowledge.”
\vs p128 7:3 \pc All this year the family affairs ran smoothly except for Jude. For years James had trouble with his youngest brother, who was not inclined to settle down to work nor was he to be depended upon for his share of the home expenses. While he would live at home, he was not conscientious about earning his share of the family upkeep.
\vs p128 7:4 Jesus was a man of peace, and ever and anon was he embarrassed by Jude’s belligerent exploits and numerous patriotic outbursts. James and Joseph were in favour of casting him out, but Jesus would not consent. When their patience would be severely tried, Jesus would only counsel: \textcolour{ubdarkred}{“Be patient. Be wise in your counsel and eloquent in your lives, that your young brother may first know the better way and then be constrained to follow you in it.”} The wise and loving counsel of Jesus prevented a break in the family; they remained together. But Jude never was brought to his sober senses until after his marriage.
\vs p128 7:5 Mary seldom spoke of Jesus’ future mission. Whenever this subject was referred to, Jesus only replied, \textcolour{ubdarkred}{“My hour has not yet come.”} Jesus had about completed the difficult task of weaning his family from dependence on the immediate presence of his personality. He was rapidly preparing for the day when he could consistently leave this Nazareth home to begin the more active prelude to his real ministry for men.
\vs p128 7:6 Never lose sight of the fact that the prime mission of Jesus in his seventh bestowal was the acquirement of creature experience, the achievement of the sovereignty of Nebadon. And in the gathering of this very experience he made the supreme revelation of the Paradise Father to Urantia and to his entire local universe. Incidental to these purposes he also undertook to untangle the complicated affairs of this planet as they were related to the Lucifer rebellion.
\vs p128 7:7 \pc This year Jesus enjoyed more than usual leisure, and he devoted much time to training James in the management of the repair shop and Joseph in the direction of home affairs. Mary sensed that he was making ready to leave them. Leave them to go where? To do what? She had about given up the thought that Jesus was the Messiah. She could not understand him; she simply could not fathom her first\hyp{}born son.
\vs p128 7:8 Jesus spent a great deal of time this year with the individual members of his family. He would take them for long and frequent strolls up the hill and through the countryside. Before harvest he took Jude to the farmer uncle south of Nazareth, but Jude did not remain long after the harvest. He ran away, and Simon later found him with the fishermen at the lake. When Simon brought him back home, Jesus talked things over with the runaway lad and, since he wanted to be a fisherman, went over to Magdala with him and put him in the care of a relative, a fisherman; and Jude worked fairly well and regularly from that time on until his marriage, and he continued as a fisherman after his marriage.
\vs p128 7:9 \pc At last the day had come when all Jesus’ brothers had chosen, and were established in, their lifework. The stage was being set for Jesus’ departure from home.
\vs p128 7:10 \pc In November a double wedding occurred. James and Esta, and Miriam and Jacob were married. It was truly a joyous occasion. Even Mary was once more happy except every now and then when she realized that Jesus was preparing to go away. She suffered under the burden of a great uncertainty: If Jesus would only sit down and talk it all over freely with her as he had done when he was a boy, but he was consistently uncommunicative; he was profoundly silent about the future.
\vs p128 7:11 James and his bride, Esta, moved into a neat little home on the west side of town, the gift of her father. While James continued his support of his mother’s home, his quota was cut in half because of his marriage, and Joseph was formally installed by Jesus as head of the family. Jude was now very faithfully sending his share of funds home each month. The weddings of James and Miriam had a very beneficial influence on Jude, and when he left for the fishing grounds, the day after the double wedding, he assured Joseph that he could depend on him “to do my full duty, and more if it is needed.” And he kept his promise.
\vs p128 7:12 Miriam lived next door to Mary in the home of Jacob, Jacob the elder having been laid to rest with his fathers. Martha took Miriam’s place in the home, and the new organization was working smoothly before the year ended.
\vs p128 7:13 \pc The day after this double wedding Jesus held an important conference with James. He told James, confidentially, that he was preparing to leave home. He presented full title to the repair shop to James, formally and solemnly abdicated as head of Joseph’s house, and most touchingly established his brother James as “head and protector of my father’s house.” He drew up, and they both signed, a secret compact in which it was stipulated that, in return for the gift of the repair shop, James would henceforth assume full financial responsibility for the family, thus releasing Jesus from all further obligations in these matters. After the contract was signed, after the budget was so arranged that the actual expenses of the family would be met without any contribution from Jesus, Jesus said to James: \textcolour{ubdarkred}{“But, my son, I will continue to send you something each month until my hour shall have come, but what I send shall be used by you as the occasion demands. Apply my funds to the family necessities or pleasures as you see fit. Use them in case of sickness or apply them to meet the unexpected emergencies which may befall any individual member of the family.”}
\vs p128 7:14 And thus did Jesus make ready to enter upon the second and home\hyp{}detached phase of his adult life before the public entrance upon his Father’s business.
\quizlink
