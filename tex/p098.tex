\upaper{98}{The Melchizedek Teachings in the Occident}
\author{Melchizedek}
\vs p098 0:1 The Melchizedek teachings entered Europe along many routes, but chiefly they came by way of Egypt and were embodied in Occidental philosophy after being thoroughly Hellenized and later Christianized. The ideals of the Western world were basically Socratic, and its later religious philosophy became that of Jesus as it was modified and compromised through contact with evolving Occidental philosophy and religion, all of which culminated in the Christian church.
\vs p098 0:2 For a long time in Europe the Salem missionaries carried on their activities, becoming gradually absorbed into many of the cults and ritual groups which periodically arose. Among those who maintained the Salem teachings in the purest form must be mentioned the Cynics. These preachers of faith and trust in God were still functioning in Roman Europe in the first century after Christ, being later incorporated into the newly forming Christian religion.
\vs p098 0:3 Much of the Salem doctrine was spread in Europe by the Jewish mercenary soldiers who fought in so many of the Occidental military struggles. In ancient times the Jews were famed as much for military valour as for theologic peculiarities.
\vs p098 0:4 The basic doctrines of Greek philosophy, Jewish theology, and Christian ethics were fundamentally repercussions of the earlier Melchizedek teachings.
\usection{1.\bibnobreakspace The Salem Religion among the Greeks}
\vs p098 1:1 The Salem missionaries might have built up a great religious structure among the Greeks had it not been for their strict interpretation of their oath of ordination, a pledge imposed by Machiventa which forbade the organization of exclusive congregations for worship, and which exacted the promise of each teacher never to function as a priest, never to receive fees for religious service, only food, clothing, and shelter. When the Melchizedek teachers penetrated to pre\hyp{}Hellenic Greece, they found a people who still fostered the traditions of Adamson and the days of the Andites, but these teachings had become greatly adulterated with the notions and beliefs of the hordes of inferior slaves that had been brought to the Greek shores in increasing numbers. This adulteration produced a reversion to a crude animism with bloody rites, the lower classes even making ceremonial out of the execution of condemned criminals.
\vs p098 1:2 The early influence of the Salem teachers was nearly destroyed by the so\hyp{}called Aryan invasion from southern Europe and the East. These Hellenic invaders brought along with them anthropomorphic God concepts similar to those which their Aryan fellows had carried to India. This importation inaugurated the evolution of the Greek family of gods and goddesses. This new religion was partly based on the cults of the incoming Hellenic barbarians, but it also shared in the myths of the older inhabitants of Greece.
\vs p098 1:3 The Hellenic Greeks found the Mediterranean world largely dominated by the mother cult, and they imposed upon these peoples their man\hyp{}god, Dyaus\hyp{}Zeus, who had already become, like Yahweh among the henotheistic Semites, head of the whole Greek pantheon of subordinate gods. And the Greeks would have eventually achieved a true monotheism in the concept of Zeus except for their retention of the overcontrol of Fate. A God of final value must, himself, be the arbiter of fate and the creator of destiny.
\vs p098 1:4 As a consequence of these factors in religious evolution, there presently developed the popular belief in the happy\hyp{}go\hyp{}lucky gods of Mount Olympus, gods more human than divine, and gods which the intelligent Greeks never did regard very seriously. They neither greatly loved nor greatly feared these divinities of their own creation. They had a patriotic and racial feeling for Zeus and his family of half men and half gods, but they hardly reverenced or worshipped them.
\vs p098 1:5 The Hellenes became so impregnated with the antipriestcraft doctrines of the earlier Salem teachers that no priesthood of any importance ever arose in Greece. Even the making of images to the gods became more of a work in art than a matter of worship.
\vs p098 1:6 The Olympian gods illustrate man’s typical anthropomorphism. But the Greek mythology was more aesthetic than ethic. The Greek religion was helpful in that it portrayed a universe governed by a deity group. But Greek morals, ethics, and philosophy presently advanced far beyond the god concept, and this imbalance between intellectual and spiritual growth was as hazardous to Greece as it had proved to be in India.
\usection{2.\bibnobreakspace Greek Philosophic Thought}
\vs p098 2:1 A lightly regarded and superficial religion cannot endure, especially when it has no priesthood to foster its forms and to fill the hearts of the devotees with fear and awe. The Olympian religion did not promise salvation, nor did it quench the spiritual thirst of its believers; therefore was it doomed to perish. Within a millennium of its inception it had nearly vanished, and the Greeks were without a national religion, the gods of Olympus having lost their hold upon the better minds.
\vs p098 2:2 This was the situation when, during the VI century before Christ, the Orient and the Levant experienced a revival of spiritual consciousness and a new awakening to the recognition of monotheism. But the West did not share in this new development; neither Europe nor northern Africa extensively participated in this religious renaissance. The Greeks, however, did engage in a magnificent intellectual advancement. They had begun to master fear and no longer sought religion as an antidote therefor, but they did not perceive that true religion is the cure for soul hunger, spiritual disquiet, and moral despair. They sought for the solace of the soul in deep thinking --- philosophy and metaphysics. They turned from the contemplation of self\hyp{}preservation --- salvation --- to self\hyp{}realization and self\hyp{}understanding.
\vs p098 2:3 By rigorous thought the Greeks attempted to attain that consciousness of security which would serve as a substitute for the belief in survival, but they utterly failed. Only the more intelligent among the higher classes of the Hellenic peoples could grasp this new teaching; the rank and file of the progeny of the slaves of former generations had no capacity for the reception of this new substitute for religion.
\vs p098 2:4 \pc The philosophers disdained all forms of worship, notwithstanding that they practically all held loosely to the background of a belief in the Salem doctrine of “the Intelligence of the universe,” “the idea of God,” and “the Great Source.” In so far as the Greek philosophers gave recognition to the divine and the superfinite, they were frankly monotheistic; they gave scant recognition to the whole galaxy of Olympian gods and goddesses.
\vs p098 2:5 The Greek poets of the fifth and sixth centuries, notably Pindar, attempted the reformation of Greek religion. They elevated its ideals, but they were more artists than religionists. They failed to develop a technique for fostering and conserving supreme values.
\vs p098 2:6 Xenophanes taught one God, but his deity concept was too pantheistic to be a personal Father to mortal man. Anaxagoras was a mechanist except that he did recognize a First Cause, an Initial Mind. Socrates and his successors, Plato and Aristotle, taught that virtue is knowledge; goodness, health of the soul; that it is better to suffer injustice than to be guilty of it, that it is wrong to return evil for evil, and that the gods are wise and good. Their cardinal virtues were: wisdom, courage, temperance, and justice.
\vs p098 2:7 \pc The evolution of religious philosophy among the Hellenic and Hebrew peoples affords a contrastive illustration of the function of the church as an institution in the shaping of cultural progress. In Palestine, human thought was so priest\hyp{}controlled and scripture\hyp{}directed that philosophy and aesthetics were entirely submerged in religion and morality. In Greece, the almost complete absence of priests and “sacred scriptures” left the human mind free and unfettered, resulting in a startling development in depth of thought. But religion as a personal experience failed to keep pace with the intellectual probings into the nature and reality of the cosmos.
\vs p098 2:8 In Greece, believing was subordinated to thinking; in Palestine, thinking was held subject to believing. Much of the strength of Christianity is due to its having borrowed heavily from both Hebrew morality and Greek thought.
\vs p098 2:9 In Palestine, religious dogma became so crystallized as to jeopardize further growth; in Greece, human thought became so abstract that the concept of God resolved itself into a misty vapour of pantheistic speculation not at all unlike the impersonal Infinity of the Brahman philosophers.
\vs p098 2:10 \pc But the average men of these times could not grasp, nor were they much interested in, the Greek philosophy of self\hyp{}realization and an abstract Deity; they rather craved promises of salvation, coupled with a personal God who could hear their prayers. They exiled the philosophers, persecuted the remnants of the Salem cult, both doctrines having become much blended, and made ready for that terrible orgiastic plunge into the follies of the mystery cults which were then overspreading the Mediterranean lands. The Eleusinian mysteries grew up within the Olympian pantheon, a Greek version of the worship of fertility; Dionysus nature worship flourished; the best of the cults was the Orphic brotherhood, whose moral preachments and promises of salvation made a great appeal to many.
\vs p098 2:11 All Greece became involved in these new methods of attaining salvation, these emotional and fiery ceremonials. No nation ever attained such heights of artistic philosophy in so short a time; none ever created such an advanced system of ethics practically without Deity and entirely devoid of the promise of human salvation; no nation ever plunged so quickly, deeply, and violently into such depths of intellectual stagnation, moral depravity, and spiritual poverty as these same Greek peoples when they flung themselves into the mad whirl of the mystery cults.
\vs p098 2:12 \pc Religions have long endured without philosophical support, but few philosophies, as such, have long persisted without some identification with religion. Philosophy is to religion as conception is to action. But the ideal human estate is that in which philosophy, religion, and science are welded into a meaningful unity by the conjoined action of wisdom, faith, and experience.
\usection{3.\bibnobreakspace The Melchizedek Teachings in Rome}
\vs p098 3:1 Having grown out of the earlier religious forms of worship of the family gods into the tribal reverence for Mars, the god of war, it was natural that the later religion of the Latins was more of a political observance than were the intellectual systems of the Greeks and Brahmans or the more spiritual religions of several other peoples.
\vs p098 3:2 In the great monotheistic renaissance of Melchizedek’s gospel during the VI century before Christ, too few of the Salem missionaries penetrated Italy, and those who did were unable to overcome the influence of the rapidly spreading Etruscan priesthood with its new galaxy of gods and temples, all of which became organized into the Roman state religion. This religion of the Latin tribes was not trivial and venal like that of the Greeks, neither was it austere and tyrannical like that of the Hebrews; it consisted for the most part in the observance of mere forms, vows, and taboos.
\vs p098 3:3 Roman religion was greatly influenced by extensive cultural importations from Greece. Eventually most of the Olympian gods were transplanted and incorporated into the Latin pantheon. The Greeks long worshipped the fire of the family hearth --- Hestia was the virgin goddess of the hearth; Vesta was the Roman goddess of the home. Zeus became Jupiter; Aphrodite, Venus; and so on down through the many Olympian deities.
\vs p098 3:4 The religious initiation of Roman youths was the occasion of their solemn consecration to the service of the state. Oaths and admissions to citizenship were in reality religious ceremonies. The Latin peoples maintained temples, altars, and shrines and, in a crisis, would consult the oracles. They preserved the bones of heroes and later on those of the Christian saints.
\vs p098 3:5 This formal and unemotional form of pseudoreligious patriotism was doomed to collapse, even as the highly intellectual and artistic worship of the Greeks had gone down before the fervid and deeply emotional worship of the mystery cults. The greatest of these devastating cults was the mystery religion of the Mother of God sect, which had its headquarters, in those days, on the exact site of the present church of St. Peter’s in Rome.
\vs p098 3:6 \pc The emerging Roman state conquered politically but was in turn conquered by the cults, rituals, mysteries, and god concepts of Egypt, Greece, and the Levant. These imported cults continued to flourish throughout the Roman state up to the time of Augustus, who, purely for political and civic reasons, made a heroic and somewhat successful effort to destroy the mysteries and revive the older political religion.
\vs p098 3:7 One of the priests of the state religion told Augustus of the earlier attempts of the Salem teachers to spread the doctrine of one God, a final Deity presiding over all supernatural beings; and this idea took such a firm hold on the emperor that he built many temples, stocked them well with beautiful images, reorganized the state priesthood, re\hyp{}established the state religion, appointed himself acting high priest of all, and as emperor did not hesitate to proclaim himself the supreme god.
\vs p098 3:8 This new religion of Augustus worship flourished and was observed throughout the empire during his lifetime except in Palestine, the home of the Jews. And this era of the human gods continued until the official Roman cult had a roster of more than twoscore self\hyp{}elevated human deities, all claiming miraculous births and other superhuman attributes.
\vs p098 3:9 \pc The last stand of the dwindling band of Salem believers was made by an earnest group of preachers, the Cynics, who exhorted the Romans to abandon their wild and senseless religious rituals and return to a form of worship embodying Melchizedek’s gospel as it had been modified and contaminated through contact with the philosophy of the Greeks. But the people at large rejected the Cynics; they preferred to plunge into the rituals of the mysteries, which not only offered hopes of personal salvation but also gratified the desire for diversion, excitement, and entertainment.
\usection{4.\bibnobreakspace The Mystery Cults}
\vs p098 4:1 The majority of people in the Gr\ae co\hyp{}Roman world, having lost their primitive family and state religions and being unable or unwilling to grasp the meaning of Greek philosophy, turned their attention to the spectacular and emotional mystery cults from Egypt and the Levant. The common people craved promises of salvation --- religious consolation for today and assurances of hope for immortality after death.
\vs p098 4:2 The three mystery cults which became most popular were:
\vs p098 4:3 \ublistelem{1.}\bibnobreakspace The Phrygian cult of Cybele and her son Attis.
\vs p098 4:4 \ublistelem{2.}\bibnobreakspace The Egyptian cult of Osiris and his mother Isis.
\vs p098 4:5 \ublistelem{3.}\bibnobreakspace The Iranian cult of the worship of Mithras as the saviour and redeemer of sinful mankind.
\vs p098 4:6 \pc The Phrygian and Egyptian mysteries taught that the divine son (respectively Attis and Osiris) had experienced death and had been resurrected by divine power, and further that all who were properly initiated into the mystery, and who reverently celebrated the anniversary of the god’s death and resurrection, would thereby become partakers of his divine nature and his immortality.
\vs p098 4:7 \pc The Phrygian ceremonies were imposing but degrading; their bloody festivals indicate how degraded and primitive these Levantine mysteries became. The most holy day was Black Friday, the “day of blood,” commemorating the self\hyp{}inflicted death of Attis. After three days of the celebration of the sacrifice and death of Attis the festival was turned to joy in honour of his resurrection.
\vs p098 4:8 The rituals of the worship of Isis and Osiris were more refined and impressive than were those of the Phrygian cult. This Egyptian ritual was built around the legend of the Nile god of old, a god who died and was resurrected, which concept was derived from the observation of the annually recurring stoppage of vegetation growth followed by the springtime restoration of all living plants. The frenzy of the observance of these mystery cults and the orgies of their ceremonials, which were supposed to lead up to the “enthusiasm” of the realization of divinity, were sometimes most revolting.
\usection{5.\bibnobreakspace The Cult of Mithras}
\vs p098 5:1 The Phrygian and Egyptian mysteries eventually gave way before the greatest of all the mystery cults, the worship of Mithras. The Mithraic cult made its appeal to a wide range of human nature and gradually supplanted both of its predecessors. Mithraism spread over the Roman Empire through the propagandizing of Roman legions recruited in the Levant, where this religion was the vogue, for they carried this belief wherever they went. And this new religious ritual was a great improvement over the earlier mystery cults.
\vs p098 5:2 The cult of Mithras arose in Iran and long persisted in its homeland despite the militant opposition of the followers of Zoroaster. But by the time Mithraism reached Rome, it had become greatly improved by the absorption of many of Zoroaster’s teachings. It was chiefly through the Mithraic cult that Zoroaster’s religion exerted an influence upon later appearing Christianity.
\vs p098 5:3 \pc The Mithraic cult portrayed a militant god taking origin in a great rock, engaging in valiant exploits, and causing water to gush forth from a rock struck with his arrows. There was a flood from which one man escaped in a specially built boat and a last supper which Mithras celebrated with the sun\hyp{}god before he ascended into the heavens. This sun\hyp{}god, or Sol Invictus, was a degeneration of the Ahura\hyp{}Mazda deity concept of Zoroastrianism. Mithras was conceived as the surviving champion of the sun\hyp{}god in his struggle with the god of darkness. And in recognition of his slaying the mythical sacred bull, Mithras was made immortal, being exalted to the station of intercessor for the human race among the gods on high.
\vs p098 5:4 The adherents of this cult worshipped in caves and other secret places, chanting hymns, mumbling magic, eating the flesh of the sacrificial animals, and drinking the blood. Three times a day they worshipped, with special weekly ceremonials on the day of the sun\hyp{}god and with the most elaborate observance of all on the annual festival of Mithras, 25\ts{th} December. It was believed that the partaking of the sacrament ensured eternal life, the immediate passing, after death, to the bosom of Mithras, there to tarry in bliss until the judgment day. On the judgment day the Mithraic keys of heaven would unlock the gates of Paradise for the reception of the faithful; whereupon all the unbaptized of the living and the dead would be annihilated upon the return of Mithras to earth. It was taught that, when a man died, he went before Mithras for judgment, and that at the end of the world Mithras would summon all the dead from their graves to face the last judgment. The wicked would be destroyed by fire, and the righteous would reign with Mithras forever.
\vs p098 5:5 At first it was a religion only for men, and there were seven different orders into which believers could be successively initiated. Later on, the wives and daughters of believers were admitted to the temples of the Great Mother, which adjoined the Mithraic temples. The women’s cult was a mixture of Mithraic ritual and the ceremonies of the Phrygian cult of Cybele, the mother of Attis.
\usection{6.\bibnobreakspace Mithraism and Christianity}
\vs p098 6:1 Prior to the coming of the mystery cults and Christianity, personal religion hardly developed as an independent institution in the civilized lands of North Africa and Europe; it was more of a family, city\hyp{}state, political, and imperial affair. The Hellenic Greeks never evolved a centralized worship system; the ritual was local; they had no priesthood and no “sacred book.” Much as the Romans, their religious institutions lacked a powerful driving agency for the preservation of higher moral and spiritual values. While it is true that the institutionalization of religion has usually detracted from its spiritual quality, it is also a fact that no religion has thus far succeeded in surviving without the aid of institutional organization of some degree, greater or lesser.
\vs p098 6:2 Occidental religion thus languished until the days of the Sceptics, Cynics, Epicureans, and Stoics, but most important of all, until the times of the great contest between Mithraism and Paul’s new religion of Christianity.
\vs p098 6:3 \pc During the third century after Christ, Mithraic and Christian churches were very similar both in appearance and in the character of their ritual. A majority of such places of worship were underground, and both contained altars whose backgrounds variously depicted the sufferings of the saviour who had brought salvation to a sin\hyp{}cursed human race.
\vs p098 6:4 Always had it been the practice of Mithraic worshippers, on entering the temple, to dip their fingers in holy water. And since in some districts there were those who at one time belonged to both religions, they introduced this custom into the majority of the Christian churches in the vicinity of Rome. Both religions employed baptism and partook of the sacrament of bread and wine. The one great difference between Mithraism and Christianity, aside from the characters of Mithras and Jesus, was that the one encouraged militarism while the other was ultrapacific. Mithraism’s tolerance for other religions (except later Christianity) led to its final undoing. But the deciding factor in the struggle between the two was the admission of women into the full fellowship of the Christian faith.
\vs p098 6:5 \pc In the end the nominal Christian faith dominated the Occident. Greek philosophy supplied the concepts of ethical value; Mithraism, the ritual of worship observance; and Christianity, as such, the technique for the conservation of moral and social values.
\usection{7.\bibnobreakspace The Christian Religion}
\vs p098 7:1 A Creator Son did not incarnate in the likeness of mortal flesh and bestow himself upon the humanity of Urantia to reconcile an angry God but rather to win all mankind to the recognition of the Father’s love and to the realization of their sonship with God. After all, even the great advocate of the atonement doctrine realized something of this truth, for he declared that “God was in Christ reconciling the world to himself.”
\vs p098 7:2 It is not the province of this paper to deal with the origin and dissemination of the Christian religion. Suffice it to say that it is built around the person of Jesus of Nazareth, the humanly incarnate Michael Son of Nebadon, known to Urantia as the Christ, the anointed one. Christianity was spread throughout the Levant and Occident by the followers of this Galilean, and their missionary zeal equalled that of their illustrious predecessors, the Sethites and Salemites, as well as that of their earnest Asiatic contemporaries, the Buddhist teachers.
\vs p098 7:3 The Christian religion, as a Urantian system of belief, arose through the compounding of the following teachings, influences, beliefs, cults, and personal individual attitudes:
\vs p098 7:4 \ublistelem{1.}\bibnobreakspace The Melchizedek teachings, which are a basic factor in all the religions of Occident and Orient that have arisen in the last 4,000 years.
\vs p098 7:5 \ublistelem{2.}\bibnobreakspace The Hebraic system of morality, ethics, theology, and belief in both Providence and the supreme Yahweh.
\vs p098 7:6 \ublistelem{3.}\bibnobreakspace The Zoroastrian conception of the struggle between cosmic good and evil, which had already left its imprint on both Judaism and Mithraism. Through prolonged contact attendant upon the struggles between Mithraism and Christianity, the doctrines of the Iranian prophet became a potent factor in determining the theologic and philosophic cast and structure of the dogmas, tenets, and cosmology of the Hellenized and Latinized versions of the teachings of Jesus.
\vs p098 7:7 \ublistelem{4.}\bibnobreakspace The mystery cults, especially Mithraism but also the worship of the Great Mother in the Phrygian cult. Even the legends of the birth of Jesus on Urantia became tainted with the Roman version of the miraculous birth of the Iranian saviour\hyp{}hero, Mithras, whose advent on earth was supposed to have been witnessed by only a handful of gift\hyp{}bearing shepherds who had been informed of this impending event by angels.
\vs p098 7:8 \ublistelem{5.}\bibnobreakspace The historic fact of the human life of Joshua ben Joseph, the reality of Jesus of Nazareth as the glorified Christ, the Son of God.
\vs p098 7:9 \ublistelem{6.}\bibnobreakspace The personal viewpoint of Paul of Tarsus. And it should be recorded that Mithraism was the dominant religion of Tarsus during his adolescence. Paul little dreamed that his well\hyp{}intentioned letters to his converts would someday be regarded by still later Christians as the “word of God.” Such well\hyp{}meaning teachers must not be held accountable for the use made of their writings by later\hyp{}day successors.
\vs p098 7:10 \ublistelem{7.}\bibnobreakspace The philosophic thought of the Hellenistic peoples, from Alexandria and Antioch through Greece to Syracuse and Rome. The philosophy of the Greeks was more in harmony with Paul’s version of Christianity than with any other current religious system and became an important factor in the success of Christianity in the Occident. Greek philosophy, coupled with Paul’s theology, still forms the basis of European ethics.
\vs p098 7:11 \pc As the original teachings of Jesus penetrated the Occident, they became Occidentalized, and as they became Occidentalized, they began to lose their potentially universal appeal to all races and kinds of men. Christianity, today, has become a religion well adapted to the social, economic, and political mores of the white races. It has long since ceased to be the religion of Jesus, although it still valiantly portrays a beautiful religion about Jesus to such individuals as sincerely seek to follow in the way of its teaching. It has glorified Jesus as the Christ, the Messianic anointed one from God, but has largely forgotten the Master’s personal gospel: the Fatherhood of God and the universal brotherhood of all men.
\vs p098 7:12 \pc And this is the long story of the teachings of Machiventa Melchizedek on Urantia. It is nearly 4,000 years since this emergency Son of Nebadon bestowed himself on Urantia, and in that time the teachings of the “priest of El Elyon, the Most High God,” have penetrated to all races and peoples. And Machiventa was successful in achieving the purpose of his unusual bestowal; when Michael made ready to appear on Urantia, the God concept was existent in the hearts of men and women, the same God concept that still flames anew in the living spiritual experience of the manifold children of the Universal Father as they live their intriguing temporal lives on the whirling planets of space.
\vsetoff
\vs p098 7:13 [Presented by a Melchizedek of Nebadon.]
\quizlink
