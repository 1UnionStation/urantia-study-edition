\upaper{68}{The Dawn of Civilization}
\author{Melchizedek}
\vs p068 0:1 This is the beginning of the narrative of the long, long forward struggle of the human species from a status that was little better than an animal existence, through the intervening ages, and down to the later times when a real, though imperfect, civilization had evolved among the higher races of mankind.
\vs p068 0:2 Civilization is a racial acquirement; it is not biologically inherent; hence must all children be reared in an environment of culture, while each succeeding generation of youth must receive anew its education. The superior qualities of civilization --- scientific, philosophic, and religious --- are not transmitted from one generation to another by direct inheritance. These cultural achievements are preserved only by the enlightened conservation of social inheritance.
\vs p068 0:3 Social evolution of the co\hyp{}operative order was initiated by the Dalamatia teachers, and for 300,000 years mankind was nurtured in the idea of group activities. The blue man most of all profited by these early social teachings, the red man to some extent, and the black man least of all. In more recent times the yellow race and the white race have presented the most advanced social development on Urantia.
\usection{1.\bibnobreakspace Protective Socialization}
\vs p068 1:1 When brought closely together, men often learn to like one another, but primitive man was not naturally overflowing with the spirit of brotherly feeling and the desire for social contact with his fellows. Rather did the early races learn by sad experience that “in union there is strength”; and it is this lack of natural brotherly attraction that now stands in the way of immediate realization of the brotherhood of man on Urantia.
\vs p068 1:2 Association early became the price of survival. The lone man was helpless unless he bore a tribal mark which testified that he belonged to a group which would certainly avenge any assault made upon him. Even in the days of Cain it was fatal to go abroad alone without some mark of group association. Civilization has become man’s insurance against violent death, while the premiums are paid by submission to society’s numerous law demands.
\vs p068 1:3 Primitive society was thus founded on the reciprocity of necessity and on the enhanced safety of association. And human society has evolved in agelong cycles as a result of this isolation fear and by means of reluctant co\hyp{}operation.
\vs p068 1:4 \pc Primitive human beings early learned that groups are vastly greater and stronger than the mere sum of their individual units. 100 men united and working in unison can move a great stone; a score of well\hyp{}trained guardians of the peace can restrain an angry mob. And so society was born, not of mere association of numbers, but rather as a result of the \bibemph{organization} of intelligent co\hyp{}operators. But co\hyp{}operation is not a natural trait of man; he learns to co\hyp{}operate first through fear and then later because he discovers it is most beneficial in meeting the difficulties of time and guarding against the supposed perils of eternity.
\vs p068 1:5 The peoples who thus early organized themselves into a primitive society became more successful in their attacks on nature as well as in defence against their fellows; they possessed greater survival possibilities; hence has civilization steadily progressed on Urantia, notwithstanding its many setbacks. And it is only because of the enhancement of survival value in association that man’s many blunders have thus far failed to stop or destroy human civilization.
\vs p068 1:6 \pc That contemporary cultural society is a rather recent phenomenon is well shown by the present\hyp{}day survival of such primitive social conditions as characterize the Australian natives and the Bushmen and Pygmies of Africa. Among these backward peoples may be observed something of the early group hostility, personal suspicion, and other highly antisocial traits which were so characteristic of all primitive races. These miserable remnants of the nonsocial peoples of ancient times bear eloquent testimony to the fact that the natural individualistic tendency of man cannot successfully compete with the more potent and powerful organizations and associations of social progression. These backward and suspicious antisocial races that speak a different dialect every 60--80 km illustrate what a world you might now be living in but for the combined teaching of the corporeal staff of the Planetary Prince and the later labours of the Adamic group of racial uplifters.
\vs p068 1:7 The modern phrase, “back to nature,” is a delusion of ignorance, a belief in the reality of the onetime fictitious “golden age.” The only basis for the legend of the golden age is the historic fact of Dalamatia and Eden. But these improved societies were far from the realization of utopian dreams.
\usection{2.\bibnobreakspace Factors in Social Progression}
\vs p068 2:1 Civilized society is the result of man’s early efforts to overcome his dislike of \bibemph{isolation.} But this does not necessarily signify mutual affection, and the present turbulent state of certain primitive groups well illustrates what the early tribes came up through. But though the individuals of a civilization may collide with each other and struggle against one another, and though civilization itself may appear to be an inconsistent mass of striving and struggling, it does evidence earnest striving, not the deadly monotony of stagnation.
\vs p068 2:2 While the level of intelligence has contributed considerably to the rate of cultural progress, society is essentially designed to lessen the risk element in the individual’s mode of living, and it has progressed just as fast as it has succeeded in lessening pain and increasing the pleasure element in life. Thus does the whole social body push on slowly toward the goal of destiny --- extinction or survival --- depending on whether that goal is self\hyp{}maintenance or self\hyp{}gratification. Self\hyp{}maintenance originates society, while excessive self\hyp{}gratification destroys civilization.
\vs p068 2:3 Society is concerned with self\hyp{}perpetuation, self\hyp{}maintenance, and self\hyp{}gratification, but human self\hyp{}realization is worthy of becoming the immediate goal of many cultural groups.
\vs p068 2:4 The herd instinct in natural man is hardly sufficient to account for the development of such a social organization as now exists on Urantia. Though this innate gregarious propensity lies at the bottom of human society, much of man’s sociability is an acquirement. Two great influences which contributed to the early association of human beings were food hunger and sex love; these instinctive urges man shares with the animal world. Two other emotions which drove human beings together and \bibemph{held} them together were vanity and fear, more particularly ghost fear.
\vs p068 2:5 \pc History is but the record of man’s agelong food struggle. \bibemph{Primitive man only thought when he was hungry;} food saving was his first self\hyp{}denial, self\hyp{}discipline. With the growth of society, food hunger ceased to be the only incentive for mutual association. Numerous other sorts of hunger, the realization of various needs, all led to the closer association of mankind. But today society is top\hyp{}heavy with the overgrowth of supposed human needs. Occidental civilization of the XX century groans wearily under the tremendous overload of luxury and the inordinate multiplication of human desires and longings. Modern society is enduring the strain of one of its most dangerous phases of far\hyp{}flung interassociation and highly complicated interdependence.
\vs p068 2:6 Hunger, vanity, and ghost fear were continuous in their social pressure, but sex gratification was transient and spasmodic. The sex urge alone did not impel primitive men and women to assume the heavy burdens of home maintenance. The early home was founded upon the sex restlessness of the male when deprived of frequent gratification and upon that devoted mother love of the human female, which in measure she shares with the females of all the higher animals. The presence of a helpless baby determined the early differentiation of male and female activities; the woman had to maintain a settled residence where she could cultivate the soil. And from earliest times, where woman was has always been regarded as the home.
\vs p068 2:7 Woman thus early became indispensable to the evolving social scheme, not so much because of the fleeting sex passion as in consequence of \bibemph{food requirement;} she was an essential partner in self\hyp{}maintenance. She was a food provider, a beast of burden, and a companion who would stand great abuse without violent resentment, and in addition to all of these desirable traits, she was an ever\hyp{}present means of sex gratification.
\vs p068 2:8 Almost everything of lasting value in civilization has its roots in the family. The family was the first successful peace group, the man and woman learning how to adjust their antagonisms while at the same time teaching the pursuits of peace to their children.
\vs p068 2:9 The function of marriage in evolution is the insurance of race survival, not merely the realization of personal happiness; self\hyp{}maintenance and self\hyp{}perpetuation are the real objects of the home. Self\hyp{}gratification is incidental and not essential except as an incentive ensuring sex association. Nature demands survival, but the arts of civilization continue to increase the pleasures of marriage and the satisfactions of family life.
\vs p068 2:10 \pc If vanity be enlarged to cover pride, ambition, and honour, then we may discern not only how these propensities contribute to the formation of human associations, but how they also hold men together, since such emotions are futile without an audience to parade before. Soon vanity associated with itself other emotions and impulses which required a social arena wherein they might exhibit and gratify themselves. This group of emotions gave origin to the early beginnings of all art, ceremonial, and all forms of sportive games and contests.
\vs p068 2:11 Vanity contributed mightily to the birth of society; but at the time of these revelations the devious strivings of a vainglorious generation threaten to swamp and submerge the whole complicated structure of a highly specialized civilization. Pleasure\hyp{}want has long since superseded hunger\hyp{}want; the legitimate social aims of self\hyp{}maintenance are rapidly translating themselves into base and threatening forms of self\hyp{}gratification. Self\hyp{}maintenance builds society; unbridled self\hyp{}gratification unfailingly destroys civilization.
\usection{3.\bibnobreakspace Socializing Influence of Ghost Fear}
\vs p068 3:1 Primitive desires produced the original society, but ghost fear held it together and imparted an extrahuman aspect to its existence. Common fear was physiological in origin: fear of physical pain, unsatisfied hunger, or some earthly calamity; but ghost fear was a new and sublime sort of terror.
\vs p068 3:2 Probably the greatest single factor in the evolution of human society was the ghost dream. Although most dreams greatly perturbed the primitive mind, the ghost dream actually terrorized early men, driving these superstitious dreamers into each other’s arms in willing and earnest association for mutual protection against the vague and unseen imaginary dangers of the spirit world. The ghost dream was one of the earliest appearing differences between the animal and human types of mind. Animals do not visualize survival after death.
\vs p068 3:3 Except for this ghost factor, all society was founded on fundamental needs and basic biologic urges. But ghost fear introduced a new factor in civilization, a fear which reaches out and away from the elemental needs of the individual, and which rises far above even the struggles to maintain the group. The dread of the departed spirits of the dead brought to light a new and amazing form of fear, an appalling and powerful terror, which contributed to whipping the loose social orders of early ages into the more thoroughly disciplined and better controlled primitive groups of ancient times. This senseless superstition, some of which still persists, prepared the minds of men, through superstitious fear of the unreal and the supernatural, for the later discovery of “the fear of the Lord which is the beginning of wisdom.” The baseless fears of evolution are designed to be supplanted by the awe for Deity inspired by revelation. The early cult of ghost fear became a powerful social bond, and ever since that far\hyp{}distant day mankind has been striving more or less for the attainment of spirituality.
\vs p068 3:4 \pc Hunger and love drove men together; vanity and ghost fear held them together. But these emotions alone, without the influence of peace\hyp{}promoting revelations, are unable to endure the strain of the suspicions and irritations of human interassociations. Without help from superhuman sources the strain of society breaks down upon reaching certain limits, and these very influences of social mobilization --- hunger, love, vanity, and fear --- conspire to plunge mankind into war and bloodshed.
\vs p068 3:5 The peace tendency of the human race is not a natural endowment; it is derived from the teachings of revealed religion, from the accumulated experience of the progressive races, but more especially from the teachings of Jesus, the Prince of Peace.
\usection{4.\bibnobreakspace Evolution of the Mores}
\vs p068 4:1 All modern social institutions arise from the evolution of the primitive customs of your savage ancestors; the conventions of today are the modified and expanded customs of yesterday. What habit is to the individual, custom is to the group; and group customs develop into folkways or tribal traditions --- mass conventions. From these early beginnings all of the institutions of present\hyp{}day human society take their humble origin.
\vs p068 4:2 It must be borne in mind that the mores originated in an effort to adjust group living to the conditions of mass existence; the mores were man’s first social institution. And all of these tribal reactions grew out of the effort to avoid pain and humiliation while at the same time seeking to enjoy pleasure and power. The origin of folkways, like the origin of languages, is always unconscious and unintentional and therefore always shrouded in mystery.
\vs p068 4:3 \pc Ghost fear drove primitive man to envision the supernatural and thus securely laid the foundations for those powerful social influences of ethics and religion which in turn preserved inviolate the mores and customs of society from generation to generation. The one thing which early established and crystallized the mores was the belief that the dead were jealous of the ways by which they had lived and died; therefore would they visit dire punishment upon those living mortals who dared to treat with careless disdain the rules of living which they had honoured when in the flesh. All this is best illustrated by the present reverence of the yellow race for their ancestors. Later developing primitive religion greatly reinforced ghost fear in stabilizing the mores, but advancing civilization has increasingly liberated mankind from the bondage of fear and the slavery of superstition.
\vs p068 4:4 Prior to the liberating and liberalizing instruction of the Dalamatia teachers, ancient man was held a helpless victim of the ritual of the mores; the primitive savage was hedged about by an endless ceremonial. Everything he did from the time of awakening in the morning to the moment he fell asleep in his cave at night had to be done just so --- in accordance with the folkways of the tribe. He was a slave to the tyranny of usage; his life contained nothing free, spontaneous, or original. There was no natural progress toward a higher mental, moral, or social existence.
\vs p068 4:5 Early man was mightily gripped by custom; the savage was a veritable slave to usage; but there have arisen ever and anon those variations from type who have dared to inaugurate new ways of thinking and improved methods of living. Nevertheless, the inertia of primitive man constitutes the biologic safety brake against precipitation too suddenly into the ruinous maladjustment of a too rapidly advancing civilization.
\vs p068 4:6 But these customs are not an unmitigated evil; their evolution should continue. It is nearly fatal to the continuance of civilization to undertake their wholesale modification by radical revolution. Custom has been the thread of continuity which has held civilization together. The path of human history is strewn with the remnants of discarded customs and obsolete social practices; but no civilization has endured which abandoned its mores except for the adoption of better and more fit customs.
\vs p068 4:7 The survival of a society depends chiefly on the progressive evolution of its mores. The process of custom evolution grows out of the desire for experimentation; new ideas are put forward --- competition ensues. A progressing civilization embraces the progressive idea and endures; time and circumstance finally select the fitter group for survival. But this does not mean that each separate and isolated change in the composition of human society has been for the better. No! indeed no! for there have been many, many retrogressions in the long forward struggle of Urantia civilization.
\usection{5.\bibnobreakspace Land Techniques --- Maintenance Arts}
\vs p068 5:1 Land is the stage of society; men are the actors. And man must ever adjust his performances to conform to the land situation. The evolution of the mores is always dependent on the land\hyp{}man ratio. This is true notwithstanding the difficulty of its discernment. Man’s land technique, or maintenance arts, plus his standards of living, equal the sum total of the folkways, the mores. And the sum of man’s adjustment to the life demands equals his cultural civilization.
\vs p068 5:2 The earliest human cultures arose along the rivers of the Eastern Hemisphere, and there were four great steps in the forward march of civilization. They were:
\vs p068 5:3 \ublistelem{1.}\bibnobreakspace \bibemph{The collection stage.} Food coercion, hunger, led to the first form of industrial organization, the primitive food\hyp{}gathering lines. Sometimes such a line of hunger march would be 16 km long as it passed over the land gleaning food. This was the primitive nomadic stage of culture and is the mode of life now followed by the African Bushmen.
\vs p068 5:4 \ublistelem{2.}\bibnobreakspace \bibemph{The hunting stage.} The invention of weapon tools enabled man to become a hunter and thus to gain considerable freedom from food slavery. A thoughtful Andonite who had severely bruised his fist in a serious combat rediscovered the idea of using a long stick for his arm and a piece of hard flint, bound on the end with sinews, for his fist. Many tribes made independent discoveries of this sort, and these various forms of hammers represented one of the great forward steps in human civilization. Today some Australian natives have progressed little beyond this stage.
\vs p068 5:5 The blue men became expert hunters and trappers; by fencing the rivers they caught fish in great numbers, drying the surplus for winter use. Many forms of ingenious snares and traps were employed in catching game, but the more primitive races did not hunt the larger animals.
\vs p068 5:6 \ublistelem{3.}\bibnobreakspace \bibemph{The pastoral stage.} This phase of civilization was made possible by the domestication of animals. The Arabs and the natives of Africa are among the more recent pastoral peoples.
\vs p068 5:7 Pastoral living afforded further relief from food slavery; man learned to live on the interest of his capital, the increase in his flocks; and this provided more leisure for culture and progress.
\vs p068 5:8 Prepastoral society was one of sex co\hyp{}operation, but the spread of animal husbandry reduced women to the depths of social slavery. In earlier times it was man’s duty to secure the animal food, woman’s business to provide the vegetable edibles. Therefore, when man entered the pastoral era of his existence, woman’s dignity fell greatly. She must still toil to produce the vegetable necessities of life, whereas the man need only go to his herds to provide an abundance of animal food. Man thus became relatively independent of woman; throughout the entire pastoral age woman’s status steadily declined. By the close of this era she had become scarcely more than a human animal, consigned to work and to bear human offspring, much as the animals of the herd were expected to labour and bring forth young. The men of the pastoral ages had great love for their cattle; all the more pity they could not have developed a deeper affection for their wives.
\vs p068 5:9 \ublistelem{4.}\bibnobreakspace \bibemph{The agricultural stage.} This era was brought about by the domestication of plants, and it represents the highest type of material civilization. Both Caligastia and Adam endeavoured to teach horticulture and agriculture. Adam and Eve were gardeners, not shepherds, and gardening was an advanced culture in those days. The growing of plants exerts an ennobling influence on all races of mankind.
\vs p068 5:10 Agriculture more than quadrupled the land\hyp{}man ratio of the world. It may be combined with the pastoral pursuits of the former cultural stage. When the three stages overlap, men hunt and women till the soil.
\vs p068 5:11 There has always been friction between the herders and the tillers of the soil. The hunter and herder were militant, warlike; the agriculturist is a more peace\hyp{}loving type. Association with animals suggests struggle and force; association with plants instils patience, quiet, and peace. Agriculture and industrialism are the activities of peace. But the weakness of both, as world social activities, is that they lack excitement and adventure.
\vs p068 5:12 \pc Human society has evolved from the hunting stage through that of the herders to the territorial stage of agriculture. And each stage of this progressive civilization was accompanied by less and less of nomadism; more and more man began to live at home.
\vs p068 5:13 And now is industry supplementing agriculture, with consequently increased urbanization and multiplication of nonagricultural groups of citizenship classes. But an industrial era cannot hope to survive if its leaders fail to recognize that even the highest social developments must ever rest upon a sound agricultural basis.
\usection{6.\bibnobreakspace Evolution of Culture}
\vs p068 6:1 Man is a creature of the soil, a child of nature; no matter how earnestly he may try to escape from the land, in the last reckoning he is certain to fail. “Dust you are and to dust shall you return” is literally true of all mankind. The basic struggle of man was, and is, and ever shall be, for land. The first social associations of primitive human beings were for the purpose of winning these land struggles. The land\hyp{}man ratio underlies all social civilization.
\vs p068 6:2 Man’s intelligence, by means of the arts and sciences, increased the land yield; at the same time the natural increase in offspring was somewhat brought under control, and thus was provided the sustenance and leisure to build a cultural civilization.
\vs p068 6:3 \pc Human society is controlled by a law which decrees that the population must vary directly in accordance with the land arts and inversely with a given standard of living. Throughout these early ages, even more than at present, the law of supply and demand as concerned men and land determined the estimated value of both. During the times of plentiful land --- unoccupied territory --- the need for men was great, and therefore the value of human life was much enhanced; hence the loss of life was more horrifying. During periods of land scarcity and associated overpopulation, human life became comparatively cheapened so that war, famine, and pestilence were regarded with less concern.
\vs p068 6:4 When the land yield is reduced or the population is increased, the inevitable struggle is renewed; the very worst traits of human nature are brought to the surface. The improvement of the land yield, the extension of the mechanical arts, and the reduction of population all tend to foster the development of the better side of human nature.
\vs p068 6:5 \pc Frontier society develops the unskilled side of humanity; the fine arts and true scientific progress, together with spiritual culture, have all thrived best in the larger centres of life when supported by an agricultural and industrial population slightly under the land\hyp{}man ratio. Cities always multiply the power of their inhabitants for either good or evil.
\vs p068 6:6 The size of the family has always been influenced by the standards of living. The higher the standard the smaller the family, up to the point of established status or gradual extinction.
\vs p068 6:7 All down through the ages the standards of living have determined the quality of a surviving population in contrast with mere quantity. Local class standards of living give origin to new social castes, new mores. When standards of living become too complicated or too highly luxurious, they speedily become suicidal. Caste is the direct result of the high social pressure of keen competition produced by dense populations.
\vs p068 6:8 The early races often resorted to practices designed to restrict population; all primitive tribes killed deformed and sickly children. Girl babies were frequently killed before the times of wife purchase. Children were sometimes strangled at birth, but the favourite method was exposure. The father of twins usually insisted that one be killed since multiple births were believed to be caused either by magic or by infidelity. As a rule, however, twins of the same sex were spared. While these taboos on twins were once well\hyp{}nigh universal, they were never a part of the Andonite mores; these peoples always regarded twins as omens of good luck.
\vs p068 6:9 Many races learned the technique of abortion, and this practice became very common after the establishment of the taboo on childbirth among the unmarried. It was long the custom for a maiden to kill her offspring, but among more civilized groups these illegitimate children became the wards of the girl’s mother. Many primitive clans were virtually exterminated by the practice of both abortion and infanticide. But regardless of the dictates of the mores, very few children were ever destroyed after having once been suckled --- maternal affection is too strong.
\vs p068 6:10 Even in the XX century there persist remnants of these primitive population controls. There is a tribe in Australia whose mothers refuse to rear more than two or three children. Not long since, one cannibalistic tribe ate every fifth child born. In Madagascar some tribes still destroy all children born on certain unlucky days, resulting in the death of about 25\%\ of all babies.
\vs p068 6:11 \pc From a world standpoint, overpopulation has never been a serious problem in the past, but if war is lessened and science increasingly controls human diseases, it may become a serious problem in the near future. At such a time the great test of the wisdom of world leadership will present itself. Will Urantia rulers have the insight and courage to foster the multiplication of the average or stabilized human being instead of the extremes of the supernormal and the enormously increasing groups of the subnormal? The normal man should be fostered; he is the backbone of civilization and the source of the mutant geniuses of the race. The subnormal man should be kept under society’s control; no more should be produced than are required to administer the lower levels of industry, those tasks requiring intelligence above the animal level but making such low\hyp{}grade demands as to prove veritable slavery and bondage for the higher types of mankind.
\vsetoff
\vs p068 6:12 [Presented by a Melchizedek sometime stationed on Urantia.]
