\newpage
\thispagestyle{empty}
\bibmark{book}{INTRODUCTION}
\fancyhead[C]{\bibheadfont INTRODUCTION}
\makeatletter
\bib@raise@anchor{\bibpdfbookmark[0]{Introduction}{Intr}}%
\makeatother

\begin{center}\bibpapertitlefont INTRODUCTION\end{center}
\introfontsize

The purpose of the Fifth Epochal Revelation is to expand cosmic consciousness and enhance spiritual perception.
The high ideals and the example of the life of Jesus are especially needed today,
in the times of planetary crisis and pandemia of fear and madness, caused by intellectual ignorance and spiritual blindness.

\bibemph{The British Study Edition of The Urantia Papers} has the following distinct features:

\begin{itemize}
\item The present text of \bibemph{The Urantia Papers} is based on that of the Fifth Epochal Revelation,
and was revised according to the present scientific knowledge and spiritual enlightenment.
\item Study notes and textual variants are printed in the apparatus at the bottom of the page.
\item The symbol \pc\ marks the first paragraph in the group as in the 1955 first printing, where such groups were delimited by blank lines.
\item All distance and temperature measures have been converted to metric units, except where there was even the slightest potential for error, such as in the use of ``Jerusem miles'', which was left intact. The idiomatic expressions like ``carry his pack for a mile'' were also left intact, obviously.
\item Long and hard to memorise phrases like ``three hundred and forty\hyp{}five thousand'' have been converted to a compact form ``345,000''. Likewise, for phrases like ``seventy\hyp{}five per cent'' a more compact form of ``75\%'' was chosen. Similarly, the time designations like ``fifteen minutes past four o’clock'' now read ``16:15''.
\item Designation of the author of each paper (and Foreword) is printed in \textit{italics} on a line by itself just before the text.
\item Paragraph numbering is used both in the superscript and in the ranges printed in the head\-er of every page.
\item Suspected textual corruptions are indicated in the footnotes.
\item The Bibliography of the human sources used in the Revelation is printed at the end of each Paper.
      When it was deemed essential the reference to the source(s) was also indicated in the footnotes.
\end{itemize}

The need for the revision of the text was predicted by the Revelation itself at \bibref[101:4.2]{p101 4:2}:
\begin{quote}
\ldots\ within a few short years many of our statements regarding the physical sciences will stand in need of revision in consequence of additional scientific developments and new discoveries.
\end{quote}
Well, those ``few short years'' have come to pass and it now devolves upon us --- its students and guardians --- to revise,
expand and fearlessly continue the living stream of the Fifth Epochal Revelation:
``My Father worketh hitherto, and I work'' (John~5:17).

Realizing the great importance of this text, a group of volunteers and I have carefully preserved
the original First Edition (1955) of the Urantia Papers, known at the time as \bibemph{The Urantia Book,}
and made it available on my website in both electronic and printed forms:
\begin{center}\myurl{http://www.bibles.org.uk/guardian-plates.html}\end{center}
For deriving the etymology of the words coined by the revelators, I acknowledge the use of the notes by Dr~Chris Halvorson.
I am grateful to my friends Jim George and Mitch Austin for their helpful suggestions and comments,
some of which have been incorporated into the study notes of the present edition.
I would like to thank Irina Chernova for creating illustrations for this edition.
For information on the human sources, upon which this Revelation is based, I am indebted almost entirely to the outstanding
research work by Matthew Block as published on his website:
\begin{center}\myurl{https://urantiabooksources.com/}\end{center}
A few of the sources related to the material of the Revelation related to the physical sciences have been discovered by myself
independently.

Last but not least I am deeply grateful to Troy R.~Bishop, whom I am honoured to call my mentor and dear friend.

\tunemarkuptwo{noquiz}{}{%
This PDF file also includes the interactive quiz (\totalcurqs\ questions in total) called \bibemph{The Cosmic Citizenship Quiz}.
To use the quiz simply click on the symbol \quizsymbol\ at the end of any Paper or click on the title
of any Paper in the text section.
Clicking on the Paper's title in the quiz section will return you back to that Paper in the text section.
}

\begin{flushleft}
\itshape
Tigran Aivazian\\
London, England, 13\ts{th} July 2011.\\
\end{flushleft}
