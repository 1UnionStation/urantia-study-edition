\newpage
\thispagestyle{empty}
\bibmark{book}{INTRODUCTION}
\fancyhead[C]{\bibheadfont INTRODUCTION}

\makeatletter
\bib@raise@anchor{\bibpdfbookmark[0]{Introduction}{Intr}}%
\makeatother

\begin{center}
\headingfont\Large\tunemarkup{pgafour}{\LARGE}\tunemarkup{pgauraone}{\LARGE}\bfseries
INTRODUCTION
\end{center}

\tunemarkup{pghanlin}{\fontsize{16}{19}\selectfont}
\tunemarkup{pgkobomini}{\fontsize{15.5}{18}\selectfont}
\tunemarkup{pgkindledx}{\fontsize{17}{21}\selectfont}
\tunemarkup{pgafour}{\fontsize{15}{17}\selectfont}
\tunemarkup{pgnexus7}{\fontsize{13}{15}\selectfont}
\tunemarkup{pgnexus10}{\fontsize{13}{16}\selectfont}
\tunemarkup{pgkoboaurahd}{\fontsize{13}{17}\selectfont}
\tunemarkup{pgauraone}{\fontsize{16}{19.5}\selectfont}
\tunemarkup{pgcrownq}{\fontsize{14}{17}\selectfont}

\bibemph{The British Study Edition of The Urantia Papers} has the following main features:

\begin{itemize}
\item The revised text of \bibemph{The Urantia Papers} (also known as the SRT) is based on the original Fifth Epochal Revelation,
revised according to the present scientific knowledge and spiritual enlightenment.
\item Study notes have been added and are marked with an asterisk, like this\ts{*}.
\item The symbol \pc\ marks the first paragraph in the group as in the 1955 first printing, where such groups were delimited by blank lines.
\item All distance and temperature measures have been converted to metric units, except where there was even the slightest potential for error, such as in the use of ``Jerusem miles'', which was left intact. The idiomatic expressions like ``carry his pack for a mile'' were also left intact, obviously.
\item Long and hard to memorise phrases like ``three hundred and forty\hyp{}five thousand'' have been converted to a compact form ``345,000''. Likewise, for phrases like ``seventy\hyp{}five per cent'' a more compact form of ``75\%'' was chosen. Similarly, the time designations like ``fifteen minutes past four o’clock'' now read ``16:15''.
\item Designation of the author of each paper (and Foreword) is printed in \textit{italics} on a line by itself just before the text.
\item Paragraph numbering is used both in the superscript and in the ranges printed in the head\-er of every page.
\item Possible textual corruptions are indicated in the footnotes.
\end{itemize}

The need for the revision of the text was predicted at \bibref[101:4.2]{p101 4:2}:

\begin{quote}
\ldots\ within a few short years many of our statements regarding the physical sciences will stand in need of revision in consequence of additional scientific developments and new discoveries.
\end{quote}

Well, those ``few short years'' have come to pass and it now devolves upon us --- its students and guardians --- to revise,
expand and fearlessly continue the living stream of the Fifth Epochal Revelation.
And to all those fearful and unbelieving who would speedily accuse me of blasphemy I say with my Lord and my Master:
``My Father worketh hitherto, and I work'' (John~5:17).
Nevertheless, I and a group of others have carefully preserved the original First Edition (1955) of \bibemph{The Urantia Book}
and made it available on my website in both electronic and printed forms:

\begin{center}
\texttt{\normalsize http://www.bibles.org.uk/guardian-plates.html}
\end{center}

For deriving the etymology of the words coined by the revelators, I acknowledge the use of the notes by Dr~Chris Halvorson.
Many thanks to my friends Jim George and Mitch Austin for their helpful suggestions and comments,
some of which have been incorporated into the study notes of the present edition.
Last, but not least, I am deeply grateful to Troy R.~Bishop, whom I am honoured to call my mentor and friend.

\tunemarkuptwo{noquiz}{}{%
This PDF file also includes the interactive quiz (\totalcurqs\ questions in total) called \bibemph{The Cosmic Citizenship Quiz}.
To use the quiz simply click on the symbol \quizsymbol\ at the end of any Paper or click on the title of any Paper in the text section.
Clicking on the Paper's title in the quiz section will return you back to that Paper in the text section.
}

\begin{center}
\authorfont\fontsize{20}{24}\selectfont\textcolor{ubdarkblue}{%
Tigran Aivazian\\
London, England\\
13\ts{th} July 2011.\\
}
\end{center}
