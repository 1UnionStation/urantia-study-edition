\upaper{167}{The Visit to Philadelphia}
\uminitoc{Breakfast with the Pharisees}
\uminitoc{Parable of the Great Supper}
\uminitoc{The Woman with the Spirit of Infirmity}
\uminitoc{The Message from Bethany}
\uminitoc{On the Way to Bethany}
\uminitoc{Blessing the Little Children}
\uminitoc{The Talk about Angels}
\author{Midwayer Commission}
\vs p167 0:1 Throughout this period of the Perean ministry, when mention is made of Jesus and the apostles visiting the various localities where the 70 were at work, it should be recalled that, as a rule, only ten were with him since it was the practice to leave at least two of the apostles at Pella to instruct the multitude. As Jesus prepared to go on to Philadelphia, Simon Peter and his brother, Andrew, returned to the Pella encampment to teach the crowds there assembled. When the Master left the camp at Pella to visit about Perea, it was not uncommon for from 300 to 500 of the campers to follow him. When he arrived at Philadelphia, he was accompanied by over 600 followers.
\vs p167 0:2 No miracles had attended the recent preaching tour through the Decapolis, and, excepting the cleansing of the ten lepers, thus far there had been no miracles on this Perean mission. This was a period when the gospel was proclaimed with power, without miracles, and most of the time without the personal presence of Jesus or even of his apostles.
\vs p167 0:3 \pc Jesus and the ten apostles arrived at Philadelphia on Wednesday, February 22, and spent Thursday and Friday resting from their recent travels and labours. That Friday night James spoke in the synagogue, and a general council was called for the following evening. They were much rejoiced over the progress of the gospel at Philadelphia and among the near\hyp{}by villages. The messengers of David also brought word of the further advancement of the kingdom throughout Palestine, as well as good news from Alexandria and Damascus.
\usection{Breakfast with the Pharisees}
\vs p167 1:1 There lived in Philadelphia a very wealthy and influential Pharisee who had accepted the teachings of Abner, and who invited Jesus to his house Sabbath morning for breakfast. It was known that Jesus was expected in Philadelphia at this time; so a large number of visitors, among them many Pharisees, had come over from Jerusalem and from elsewhere. Accordingly, about 40 of these leading men and a few lawyers were bidden to this breakfast, which had been arranged in honour of the Master.
\vs p167 1:2 As Jesus lingered by the door, speaking with Abner, and after the host had seated himself, there came into the room one of the leading Pharisees of Jerusalem, a member of the Sanhedrin, and as was his habit, he made straight for the seat of honour at the left of the host. But since this place had been reserved for the Master and that on the right for Abner, the host beckoned the Jerusalem Pharisee to sit four seats to the left, and this dignitary was much offended because he did not receive the seat of honour.
\vs p167 1:3 Soon they were all seated and enjoying the visiting among themselves since the majority of those present were disciples of Jesus or else were friendly to the gospel. Only his enemies took notice of the fact that he did not observe the ceremonial washing of his hands before he sat down to eat. Abner washed his hands at the beginning of the meal but not during the serving.
\vs p167 1:4 Near the end of the meal there came in from the street a man long afflicted with a chronic disease and now in a dropsical condition. This man was a believer, having recently been baptized by Abner’s associates. He made no request of Jesus for healing, but the Master knew full well that this afflicted man came to this breakfast hoping thereby to escape the crowds which thronged him and thus be more likely to engage his attention. This man knew that few miracles were then being performed; however, he had reasoned in his heart that his sorry plight might possibly appeal to the Master’s compassion. And he was not mistaken, for, when he entered the room, both Jesus and the self\hyp{}righteous Pharisee from Jerusalem took notice of him. The Pharisee was not slow to voice his resentment that such a one should be permitted to enter the room. But Jesus looked upon the sick man and smiled so benignly that he drew near and sat down upon the floor. As the meal was ending, the Master looked over his fellow guests and then, after glancing significantly at the man with dropsy, said: \textcolour{ubdarkred}{“My friends, teachers in Israel and learned lawyers, I would like to ask you a question: Is it lawful to heal the sick and afflicted on the Sabbath day, or not?”} But those who were there present knew Jesus too well; they held their peace; they answered not his question.
\vs p167 1:5 \pc Then went Jesus over to where the sick man sat and, taking him by the hand, said: \textcolour{ubdarkred}{“Arise and go your way. You have not asked to be healed, but I know the desire of your heart and the faith of your soul.”} Before the man left the room, Jesus returned to his seat and, addressing those at the table, said: \textcolour{ubdarkred}{“Such works my Father does, not to tempt you into the kingdom, but to reveal himself to those who are already in the kingdom. You can perceive that it would be like the Father to do just such things because which one of you, having a favourite animal that fell in the well on the Sabbath day, would not go right out and draw him up?”} And since no one would answer him, and inasmuch as his host evidently approved of what was going on, Jesus stood up and spoke to all present: \textcolour{ubdarkred}{“My brethren, when you are bidden to a marriage feast, sit not down in the chief seat, lest, perchance, a more honoured man than you has been invited, and the host will have to come to you and request that you give your place to this other and honoured guest. In this event, with shame you will be required to take a lower place at the table. When you are bidden to a feast, it would be the part of wisdom, on arriving at the festive table, to seek for the lowest place and take your seat therein, so that, when the host looks over the guests, he may say to you: ‘My friend, why sit in the seat of the least? come up higher’; and thus will such a one have glory in the presence of his fellow guests. Forget not, every one who exalts himself shall be humbled, while he who truly humbles himself shall be exalted. Therefore, when you entertain at dinner or give a supper, invite not always your friends, your brethren, your kinsmen, or your rich neighbours that they in return may bid you to their feasts, and thus will you be recompensed. When you give a banquet, sometimes bid the poor, the maimed, and the blind. In this way you shall be blessed in your heart, for you well know that the lame and the halt cannot repay you for your loving ministry.”}
\usection{Parable of the Great Supper}
\vs p167 2:1 As Jesus finished speaking at the breakfast table of the Pharisee, one of the lawyers present, desiring to relieve the silence, thoughtlessly said: “Blessed is he who shall eat bread in the kingdom of God” --- that being a common saying of those days. And then Jesus spoke a parable, which even his friendly host was compelled to take to heart. He said:
\vs p167 2:2 \textcolour{ubdarkred}{“A certain ruler gave a great supper, and having bidden many guests, he dispatched his servants at suppertime to say to those who were invited, ‘Come, for everything is now ready.’ And they all with one accord began to make excuses. The first said, ‘I have just bought a farm, and I must needs go to prove it\fnc{I must needs \bibtextul{to go} prove it. \bibexpl{This is simply bad English and has to be corrected.}}; I pray you have me excused.’ Another said, ‘I have bought five yoke of oxen, and I must go to receive them; I pray you have me excused.’ And another said, ‘I have just married a wife, and therefore I cannot come.’ So the servants went back and reported this to their master. When the master of the house heard this, he was very angry, and turning to his servants, he said: ‘I have made ready this marriage feast; the fatlings are killed, and all is in readiness for my guests, but they have spurned my invitation; they have gone every man after his lands and his merchandise, and they even show disrespect to my servants who bid them come to my feast. Go out quickly, therefore, into the streets and lanes of the city, out into the highways and the byways, and bring hither the poor and the outcast, the blind and the lame, that the marriage feast may have guests.’ And the servants did as their lord commanded, and even then there was room for more guests. Then said the lord to his servants: ‘Go now out into the roads and the countryside and constrain those who are there to come in that my house may be filled. I declare that none of those who were first bidden shall taste of my supper.’ And the servants did as their master commanded, and the house was filled.”}
\vs p167 2:3 \pc And when they heard these words, they departed; every man went to his own place. At least one of the sneering Pharisees present that morning comprehended the meaning of this parable, for he was baptized that day and made public confession of his faith in the gospel of the kingdom. Abner preached on this parable that night at the general council of believers.
\vs p167 2:4 The next day all of the apostles engaged in the philosophic exercise of endeavouring to interpret the meaning of this parable of the great supper. Though Jesus listened with interest to all of these differing interpretations, he steadfastly refused to offer them further help in understanding the parable. He would only say, \textcolour{ubdarkred}{“Let every man find out the meaning for himself and in his own soul.”}
\usection{The Woman with the Spirit of Infirmity}
\vs p167 3:1 Abner had arranged for the Master to teach in the synagogue on this Sabbath day, the first time Jesus had appeared in a synagogue since they had all been closed to his teachings by order of the Sanhedrin. At the conclusion of the service Jesus looked down before him upon an elderly woman who wore a downcast expression, and who was much bent in form. This woman had long been fear\hyp{}ridden, and all joy had passed out of her life. As Jesus stepped down from the pulpit, he went over to her and, touching her bowed\hyp{}over form on the shoulder, said: \textcolour{ubdarkred}{“Woman, if you would only believe, you could be wholly loosed from your spirit of infirmity.”} And this woman, who had been bowed down and bound up by the depressions of fear for more than 18 years, believed the words of the Master and by faith straightened up immediately. When this woman saw that she had been made straight, she lifted up her voice and glorified God.
\vs p167 3:2 Notwithstanding that this woman’s affliction was wholly mental, her bowed\hyp{}over form being the result of her depressed mind, the people thought that Jesus had healed a real physical disorder. Although the congregation of the synagogue at Philadelphia was friendly toward the teachings of Jesus, the chief ruler of the synagogue was an unfriendly Pharisee. And as he shared the opinion of the congregation that Jesus had healed a physical disorder, and being indignant because Jesus had presumed to do such a thing on the Sabbath, he stood up before the congregation and said: “Are there not six days in which men should do all their work? In these working days come, therefore, and be healed, but not on the Sabbath day.”
\vs p167 3:3 When the unfriendly ruler had thus spoken, Jesus returned to the speaker’s platform and said: \textcolour{ubdarkred}{“Why play the part of hypocrites? Does not every one of you, on the Sabbath, loose his ox from the stall and lead him forth for watering? If such a service is permissible on the Sabbath day, should not this woman, a daughter of Abraham who has been bound down by evil these 18 years, be loosed from this bondage and led forth to partake of the waters of liberty and life, even on this Sabbath day?”} And as the woman continued to glorify God, his critic was put to shame, and the congregation rejoiced with her that she had been healed.
\vs p167 3:4 As a result of his public criticism of Jesus on this Sabbath the chief ruler of the synagogue was deposed, and a follower of Jesus was put in his place.
\vs p167 3:5 \pc Jesus frequently delivered such victims of fear from their spirit of infirmity, from their depression of mind, and from their bondage of fear. But the people thought that all such afflictions were either physical disorders or possession of evil spirits.
\vs p167 3:6 \pc Jesus taught again in the synagogue on Sunday, and many were baptized by Abner at noon on that day in the river which flowed south of the city. On the morrow Jesus and the ten apostles would have started back to the Pella encampment but for the arrival of one of David’s messengers, who brought an urgent message to Jesus from his friends at Bethany, near Jerusalem.
\usection{The Message from Bethany}
\vs p167 4:1 Very late on Sunday night, February 26, a runner from Bethany arrived at Philadelphia, bringing a message from Martha and Mary which said, “Lord, he whom you love is very sick.” This message reached Jesus at the close of the evening conference and just as he was taking leave of the apostles for the night. At first Jesus made no reply. There occurred one of those strange interludes, a time when he appeared to be in communication with something outside of, and beyond, himself. And then, looking up, he addressed the messenger in the hearing of the apostles, saying: \textcolour{ubdarkred}{“This sickness is really not to the death. Doubt not that it may be used to glorify God and exalt the Son.”}
\vs p167 4:2 \pc Jesus was very fond of Martha, Mary, and their brother, Lazarus; he loved them with a fervent affection. His first and human thought was to go to their assistance at once, but another idea came into his combined mind. He had almost given up hope that the Jewish leaders at Jerusalem would ever accept the kingdom, but he still loved his people, and there now occurred to him a plan whereby the scribes and Pharisees of Jerusalem might have one more chance to accept his teachings; and he decided, his Father willing, to make this last appeal to Jerusalem the most profound and stupendous outward working of his entire earth career. The Jews clung to the idea of a wonder\hyp{}working deliverer. And though he refused to stoop to the performance of material wonders or to the enactment of temporal exhibitions of political power, he did now ask the Father’s consent for the manifestation of his hitherto unexhibited power over life and death.
\vs p167 4:3 \pc The Jews were in the habit of burying their dead on the day of their demise; this was a necessary practice in such a warm climate. It often happened that they put in the tomb one who was merely comatose, so that on the second or even the third day, such a one would come forth from the tomb. But it was the belief of the Jews that, while the spirit or soul might linger near the body for two or three days, it never tarried after the third day; that decay was well advanced by the fourth day, and that no one ever returned from the tomb after the lapse of such a period. And it was for these reasons that Jesus tarried yet two full days in Philadelphia before he made ready to start for Bethany.
\vs p167 4:4 \pc Accordingly, early on Wednesday morning he said to his apostles: \textcolour{ubdarkred}{“Let us prepare at once to go into Judea again.”} And when the apostles heard their Master say this, they drew off by themselves for a time to take counsel of one another. James assumed the direction of the conference, and they all agreed that it was only folly to allow Jesus to go again into Judea, and they came back as one man and so informed him. Said James: “Master, you were in Jerusalem a few weeks back, and the leaders sought your death, while the people were minded to stone you. At that time you gave these men their chance to receive the truth, and we will not permit you to go again into Judea.”
\vs p167 4:5 Then said Jesus: \textcolour{ubdarkred}{“But do you not understand that there are twelve hours of the day in which work may safely be done? If a man walks in the day, he does not stumble inasmuch as he has light. If a man walks in the night, he is liable to stumble since he is without light. As long as my day lasts, I fear not to enter Judea. I would do one more mighty work for these Jews; I would give them one more chance to believe, even on their own terms --- conditions of outward glory and the visible manifestation of the power of the Father and the love of the Son. Besides, do you not realize that our friend Lazarus has fallen asleep, and I would go to awake him out of this sleep!”}
\vs p167 4:6 Then said one of the apostles: “Master, if Lazarus has fallen asleep, then will he the more surely recover.” It was the custom of the Jews at that time to speak of death as a form of sleep, but as the apostles did not understand that Jesus meant that Lazarus had departed from this world, he now said plainly: \textcolour{ubdarkred}{“Lazarus is dead. And I am glad for your sakes, even if the others are not thereby saved, that I was not there, to the end that you shall now have new cause to believe in me; and by that which you will witness, you should all be strengthened in preparation for that day when I shall take leave of you and go to the Father.”}
\vs p167 4:7 When they could not persuade him to refrain from going into Judea, and when some of the apostles were loath even to accompany him, Thomas addressed his fellows, saying: “We have told the Master our fears, but he is determined to go to Bethany. I am satisfied it means the end; they will surely kill him, but if that is the Master’s choice, then let us acquit ourselves like men of courage; let us go also that we may die with him.” And it was ever so; in matters requiring deliberate and sustained courage, Thomas was always the mainstay of the twelve apostles.
\usection{On the Way to Bethany}
\vs p167 5:1 On the way to Judea Jesus was followed by a company of almost 50 of his friends and enemies. At their noon lunchtime, on Wednesday, he talked to his apostles and this group of followers on the “Terms of Salvation,” and at the end of this lesson told the parable of the Pharisee and the publican (a tax collector). Said Jesus: \textcolour{ubdarkred}{“You see, then, that the Father gives salvation to the children of men, and this salvation is a free gift to all who have the faith to receive sonship in the divine family. There is nothing man can do to earn this salvation. Works of self\hyp{}righteousness cannot buy the favour of God, and much praying in public will not atone for lack of living faith in the heart. Men you may deceive by your outward service, but God looks into your souls. What I am telling you is well illustrated by two men who went into the temple to pray, the one a Pharisee and the other a publican. The Pharisee stood and prayed to himself: ‘O God, I thank you that I am not like the rest of men, extortioners, unlearned, unjust, adulterers, or even like this publican. I fast twice a week; I give tithes of all that I get.’ But the publican, standing afar off, would not so much as lift his eyes to heaven but smote his breast, saying, ‘God be merciful to me a sinner.’ I tell you that the publican went home with God’s approval rather than the Pharisee, for every one who exalts himself shall be humbled, but he who humbles himself shall be exalted.”}
\vs p167 5:2 \pc That night, in Jericho, the unfriendly Pharisees sought to entrap the Master by inducing him to discuss marriage and divorce, as did their fellows one time in Galilee, but Jesus artfully avoided their efforts to bring him into conflict with their laws concerning divorce. As the publican and the Pharisee illustrated good and bad religion, their divorce practices served to contrast the better marriage laws of the Jewish code with the disgraceful laxity of the Pharisaic interpretations of these Mosaic divorce statutes. The Pharisee judged himself by the lowest standard; the publican squared himself by the highest ideal. Devotion, to the Pharisee, was a means of inducing self\hyp{}righteous inactivity and the assurance of false spiritual security; devotion, to the publican, was a means of stirring up his soul to the realization of the need for repentance, confession, and the acceptance, by faith, of merciful forgiveness. The Pharisee sought justice; the publican sought mercy. The law of the universe is: Ask and you shall receive; seek and you shall find.
\vs p167 5:3 Though Jesus refused to be drawn into a controversy with the Pharisees concerning divorce, he did proclaim a positive teaching of the highest ideals regarding marriage. He exalted marriage as the most ideal and highest of all human relationships. Likewise, he intimated strong disapproval of the lax and unfair divorce practices of the Jerusalem Jews, who at that time permitted a man to divorce his wife for the most trifling of reasons, such as being a poor cook, a faulty housekeeper, or for no better reason than that he had become enamoured of a better\hyp{}looking woman.
\vs p167 5:4 The Pharisees had even gone so far as to teach that divorce of this easy variety was a special dispensation granted the Jewish people, particularly the Pharisees. And so, while Jesus refused to make pronouncements dealing with marriage and divorce, he did most bitterly denounce these shameful floutings of the marriage relationship and pointed out their injustice to women and children. He never sanctioned any divorce practice which gave man any advantage over woman; the Master countenanced only those teachings which accorded women equality with men.
\vs p167 5:5 Although Jesus did not offer new mandates governing marriage and divorce, he did urge the Jews to live up to their own laws and higher teachings. He constantly appealed to the written Scriptures in his effort to improve their practices along these social lines. While thus upholding the high and ideal concepts of marriage, Jesus skillfully avoided clashing with his questioners about the social practices represented by either their written laws or their much\hyp{}cherished divorce privileges.
\vs p167 5:6 It was very difficult for the apostles to understand the Master’s reluctance to make positive pronouncements relative to scientific, social, economic, and political problems. They did not fully realize that his earth mission was exclusively concerned with revelations of spiritual and religious truths.
\vs p167 5:7 After Jesus had talked about marriage and divorce, later on that evening his apostles privately asked many additional questions, and his answers to these inquiries relieved their minds of many misconceptions. At the conclusion of this conference Jesus said: \textcolour{ubdarkred}{“Marriage is honourable and is to be desired by all men. The fact that the Son of Man pursues his earth mission alone is in no way a reflection on the desirability of marriage. That I should so work is the Father’s will, but this same Father has directed the creation of male and female, and it is the divine will that men and women should find their highest service and consequent joy in the establishment of homes for the reception and training of children, in the creation of whom these parents become copartners with the Makers of heaven and earth. And for this cause shall a man leave his father and mother and shall cleave to his wife, and they two shall become as one.”}
\vs p167 5:8 And in this way Jesus relieved the minds of the apostles of many worries about marriage and cleared up many misunderstandings regarding divorce; at the same time he did much to exalt their ideals of social union and to augment their respect for women and children and for the home.
\usection{Blessing the Little Children}
\vs p167 6:1 That evening Jesus’ message regarding marriage and the blessedness of children spread all over Jericho, so that the next morning, long before Jesus and the apostles prepared to leave, even before breakfast time, scores of mothers came to where Jesus lodged, bringing their children in their arms and leading them by their hands, and desired that he bless the little ones. When the apostles went out to view this assemblage of mothers with their children, they endeavoured to send them away, but these women refused to depart until the Master laid his hands on their children and blessed them. And when the apostles loudly rebuked these mothers, Jesus, hearing the tumult, came out and indignantly reproved them, saying: \textcolour{ubdarkred}{“Suffer little children to come to me; forbid them not, for of such is the kingdom of heaven. Verily, verily, I say to you, whosoever receives not the kingdom of God as a little child shall hardly enter therein to grow up to the full stature of spiritual manhood.”}
\vs p167 6:2 And when the Master had spoken to his apostles, he received all of the children, laying his hands on them, while he spoke words of courage and hope to their mothers.
\vs p167 6:3 \pc Jesus often talked to his apostles about the celestial mansions and taught that the advancing children of God must there grow up spiritually as children grow up physically on this world. And so does the sacred oftentimes appear to be the common, as on this day these children and their mothers little realized that the onlooking intelligences of Nebadon beheld the children of Jericho playing with the Creator of a universe.
\vs p167 6:4 \pc Woman’s status in Palestine was much improved by Jesus’ teaching; and so it would have been throughout the world if his followers had not departed so far from that which he painstakingly taught them.
\vs p167 6:5 \pc It was also at Jericho, in connection with the discussion of the early religious training of children in habits of divine worship, that Jesus impressed upon his apostles the great value of beauty as an influence leading to the urge to worship, especially with children. The Master by precept and example taught the value of worshipping the Creator in the midst of the natural surroundings of creation. He preferred to commune with the heavenly Father amidst the trees and among the lowly creatures of the natural world. He rejoiced to contemplate the Father through the inspiring spectacle of the starry realms of the Creator Sons.
\vs p167 6:6 When it is not possible to worship God in the tabernacles of nature, men should do their best to provide houses of beauty, sanctuaries of appealing simplicity and artistic embellishment, so that the highest of human emotions may be aroused in association with the intellectual approach to spiritual communion with God. Truth, beauty, and holiness are powerful and effective aids to true worship. But spirit communion is not promoted by mere massive ornateness and overmuch embellishment with man’s elaborate and ostentatious art. Beauty is most religious when it is most simple and naturelike. How unfortunate that little children should have their first introduction to concepts of public worship in cold and barren rooms so devoid of the beauty appeal and so empty of all suggestion of good cheer and inspiring holiness! The child should be introduced to worship in nature’s outdoors and later accompany his parents to public houses of religious assembly which are at least as materially attractive and artistically beautiful as the home in which he is daily domiciled.
\usection{The Talk about Angels}
\vs p167 7:1 As they journeyed up the hills from Jericho to Bethany, Nathaniel walked most of the way by the side of Jesus, and their discussion of children in relation to the kingdom of heaven led indirectly to the consideration of the ministry of angels. Nathaniel finally asked the Master this question: “Seeing that the high priest is a Sadducee, and since the Sadducees do not believe in angels, what shall we teach the people regarding the heavenly ministers?” Then, among other things, Jesus said:
\vs p167 7:2 \pc \textcolour{ubdarkred}{“The angelic hosts are a separate order of created beings; they are entirely different from the material order of mortal creatures, and they function as a distinct group of universe intelligences. Angels are not of that group of creatures called ‘the Sons of God’ in the Scriptures; neither are they the glorified spirits of mortal men who have gone on to progress through the mansions on high. Angels are a direct creation, and they do not reproduce themselves. The angelic hosts have only a spiritual kinship with the human race. As man progresses in the journey to the Father in Paradise, he does traverse a state of being at one time analogous to the state of the angels, but mortal man never becomes an angel.}
\vs p167 7:3 \textcolour{ubdarkred}{“The angels never die, as man does. The angels are immortal unless, perchance, they become involved in sin as did some of them with the deceptions of Lucifer. The angels are the spirit servants in heaven, and they are neither all\hyp{}wise nor all\hyp{}powerful. But all of the loyal angels are truly pure and holy.}
\vs p167 7:4 \textcolour{ubdarkred}{“And do you not remember that I said to you once before that, if you had your spiritual eyes anointed, you would then see the heavens opened and behold the angels of God ascending and descending? It is by the ministry of the angels that one world may be kept in touch with other worlds, for have I not repeatedly told you that I have other sheep not of this fold? And these angels are not the spies of the spirit world who watch upon you and then go forth to tell the Father the thoughts of your heart and to report on the deeds of the flesh. The Father has no need of such service inasmuch as his own spirit lives within you. But these angelic spirits do function to keep one part of the heavenly creation informed concerning the doings of other and remote parts of the universe. And many of the angels, while functioning in the government of the Father and the universes of the Sons, are assigned to the service of the human races. When I taught you that many of these seraphim are ministering spirits, I spoke not in figurative language nor in poetic strains. And all this is true, regardless of your difficulty in comprehending such matters.}
\vs p167 7:5 \textcolour{ubdarkred}{“Many of these angels are engaged in the work of saving men, for have I not told you of the seraphic joy when one soul elects to forsake sin and begin the search for God? I did even tell you of the joy in the \bibemph{presence of the angels} of heaven over one sinner who repents, thereby indicating the existence of other and higher orders of celestial beings who are likewise concerned in the spiritual welfare and with the divine progress of mortal man.}
\vs p167 7:6 \textcolour{ubdarkred}{“Also are these angels very much concerned with the means whereby man’s spirit is released from the tabernacles of the flesh and his soul escorted to the mansions in heaven. Angels are the sure and heavenly guides of the soul of man during that uncharted and indefinite period of time which intervenes between the death of the flesh and the new life in the spirit abodes.”}
\vs p167 7:7 \pc And he would have spoken further with Nathaniel regarding the ministry of angels, but he was interrupted by the approach of Martha, who had been informed that the Master was drawing near to Bethany by friends who had observed him ascending the hills to the east. And she now hastened to greet him.
\quizlink
