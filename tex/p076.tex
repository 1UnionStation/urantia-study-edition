\upaper{76}{The Second Garden}
\author{Solonia}
\vs p076 0:1 When Adam elected to leave the first garden to the Nodites unopposed, he and his followers could not go west, for the Edenites had no boats suitable for such a marine adventure. They could not go north; the northern Nodites were already on the march toward Eden. They feared to go south; the hills of that region were infested with hostile tribes. The only way open was to the east, and so they journeyed eastward toward the then pleasant regions between the Tigris and Euphrates rivers. And many of those who were left behind later journeyed eastward to join the Adamites in their new valley home.
\vs p076 0:2 \pc Cain and Sansa were both born before the Adamic caravan had reached its destination between the rivers in Mesopotamia. Laotta, the mother of Sansa, perished at the birth of her daughter; Eve suffered much but survived, owing to superior strength. Eve took Sansa, the child of Laotta, to her bosom, and she was reared along with Cain. Sansa grew up to be a woman of great ability. She became the wife of Sargan, the chief of the northern blue races, and contributed to the advancement of the blue men of those times.
\usection{1.\bibnobreakspace The Edenites Enter Mesopotamia}
\vs p076 1:1 It required almost a full year for the caravan of Adam to reach the Euphrates River. Finding it in flood tide, they remained camped on the plains west of the stream almost six weeks before they made their way across to the land between the rivers which was to become the second garden.
\vs p076 1:2 When word had reached the dwellers in the land of the second garden that the king and high priest of the Garden of Eden was marching on them, they had fled in haste to the eastern mountains. Adam found all of the desired territory vacated when he arrived. And here in this new location Adam and his helpers set themselves to work to build new homes and establish a new centre of culture and religion.
\vs p076 1:3 This site was known to Adam as one of the three original selections of the committee assigned to choose possible locations for the Garden proposed by Van and Amadon. The two rivers themselves were a good natural defence in those days, and a short way north of the second garden the Euphrates and Tigris came close together so that a defence wall extending 90 km could be built for the protection of the territory to the south and between the rivers.
\vs p076 1:4 \pc After getting settled in the new Eden, it became necessary to adopt crude methods of living; it seemed entirely true that the ground had been cursed. Nature was once again taking its course. Now were the Adamites compelled to wrest a living from unprepared soil and to cope with the realities of life in the face of the natural hostilities and incompatibilities of mortal existence. They found the first garden partially prepared for them, but the second had to be created by the labour of their own hands and in the “sweat of their faces.”
\usection{2.\bibnobreakspace Cain and Abel}
\vs p076 2:1 Less than two years after Cain’s birth, Abel was born, the first child of Adam and Eve to be born in the second garden. When Abel grew up to the age of 12 years, he elected to be a herder; Cain had chosen to follow agriculture.
\vs p076 2:2 Now, in those days it was customary to make offerings to the priesthood of the things at hand. Herders would bring of their flocks, farmers of the fruits of the fields; and in accordance with this custom, Cain and Abel likewise made periodic offerings to the priests. The two boys had many times argued about the relative merits of their vocations, and Abel was not slow to note that preference was shown for his animal sacrifices. In vain did Cain appeal to the traditions of the first Eden, to the former preference for the fruits of the fields. But this Abel would not allow, and he taunted his older brother in his discomfiture.
\vs p076 2:3 In the days of the first Eden, Adam had indeed sought to discourage the offering of animal sacrifice so that Cain had a justifiable precedent for his contentions. It was, however, difficult to organize the religious life of the second Eden. Adam was burdened with a thousand and one details associated with the work of building, defence, and agriculture. Being much depressed spiritually, he intrusted the organization of worship and education to those of Nodite extraction who had served in these capacities in the first garden; and in even so short a time the officiating Nodite priests were reverting to the standards and rulings of pre\hyp{}Adamic times.
\vs p076 2:4 The two boys never got along well, and this matter of sacrifices further contributed to the growing hatred between them. Abel knew he was the son of both Adam and Eve and never failed to impress upon Cain that Adam was not his father. Cain was not pure violet as his father was of the Nodite race later admixed with the blue and the red man and with the aboriginal Andonic stock. And all of this, with Cain’s natural bellicose inheritance, caused him to nourish an ever\hyp{}increasing hatred for his younger brother.
\vs p076 2:5 The boys were respectively 18 and 20 years of age when the tension between them was finally resolved, one day, when Abel’s taunts so infuriated his bellicose brother that Cain turned upon him in wrath and slew him.
\vs p076 2:6 \pc The observation of Abel’s conduct establishes the value of environment and education as factors in character development. Abel had an ideal inheritance, and heredity lies at the bottom of all character; but the influence of an inferior environment virtually neutralized this magnificent inheritance. Abel, especially during his younger years, was greatly influenced by his unfavourable surroundings. He would have become an entirely different person had he lived to be 25 or 30; his superb inheritance would then have shown itself. While a good environment cannot contribute much toward really overcoming the character handicaps of a base heredity, a bad environment can very effectively spoil an excellent inheritance, at least during the younger years of life. Good social environment and proper education are indispensable soil and atmosphere for getting the most out of a good inheritance.
\vs p076 2:7 \pc The death of Abel became known to his parents when his dogs brought the flocks home without their master. To Adam and Eve, Cain was fast becoming the grim reminder of their folly, and they encouraged him in his decision to leave the garden.
\vs p076 2:8 Cain’s life in Mesopotamia had not been exactly happy since he was in such a peculiar way symbolic of the default. It was not that his associates were unkind to him, but he had not been unaware of their subconscious resentment of his presence. But Cain knew that, since he bore no tribal mark, he would be killed by the first neighbouring tribesmen who might chance to meet him. Fear, and some remorse, led him to repent. Cain had never been indwelt by an Adjuster, had always been defiant of the family discipline and disdainful of his father’s religion. But he now went to Eve, his mother, and asked for spiritual help and guidance, and when he honestly sought divine assistance, an Adjuster indwelt him. And this Adjuster, dwelling within and looking out, gave Cain a distinct advantage of superiority which classed him with the greatly feared tribe of Adam.
\vs p076 2:9 And so Cain departed for the land of Nod, east of the second Eden. He became a great leader among one group of his father’s people and did, to a certain degree, fulfil the predictions of Serapatatia, for he did promote peace between this division of the Nodites and the Adamites throughout his lifetime. Cain married Remona, his distant cousin, and their first son, Enoch, became the head of the Elamite Nodites. And for hundreds of years the Elamites and the Adamites continued to be at peace.
\usection{3.\bibnobreakspace Life in Mesopotamia}
\vs p076 3:1 As time passed in the second garden, the consequences of default became increasingly apparent. Adam and Eve greatly missed their former home of beauty and tranquillity as well as their children who had been deported to Edentia. It was indeed pathetic to observe this magnificent couple reduced to the status of the common flesh of the realm; but they bore their diminished estate with grace and fortitude.
\vs p076 3:2 Adam wisely spent most of the time training his children and their associates in civil administration, educational methods, and religious devotions. Had it not been for this foresight, pandemonium would have broken loose upon his death. As it was, the death of Adam made little difference in the conduct of the affairs of his people. But long before Adam and Eve passed away, they recognized that their children and followers had gradually learned to forget the days of their glory in Eden. And it was better for the majority of their followers that they did forget the grandeur of Eden; they were not so likely to experience undue dissatisfaction with their less fortunate environment.
\vs p076 3:3 \pc The civil rulers of the Adamites were derived hereditarily from the sons of the first garden. Adam’s first son, Adamson (Adam ben Adam), founded a secondary centre of the violet race to the north of the second Eden. Adam’s second son, Eveson, became a masterly leader and administrator; he was the great helper of his father. Eveson lived not quite so long as Adam, and his eldest son, Jansad, became the successor of Adam as the head of the Adamite tribes.
\vs p076 3:4 \pc The religious rulers, or priesthood, originated with Seth, the eldest surviving son of Adam and Eve born in the second garden. He was born 129 years after Adam’s arrival on Urantia. Seth became absorbed in the work of improving the spiritual status of his father’s people, becoming the head of the new priesthood of the second garden. His son, Enos, founded the new order of worship, and his grandson, Kenan, instituted the foreign missionary service to the surrounding tribes, near and far.
\vs p076 3:5 The Sethite priesthood was a threefold undertaking, embracing religion, health, and education. The priests of this order were trained to officiate at religious ceremonies, to serve as physicians and sanitary inspectors, and to act as teachers in the schools of the garden.
\vs p076 3:6 \pc Adam’s caravan had carried the seeds and bulbs of hundreds of plants and cereals of the first garden with them to the land between the rivers; they also had brought along extensive herds and some of all the domesticated animals. Because of this they possessed great advantages over the surrounding tribes. They enjoyed many of the benefits of the previous culture of the original Garden.
\vs p076 3:7 Up to the time of leaving the first garden, Adam and his family had always subsisted on fruits, cereals, and nuts. On the way to Mesopotamia they had, for the first time, partaken of herbs and vegetables. The eating of meat was early introduced into the second garden, but Adam and Eve never partook of flesh as a part of their regular diet. Neither did Adamson nor Eveson nor the other children of the first generation of the first garden become flesh eaters.
\vs p076 3:8 \pc The Adamites greatly excelled the surrounding peoples in cultural achievement and intellectual development. They produced the third alphabet and otherwise laid the foundations for much that was the forerunner of modern art, science, and literature. Here in the lands between the Tigris and Euphrates they maintained the arts of writing, metalworking, pottery making, and weaving and produced a type of architecture that was not excelled in thousands of years.
\vs p076 3:9 The home life of the violet peoples was, for their day and age, ideal. Children were subjected to courses of training in agriculture, craftsmanship, and animal husbandry or else were educated to perform the threefold duty of a Sethite: to be priest, physician, and teacher.
\vs p076 3:10 And when thinking of the Sethite priesthood, do not confuse those high\hyp{}minded and noble teachers of health and religion, those true educators, with the debased and commercial priesthoods of the later tribes and surrounding nations. Their religious concepts of Deity and the universe were advanced and more or less accurate, their health provisions were, for their time, excellent, and their methods of education have never since been surpassed.
\usection{4.\bibnobreakspace The Violet Race}
\vs p076 4:1 Adam and Eve were the founders of the violet race of men, the ninth human race to appear on Urantia. Adam and his offspring had blue eyes, and the violet peoples were characterized by fair complexions and light hair colour --- yellow, red, and brown.
\vs p076 4:2 Eve did not suffer pain in childbirth; neither did the early evolutionary races. Only the mixed races produced by the union of evolutionary man with the Nodites and later with the Adamites suffered the severe pangs of childbirth.
\vs p076 4:3 Adam and Eve, like their brethren on Jerusem, were energized by dual nutrition, subsisting on both food and light, supplemented by certain superphysical energies unrevealed on Urantia. Their Urantia offspring did not inherit the parental endowment of energy intake and light circulation. They had a single circulation, the human type of blood sustenance. They were designedly mortal though long\hyp{}lived, albeit longevity gravitated toward the human norm with each succeeding generation.
\vs p076 4:4 Adam and Eve and their first generation of children did not use the flesh of animals for food. They subsisted wholly upon “the fruits of the trees.” After the first generation all of the descendants of Adam began to partake of dairy products, but many of them continued to follow a nonflesh diet. Many of the southern tribes with whom they later united were also nonflesh eaters. Later on, most of these vegetarian tribes migrated to the east and survived as now admixed in the peoples of India.
\vs p076 4:5 Both the physical and spiritual visions of Adam and Eve were far superior to those of the present\hyp{}day peoples. Their special senses were much more acute, and they were able to see the midwayers and the angelic hosts, the Melchizedeks, and the fallen Prince Caligastia, who several times came to confer with his noble successor. They retained the ability to see these celestial beings for over 100 years after the default. These special senses were not so acutely present in their children and tended to diminish with each succeeding generation.
\vs p076 4:6 The Adamic children were usually Adjuster indwelt since they all possessed undoubted survival capacity. These superior offspring were not so subject to fear as the children of evolution. So much of fear persists in the present\hyp{}day races of Urantia because your ancestors received so little of Adam’s life plasm, owing to the early miscarriage of the plans for racial physical uplift.
\vs p076 4:7 The body cells of the Material Sons and their progeny are far more resistant to disease than are those of the evolutionary beings indigenous to the planet. The body cells of the native races are akin to the living disease\hyp{}producing microscopic and ultramicroscopic organisms of the realm. These facts explain why the Urantia peoples must do so much by way of scientific effort to withstand so many physical disorders. You would be far more disease resistant if your races carried more of the Adamic life.
\vs p076 4:8 \pc After becoming established in the second garden on the Euphrates, Adam elected to leave behind as much of his life plasm as possible to benefit the world after his death. Accordingly, Eve was made the head of a commission of 12 on race improvement, and before Adam died this commission had selected 1,682 of the highest type of women on Urantia, and these women were impregnated with the Adamic life plasm. Their children all grew up to maturity except 112, so that the world, in this way, was benefited by the addition of 1,570 superior men and women. Though these candidate mothers were selected from all the surrounding tribes and represented most of the races on earth, the majority were chosen from the highest strains of the Nodites, and they constituted the early beginnings of the mighty Andite race. These children were born and reared in the tribal surroundings of their respective mothers.
\usection{5.\bibnobreakspace Death of Adam and Eve}
\vs p076 5:1 Not long after the establishment of the second Eden, Adam and Eve were duly informed that their repentance was acceptable, and that, while they were doomed to suffer the fate of the mortals of their world, they should certainly become eligible for admission to the ranks of the sleeping survivors of Urantia. They fully believed this gospel of resurrection and rehabilitation which the Melchizedeks so touchingly proclaimed to them. Their transgression had been an error of judgment and not the sin of conscious and deliberate rebellion.
\vs p076 5:2 Adam and Eve did not, as citizens of Jerusem, have Thought Adjusters, nor were they Adjuster indwelt when they functioned on Urantia in the first garden. But shortly after their reduction to mortal status they became conscious of a new presence within them and awakened to the realization that human status coupled with sincere repentance had made it possible for Adjusters to indwell them. It was this knowledge of being Adjuster indwelt that greatly heartened Adam and Eve throughout the remainder of their lives; they knew that they had failed as Material Sons of Satania, but they also knew that the Paradise career was still open to them as ascending sons of the universe.
\vs p076 5:3 \pc Adam knew about the dispensational resurrection which occurred simultaneously with his arrival on the planet, and he believed that he and his companion would probably be repersonalized in connection with the advent of the next order of sonship. He did not know that Michael, the sovereign of this universe, was so soon to appear on Urantia; he expected that the next Son to arrive would be of the Avonal order. Even so, it was always a comfort to Adam and Eve, as well as something difficult for them to understand, to ponder the only personal message they ever received from Michael. This message, among other expressions of friendship and comfort, said: “I have given consideration to the circumstances of your default, I have remembered the desire of your hearts ever to be loyal to my Father’s will, and you will be called from the embrace of mortal slumber when I come to Urantia if the subordinate Sons of my realm do not send for you before that time.”
\vs p076 5:4 And this was a great mystery to Adam and Eve. They could comprehend the veiled promise of a possible special resurrection in this message, and such a possibility greatly cheered them, but they could not grasp the meaning of the intimation that they might rest until the time of a resurrection associated with Michael’s personal appearance on Urantia. And so the Edenic pair always proclaimed that a Son of God would sometime come, and they communicated to their loved ones the belief, at least the longing hope, that the world of their blunders and sorrows might possibly be the realm whereon the ruler of this universe would elect to function as the Paradise bestowal Son. It seemed too good to be true, but Adam did entertain the thought that strife\hyp{}torn Urantia might, after all, turn out to be the most fortunate world in the system of Satania, the envied planet of all Nebadon.
\vs p076 5:5 \pc Adam lived for 530 years; he died of what might be termed old age. His physical mechanism simply wore out; the process of disintegration gradually gained on the process of repair, and the inevitable end came. Eve had died 19 years previously of a weakened heart. They were both buried in the centre of the temple of divine service which had been built in accordance with their plans soon after the wall of the colony had been completed. And this was the origin of the practice of burying noted and pious men and women under the floors of the places of worship.
\vs p076 5:6 \pc The supermaterial government of Urantia, under the direction of the Melchizedeks, continued, but direct physical contact with the evolutionary races had been severed. From the distant days of the arrival of the corporeal staff of the Planetary Prince, down through the times of Van and Amadon to the arrival of Adam and Eve, physical representatives of the universe government had been stationed on the planet. But with the Adamic default this regime, extending over a period of more than 450,000 years, came to an end. In the spiritual spheres, angelic helpers continued to struggle in conjunction with the Thought Adjusters, both working heroically for the salvage of the individual; but no comprehensive plan for far\hyp{}reaching world welfare was promulgated to the mortals of earth until the arrival of Machiventa Melchizedek, in the times of Abraham, who, with the power, patience, and authority of a Son of God, did lay the foundations for the further uplift and spiritual rehabilitation of unfortunate Urantia.
\vs p076 5:7 Misfortune has not, however, been the sole lot of Urantia; this planet has also been the most fortunate in the local universe of Nebadon. Urantians should count it all gain if the blunders of their ancestors and the mistakes of their early world rulers so plunged the planet into such a hopeless state of confusion, all the more confounded by evil and sin, that this very background of darkness should so appeal to Michael of Nebadon that he selected this world as the arena wherein to reveal the loving personality of the Father in heaven. It is not that Urantia needed a Creator Son to set its tangled affairs in order; it is rather that the evil and sin on Urantia afforded the Creator Son a more striking background against which to reveal the matchless love, mercy, and patience of the Paradise Father.
\usection{6.\bibnobreakspace Survival of Adam and Eve}
\vs p076 6:1 Adam and Eve went to their mortal rest with strong faith in the promises made to them by the Melchizedeks that they would sometime awake from the sleep of death to resume life on the mansion worlds, worlds all so familiar to them in the days preceding their mission in the material flesh of the violet race on Urantia.
\vs p076 6:2 They did not long rest in the oblivion of the unconscious sleep of the mortals of the realm. On the third day after Adam’s death, the second following his reverent burial, the orders of Lanaforge, sustained by the acting Most High of Edentia and concurred in by the Union of Days on Salvington, acting for Michael, were placed in Gabriel’s hands, directing the special roll call of the distinguished survivors of the Adamic default on Urantia. And in accordance with this mandate of special resurrection, number 26 of the Urantia series, Adam and Eve were repersonalized and reassembled in the resurrection halls of the mansion worlds of Satania together with 1,316 of their associates in the experience of the first garden. Many other loyal souls had already been translated at the time of Adam’s arrival, which was attended by a dispensational adjudication of both the sleeping survivors and of the living qualified ascenders.
\vs p076 6:3 \pc Adam and Eve quickly passed through the worlds of progressive ascension until they attained citizenship on Jerusem, once again to be residents of the planet of their origin but this time as members of a different order of universe personalities. They left Jerusem as permanent citizens --- Sons of God; they returned as ascendant citizens --- sons of man. They were immediately attached to the Urantia service on the system capital, later being assigned membership among the four and twenty counsellors who constitute the present advisory\hyp{}control body of Urantia.
\vs p076 6:4 \pc And thus ends the story of the Planetary Adam and Eve of Urantia, a story of trial, tragedy, and triumph, at least personal triumph for your well\hyp{}meaning but deluded Material Son and Daughter and undoubtedly, in the end, a story of ultimate triumph for their world and its rebellion\hyp{}tossed and evil\hyp{}harassed inhabitants. When all is summed up, Adam and Eve made a mighty contribution to the speedy civilization and accelerated biologic progress of the human race. They left a great culture on earth, but it was not possible for such an advanced civilization to survive in the face of the early dilution and the eventual submergence of the Adamic inheritance. It is the people who make a civilization; civilization does not make the people.
\vsetoff
\vs p076 6:5 [Presented by Solonia, the seraphic “voice in the Garden.”]
