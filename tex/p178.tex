\upaper{178}{Last Day at the Camp}
\author{Midwayer Commission}
\vs p178 0:1 Jesus planned to spend this Thursday, his last free day on earth as a divine Son incarnated in the flesh, with his apostles and a few loyal and devoted disciples. Soon after the breakfast hour on this beautiful morning, the Master led them to a secluded spot a short distance above their camp and there taught them many new truths. Although Jesus delivered other discourses to the apostles during the early evening hours of the day, this talk of Thursday forenoon was his farewell address to the combined camp group of apostles and chosen disciples, both Jews and gentiles. The 12 were all present save Judas. Peter and several of the apostles remarked about his absence, and some of them thought Jesus had sent him into the city to attend to some matter, probably to arrange the details of their forthcoming celebration of the Passover. Judas did not return to the camp until midafternoon, a short time before Jesus led the 12 into Jerusalem to partake of the Last Supper.
\usection{1.\bibnobreakspace Discourse on Sonship and Citizenship}
\vs p178 1:1 Jesus talked to about 50 of his trusted followers for almost 2 hours and answered a score of questions regarding the relation of the kingdom of heaven to the kingdoms of this world, concerning the relation of sonship with God to citizenship in earthly governments. This discourse, together with his answers to questions, may be summarized and restated in modern language as follows:
\vs p178 1:2 \pc The kingdoms of this world, being material, may often find it necessary to employ physical force in the execution of their laws and for the maintenance of order. In the kingdom of heaven true believers will not resort to the employment of physical force. The kingdom of heaven, being a spiritual brotherhood of the spirit\hyp{}born sons of God, may be promulgated only by the power of the spirit. This distinction of procedure refers to the relations of the kingdom of believers to the kingdoms of secular government and does not nullify the right of social groups of believers to maintain order in their ranks and administer discipline upon unruly and unworthy members.
\vs p178 1:3 There is nothing incompatible between sonship in the spiritual kingdom and citizenship in the secular or civil government. It is the believer’s duty to render to Caesar the things which are Caesar’s and to God the things which are God’s. There cannot be any disagreement between these two requirements, the one being material and the other spiritual, unless it should develop that a Caesar presumes to usurp the prerogatives of God and demand that spiritual homage and supreme worship be rendered to him. In such a case you shall worship only God while you seek to enlighten such misguided earthly rulers and in this way lead them also to the recognition of the Father in heaven. You shall not render spiritual worship to earthly rulers; neither should you employ the physical forces of earthly governments, whose rulers may sometime become believers, in the work of furthering the mission of the spiritual kingdom.
\vs p178 1:4 Sonship in the kingdom, from the standpoint of advancing civilization, should assist you in becoming the ideal citizens of the kingdoms of this world since brotherhood and service are the cornerstones of the gospel of the kingdom. The love call of the spiritual kingdom should prove to be the effective destroyer of the hate urge of the unbelieving and war\hyp{}minded citizens of the earthly kingdoms. But these material\hyp{}minded sons in darkness will never know of your spiritual light of truth unless you draw very near them with that unselfish social service which is the natural outgrowth of the bearing of the fruits of the spirit in the life experience of each individual believer.
\vs p178 1:5 As mortal and material men, you are indeed citizens of the earthly kingdoms, and you should be good citizens, all the better for having become reborn spirit sons of the heavenly kingdom. As faith\hyp{}enlightened and spirit\hyp{}liberated sons of the kingdom of heaven, you face a double responsibility of duty to man and duty to God while you voluntarily assume a third and sacred obligation: service to the brotherhood of God\hyp{}knowing believers.
\vs p178 1:6 You may not worship your temporal rulers, and you should not employ temporal power in the furtherance of the spiritual kingdom; but you should manifest the righteous ministry of loving service to believers and unbelievers alike. In the gospel of the kingdom there resides the mighty Spirit of Truth, and presently I will pour out this same spirit upon all flesh. The fruits of the spirit, your sincere and loving service, are the mighty social lever to uplift the races of darkness, and this Spirit of Truth will become your power\hyp{}multiplying fulcrum.
\vs p178 1:7 Display wisdom and exhibit sagacity in your dealings with unbelieving civil rulers. By discretion show yourselves to be expert in ironing out minor disagreements and in adjusting trifling misunderstandings. In every possible way --- in everything short of your spiritual allegiance to the rulers of the universe --- seek to live peaceably with all men. Be you always as wise as serpents but as harmless as doves.
\vs p178 1:8 You should be made all the better citizens of the secular government as a result of becoming enlightened sons of the kingdom; so should the rulers of earthly governments become all the better rulers in civil affairs as a result of believing this gospel of the heavenly kingdom. The attitude of unselfish service of man and intelligent worship of God should make all kingdom believers better world citizens, while the attitude of honest citizenship and sincere devotion to one’s temporal duty should help to make such a citizen the more easily reached by the spirit call to sonship in the heavenly kingdom.
\vs p178 1:9 So long as the rulers of earthly governments seek to exercise the authority of religious dictators, you who believe this gospel can expect only trouble, persecution, and even death. But the very light which you bear to the world, and even the very manner in which you will suffer and die for this gospel of the kingdom, will, in themselves, eventually enlighten the whole world and result in the gradual divorcement of politics and religion. The persistent preaching of this gospel of the kingdom will some day bring to all nations a new and unbelievable liberation, intellectual freedom, and religious liberty.
\vs p178 1:10 Under the soon\hyp{}coming persecutions by those who hate this gospel of joy and liberty, you will thrive and the kingdom will prosper. But you will stand in grave danger in subsequent times when most men will speak well of kingdom believers and many in high places nominally accept the gospel of the heavenly kingdom. Learn to be faithful to the kingdom even in times of peace and prosperity. Tempt not the angels of your supervision to lead you in troublous ways as a loving discipline designed to save your ease\hyp{}drifting souls.
\vs p178 1:11 Remember that you are commissioned to preach this gospel of the kingdom --- the supreme desire to do the Father’s will coupled with the supreme joy of the faith realization of sonship with God --- and you must not allow anything to divert your devotion to this one duty. Let all mankind benefit from the overflow of your loving spiritual ministry, enlightening intellectual communion, and uplifting social service; but none of these humanitarian labours, nor all of them, should be permitted to take the place of proclaiming the gospel. These mighty ministrations are the social by\hyp{}products of the still more mighty and sublime ministrations and transformations wrought in the heart of the kingdom believer by the living Spirit of Truth and by the personal realization that the faith of a spirit\hyp{}born man confers the assurance of living fellowship with the eternal God.
\vs p178 1:12 You must not seek to promulgate truth nor to establish righteousness by the power of civil governments or by the enaction of secular laws. You may always labour to persuade men’s minds, but you must never dare to compel them. You must not forget the great law of human fairness which I have taught you in positive form: Whatsoever you would that men should do to you, do even so to them.
\vs p178 1:13 When a kingdom believer is called upon to serve the civil government, let him render such service as a temporal citizen of such a government, albeit such a believer should display in his civil service all of the ordinary traits of citizenship as these have been enhanced by the spiritual enlightenment of the ennobling association of the mind of mortal man with the indwelling spirit of the eternal God. If the unbeliever can qualify as a superior civil servant, you should seriously question whether the roots of truth in your heart have not died from the lack of the living waters of combined spiritual communion and social service. The consciousness of sonship with God should quicken the entire life service of every man, woman, and child who has become the possessor of such a mighty stimulus to all the inherent powers of a human personality.
\vs p178 1:14 You are not to be passive mystics or colourless ascetics; you should not become dreamers and drifters, supinely trusting in a fictitious Providence to provide even the necessities of life. You are indeed to be gentle in your dealings with erring mortals, patient in your intercourse with ignorant men, and forbearing under provocation; but you are also to be valiant in defence of righteousness, mighty in the promulgation of truth, and aggressive in the preaching of this gospel of the kingdom, even to the ends of the earth.
\vs p178 1:15 This gospel of the kingdom is a living truth. I have told you it is like the leaven in the dough, like the grain of mustard seed; and now I declare that it is like the seed of the living being, which, from generation to generation, while it remains the same living seed, unfailingly unfolds itself in new manifestations and grows acceptably in channels of new adaptation to the peculiar needs and conditions of each successive generation. The revelation I have made to you is a \bibemph{living revelation,} and I desire that it shall bear appropriate fruits in each individual and in each generation in accordance with the laws of spiritual growth, increase, and adaptative development. From generation to generation this gospel must show increasing vitality and exhibit greater depth of spiritual power. It must not be permitted to become merely a sacred memory, a mere tradition about me and the times in which we now live.
\vs p178 1:16 And forget not: We have made no direct attack upon the persons or upon the authority of those who sit in Moses’ seat; we only offered them the new light, which they have so vigorously rejected. We have assailed them only by the denunciation of their spiritual disloyalty to the very truths which they profess to teach and safeguard. We clashed with these established leaders and recognized rulers only when they threw themselves directly in the way of the preaching of the gospel of the kingdom to the sons of men. And even now, it is not we who assail them, but they who seek our destruction. Do not forget that you are commissioned to go forth preaching only the good news. You are not to attack the old ways; you are skillfully to put the leaven of new truth in the midst of the old beliefs. Let the Spirit of Truth do his own work. Let controversy come only when they who despise the truth force it upon you. But when the wilful unbeliever attacks you, do not hesitate to stand in vigorous defence of the truth which has saved and sanctified you.
\vs p178 1:17 Throughout the vicissitudes of life, remember always to love one another. Do not strive with men, even with unbelievers. Show mercy even to those who despitefully abuse you. Show yourselves to be loyal citizens, upright artisans, praiseworthy neighbours, devoted kinsmen, understanding parents, and sincere believers in the brotherhood of the Father’s kingdom. And my spirit shall be upon you, now and even to the end of the world.
\vs p178 1:18 \pc When Jesus had concluded his teaching, it was almost 13:00, and they immediately went back to the camp, where David and his associates had lunch ready for them.
\usection{2.\bibnobreakspace After the Noontime Meal}
\vs p178 2:1 Not many of the Master’s hearers were able to take in even a part of his forenoon address. Of all who heard him, the Greeks comprehended most. Even the 11 apostles were bewildered by his allusions to future political kingdoms and to successive generations of kingdom believers. Jesus’ most devoted followers could not reconcile the impending end of his earthly ministry with these references to an extended future of gospel activities. Some of these Jewish believers were beginning to sense that earth’s greatest tragedy was about to take place, but they could not reconcile such an impending disaster with either the Master’s cheerfully indifferent personal attitude or his forenoon discourse, wherein he repeatedly alluded to the future transactions of the heavenly kingdom, extending over vast stretches of time and embracing relations with many and successive temporal kingdoms on earth.
\vs p178 2:2 By noon of this day all the apostles and disciples had learned about the hasty flight of Lazarus from Bethany. They began to sense the grim determination of the Jewish rulers to exterminate Jesus and his teachings.
\vs p178 2:3 David Zebedee, through the work of his secret agents in Jerusalem, was fully advised concerning the progress of the plan to arrest and kill Jesus. He knew all about the part of Judas in this plot, but he never disclosed this knowledge to the other apostles nor to any of the disciples. Shortly after lunch he did lead Jesus aside and, making bold, asked him whether he knew --- but he never got further with his question. The Master, holding up his hand, stopped him, saying: \textcolour{ubdarkred}{“Yes, David, I know all about it, and I know that you know, but see to it that you tell no man. Only doubt not in your own heart that the will of God will prevail in the end.”}
\vs p178 2:4 This conversation with David was interrupted by the arrival of a messenger from Philadelphia bringing word that Abner had heard of the plot to kill Jesus and asking if he should depart for Jerusalem. This runner hastened off for Philadelphia with this word for Abner: \textcolour{ubdarkred}{“Go on with your work. If I depart from you in the flesh, it is only that I may return in the spirit. I will not forsake you. I will be with you to the end.”}
\vs p178 2:5 About this time Philip came to the Master and asked: “Master, seeing that the time of the Passover draws near, where would you have us prepare to eat it?” And when Jesus heard Philip’s question, he answered: \textcolour{ubdarkred}{“Go and bring Peter and John, and I will give you directions concerning the supper we will eat together this night. As for the Passover, that you will have to consider after we have first done this.”}
\vs p178 2:6 When Judas heard the Master speaking with Philip about these matters, he drew closer that he might overhear their conversation. But David Zebedee, who was standing near, stepped up and engaged Judas in conversation while Philip, Peter, and John went to one side to talk with the Master.
\vs p178 2:7 Said Jesus to the three: \textcolour{ubdarkred}{“Go immediately into Jerusalem, and as you enter the gate, you will meet a man bearing a water pitcher. He will speak to you, and then shall you follow him. When he leads you to a certain house, go in after him and ask of the good man of that house, ‘Where is the guest chamber wherein the Master is to eat supper with his apostles?’ And when you have thus inquired, this householder will show you a large upper room all furnished and ready for us.”}
\vs p178 2:8 When the apostles reached the city, they met the man with the water pitcher near the gate and followed on after him to the home of John Mark, where the lad’s father met them and showed them the upper room in readiness for the evening meal.
\vs p178 2:9 And all of this came to pass as the result of an understanding arrived at between the Master and John Mark during the afternoon of the preceding day when they were alone in the hills. Jesus wanted to be sure he would have this one last meal undisturbed with his apostles, and believing if Judas knew beforehand of their place of meeting he might arrange with his enemies to take him, he made this secret arrangement with John Mark. In this way Judas did not learn of their place of meeting until later on when he arrived there in company with Jesus and the other apostles.
\vs p178 2:10 \pc David Zebedee had much business to transact with Judas so that he was easily prevented from following Peter, John, and Philip, as he so much desired to do. When Judas gave David a certain sum of money for provisions, David said to him: “Judas, might it not be well, under the circumstances, to provide me with a little money in advance of my actual needs?” And after Judas had reflected for a moment, he answered: “Yes, David, I think it would be wise. In fact, in view of the disturbed conditions in Jerusalem, I think it would be best for me to turn over all the money to you. They plot against the Master, and in case anything should happen to me, you would not be hampered.”
\vs p178 2:11 And so David received all the apostolic cash funds and receipts for all money on deposit. Not until the evening of the next day did the apostles learn of this transaction.
\vs p178 2:12 \pc It was about 16:30 when the three apostles returned and informed Jesus that everything was in readiness for the supper. The Master immediately prepared to lead his 12 apostles over the trail to the Bethany road and on into Jerusalem. And this was the last journey he ever made with all 12 of them.
\usection{3.\bibnobreakspace On the Way to the Supper}
\vs p178 3:1 Seeking again to avoid the crowds passing through the Kidron valley back and forth between Gethsemane Park and Jerusalem, Jesus and the 12 walked over the western brow of Mount Olivet to meet the road leading from Bethany down to the city. As they drew near the place where Jesus had tarried the previous evening to discourse on the destruction of Jerusalem, they unconsciously paused while they stood and looked down in silence upon the city. As they were a little early, and since Jesus did not wish to pass through the city until after sunset, he said to his associates:
\vs p178 3:2 \pc \textcolour{ubdarkred}{“Sit down and rest yourselves while I talk with you about what must shortly come to pass. All these years have I lived with you as brethren, and I have taught you the truth concerning the kingdom of heaven and have revealed to you the mysteries thereof. And my Father has indeed done many wonderful works in connection with my mission on earth. You have been witnesses of all this and partakers in the experience of being labourers together with God. And you will bear me witness that I have for some time warned you that I must presently return to the work the Father has given me to do; I have plainly told you that I must leave you in the world to carry on the work of the kingdom. It was for this purpose that I set you apart, in the hills of Capernaum. The experience you have had with me, you must now make ready to share with others. As the Father sent me into this world, so am I about to send you forth to represent me and finish the work I have begun.}
\vs p178 3:3 \textcolour{ubdarkred}{“You look down on yonder city in sorrow, for you have heard my words telling of the end of Jerusalem. I have forewarned you lest you should perish in her destruction and so delay the proclamation of the gospel of the kingdom. Likewise do I warn you to take heed lest you needlessly expose yourselves to peril when they come to take the Son of Man. I must go, but you are to remain to witness to this gospel when I have gone, even as I directed that Lazarus flee from the wrath of man that he might live to make known the glory of God. If it is the Father’s will that I depart, nothing you may do can frustrate the divine plan. Take heed to yourselves lest they kill you also. Let your souls be valiant in defence of the gospel by spirit power but be not misled into any foolish attempt to defend the Son of Man. I need no defence by the hand of man; the armies of heaven are even now near at hand; but I am determined to do the will of my Father in heaven, and therefore must we submit to that which is so soon to come upon us.}
\vs p178 3:4 \textcolour{ubdarkred}{“When you see this city destroyed, forget not that you have entered already upon the eternal life of endless service in the ever\hyp{}advancing kingdom of heaven, even of the heaven of heavens. You should know that in my Father’s universe and in mine are many abodes, and that there awaits the children of light the revelation of cities whose builder is God and worlds whose habit of life is righteousness and joy in the truth. I have brought the kingdom of heaven to you here on earth, but I declare that all of you who by faith enter therein and remain therein by the living service of truth, shall surely ascend to the worlds on high and sit with me in the spirit kingdom of our Father. But first must you gird yourselves and complete the work which you have begun with me. You must first pass through much tribulation and endure many sorrows --- and these trials are even now upon us --- and when you have finished your work on earth, you shall come to my joy, even as I have finished my Father’s work on earth and am about to return to his embrace.”}
\vs p178 3:5 \pc When the Master had spoken, he arose, and they all followed him down Olivet and into the city. None of the apostles, save three, knew where they were going as they made their way along the narrow streets in the approaching darkness. The crowds jostled them, but no one recognized them nor knew that the Son of God was passing by on his way to the last mortal rendezvous with his chosen ambassadors of the kingdom. And neither did the apostles know that one of their own number had already entered into a conspiracy to betray the Master into the hands of his enemies.
\vs p178 3:6 John Mark had followed them all the way into the city, and after they had entered the gate, he hurried on by another street so that he was waiting to welcome them to his father’s home when they arrived.
