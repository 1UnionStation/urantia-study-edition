\upaper{123}{The Early Childhood of Jesus}
\uminitoc{Back in Nazareth}
\uminitoc{The Fifth Year (2\,B.C.)}
\uminitoc{Events of the Sixth Year (1\,B.C.)}
\uminitoc{The Seventh Year (A.D.\,1)}
\uminitoc{School Days in Nazareth}
\uminitoc{His Eighth Year (A.D.\,2)}
\author{Midwayer Commission}
\vs p123 0:1 Owing to the uncertainties and anxieties of their sojourn in Bethlehem, Mary did not wean the babe until they had arrived safely in Alexandria, where the family was able to settle down to a normal life. They lived with kinsfolk, and Joseph was well able to support his family as he secured work shortly after their arrival. He was employed as a carpenter for several months and then elevated to the position of foreman of a large group of workmen employed on one of the public buildings then in process of construction. This new experience gave him the idea of becoming a contractor and builder after their return to Nazareth.
\vs p123 0:2 \pc All through these early years of Jesus’ helpless infancy, Mary maintained one long and constant vigil lest anything befall her child which might jeopardize his welfare or in any way interfere with his future mission on earth; no mother was ever more devoted to her child. In the home where Jesus chanced to be there were two other children about his age, and among the near neighbours there were six others whose ages were sufficiently near his own to make them acceptable play\hyp{}fellows. At first Mary was disposed to keep Jesus close by her side. She feared something might happen to him if he were allowed to play in the garden with the other children, but Joseph, with the assistance of his kinsfolk, was able to convince her that such a course would deprive Jesus of the helpful experience of learning how to adjust himself to children of his own age. And Mary, realizing that such a program of undue sheltering and unusual protection might tend to make him self\hyp{}conscious and somewhat self\hyp{}centred, finally gave assent to the plan of permitting the child of promise to grow up just like any other child; and though she was obedient to this decision, she made it her business always to be on watch while the little folks were at play about the house or in the garden. Only an affectionate mother can know the burden that Mary carried in her heart for the safety of her son during these years of his infancy and early childhood.
\vs p123 0:3 Throughout the two years of their sojourn at Alexandria, Jesus enjoyed good health and continued to grow normally. Aside from a few friends and relatives no one was told about Jesus’ being a “child of promise.” One of Joseph’s relatives revealed this to a few friends in Memphis, descendants of the distant Ikhnaton, and they, with a small group of Alexandrian believers, assembled at the palatial home of Joseph’s relative\hyp{}benefactor a short time before the return to Palestine to wish the Nazareth family well and to pay their respects to the child. On this occasion the assembled friends presented Jesus with a complete copy of the Greek translation of the Hebrew scriptures. But this copy of the Jewish sacred writings was not placed in Joseph’s hands until both he and Mary had finally declined the invitation of their Memphis and Alexandrian friends to remain in Egypt. These believers insisted that the child of destiny would be able to exert a far greater world influence as a resident of Alexandria than of any designated place in Palestine. These persuasions delayed their departure for Palestine for some time after they received the news of Herod’s death.
\vs p123 0:4 \pc Joseph and Mary finally took leave of Alexandria on a boat belonging to their friend Ezraeon, bound for Joppa, arriving at that port late in August of the year 4\,B.C. They went directly to Bethlehem, where they spent the entire month of September in counsel with their friends and relatives concerning whether they should remain there or return to Nazareth.
\vs p123 0:5 Mary had never fully given up the idea that Jesus ought to grow up in Bethlehem, the City of David. Joseph did not really believe that their son was to become a kingly deliverer of Israel. Besides, he knew that he himself was not really a descendant of David; that his being reckoned among the offspring of David was due to the adoption of one of his ancestors into the Davidic line of descent. Mary, of course, thought the City of David the most appropriate place in which the new candidate for David’s throne could be reared, but Joseph preferred to take chances with Herod Antipas rather than with his brother Archelaus. He entertained great fears for the child’s safety in Bethlehem or in any other city in Judea, and surmised that Archelaus would be more likely to pursue the menacing policies of his father, Herod, than would Antipas in Galilee. And besides all these reasons, Joseph was outspoken in his preference for Galilee as a better place in which to rear and educate the child, but it required three weeks to overcome Mary’s objections.
\vs p123 0:6 By the first of October Joseph had convinced Mary and all their friends that it was best for them to return to Nazareth. Accordingly, early in October, 4\,B.C., they departed from Bethlehem for Nazareth, going by way of Lydda and Scythopolis. They started out early one Sunday morning, Mary and the child riding on their newly acquired beast of burden, while Joseph and five accompanying kinsmen proceeded on foot; Joseph’s relatives refused to permit them to make the trip to Nazareth alone. They feared to go to Galilee by Jerusalem and the Jordan valley, and the western routes were not altogether safe for two lone travellers with a child of tender years.
\usection{Back in Nazareth}
\vs p123 1:1 On the fourth day of the journey the party reached its destination in safety. They arrived unannounced at the Nazareth home, which had been occupied for more than three years by one of Joseph’s married brothers, who was indeed surprised to see them; so quietly had they gone about their business that neither the family of Joseph nor that of Mary knew they had even left Alexandria. The next day Joseph’s brother moved his family, and Mary, for the first time since Jesus’ birth, settled down with her little family to enjoy life in their own home. In less than a week Joseph secured work as a carpenter, and they were supremely happy.
\vs p123 1:2 Jesus was about three years and two months old at the time of their return to Nazareth. He had stood all these travels very well and was in excellent health and full of childish glee and excitement at having premises of his own to run about in and to enjoy. But he greatly missed the association of his Alexandrian playmates.
\vs p123 1:3 On the way to Nazareth Joseph had persuaded Mary that it would be unwise to spread the word among their Galilean friends and relatives that Jesus was a child of promise. They agreed to refrain from all mention of these matters to anyone. And they were both very faithful in keeping this promise.
\vs p123 1:4 Jesus’ entire fourth year was a period of normal physical development and of unusual mental activity. Meantime he had formed a very close attachment for a neighbour boy about his own age named Jacob. Jesus and Jacob were always happy in their play, and they grew up to be great friends and loyal companions.
\vs p123 1:5 The next important event in the life of this Nazareth family was the birth of the second child, James, in the early morning hours of April 2, 3\,B.C. Jesus was thrilled by the thought of having a baby brother, and he would stand around by the hour just to observe the baby’s early activities.
\vs p123 1:6 It was midsummer of this same year that Joseph built a small workshop close to the village spring and near the caravan tarrying lot. After this he did very little carpenter work by the day. He had as associates two of his brothers and several other mechanics, whom he sent out to work while he remained at the shop making yokes and ploughs and doing other woodwork. He also did some work in leather and with rope and canvas. And Jesus, as he grew up, when not at school, spent his time about equally between helping his mother with home duties and watching his father work at the shop, meanwhile listening to the conversation and gossip of the caravan conductors and passengers from the four corners of the earth.
\vs p123 1:7 In July of this year, one month before Jesus was four years old, an outbreak of malignant intestinal trouble spread over all Nazareth from contact with the caravan travellers. Mary became so alarmed by the danger of Jesus being exposed to this epidemic of disease that she bundled up both her children and fled to the country home of her brother, several kilometres south of Nazareth on the Megiddo road near Sarid. They did not return to Nazareth for more than two months; Jesus greatly enjoyed this, his first experience on a farm.
\usection{The Fifth Year (2\,B.C.)}
\vs p123 2:1 In something more than a year after the return to Nazareth the boy Jesus arrived at the age of his first personal and wholehearted moral decision; and there came to abide with him a Thought Adjuster, a divine gift of the Paradise Father, which had aforetime served with Machiventa Melchizedek, thus gaining the experience of functioning in connection with the incarnation of a supermortal being living in the likeness of mortal flesh. This event occurred on February 11, 2\,B.C. Jesus was no more aware of the coming of the divine Monitor than are the millions upon millions of other children who, before and since that day, have likewise received these Thought Adjusters to indwell their minds and work for the ultimate spiritualization of these minds and the eternal survival of their evolving immortal souls.
\vs p123 2:2 On this day in February the direct and personal supervision of the Universe Rulers, as it was related to the integrity of the childlike incarnation of Michael, terminated. From that time on throughout the human unfolding of the incarnation, the guardianship of Jesus was destined to rest in the keeping of this indwelling Adjuster and the associated seraphic guardians, supplemented from time to time by the ministry of midway creatures assigned for the performance of certain definite duties in accordance with the instruction of their planetary superiors.
\vs p123 2:3 \pc Jesus was five years old in August of this year, and we will, therefore, refer to this as his fifth (calendar) year of life. In this year, 2\,B.C., a little more than one month before his fifth birthday anniversary, Jesus was made very happy by the coming of his sister Miriam, who was born on the night of July 11. During the evening of the following day Jesus had a long talk with his father concerning the manner in which various groups of living things are born into the world as separate individuals. The most valuable part of Jesus’ early education was secured from his parents in answer to his thoughtful and searching inquiries. Joseph never failed to do his full duty in taking pains and spending time answering the boy’s numerous questions. From the time Jesus was five years old until he was ten, he was one continuous question mark. While Joseph and Mary could not always answer his questions, they never failed fully to discuss his inquiries and in every other possible way to assist him in his efforts to reach a satisfactory solution of the problem which his alert mind had suggested.
\vs p123 2:4 Since returning to Nazareth, theirs had been a busy household, and Joseph had been unusually occupied building his new shop and getting his business started again. So fully was he occupied that he had found no time to build a cradle for James, but this was corrected long before Miriam came, so that she had a very comfortable crib in which to nestle while the family admired her. And the child Jesus heartily entered into all these natural and normal home experiences. He greatly enjoyed his little brother and his baby sister and was of great help to Mary in their care.
\vs p123 2:5 There were few homes in the gentile world of those days that could give a child a better intellectual, moral, and religious training than the Jewish homes of Galilee. These Jews had a systematic program for rearing and educating their children. They divided a child’s life into seven stages:
\vs p123 2:6 \ublistelem{1.}\bibnobreakspace The newborn child, the first to the eighth day.
\vs p123 2:7 \ublistelem{2.}\bibnobreakspace The suckling child.
\vs p123 2:8 \ublistelem{3.}\bibnobreakspace The weaned child.
\vs p123 2:9 \ublistelem{4.}\bibnobreakspace The period of dependence on the mother, lasting up to the end of the fifth year.
\vs p123 2:10 \ublistelem{5.}\bibnobreakspace The beginning independence of the child and, with sons, the father assuming responsibility for their education.
\vs p123 2:11 \ublistelem{6.}\bibnobreakspace The adolescent youths and maidens.
\vs p123 2:12 \ublistelem{7.}\bibnobreakspace The young men and the young women.
\vs p123 2:13 \pc It was the custom of the Galilean Jews for the mother to bear the responsibility for a child’s training until the fifth birthday, and then, if the child were a boy, to hold the father responsible for the lad’s education from that time on. This year, therefore, Jesus entered upon the fifth stage of a Galilean Jewish child’s career, and accordingly on August 21, 2\,B.C., Mary formally turned him over to Joseph for further instruction.
\vs p123 2:14 Though Joseph was now assuming the direct responsibility for Jesus’ intellectual and religious education, his mother still interested herself in his home training. She taught him to know and care for the vines and flowers growing about the garden walls which completely surrounded the home plot. She also provided on the roof of the house (the summer bedroom) shallow boxes of sand in which Jesus worked out maps and did much of his early practice at writing Aramaic, Greek, and later on, Hebrew, for in time he learned to read, write, and speak, fluently, all three languages.
\vs p123 2:15 Jesus appeared to be a well\hyp{}nigh perfect child physically and continued to make normal progress mentally and emotionally. He experienced a mild digestive upset, his first minor illness, in the latter part of this, his fifth (calendar) year.
\vs p123 2:16 Though Joseph and Mary often talked about the future of their eldest child, had you been there, you would only have observed the growing up of a normal, healthy, carefree, but exceedingly inquisitive child of that time and place.
\usection{Events of the Sixth Year (1\,B.C.)}
\vs p123 3:1 Already, with his mother’s help, Jesus had mastered the Galilean dialect of the Aramaic tongue; and now his father began teaching him Greek. Mary spoke little Greek, but Joseph was a fluent speaker of both Aramaic and Greek. The textbook for the study of the Greek language was the copy of the Hebrew scriptures --- a complete version of the law and the prophets, including the Psalms --- which had been presented to them on leaving Egypt. There were only two complete copies of the Scriptures in Greek in all Nazareth, and the possession of one of them by the carpenter’s family made Joseph’s home a much\hyp{}sought place and enabled Jesus, as he grew up, to meet an almost endless procession of earnest students and sincere truth seekers. Before this year ended, Jesus had assumed custody of this priceless manuscript, having been told on his sixth birthday that the sacred book had been presented to him by Alexandrian friends and relatives. And in a very short time he could read it readily.
\vs p123 3:2 \pc The first great shock of Jesus’ young life occurred when he was not quite six years old. It had seemed to the lad that his father --- at least his father and mother together --- knew everything. Imagine, therefore, the surprise of this inquiring child, when he asked his father the cause of a mild earthquake which had just occurred, to hear Joseph say, “My son, I really do not know.” Thus began that long and disconcerting disillusionment in the course of which Jesus found out that his earthly parents were not all\hyp{}wise and all\hyp{}knowing.
\vs p123 3:3 Joseph’s first thought was to tell Jesus that the earthquake had been caused by God, but a moment’s reflection admonished him that such an answer would immediately be provocative of further and still more embarrassing inquiries. Even at an early age it was very difficult to answer Jesus’ questions about physical or social phenomena by thoughtlessly telling him that either God or the devil was responsible. In harmony with the prevailing belief of the Jewish people, Jesus was long willing to accept the doctrine of good spirits and evil spirits as the possible explanation of mental and spiritual phenomena, but he very early became doubtful that such unseen influences were responsible for the physical happenings of the natural world.
\vs p123 3:4 \pc Before Jesus was 6 years of age, in the early summer of 1\,B.C., Zacharias and Elizabeth and their son John came to visit the Nazareth family. Jesus and John had a happy time during this, their first visit within their memories. Although the visitors could remain only a few days, the parents talked over many things, including the future plans for their sons. While they were thus engaged, the lads played with blocks in the sand on top of the house and in many other ways enjoyed themselves in true boyish fashion.
\vs p123 3:5 \pc Having met John, who came from near Jerusalem, Jesus began to evince an unusual interest in the history of Israel and to inquire in great detail as to the meaning of the Sabbath rites, the synagogue sermons, and the recurring feasts of commemoration. His father explained to him the meaning of all these seasons. The first was the midwinter festive illumination, lasting eight days, starting out with one candle the first night and adding one each successive night; this commemorated the dedication of the temple after the restoration of the Mosaic services by Judas Maccabee. Next came the early springtime celebration of Purim, the feast of Esther and Israel’s deliverance through her. Then followed the solemn Passover, which the adults celebrated in Jerusalem whenever possible, while at home the children would remember that no leavened bread was to be eaten for the whole week. Later came the feast of the first\hyp{}fruits, the harvest ingathering; and last, the most solemn of all, the feast of the new year, the day of atonement. While some of these celebrations and observances were difficult for Jesus’ young mind to understand, he pondered them seriously and then entered fully into the joy of the feast of tabernacles, the annual vacation season of the whole Jewish people, the time when they camped out in leafy booths and gave themselves up to mirth and pleasure.
\vs p123 3:6 \pc During this year Joseph and Mary had trouble with Jesus about his prayers. He insisted on talking to his heavenly Father much as he would talk to Joseph, his earthly father. This departure from the more solemn and reverent modes of communication with Deity was a bit disconcerting to his parents, especially to his mother, but there was no persuading him to change; he would say his prayers just as he had been taught, after which he insisted on having \textcolour{ubdarkred}{“just a little talk with my Father in heaven.”}
\vs p123 3:7 In June of this year Joseph turned the shop in Nazareth over to his brothers and formally entered upon his work as a builder. Before the year was over, the family income had more than trebled. Never again, until after Joseph’s death, did the Nazareth family feel the pinch of poverty. The family grew larger and larger, and they spent much money on extra education and travel, but always Joseph’s increasing income kept pace with the growing expenses.
\vs p123 3:8 The next few years Joseph did considerable work at Cana, Bethlehem (of Galilee), Magdala, Nain, Sepphoris, Capernaum, and Endor, as well as much building in and near Nazareth. As James grew up to be old enough to help his mother with the housework and care of the younger children, Jesus made frequent trips away from home with his father to these surrounding towns and villages. Jesus was a keen observer and gained much practical knowledge from these trips away from home; he was assiduously storing up knowledge regarding man and the way he lived on earth.
\vs p123 3:9 \pc This year Jesus made great progress in adjusting his strong feelings and vigorous impulses to the demands of family co\hyp{}operation and home discipline. Mary was a loving mother but a fairly strict disciplinarian. In many ways, however, Joseph exerted the greater control over Jesus as it was his practice to sit down with the boy and fully explain the real and underlying reasons for the necessity of disciplinary curtailment of personal desires in deference to the welfare and tranquillity of the entire family. When the situation had been explained to Jesus, he was always intelligently and willingly co\hyp{}operative with parental wishes and family regulations.
\vs p123 3:10 \pc Much of his spare time --- when his mother did not require his help about the house --- was spent studying the flowers and plants by day and the stars by night. He evinced a troublesome penchant for lying on his back and gazing wonderingly up into the starry heavens long after his usual bedtime in this well\hyp{}ordered Nazareth household.
\usection{The Seventh Year (A.D.\,1)}
\vs p123 4:1 This was, indeed, an eventful year in Jesus’ life. Early in January a great snowstorm occurred in Galilee. Snow fell 60\,cm deep, the heaviest snowfall Jesus saw during his lifetime and one of the deepest at Nazareth in 100 years.
\vs p123 4:2 The play life of Jewish children in the times of Jesus was rather circumscribed; all too often the children played at the more serious things they observed their elders doing. They played much at weddings and funerals, ceremonies which they so frequently saw and which were so spectacular. They danced and sang but had few organized games, such as children of later days so much enjoy.
\vs p123 4:3 Jesus, in company with a neighbour boy and later his brother James, delighted to play in the far corner of the family carpenter shop, where they had great fun with the shavings and the blocks of wood. It was always difficult for Jesus to comprehend the harm of certain sorts of play which were forbidden on the Sabbath, but he never failed to conform to his parents’ wishes. He had a capacity for humour and play which was afforded little opportunity for expression in the environment of his day and generation, but up to the age of 14 he was cheerful and light\hyp{}hearted most of the time.
\vs p123 4:4 Mary maintained a dovecote on top of the animal house adjoining the home, and they used the profits from the sale of doves as a special charity fund, which Jesus administered after he deducted the tithe and turned it over to the officer of the synagogue.\tunemarkup{private}{\begin{figure}[H]\centering\includegraphics[width=0.9\columnwidth]{images/Jesus-Fall.png}\caption{Jesus falling down the steps by Slawa~Radziszewska}\end{figure}}
\vs p123 4:5 \pc The only real accident Jesus had up to this time was a fall down the back\hyp{}yard stone stairs which led up to the canvas\hyp{}roofed bedroom. It happened during an unexpected July sandstorm from the east. The hot winds, carrying blasts of fine sand, usually blew during the rainy season, especially in March and April. It was extraordinary to have such a storm in July. When the storm came up, Jesus was on the housetop playing, as was his habit, for during much of the dry season this was his accustomed playroom. He was blinded by the sand when descending the stairs and fell. After this accident Joseph built a balustrade up both sides of the stairway.
\vs p123 4:6 There was no way in which this accident could have been prevented. It was not chargeable to neglect by the midway temporal guardians, one primary and one secondary midwayer having been assigned to the watchcare of the lad; neither was it chargeable to the guardian seraphim. It simply could not have been avoided. But this slight accident, occurring while Joseph was absent in Endor, caused such great anxiety to develop in Mary’s mind that she unwisely tried to keep Jesus very close to her side for some months.
\vs p123 4:7 Material accidents, commonplace occurrences of a physical nature, are not arbitrarily interfered with by celestial personalities. Under ordinary circumstances only midway creatures can intervene in material conditions to safeguard the persons of men and women of destiny, and even in special situations these beings can so act only in obedience to the specific mandates of their superiors.
\vs p123 4:8 And this was but one of a number of such minor accidents which subsequently befell this inquisitive and adventurous youth. If you envisage the average childhood and youth of an aggressive boy, you will have a fairly good idea of the youthful career of Jesus, and you will be able to imagine just about how much anxiety he caused his parents, particularly his mother.
\vs p123 4:9 \pc The fourth member of the Nazareth family, Joseph, was born Wednesday morning, March 16, A.D.\,1.
\usection{School Days in Nazareth}
\vs p123 5:1 Jesus was now seven years old, the age when Jewish children were supposed to begin their formal education in the synagogue schools. Accordingly, in August of this year he entered upon his eventful school life at Nazareth. Already this lad was a fluent reader, writer, and speaker of two languages, Aramaic and Greek. He was now to acquaint himself with the task of learning to read, write, and speak the Hebrew language. And he was truly eager for the new school life which was ahead of him.
\vs p123 5:2 For three years --- until he was ten --- he attended the elementary school of the Nazareth synagogue. For these three years he studied the rudiments of the Book of the Law as it was recorded in the Hebrew tongue. For the following three years he studied in the advanced school and committed to memory, by the method of repeating aloud, the deeper teachings of the sacred law. He graduated from this school of the synagogue during his 13\ts{th} year and was turned over to his parents by the synagogue rulers as an educated “son of the commandment” --- henceforth a responsible citizen of the commonwealth of Israel, all of which entailed his attendance at the Passovers in Jerusalem; accordingly, he attended his first Passover that year in company with his father and mother.
\vs p123 5:3 \pc At Nazareth the pupils sat on the floor in a semicircle, while their teacher, the chazan\fnst{\textbf{chazan}, A more accurate rendition of the Hebrew \textheb{חַזָּן} would have been ``chazzan''.}, an officer of the synagogue, sat facing them. Beginning with the Book of Leviticus, they passed on to the study of the other books of the Law\fnc{books of the \bibtextul{law}. \bibexpl{Here and in the paragraph 5 below, the ``Law'' being spoken of is the Hebrew Torah, i.e. the first part of the Hebrew Scriptures: \textheb{כתובים נביאים תורה}}.}, followed by the study of the Prophets and the Psalms\fnst{\textbf{Psalms}, Here the Writings (Hebrew \textheb{כתובים}) are called the ``Psalms'' by the name of the first book of this section of Hebrew Scriptures, according to many Hebrew MSS, though not that of Talmud (which places Ruth first) and some of the early MSS I have examined, such as The Leningrad Codex (B19A), which places Chronicles first.}. The Nazareth synagogue possessed a complete copy of the Scriptures in Hebrew. Nothing but the Scriptures was studied prior to the 12\ts{th} year. In the summer months the hours for school were greatly shortened.
\vs p123 5:4 Jesus early became a master of Hebrew, and as a young man, when no visitor of prominence happened to be sojourning in Nazareth, he would often be asked to read the Hebrew scriptures to the faithful assembled in the synagogue at the regular Sabbath services.
\vs p123 5:5 These synagogue schools, of course, had no textbooks. In teaching, the chazan would utter a statement while the pupils would in unison repeat it after him. When having access to the written books of the Law, the student learned his lesson by reading aloud and by constant repetition.
\vs p123 5:6 \pc Next, in addition to his more formal schooling, Jesus began to make contact with human nature from the four quarters of the earth as men from many lands passed in and out of his father’s repair shop. When he grew older, he mingled freely with the caravans as they tarried near the spring for rest and nourishment. Being a fluent speaker of Greek, he had little trouble in conversing with the majority of the caravan travellers and conductors.
\vs p123 5:7 Nazareth was a caravan way station and crossroads of travel and largely gentile in population; at the same time it was widely known as a centre of liberal interpretation of Jewish traditional law. In Galilee the Jews mingled more freely with the gentiles than was their practice in Judea. And of all the cities of Galilee, the Jews of Nazareth were most liberal in their interpretation of the social restrictions based on the fears of contamination as a result of contact with the gentiles. And these conditions gave rise to the common saying in Jerusalem, “Can any good thing come out of Nazareth?”
\vs p123 5:8 Jesus received his moral training and spiritual culture chiefly in his own home. He secured much of his intellectual and theological education from the chazan. But his real education --- that equipment of mind and heart for the actual test of grappling with the difficult problems of life --- he obtained by mingling with his fellow men. It was this close association with his fellow men, young and old, Jew and gentile, that afforded him the opportunity to know the human race. Jesus was highly educated in that he thoroughly understood men and devotedly loved them.
\vs p123 5:9 \pc Throughout his years at the synagogue he was a brilliant student, possessing a great advantage since he was conversant with three languages. The Nazareth chazan, on the occasion of Jesus’ finishing the course in his school, remarked to Joseph that he feared he “had learned more from Jesus’ searching questions” than he had “been able to teach the lad.”
\vs p123 5:10 Throughout his course of study Jesus learned much and derived great inspiration from the regular Sabbath sermons in the synagogue. It was customary to ask distinguished visitors, stopping over the Sabbath in Nazareth, to address the synagogue. As Jesus grew up, he heard many great thinkers of the entire Jewish world expound their views, and many also who were hardly orthodox Jews since the synagogue of Nazareth was an advanced and liberal centre of Hebrew thought and culture.
\vs p123 5:11 When entering school at seven years (at this time the Jews had just inaugurated a compulsory education law), it was customary for the pupils to choose their “birthday text,” a sort of golden rule to guide them throughout their studies, one upon which they often expatiated at their graduation when 13 years old. The text which Jesus chose was from the Prophet Isaiah: “The spirit of the Lord God is upon me, for the Lord has anointed me; he has sent me to bring good news to the meek, to bind up the brokenhearted, to proclaim liberty to the captives, and to set the spiritual prisoners free.”\tunemarkup{private}{\begin{figure}[H]\centering\includegraphics[width=\columnwidth]{images/Jesus-with-Joseph.jpg}\caption{Young Jesus with Joseph by Michael~Malm}\end{figure}}
\vs p123 5:12 \pc Nazareth was one of the 24 priest centres of the Hebrew nation. But the Galilean priesthood was more liberal in the interpretation of the traditional laws than were the Judean scribes and rabbis. And at Nazareth they were also more liberal regarding the observance of the Sabbath. It was therefore the custom for Joseph to take Jesus out for walks on Sabbath afternoons, one of their favourite jaunts being to climb the high hill near their home, from which they could obtain a panoramic view of all Galilee. To the north\hyp{}west, on clear days, they could see the long ridge of Mount Carmel running down to the sea; and many times Jesus heard his father relate the story of Elijah, one of the first of that long line of Hebrew prophets, who reproved Ahab and exposed the priests of Baal. To the north Mount Hermon raised its snowy peak in majestic splendour and monopolized the skyline, almost 900\,m of the upper slopes glistening white with perpetual snow. Far to the east they could discern the Jordan valley and, far beyond, the rocky hills of Moab. Also to the south and the east, when the sun shone upon their marble walls, they could see the Gr\ae co\hyp{}Roman cities of the Decapolis, with their amphitheatres and pretentious temples. And when they lingered toward the going down of the sun, to the west they could make out the sailing vessels on the distant Mediterranean.\fnc{ \bibexpl{Informational: first printing; Far to the east they could discern the Jordan valley and, far beyond, the rocky hills of Moab. Also to the south and the east\ldots{} --- Punctuation and wording changes were rejected by the committee. The context for this sentence is the “panoramic view” from atop the Nazareth hill: Jesus and his father are standing on top of the hill and are moving their gaze from Mt. Carmel in the north\hyp{}west around an arc to the north, east, south and west. Mt. Hermon is to their north, and from springs in its foothills near Dan (north\hyp{}east of Nazareth) the Jordan valley extends to the Dead Sea in the south. Thus, as Jesus and Joseph follow the line of the river valley along the arc of their survey, as the Jordan approaches the Dead Sea, father and son “discern\ldots{}far beyond, the rocky hills of Moab.” This interpretation is further supported by the punctuation of the following sentence which does not read “Also, to the south and the east,\ldots{}” (suggesting a change in direction from the last reference), but rather, “Also to the south and the east,\ldots{}” which implies that the last referenced location (Moab) was in the same direction.}}
\vs p123 5:13 From four directions Jesus could observe the caravan trains as they wended their way in and out of Nazareth, and to the south he could overlook the broad and fertile plain country of Esdraelon, stretching off toward Mount Gilboa and Samaria.
\vs p123 5:14 When they did not climb the heights to view the distant landscape, they strolled through the countryside and studied nature in her various moods in accordance with the seasons. Jesus’ earliest training, aside from that of the home hearth, had to do with a reverent and sympathetic contact with nature.
\vs p123 5:15 \pc Before he was 8 years of age, he was known to all the mothers and young women of Nazareth, who had met him and talked with him at the spring, which was not far from his home, and which was one of the social centres of contact and gossip for the entire town. This year Jesus learned to milk the family cow and care for the other animals. During this and the following year he also learned to make cheese and to weave. When he was 10 years of age, he was an expert loom operator. It was about this time that Jesus and the neighbour boy Jacob became great friends of the potter who worked near the flowing spring; and as they watched Nathan’s deft fingers mould the clay on the potter’s wheel, many times both of them determined to be potters when they grew up. Nathan was very fond of the lads and often gave them clay to play with, seeking to stimulate their creative imaginations by suggesting competitive efforts in modelling various objects and animals.
\usection{His Eighth Year (A.D.\,2)}
\vs p123 6:1 This was an interesting year at school. Although Jesus was not an unusual student, he was a diligent pupil and belonged to the more progressive third of the class, doing his work so well that he was excused from attendance one week out of each month. This week he usually spent either with his fisherman uncle on the shores of the Sea of Galilee near Magdala or on the farm of another uncle (his mother’s brother) 8\,km south of Nazareth.
\vs p123 6:2 Although his mother had become unduly anxious about his health and safety, she gradually became reconciled to these trips away from home. Jesus’ uncles and aunts were all very fond of him, and there ensued a lively competition among them to secure his company for these monthly visits throughout this and immediately subsequent years. His first week’s sojourn on his uncle’s farm (since infancy) was in January of this year; the first week’s fishing experience on the Sea of Galilee occurred in the month of May.
\vs p123 6:3 About this time Jesus met a teacher of mathematics from Damascus, and learning some new techniques of numbers, he spent much time on mathematics for several years. He developed a keen sense of numbers, distances, and proportions.
\vs p123 6:4 Jesus began to enjoy his brother James very much and by the end of this year had begun to teach him the alphabet.
\vs p123 6:5 This year Jesus made arrangements to exchange dairy products for lessons on the harp. He had an unusual liking for everything musical. Later on he did much to promote an interest in vocal music among his youthful associates. By the time he was 11 years of age, he was a skillful harpist and greatly enjoyed entertaining both family and friends with his extraordinary interpretations and able improvisations.
\vs p123 6:6 While Jesus continued to make enviable progress at school, all did not run smoothly for either parents or teachers. He persisted in asking many embarrassing questions concerning both science and religion, particularly regarding geography and astronomy. He was especially insistent on finding out why there was a dry season and a rainy season in Palestine. Repeatedly he sought the explanation for the great difference between the temperatures of Nazareth and the Jordan valley. He simply never ceased to ask such intelligent but perplexing questions.
\vs p123 6:7 \pc His third brother, Simon, was born on Friday evening, April 14, of this year, A.D.\,2.
\vs p123 6:8 \pc In February, Nahor, one of the teachers in a Jerusalem academy of the rabbis, came to Nazareth to observe Jesus, having been on a similar mission to Zacharias’s home near Jerusalem. He came to Nazareth at the instigation of John’s father. While at first he was somewhat shocked by Jesus’ frankness and unconventional manner of relating himself to things religious, he attributed it to the remoteness of Galilee from the centres of Hebrew learning and culture and advised Joseph and Mary to allow him to take Jesus back with him to Jerusalem, where he could have the advantages of education and training at the centre of Jewish culture. Mary was half persuaded to consent; she was convinced her eldest son was to become the Messiah, the Jewish deliverer; Joseph hesitated; he was equally persuaded that Jesus was to grow up to become a man of destiny, but what that destiny would prove to be he was profoundly uncertain. But he never really doubted that his son was to fulfil some great mission on earth. The more he thought about Nahor’s advice, the more he questioned the wisdom of the proposed sojourn in Jerusalem.
\vs p123 6:9 Because of this difference of opinion between Joseph and Mary, Nahor requested permission to lay the whole matter before Jesus. Jesus listened attentively, talked with Joseph, Mary, and a neighbour, Jacob the stone mason, whose son was his favourite playmate, and then, two days later, reported that since there was such a difference of opinion among his parents and advisers, and since he did not feel competent to assume the responsibility for such a decision, not feeling strongly one way or the other, in view of the whole situation, he had finally decided to \textcolour{ubdarkred}{“talk with my Father who is in heaven”;} and while he was not perfectly sure about the answer, he rather felt he should remain at home “with my father and mother,” adding, “they who love me so much should be able to do more for me and guide me more safely than strangers who can only view my body and observe my mind but can hardly truly know me.” They all marveled, and Nahor went his way, back to Jerusalem. And it was many years before the subject of Jesus’ going away from home again came up for consideration.
\quizlink
