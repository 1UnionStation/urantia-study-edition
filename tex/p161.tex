\upaper{161}{Further Discussions with Rodan}
\author{Midwayer Commission}
\vs p161 0:1 On Sunday, September 25, A.D.\,29, the apostles and the evangelists assembled at Magadan. After a long conference that evening with his associates, Jesus surprised all by announcing that early the next day he and the 12 apostles would start for Jerusalem to attend the feast of tabernacles. He directed that the evangelists visit the believers in Galilee, and that the women’s corps return for a while to Bethsaida.
\vs p161 0:2 When the hour came to leave for Jerusalem, Nathaniel and Thomas were still in the midst of their discussions with Rodan of Alexandria, and they secured the Master’s permission to remain at Magadan for a few days. And so, while Jesus and the ten were on their way to Jerusalem, Nathaniel and Thomas were engaged in earnest debate with Rodan. The week prior, in which Rodan had expounded his philosophy, Thomas and Nathaniel had alternated in presenting the gospel of the kingdom to the Greek philosopher. Rodan discovered that he had been well instructed in Jesus’ teachings by one of the former apostles of John the Baptist who had been his teacher at Alexandria.
\usection{1.\bibnobreakspace The Personality of God}
\vs p161 1:1 There was one matter on which Rodan and the two apostles did not see alike, and that was the personality of God. Rodan readily accepted all that was presented to him regarding the attributes of God, but he contended that the Father in heaven is not, cannot be, a person as man conceives personality. While the apostles found themselves in difficulty trying to prove that God is a person, Rodan found it still more difficult to prove he is not a person.
\vs p161 1:2 Rodan contended that the fact of personality consists in the coexistent fact of full and mutual communication between beings of equality, beings who are capable of sympathetic understanding. Said Rodan: “In order to be a person, God must have symbols of spirit communication which would enable him to become fully understood by those who make contact with him. But since God is infinite and eternal, the Creator of all other beings, it follows that, as regards beings of equality, God is alone in the universe. There are none equal to him; there are none with whom he can communicate as an equal. God indeed may be the source of all personality, but as such he is transcendent to personality, even as the Creator is above and beyond the creature.”
\vs p161 1:3 This contention greatly troubled Thomas and Nathaniel, and they had asked Jesus to come to their rescue, but the Master refused to enter into their discussions. He did say to Thomas: \textcolour{ubdarkred}{“It matters little what \bibemph{idea} of the Father you may entertain as long as you are spiritually acquainted with the \bibemph{ideal} of his infinite and eternal nature.”}
\vs p161 1:4 Thomas contended that God does communicate with man, and therefore that the Father is a person, even within the definition of Rodan. This the Greek rejected on the ground that God does not reveal himself personally; that he is still a mystery. Then Nathaniel appealed to his own personal experience with God, and that Rodan allowed, affirming that he had recently had similar experiences, but these experiences, he contended, proved only the \bibemph{reality} of God, not his \bibemph{personality.}
\vs p161 1:5 By Monday night Thomas gave up. But by Tuesday night Nathaniel had won Rodan to believe in the personality of the Father, and he effected this change in the Greek’s views by the following steps of reasoning:
\vs p161 1:6 \ublistelem{1.}\bibnobreakspace The Father in Paradise does enjoy equality of communication with at least two other beings who are fully equal to himself and wholly like himself --- the Eternal Son and the Infinite Spirit. In view of the doctrine of the Trinity, the Greek was compelled to concede the personality possibility of the Universal Father. (It was the later consideration of these discussions which led to the enlarged conception of the Trinity in the minds of the 12 apostles. Of course, it was the general belief that Jesus was the Eternal Son.)
\vs p161 1:7 \ublistelem{2.}\bibnobreakspace Since Jesus was equal with the Father, and since this Son had achieved the manifestation of personality to his earth children, such a phenomenon constituted proof of the fact, and demonstration of the possibility, of the possession of personality by all three of the Godheads and forever settled the question regarding the ability of God to communicate with man and the possibility of man’s communicating with God.
\vs p161 1:8 \ublistelem{3.}\bibnobreakspace That Jesus was on terms of mutual association and perfect communication with man; that Jesus was the Son of God. That the relation of Son and Father presupposes equality of communication and mutuality of sympathetic understanding; that Jesus and the Father were one. That Jesus maintained at one and the same time understanding communication with both God and man, and that, since both God and man comprehended the meaning of the symbols of Jesus’ communication, both God and man possessed the attributes of personality in so far as the requirements of the ability of intercommunication were concerned. That the personality of Jesus demonstrated the personality of God, while it proved conclusively the presence of God in man. That two things which are related to the same thing are related to each other.
\vs p161 1:9 \ublistelem{4.}\bibnobreakspace That personality represents man’s highest concept of human reality and divine values; that God also represents man’s highest concept of divine reality and infinite values; therefore, that God must be a divine and infinite personality, a personality in reality although infinitely and eternally transcending man’s concept and definition of personality, but nevertheless always and universally a personality.
\vs p161 1:10 \ublistelem{5.}\bibnobreakspace That God must be a personality since he is the Creator of all personality and the destiny of all personality. Rodan had been tremendously influenced by the teaching of Jesus, \textcolour{ubdarkred}{“Be you therefore perfect, even as your Father in heaven is perfect.”}
\vs p161 1:11 \pc When Rodan heard these arguments, he said: “I am convinced. I will confess God as a person if you will permit me to qualify my confession of such a belief by attaching to the meaning of personality a group of extended values, such as superhuman, transcendent, supreme, infinite, eternal, final, and universal. I am now convinced that, while God must be infinitely more than a personality, he cannot be anything less. I am satisfied to end the argument and to accept Jesus as the personal revelation of the Father and the satisfaction of all unsatisfied factors in logic, reason, and philosophy.”
\usection{2.\bibnobreakspace The Divine Nature of Jesus}
\vs p161 2:1 Since Nathaniel and Thomas had so fully approved Rodan’s views of the gospel of the kingdom, there remained only one more point to consider, the teaching dealing with the divine nature of Jesus, a doctrine only so recently publicly announced. Nathaniel and Thomas jointly presented their views of the divine nature of the Master, and the following narrative is a condensed, rearranged, and restated presentation of their teaching:
\vs p161 2:2 \ublistelem{1.}\bibnobreakspace Jesus has admitted his divinity, and we believe him. Many remarkable things have happened in connection with his ministry which we can understand only by believing that he is the Son of God as well as the Son of Man.
\vs p161 2:3 \ublistelem{2.}\bibnobreakspace His life association with us exemplifies the ideal of human friendship; only a divine being could possibly be such a human friend. He is the most truly unselfish person we have ever known. He is the friend even of sinners; he dares to love his enemies. He is very loyal to us. While he does not hesitate to reprove us, it is plain to all that he truly loves us. The better you know him, the more you will love him. You will be charmed by his unswerving devotion. Through all these years of our failure to comprehend his mission, he has been a faithful friend. While he makes no use of flattery, he does treat us all with equal kindness; he is invariably tender and compassionate. He has shared his life and everything else with us. We are a happy community; we share all things in common. We do not believe that a mere human could live such a blameless life under such trying circumstances.
\vs p161 2:4 \ublistelem{3.}\bibnobreakspace We think Jesus is divine because he never does wrong; he makes no mistakes. His wisdom is extraordinary; his piety superb. He lives day by day in perfect accord with the Father’s will. He never repents of misdeeds because he transgresses none of the Father’s laws. He prays for us and with us, but he never asks us to pray for him. We believe that he is consistently sinless. We do not think that one who is only human ever professed to live such a life. He claims to live a perfect life, and we acknowledge that he does. Our piety springs from repentance, but his piety springs from righteousness. He even professes to forgive sins and does heal diseases. No mere man would sanely profess to forgive sin; that is a divine prerogative. And he has seemed to be thus perfect in his righteousness from the times of our first contact with him. We grow in grace and in the knowledge of the truth, but our Master exhibits maturity of righteousness to start with. All men, good and evil, recognize these elements of goodness in Jesus. And yet never is his piety obtrusive or ostentatious. He is both meek and fearless. He seems to approve of our belief in his divinity. He is either what he professes to be, or else he is the greatest hypocrite and fraud the world has ever known. We are persuaded that he is just what he claims to be.
\vs p161 2:5 \ublistelem{4.}\bibnobreakspace The uniqueness of his character and the perfection of his emotional control convince us that he is a combination of humanity and divinity. He unfailingly responds to the spectacle of human need; suffering never fails to appeal to him. His compassion is moved alike by physical suffering, mental anguish, or spiritual sorrow. He is quick to recognize and generous to acknowledge the presence of faith or any other grace in his fellow men. He is so just and fair and at the same time so merciful and considerate. He grieves over the spiritual obstinacy of the people and rejoices when they consent to see the light of truth.
\vs p161 2:6 \ublistelem{5.}\bibnobreakspace He seems to know the thoughts of men’s minds and to understand the longings of their hearts. And he is always sympathetic with our troubled spirits. He seems to possess all our human emotions, but they are magnificently glorified. He strongly loves goodness and equally hates sin. He possesses a superhuman consciousness of the presence of Deity. He prays like a man but performs like a God. He seems to foreknow things; he even now dares to speak about his death, some mystic reference to his future glorification. While he is kind, he is also brave and courageous. He never falters in doing his duty.
\vs p161 2:7 \ublistelem{6.}\bibnobreakspace We are constantly impressed by the phenomenon of his superhuman knowledge. Hardly does a day pass but something transpires to disclose that the Master knows what is going on away from his immediate presence. He also seems to know about the thoughts of his associates. He undoubtedly has communion with celestial personalities; he unquestionably lives on a spiritual plane far above the rest of us. Everything seems to be open to his unique understanding. He asks us questions to draw us out, not to gain information.
\vs p161 2:8 \ublistelem{7.}\bibnobreakspace Recently the Master does not hesitate to assert his superhumanity. From the day of our ordination as apostles right on down to recent times, he has never denied that he came from the Father above. He speaks with the authority of a divine teacher. The Master does not hesitate to refute the religious teachings of today and to declare the new gospel with positive authority. He is assertive, positive, and authoritative. Even John the Baptist, when he heard Jesus speak, declared that he was the Son of God. He seems to be so sufficient within himself. He craves not the support of the multitude; he is indifferent to the opinions of men. He is brave and yet so free from pride.
\vs p161 2:9 \ublistelem{8.}\bibnobreakspace He constantly talks about God as an ever\hyp{}present associate in all that he does. He goes about doing good, for God seems to be in him. He makes the most astounding assertions about himself and his mission on earth, statements which would be absurd if he were not divine. He once declared, \textcolour{ubdarkred}{“Before Abraham was, I am.”} He has definitely claimed divinity; he professes to be in partnership with God. He well\hyp{}nigh exhausts the possibilities of language in the reiteration of his claims of intimate association with the heavenly Father. He even dares to assert that he and the Father are one. He says that anyone who has seen him has seen the Father. And he says and does all these tremendous things with such childlike naturalness. He alludes to his association with the Father in the same manner that he refers to his association with us. He seems to be so sure about God and speaks of these relations in such a matter\hyp{}of\hyp{}fact way.\fnc{He says that \bibtextul{any one} who has seen him has seen the Father. \bibexpl{See note for \bibref[133:1.5]{p0133 1:5}.}}
\vs p161 2:10 \ublistelem{9.}\bibnobreakspace In his prayer life he appears to communicate directly with his Father. We have heard few of his prayers, but these few would indicate that he talks with God, as it were, face to face. He seems to know the future as well as the past. He simply could not be all of this and do all of these extraordinary things unless he were something more than human. We know he is human, we are sure of that, but we are almost equally sure that he is also divine. We believe that he is divine. We are convinced that he is the Son of Man and the Son of God.
\vs p161 2:11 \pc When Nathaniel and Thomas had concluded their conferences with Rodan, they hurried on toward Jerusalem to join their fellow apostles, arriving on Friday of that week. This had been a great experience in the lives of all three of these believers, and the other apostles learned much from the recounting of these experiences by Nathaniel and Thomas.
\vs p161 2:12 Rodan made his way back to Alexandria, where he long taught his philosophy in the school of Meganta. He became a mighty man in the later affairs of the kingdom of heaven; he was a faithful believer to the end of his earth days, yielding up his life in Greece with others when the persecutions were at their height.
\usection{3.\bibnobreakspace Jesus’ Human and Divine Minds}
\vs p161 3:1 Consciousness of divinity was a gradual growth in the mind of Jesus up to the occasion of his baptism. After he became fully self\hyp{}conscious of his divine nature, prehuman existence, and universe prerogatives, he seems to have possessed the power of variously limiting his human consciousness of his divinity. It appears to us that from his baptism until the crucifixion it was entirely optional with Jesus whether to depend only on the human mind or to utilize the knowledge of both the human and the divine minds. At times he appeared to avail himself of only that information which was resident in the human intellect. On other occasions he appeared to act with such fullness of knowledge and wisdom as could be afforded only by the utilization of the superhuman content of his divine consciousness.
\vs p161 3:2 We can understand his unique performances only by accepting the theory that he could, at will, self\hyp{}limit his divinity consciousness. We are fully cognizant that he frequently withheld from his associates his foreknowledge of events, and that he was aware of the nature of their thinking and planning. We understand that he did not wish his followers to know too fully that he was able to discern their thoughts and to penetrate their plans. He did not desire too far to transcend the concept of the human as it was held in the minds of his apostles and disciples.
\vs p161 3:3 We are utterly at a loss to differentiate between his practice of self\hyp{}limiting his divine consciousness and his technique of concealing his preknowledge and thought discernment from his human associates. We are convinced that he used both of these techniques, but we are not always able, in a given instance, to specify which method he may have employed. We frequently observed him acting with only the human content of consciousness; then would we behold him in conference with the directors of the celestial hosts of the universe and discern the undoubted functioning of the divine mind. And then on almost numberless occasions did we witness the working of this combined personality of man and God as it was activated by the apparent perfect union of the human and the divine minds. This is the limit of our knowledge of such phenomena; we really do not actually know the full truth about this mystery.
