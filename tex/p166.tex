\upaper{166}{Last Visit to Northern Perea}
\author{Midwayer Commission}
\vs p166 0:1 From February 11 to 20, Jesus and the 12 made a tour of all the cities and villages of northern Perea where the associates of Abner and the members of the women’s corps were working. They found these messengers of the gospel meeting with success, and Jesus repeatedly called the attention of his apostles to the fact that the gospel of the kingdom could spread without the accompaniment of miracles and wonders.
\vs p166 0:2 This entire mission of three months in Perea was successfully carried on with little help from the 12 apostles, and the gospel from this time on reflected, not so much Jesus’ personality, as his \bibemph{teachings.} But his followers did not long follow his instructions, for soon after Jesus’ death and resurrection they departed from his teachings and began to build the early church around the miraculous concepts and the glorified memories of his divine\hyp{}human personality.
\usection{1.\bibnobreakspace The Pharisees at Ragaba}
\vs p166 1:1 On Sabbath, February 18, Jesus was at Ragaba, where there lived a wealthy Pharisee named Nathaniel; and since quite a number of his fellow Pharisees were following Jesus and the 12 around the country, he made a breakfast on this Sabbath morning for all of them, about 20 in number, and invited Jesus as the guest of honour.
\vs p166 1:2 By the time Jesus arrived at this breakfast, most of the Pharisees, with two or three lawyers, were already there and seated at the table. The Master immediately took his seat at the left of Nathaniel without going to the water basins to wash his hands. Many of the Pharisees, especially those favourable to Jesus’ teachings, knew that he washed his hands only for purposes of cleanliness, that he abhorred these purely ceremonial performances; so they were not surprised at his coming directly to the table without having twice washed his hands. But Nathaniel was shocked by this failure of the Master to comply with the strict requirements of Pharisaic practice. Neither did Jesus wash his hands, as did the Pharisees, after each course of food nor at the end of the meal.
\vs p166 1:3 After considerable whispering between Nathaniel and an unfriendly Pharisee on his right and after much lifting of eyebrows and sneering curling of lips by those who sat opposite the Master, Jesus finally said: \textcolour{ubdarkred}{“I had thought that you invited me to this house to break bread with you and perchance to inquire of me concerning the proclamation of the new gospel of the kingdom of God; but I perceive that you have brought me here to witness an exhibition of ceremonial devotion to your own self\hyp{}righteousness. That service you have now done me; what next will you honour me with as your guest on this occasion?”}
\vs p166 1:4 When the Master had thus spoken, they cast their eyes upon the table and remained silent. And since no one spoke, Jesus continued: \textcolour{ubdarkred}{“Many of you Pharisees are here with me as friends, some are even my disciples, but the majority of the Pharisees are persistent in their refusal to see the light and acknowledge the truth, even when the work of the gospel is brought before them in great power. How carefully you cleanse the outside of the cups and the platters while the spiritual\hyp{}food vessels are filthy and polluted! You make sure to present a pious and holy appearance to the people, but your inner souls are filled with self\hyp{}righteousness, covetousness, extortion, and all manner of spiritual wickedness. Your leaders even dare to plot and plan the murder of the Son of Man. Do not you foolish men understand that the God of heaven looks at the inner motives of the soul as well as on your outer pretenses and your pious professions? Think not that the giving of alms and the paying of tithes will cleanse you from unrighteousness and enable you to stand clean in the presence of the Judge of all men. Woe upon you Pharisees who have persisted in rejecting the light of life! You are meticulous in tithing and ostentatious in almsgiving, but you knowingly spurn the visitation of God and reject the revelation of his love. Though it is all right for you to give attention to these minor duties, you should not have left these weightier requirements undone. Woe upon all who shun justice, spurn mercy, and reject truth! Woe upon all those who despise the revelation of the Father while they seek the chief seats in the synagogue and crave flattering salutations in the market places!”}
\vs p166 1:5 \pc When Jesus would have risen to depart, one of the lawyers who was at the table, addressing him, said: “But, Master, in some of your statements you reproach us also. Is there nothing good in the scribes, the Pharisees, or the lawyers?” And Jesus, standing, replied to the lawyer: \textcolour{ubdarkred}{“You, like the Pharisees, delight in the first places at the feasts and in wearing long robes while you put heavy burdens, grievous to be borne, on men’s shoulders. And when the souls of men stagger under these heavy burdens, you will not so much as lift with one of your fingers. Woe upon you who take your greatest delight in building tombs for the prophets your fathers killed! And that you consent to what your fathers did is made manifest when you now plan to kill those who come in this day doing what the prophets did in their day --- proclaiming the righteousness of God and revealing the mercy of the heavenly Father. But of all the generations that are past, the blood of the prophets and the apostles shall be required of this perverse and self\hyp{}righteous generation. Woe upon all of you lawyers who have taken away the key of knowledge from the common people! You yourselves refuse to enter into the way of truth, and at the same time you would hinder all others who seek to enter therein. But you cannot thus shut up the doors of the kingdom of heaven; these we have opened to all who have the faith to enter, and these portals of mercy shall not be closed by the prejudice and arrogance of false teachers and untrue shepherds who are like whited sepulchres which, while outwardly they appear beautiful, are inwardly full of dead men’s bones and all manner of spiritual uncleanness.”}
\vs p166 1:6 And when Jesus had finished speaking at Nathaniel’s table, he went out of the house without partaking of food. And of the Pharisees who heard these words, some became believers in his teaching and entered into the kingdom, but the larger number persisted in the way of darkness, becoming all the more determined to lie in wait for him that they might catch some of his words which could be used to bring him to trial and judgment before the Sanhedrin at Jerusalem.
\vs p166 1:7 \pc There were just three things to which the Pharisees paid particular attention:
\vs p166 1:8 \ublistelem{1.}\bibnobreakspace The practice of strict tithing.
\vs p166 1:9 \ublistelem{2.}\bibnobreakspace Scrupulous observance of the laws of purification.
\vs p166 1:10 \ublistelem{3.}\bibnobreakspace Avoidance of association with all non\hyp{}Pharisees.
\vs p166 1:11 \pc At this time Jesus sought to expose the spiritual barrenness of the first two practices, while he reserved his remarks designed to rebuke the Pharisees’ refusal to engage in social intercourse with non\hyp{}Pharisees for another and subsequent occasion when he would again be dining with many of these same men.
\usection{2.\bibnobreakspace The Ten Lepers}
\vs p166 2:1 The next day Jesus went with the 12 over to Amathus, near the border of Samaria, and as they approached the city, they encountered a group of ten lepers who sojourned near this place. Nine of this group were Jews, one a Samaritan. Ordinarily these Jews would have refrained from all association or contact with this Samaritan, but their common affliction was more than enough to overcome all religious prejudice. They had heard much of Jesus and his earlier miracles of healing, and since the 70 made a practice of announcing the time of Jesus’ expected arrival when the Master was out with the 12 on these tours, the ten lepers had been made aware that he was expected to appear in this vicinity at about this time; and they were, accordingly, posted here on the outskirts of the city where they hoped to attract his attention and ask for healing. When the lepers saw Jesus drawing near them, not daring to approach him, they stood afar off and cried to him: “Master, have mercy on us; cleanse us from our affliction. Heal us as you have healed others.”
\vs p166 2:2 Jesus had just been explaining to the 12 why the gentiles of Perea, together with the less orthodox Jews, were more willing to believe the gospel preached by the 70 than were the more orthodox and tradition\hyp{}bound Jews of Judea. He had called their attention to the fact that their message had likewise been more readily received by the Galileans, and even by the Samaritans. But the 12 apostles were hardly yet willing to entertain kind feelings for the long\hyp{}despised Samaritans.
\vs p166 2:3 Accordingly, when Simon Zelotes observed the Samaritan among the lepers, he sought to induce the Master to pass on into the city without even hesitating to exchange greetings with them. Said Jesus to Simon: \textcolour{ubdarkred}{“But what if the Samaritan loves God as well as the Jews? Should we sit in judgment on our fellow men? Who can tell? if we make these ten men whole, perhaps the Samaritan will prove more grateful even than the Jews. Do you feel certain about your opinions, Simon?”} And Simon quickly replied, “If you cleanse them, you will soon find out.” And Jesus replied: \textcolour{ubdarkred}{“So shall it be, Simon, and you will soon know the truth regarding the gratitude of men and the loving mercy of God.”}
\vs p166 2:4 Jesus, going near the lepers, said: \textcolour{ubdarkred}{“If you would be made whole, go forthwith and show yourselves to the priests as required by the law of Moses.”} And as they went, they were made whole. But when the Samaritan saw that he was being healed, he turned back and, going in quest of Jesus, began to glorify God with a loud voice. And when he had found the Master, he fell on his knees at his feet and gave thanks for his cleansing. The nine others, the Jews, had also discovered their healing, and while they also were grateful for their cleansing, they continued on their way to show themselves to the priests.
\vs p166 2:5 As the Samaritan remained kneeling at Jesus’ feet, the Master, looking about at the 12, especially at Simon Zelotes, said: \textcolour{ubdarkred}{“Were not ten cleansed? Where, then, are the other nine, the Jews? Only one, this alien, has returned to give glory to God.”} And then he said to the Samaritan, \textcolour{ubdarkred}{“Arise and go your way; your faith has made you whole.”}
\vs p166 2:6 Jesus looked again at his apostles as the stranger departed. And the apostles all looked at Jesus, save Simon Zelotes, whose eyes were downcast. The 12 said not a word. Neither did Jesus speak; it was not necessary that he should.
\vs p166 2:7 \pc Though all ten of these men really believed they had leprosy, only four were thus afflicted. The other six were cured of a skin disease which had been mistaken for leprosy. But the Samaritan really had leprosy.
\vs p166 2:8 \pc Jesus enjoined the 12 to say nothing about the cleansing of the lepers, and as they went on into Amathus, he remarked: \textcolour{ubdarkred}{“You see how it is that the children of the house, even when they are insubordinate to their Father’s will, take their blessings for granted. They think it a small matter if they neglect to give thanks when the Father bestows healing upon them, but the strangers, when they receive gifts from the head of the house, are filled with wonder and are constrained to give thanks in recognition of the good things bestowed upon them.”} And still the apostles said nothing in reply to the Master’s words.
\usection{3.\bibnobreakspace The Sermon at Gerasa}
\vs p166 3:1 As Jesus and the 12 visited with the messengers of the kingdom at Gerasa, one of the Pharisees who believed in him asked this question: “Lord, will there be few or many really saved?” And Jesus, answering, said:
\vs p166 3:2 \pc \textcolour{ubdarkred}{“You have been taught that only the children of Abraham will be saved; that only the gentiles of adoption can hope for salvation. Some of you have reasoned that, since the Scriptures record that only Caleb and Joshua from among all the hosts that went out of Egypt lived to enter the promised land, only a comparatively few of those who seek the kingdom of heaven shall find entrance thereto.}
\vs p166 3:3 \textcolour{ubdarkred}{“You also have another saying among you, and one that contains much truth: That the way which leads to eternal life is straight and narrow, that the door which leads thereto is likewise narrow so that, of those who seek salvation, few can find entrance through this door. You also have a teaching that the way which leads to destruction is broad, that the entrance thereto is wide, and that there are many who choose to go this way. And this proverb is not without its meaning. But I declare that salvation is first a matter of your personal choosing. Even if the door to the way of life is narrow, it is wide enough to admit all who sincerely seek to enter, for I am that door. And the Son will never refuse entrance to any child of the universe who, by faith, seeks to find the Father through the Son.}
\vs p166 3:4 “But herein is the danger to all who would postpone their entrance into the kingdom while they continue to pursue the pleasures of immaturity and indulge the satisfactions of selfishness: Having refused to enter the kingdom as a spiritual experience, they may subsequently seek entrance thereto when the glory of the better way becomes revealed in the age to come. And when, therefore, those who spurned the kingdom when I came in the likeness of humanity seek to find an entrance when it is revealed in the likeness of divinity, then will I say to all such selfish ones: I know not whence you are. You had your chance to prepare for this heavenly citizenship, but you refused all such proffers of mercy; you rejected all invitations to come while the door was open. Now, to you who have refused salvation, the door is shut. This door is not open to those who would enter the kingdom for selfish glory. Salvation is not for those who are unwilling to pay the price of wholehearted dedication to doing my Father’s will. When in spirit and soul you have turned your backs upon the Father’s kingdom, it is useless in mind and body to stand before this door and knock, saying, ‘Lord, open to us; we would also be great in the kingdom.’ Then will I declare that you are not of my fold. I will not receive you to be among those who have fought the good fight of faith and won the reward of unselfish service in the kingdom on earth. And when you say, ‘Did we not eat and drink with you, and did you not teach in our streets?’ then shall I again declare that you are spiritual strangers; that we were not fellow servants in the Father’s ministry of mercy on earth; that I do not know you; and then shall the Judge of all the earth say to you: ‘Depart from us, all you who have taken delight in the works of iniquity.’\fnc{\bibtextul{Lord} open to us; we would also be great in the kingdom. \bibexpl{In the original format, Lord was the last word in the line, making a dropped comma not unlikely. It is possible that the comma was simply viewed as unnecessary within such a short phrase, and it should also be noted that while the use of the comma in direct address is now regarded as standard, the \bibemph{Chicago Manual of Style} was silent on the matter until its 12\ts{th} edition (1969). The committee decided to adopt the modern format and insert the comma.}}
\vs p166 3:5 \textcolour{ubdarkred}{“But fear not; every one who sincerely desires to find eternal life by entrance into the kingdom of God shall certainly find such everlasting salvation. But you who refuse this salvation will some day see the prophets of the seed of Abraham sit down with the believers of the gentile nations in this glorified kingdom to partake of the bread of life and to refresh themselves with the water thereof. And they who shall thus take the kingdom in spiritual power and by the persistent assaults of living faith will come from the north and the south and from the east and the west. And, behold, many who are first will be last, and those who are last will many times be first.”}
\vs p166 3:6 This was indeed a new and strange version of the old and familiar proverb of the straight and narrow way.
\vs p166 3:7 Slowly the apostles and many of the disciples were learning the meaning of Jesus’ early declaration: \textcolour{ubdarkred}{“Unless you are born again, born of the spirit, you cannot enter the kingdom of God.”} Nevertheless, to all who are honest of heart and sincere in faith, it remains eternally true: \textcolour{ubdarkred}{“Behold, I stand at the doors of men’s hearts and knock, and if any man will open to me, I will come in and sup with him and will feed him with the bread of life; we shall be one in spirit and purpose, and so shall we ever be brethren in the long and fruitful service of the search for the Paradise Father.”} And so, whether few or many are to be saved altogether depends on whether few or many will heed the invitation: \textcolour{ubdarkred}{“I am the door, I am the new and living way, and whosoever wills may enter to embark upon the endless truth\hyp{}search for eternal life.”}
\vs p166 3:8 Even the apostles were unable fully to comprehend his teaching as to the necessity for using spiritual force for the purpose of breaking through all material resistance and for surmounting every earthly obstacle which might chance to stand in the way of grasping the all\hyp{}important spiritual values of the new life in the spirit as the liberated sons of God.
\usection{4.\bibnobreakspace Teaching about Accidents}
\vs p166 4:1 While most Palestinians ate only two meals a day, it was the custom of Jesus and the apostles, when on a journey, to pause at midday for rest and refreshment. And it was at such a noontide stop on the way to Philadelphia that Thomas asked Jesus: “Master, from hearing your remarks as we journeyed this morning, I would like to inquire whether spiritual beings are concerned in the production of strange and extraordinary events in the material world and, further, to ask whether the angels and other spirit beings are able to prevent accidents.”
\vs p166 4:2 \pc In answer to Thomas’s inquiry, Jesus said: \textcolour{ubdarkred}{“Have I been so long with you, and yet you continue to ask me such questions? Have you failed to observe how the Son of Man lives as one with you and consistently refuses to employ the forces of heaven for his personal sustenance? Do we not all live by the same means whereby all men exist? Do you see the power of the spiritual world manifested in the material life of this world, save for the revelation of the Father and the sometime healing of his afflicted children?}
\vs p166 4:3 \textcolour{ubdarkred}{“All too long have your fathers believed that prosperity was the token of divine approval; that adversity was the proof of God’s displeasure. I declare that such beliefs are superstitions. Do you not observe that far greater numbers of the poor joyfully receive the gospel and immediately enter the kingdom? If riches evidence divine favour, why do the rich so many times refuse to believe this good news from heaven?}
\vs p166 4:4 \textcolour{ubdarkred}{“The Father causes his rain to fall on the just and the unjust; the sun likewise shines on the righteous and the unrighteous. You know about those Galileans whose blood Pilate mingled with the sacrifices, but I tell you these Galileans were not in any manner sinners above all their fellows just because this happened to them. You also know about the 18 men upon whom the tower of Siloam fell, killing them. Think not that these men who were thus destroyed were offenders above all their brethren in Jerusalem. These folks were simply innocent victims of one of the accidents of time.}
\vs p166 4:5 \textcolour{ubdarkred}{“There are three groups of events which may occur in your lives:}
\vs p166 4:6 \textcolour{ubdarkred}{“\ublistelem{1.}\bibnobreakspace You may share in those normal happenings which are a part of the life you and your fellows live on the face of the earth.}
\vs p166 4:7 \textcolour{ubdarkred}{“\ublistelem{2.}\bibnobreakspace You may chance to fall victim to one of the accidents of nature, one of the mischances of men, knowing full well that such occurrences are in no way prearranged or otherwise produced by the spiritual forces of the realm.}
\vs p166 4:8 \textcolour{ubdarkred}{“\ublistelem{3.}\bibnobreakspace You may reap the harvest of your direct efforts to comply with the natural laws governing the world.}
\vs p166 4:9 \pc \textcolour{ubdarkred}{“There was a certain man who planted a fig tree in his yard, and when he had many times sought fruit thereon and found none, he called the vinedressers before him and said: ‘Here have I come these three seasons looking for fruit on this fig tree and have found none. Cut down this barren tree; why should it encumber the ground?’ But the head gardener answered his master: ‘Let it alone for one more year so that I may dig around it and put on fertilizer, and then, next year, if it bears no fruit, it shall be cut down.’ And when they had thus complied with the laws of fruitfulness, since the tree was living and good, they were rewarded with an abundant yield.}
\vs p166 4:10 \textcolour{ubdarkred}{“In the matter of sickness and health, you should know that these bodily states are the result of material causes; health is not the smile of heaven, neither is affliction the frown of God.}
\vs p166 4:11 \textcolour{ubdarkred}{“The Father’s human children have equal capacity for the reception of material blessings; therefore does he bestow things physical upon the children of men without discrimination. When it comes to the bestowal of spiritual gifts, the Father is limited by man’s capacity for receiving these divine endowments. Although the Father is no respecter of persons, in the bestowal of spiritual gifts he is limited by man’s faith and by his willingness always to abide by the Father’s will.”}
\vs p166 4:12 \pc As they journeyed on toward Philadelphia, Jesus continued to teach them and to answer their questions having to do with accidents, sickness, and miracles, but they were not able fully to comprehend this instruction. One hour of teaching will not wholly change the beliefs of a lifetime, and so Jesus found it necessary to reiterate his message, to tell again and again that which he wished them to understand; and even then they failed to grasp the meaning of his earth mission until after his death and resurrection.
\usection{5.\bibnobreakspace The Congregation at Philadelphia}
\vs p166 5:1 Jesus and the 12 were on their way to visit Abner and his associates, who were preaching and teaching in Philadelphia. Of all the cities of Perea, in Philadelphia the largest group of Jews and gentiles, rich and poor, learned and unlearned, embraced the teachings of the 70, thereby entering into the kingdom of heaven. The synagogue of Philadelphia had never been subject to the supervision of the Sanhedrin at Jerusalem and therefore had never been closed to the teachings of Jesus and his associates. At this very time, Abner was teaching three times a day in the Philadelphia synagogue.
\vs p166 5:2 This very synagogue later on became a Christian church and was the missionary headquarters for the promulgation of the gospel through the regions to the east. It was long a stronghold of the Master’s teachings and stood alone in this region as a centre of Christian learning for centuries.
\vs p166 5:3 The Jews at Jerusalem had always had trouble with the Jews of Philadelphia. And after the death and resurrection of Jesus the Jerusalem church, of which James the Lord’s brother was head, began to have serious difficulties with the Philadelphia congregation of believers. Abner became the head of the Philadelphia church, continuing as such until his death. And this estrangement with Jerusalem explains why nothing is heard of Abner and his work in the Gospel records of the New Testament. This feud between Jerusalem and Philadelphia lasted throughout the lifetimes of James and Abner and continued for some time after the destruction of Jerusalem. Philadelphia was really the headquarters of the early church in the south and east as Antioch was in the north and west.
\vs p166 5:4 \pc It was the apparent misfortune of Abner to be at variance with all of the leaders of the early Christian church. He fell out with Peter and James (Jesus’ brother) over questions of administration and the jurisdiction of the Jerusalem church; he parted company with Paul over differences of philosophy and theology. Abner was more Babylonian than Hellenic in his philosophy, and he stubbornly resisted all attempts of Paul to remake the teachings of Jesus so as to present less that was objectionable, first to the Jews, then to the Gr\ae co\hyp{}Roman believers in the mysteries.
\vs p166 5:5 Thus was Abner compelled to live a life of isolation. He was head of a church which was without standing at Jerusalem. He had dared to defy James the Lord’s brother, who was subsequently supported by Peter. Such conduct effectively separated him from all his former associates. Then he dared to withstand Paul. Although he was wholly sympathetic with Paul in his mission to the gentiles, and though he supported him in his contentions with the church at Jerusalem, he bitterly opposed the version of Jesus’ teachings which Paul elected to preach. In his last years Abner denounced Paul as the “clever corrupter of the life teachings of Jesus of Nazareth, the Son of the living God.”
\vs p166 5:6 During the later years of Abner and for some time thereafter, the believers at Philadelphia held more strictly to the religion of Jesus, as he lived and taught, than any other group on earth.
\vs p166 5:7 Abner lived to be 89 years old, dying at Philadelphia on the 21\ts{st} day of November, A.D.\,74. And to the very end he was a faithful believer in, and teacher of, the gospel of the heavenly kingdom.
\quizlink
