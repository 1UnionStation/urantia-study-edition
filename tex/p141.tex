\upaper{141}{Beginning the Public Work}
\author{Midwayer Commission}
\vs p141 0:1 On the first day of the week, January 19, A.D.\,27, Jesus and the 12 apostles made ready to depart from their headquarters in Bethsaida. The 12 knew nothing of their Master’s plans except that they were going up to Jerusalem to attend the Passover feast in April, and that it was the intention to journey by way of the Jordan valley. They did not get away from Zebedee’s house until near noon because the families of the apostles and others of the disciples had come to say good\hyp{}bye and wish them well in the new work they were about to begin.
\vs p141 0:2 Just before leaving, the apostles missed the Master, and Andrew went out to find him. After a brief search he found Jesus sitting in a boat down the beach, and he was weeping. The 12 had often seen their Master when he seemed to grieve, and they had beheld his brief seasons of serious preoccupation of mind, but none of them had ever seen him weep. Andrew was somewhat startled to see the Master thus affected on the eve of their departure for Jerusalem, and he ventured to approach Jesus and ask: “On this great day, Master, when we are to depart for Jerusalem to proclaim the Father’s kingdom, why is it that you weep? Which of us has offended you?” And Jesus, going back with Andrew to join the 12, answered him: \textcolour{ubdarkred}{“No one of you has grieved me. I am saddened only because none of my father Joseph’s family have remembered to come over to bid us Godspeed.”} At this time Ruth was on a visit to her brother Joseph at Nazareth. Other members of his family were kept away by pride, disappointment, misunderstanding, and petty resentment indulged as a result of hurt feelings.
\usection{1.\bibnobreakspace Leaving Galilee}
\vs p141 1:1 Capernaum was not far from Tiberias, and the fame of Jesus had begun to spread well over all of Galilee and even to parts beyond. Jesus knew that Herod would soon begin to take notice of his work; so he thought best to journey south and into Judea with his apostles. A company of over 100 believers desired to go with them, but Jesus spoke to them and besought them not to accompany the apostolic group on their way down the Jordan. Though they consented to remain behind, many of them followed after the Master within a few days.
\vs p141 1:2 The first day Jesus and the apostles only journeyed as far as Tarichea, where they rested for the night. The next day they travelled to a point on the Jordan near Pella where John had preached about one year before, and where Jesus had received baptism. Here they tarried for more than two weeks, teaching and preaching. By the end of the first week several hundred people had assembled in a camp near where Jesus and the 12 dwelt, and they had come from Galilee, Phoenicia, Syria, the Decapolis, Perea, and Judea.
\vs p141 1:3 Jesus did no public preaching. Andrew divided the multitude and assigned the preachers for the forenoon and afternoon assemblies; after the evening meal Jesus talked with the 12. He taught them nothing new but reviewed his former teaching and answered their many questions. On one of these evenings he told the 12 something about the 40 days which he spent in the hills near this place.
\vs p141 1:4 Many of those who came from Perea and Judea had been baptized by John and were interested in finding out more about Jesus’ teachings. The apostles made much progress in teaching the disciples of John inasmuch as they did not in any way detract from John’s preaching, and since they did not at this time even baptize their new disciples. But it was always a stumbling stone to John’s followers that Jesus, if he were all that John had announced, did nothing to get him out of prison. John’s disciples never could understand why Jesus did not prevent the cruel death of their beloved leader.
\vs p141 1:5 From night to night Andrew carefully instructed his fellow apostles in the delicate and difficult task of getting along smoothly with the followers of John the Baptist. During this first year of Jesus’ public ministry more than \bibfrac{3}{4} of his followers had previously followed John and had received his baptism. This entire year of A.D.\,27 was spent in quietly taking over John’s work in Perea and Judea.
\usection{2.\bibnobreakspace God’s Law and the Father’s Will}
\vs p141 2:1 The night before they left Pella, Jesus gave the apostles some further instruction with regard to the new kingdom. Said the Master: \textcolour{ubdarkred}{“You have been taught to look for the coming of the kingdom of God, and now I come announcing that this long\hyp{}looked\hyp{}for kingdom is near at hand, even that it is already here and in our midst. In every kingdom there must be a king seated upon his throne and decreeing the laws of the realm. And so have you developed a concept of the kingdom of heaven as a glorified rule of the Jewish people over all the peoples of the earth with Messiah sitting on David’s throne and from this place of miraculous power promulgating the laws of all the world. But, my children, you see not with the eye of faith, and you hear not with the understanding of the spirit. I declare that the kingdom of heaven is the realization and acknowledgement of God’s rule within the hearts of men. True, there is a King in this kingdom, and that King is my Father and your Father. We are indeed his loyal subjects, but far transcending that fact is the transforming truth that we are his \bibemph{sons.} In my life this truth is to become manifest to all. Our Father also sits upon a throne, but not one made with hands. The throne of the Infinite is the eternal dwelling place of the Father in the heaven of heavens; he fills all things and proclaims his laws to universes upon universes. And the Father also rules within the hearts of his children on earth by the spirit which he has sent to live within the souls of mortal men.}
\vs p141 2:2 \textcolour{ubdarkred}{“When you are the subjects of this kingdom, you indeed are made to hear the law of the Universe Ruler; but when, because of the gospel of the kingdom which I have come to declare, you faith\hyp{}discover yourselves as sons, you henceforth look not upon yourselves as law\hyp{}subject creatures of an all\hyp{}powerful king but as privileged sons of a loving and divine Father. Verily, verily, I say to you, when the Father’s will is your \bibemph{law,} you are hardly in the kingdom. But when the Father’s will becomes truly your \bibemph{will,} then are you in very truth in the kingdom because the kingdom has thereby become an established experience in you. When God’s will is your law, you are noble slave subjects; but when you believe in this new gospel of divine sonship, my Father’s will becomes your will, and you are elevated to the high position of the free children of God, liberated sons of the kingdom.”}
\vs p141 2:3 Some of the apostles grasped something of this teaching, but none of them comprehended the full significance of this tremendous announcement, unless it was James Zebedee. But these words sank into their hearts and came forth to gladden their ministry during later years of service.
\usection{3.\bibnobreakspace The Sojourn at Amathus}
\vs p141 3:1 The Master and his apostles remained near Amathus for almost three weeks. The apostles continued to preach twice daily to the multitude, and Jesus preached each Sabbath afternoon. It became impossible to continue the Wednesday playtime; so Andrew arranged that two apostles should rest each day of the six days in the week, while all were on duty during the Sabbath services.
\vs p141 3:2 Peter, James, and John did most of the public preaching. Philip, Nathaniel, Thomas, and Simon did much of the personal work and conducted classes for special groups of inquirers; the twins continued their general police supervision, while Andrew, Matthew, and Judas developed into a general managerial committee of three, although each of these three also did considerable religious work.
\vs p141 3:3 Andrew was much occupied with the task of adjusting the constantly recurring misunderstandings and disagreements between the disciples of John and the newer disciples of Jesus. Serious situations would arise every few days, but Andrew, with the assistance of his apostolic associates, managed to induce the contending parties to come to some sort of agreement, at least temporarily. Jesus refused to participate in any of these conferences; neither would he give any advice about the proper adjustment of these difficulties. He never once offered a suggestion as to how the apostles should solve these perplexing problems. When Andrew came to Jesus with these questions, he would always say: \textcolour{ubdarkred}{“It is not wise for the host to participate in the family troubles of his guests; a wise parent never takes sides in the petty quarrels of his own children.”}
\vs p141 3:4 \pc The Master displayed great wisdom and manifested perfect fairness in all of his dealings with his apostles and with all of his disciples. Jesus was truly a master of men; he exercised great influence over his fellow men because of the combined charm and force of his personality. There was a subtle commanding influence in his rugged, nomadic, and homeless life. There was intellectual attractiveness and spiritual drawing power in his authoritative manner of teaching, in his lucid logic, his strength of reasoning, his sagacious insight, his alertness of mind, his matchless poise, and his sublime tolerance. He was simple, manly, honest, and fearless. With all of this physical and intellectual influence manifest in the Master’s presence, there were also all those spiritual charms of being which have become associated with his personality --- patience, tenderness, meekness, gentleness, and humility.
\vs p141 3:5 Jesus of Nazareth was indeed a strong and forceful personality; he was an intellectual power and a spiritual stronghold. His personality not only appealed to the spiritually minded women among his followers, but also to the educated and intellectual Nicodemus and to the hardy Roman soldier, the captain stationed on guard at the cross, who, when he had finished watching the Master die, said, “Truly, this was a Son of God.” And red\hyp{}blooded, rugged Galilean fishermen called him Master.
\vs p141 3:6 The pictures of Jesus have been most unfortunate. These paintings of the Christ have exerted a deleterious influence on youth; the temple merchants would hardly have fled before Jesus if he had been such a man as your artists usually have depicted. His was a dignified manhood; he was good, but natural. Jesus did not pose as a mild, sweet, gentle, and kindly mystic. His teaching was thrillingly dynamic. He not only \bibemph{meant well,} but he went about actually \bibemph{doing good.}
\vs p141 3:7 The Master never said, “Come to me all you who are indolent and all who are dreamers.” But he did many times say, \textcolour{ubdarkred}{“Come to me all you who \bibemph{labour,} and I will give you rest --- spiritual strength.”} The Master’s yoke is, indeed, easy, but even so, he never imposes it; every individual must take this yoke of his own free will.
\vs p141 3:8 Jesus portrayed conquest by sacrifice, the sacrifice of pride and selfishness. By showing mercy, he meant to portray spiritual deliverance from all grudges, grievances, anger, and the lust for selfish power and revenge. And when he said, \textcolour{ubdarkred}{“Resist not evil,”} he later explained that he did not mean to condone sin or to counsel fraternity with iniquity. He intended the more to teach forgiveness, to \textcolour{ubdarkred}{“resist not evil treatment of one’s personality, evil injury to one’s feelings of personal dignity.”}
\usection{4.\bibnobreakspace Teaching about the Father}
\vs p141 4:1 While sojourning at Amathus, Jesus spent much time with the apostles instructing them in the new concept of God; again and again did he impress upon them that \bibemph{God is a Father,} not a great and supreme bookkeeper who is chiefly engaged in making damaging entries against his erring children on earth, recordings of sin and evil to be used against them when he subsequently sits in judgment upon them as the just Judge of all creation. The Jews had long conceived of God as a king over all, even as a Father of the nation, but never before had large numbers of mortal men held the idea of God as a loving Father of the \bibemph{individual.}
\vs p141 4:2 In answer to Thomas’s question, “Who is this God of the kingdom?” Jesus replied: \textcolour{ubdarkred}{“God is \bibemph{your} Father, and religion --- my gospel --- is nothing more nor less than the believing recognition of the truth that you are his son. And I am here among you in the flesh to make clear both of these ideas in my life and teachings.”}
\vs p141 4:3 Jesus also sought to free the minds of his apostles from the idea of offering animal sacrifices as a religious duty. But these men, trained in the religion of the daily sacrifice, were slow to comprehend what he meant. Nevertheless, the Master did not grow weary in his teaching. When he failed to reach the minds of all of the apostles by means of one illustration, he would restate his message and employ another type of parable for purposes of illumination.
\vs p141 4:4 \pc At this same time Jesus began to teach the 12 more fully concerning their mission \textcolour{ubdarkred}{“to comfort the afflicted and minister to the sick.”} The Master taught them much about the whole man --- the union of body, mind, and spirit to form the individual man or woman. Jesus told his associates about the three forms of affliction they would meet and went on to explain how they should minister to all who suffer the sorrows of human sickness. He taught them to recognize:
\vs p141 4:5 \ublistelem{1.}\bibnobreakspace Diseases of the flesh --- those afflictions commonly regarded as physical sickness.
\vs p141 4:6 \ublistelem{2.}\bibnobreakspace Troubled minds --- those nonphysical afflictions which were subsequently looked upon as emotional and mental difficulties and disturbances.
\vs p141 4:7 \ublistelem{3.}\bibnobreakspace The possession of evil spirits.
\vs p141 4:8 \pc Jesus explained to his apostles on several occasions the nature, and something concerning the origin, of these evil spirits, in that day often also called unclean spirits. The Master well knew the difference between the possession of evil spirits and insanity, but the apostles did not. Neither was it possible, in view of their limited knowledge of the early history of Urantia, for Jesus to undertake to make this matter fully understandable. But he many times said to them, alluding to these evil spirits: \textcolour{ubdarkred}{“They shall no more molest men when I shall have ascended to my Father in heaven, and after I shall have poured out my spirit upon all flesh in those times when the kingdom will come in great power and spiritual glory.”}
\vs p141 4:9 From week to week and from month to month, throughout this entire year, the apostles paid more and more attention to the healing ministry of the sick.
\usection{5.\bibnobreakspace Spiritual Unity}
\vs p141 5:1 One of the most eventful of all the evening conferences at Amathus was the session having to do with the discussion of spiritual unity. James Zebedee had asked, “Master, how shall we learn to see alike and thereby enjoy more harmony among ourselves?” When Jesus heard this question, he was stirred within his spirit, so much so that he replied: \textcolour{ubdarkred}{“James, James, when did I teach you that you should all see alike? I have come into the world to proclaim spiritual liberty to the end that mortals may be empowered to live individual lives of originality and freedom before God. I do not desire that social harmony and fraternal peace shall be purchased by the sacrifice of free personality and spiritual originality. What I require of you, my apostles, is \bibemph{spirit unity ---} and that you can experience in the joy of your united dedication to the wholehearted doing of the will of my Father in heaven. You do not have to see alike or feel alike or even think alike in order spiritually to \bibemph{be alike.} Spiritual unity is derived from the consciousness that each of you is indwelt, and increasingly dominated, by the spirit gift of the heavenly Father. Your apostolic harmony must grow out of the fact that the spirit hope of each of you is identical in origin, nature, and destiny.}
\vs p141 5:2 \textcolour{ubdarkred}{“In this way you may experience a perfected unity of spirit purpose and spirit understanding growing out of the mutual consciousness of the identity of each of your indwelling Paradise spirits; and you may enjoy all of this profound spiritual unity in the very face of the utmost diversity of your individual attitudes of intellectual thinking, temperamental feeling, and social conduct. Your personalities may be refreshingly diverse and markedly different, while your spiritual natures and spirit fruits of divine worship and brotherly love may be so unified that all who behold your lives will of a surety take cognizance of this spirit identity and soul unity; they will recognize that you have been with me and have thereby learned, and acceptably, how to do the will of the Father in heaven. You can achieve the unity of the service of God even while you render such service in accordance with the technique of your own original endowments of mind, body, and soul.}
\vs p141 5:3 \textcolour{ubdarkred}{“Your spirit unity implies two things, which always will be found to harmonize in the lives of individual believers: First, you are possessed with a common motive for life service; you all desire above everything to do the will of the Father in heaven. Second, you all have a common goal of existence; you all purpose to find the Father in heaven, thereby proving to the universe that you have become like him.”}
\vs p141 5:4 Many times during the training of the 12 Jesus reverted to this theme. Repeatedly he told them it was not his desire that those who believed in him should become dogmatized and standardized in accordance with the religious interpretations of even good men. Again and again he warned his apostles against the formulation of creeds and the establishment of traditions as a means of guiding and controlling believers in the gospel of the kingdom.
\usection{6.\bibnobreakspace Last Week at Amathus}
\vs p141 6:1 Near the end of the last week at Amathus, Simon Zelotes brought to Jesus one Teherma, a Persian doing business at Damascus. Teherma had heard of Jesus and had come to Capernaum to see him, and there learning that Jesus had gone with his apostles down the Jordan on the way to Jerusalem, he set out to find him. Andrew had presented Teherma to Simon for instruction. Simon looked upon the Persian as a “fire worshipper,” although Teherma took great pains to explain that fire was only the visible symbol of the Pure and Holy One. After talking with Jesus, the Persian signified his intention of remaining for several days to hear the teaching and listen to the preaching.
\vs p141 6:2 When Simon Zelotes and Jesus were alone, Simon asked the Master: “Why is it that I could not persuade him? Why did he so resist me and so readily lend an ear to you?” Jesus answered: \textcolour{ubdarkred}{“Simon, Simon, how many times have I instructed you to refrain from all efforts to take something \bibemph{out} of the hearts of those who seek salvation? How often have I told you to labour only to put something \bibemph{into} these hungry souls? Lead men into the kingdom, and the great and living truths of the kingdom will presently drive out all serious error. When you have presented to mortal man the good news that God is his Father, you can the easier persuade him that he is in reality a son of God. And having done that, you have brought the light of salvation to the one who sits in darkness. Simon, when the Son of Man came first to you, did he come denouncing Moses and the prophets and proclaiming a new and better way of life? No. I came not to take away that which you had from your forefathers but to show you the perfected vision of that which your fathers saw only in part. Go then, Simon, teaching and preaching the kingdom, and when you have a man safely and securely within the kingdom, then is the time, when such a one shall come to you with inquiries, to impart instruction having to do with the progressive advancement of the soul within the divine kingdom.”}
\vs p141 6:3 Simon was astonished at these words, but he did as Jesus had instructed him, and Teherma, the Persian, was numbered among those who entered the kingdom.
\vs p141 6:4 \pc That night Jesus discoursed to the apostles on the new life in the kingdom. He said in part: \textcolour{ubdarkred}{“When you enter the kingdom, you are reborn. You cannot teach the deep things of the spirit to those who have been born only of the flesh; first see that men are born of the spirit before you seek to instruct them in the advanced ways of the spirit. Do not undertake to show men the beauties of the temple until you have first taken them into the temple. Introduce men to God and \bibemph{as} the sons of God before you discourse on the doctrines of the fatherhood of God and the sonship of men. Do not strive with men --- always be patient. It is not your kingdom; you are only ambassadors. Simply go forth proclaiming: This is the kingdom of heaven --- God is your Father and you are his sons, and this good news, if you wholeheartedly believe it, \bibemph{is} your eternal salvation.”}
\vs p141 6:5 The apostles made great progress during the sojourn at Amathus. But they were very much disappointed that Jesus would give them no suggestions about dealing with John’s disciples. Even in the important matter of baptism, all that Jesus said was: \textcolour{ubdarkred}{“John did indeed baptize with water, but when you enter the kingdom of heaven, you shall be baptized with the Spirit.”}
\usection{7.\bibnobreakspace At Bethany Beyond Jordan}
\vs p141 7:1 On February 26, Jesus, his apostles, and a large group of followers journeyed down the Jordan to the ford near Bethany in Perea, the place where John first made proclamation of the coming kingdom. Jesus with his apostles remained here, teaching and preaching, for four weeks before they went on up to Jerusalem.
\vs p141 7:2 The second week of the sojourn at Bethany beyond Jordan, Jesus took Peter, James, and John into the hills across the river and south of Jericho for a three days’ rest. The Master taught these three many new and advanced truths about the kingdom of heaven. For the purpose of this record we will reorganize and classify these teachings as follows:
\vs p141 7:3 \pc Jesus endeavoured to make clear that he desired his disciples, having tasted of the good spirit realities of the kingdom, so to live in the world that men, by \bibemph{seeing} their lives, would become kingdom conscious and hence be led to inquire of believers concerning the ways of the kingdom. All such sincere seekers for the truth are always glad to \bibemph{hear} the glad tidings of the faith gift which ensures admission to the kingdom with its eternal and divine spirit realities.
\vs p141 7:4 The Master sought to impress upon all teachers of the gospel of the kingdom that their only business was to reveal God to the individual man as his Father --- to lead this individual man to become son\hyp{}conscious; then to present this same man to God as his faith son. Both of these essential revelations are accomplished in Jesus. He became, indeed, \textcolour{ubdarkred}{“the way, the truth, and the life.”} The religion of Jesus was wholly based on the living of his bestowal life on earth. When Jesus departed from this world, he left behind no books, laws, or other forms of human organization affecting the religious life of the individual.
\vs p141 7:5 Jesus made it plain that he had come to establish personal and eternal relations with men which should forever take precedence over all other human relationships. And he emphasized that this intimate spiritual fellowship was to be extended to all men of all ages and of all social conditions among all peoples. The only reward which he held out for his children was: in this world --- spiritual joy and divine communion; in the next world --- eternal life in the progress of the divine spirit realities of the Paradise Father.
\vs p141 7:6 Jesus laid great emphasis upon what he called the two truths of first import in the teachings of the kingdom, and they are: the attainment of salvation by faith, and faith alone, associated with the revolutionary teaching of the attainment of human liberty through the sincere recognition of truth, \textcolour{ubdarkred}{“You shall know the truth, and the truth shall make you free.”} Jesus was the truth made manifest in the flesh, and he promised to send his Spirit of Truth into the hearts of all his children after his return to the Father in heaven.
\vs p141 7:7 The Master was teaching these apostles the essentials of truth for an entire age on earth. They often listened to his teachings when in reality what he said was intended for the inspiration and edification of other worlds. He exemplified a new and original plan of life. From the human standpoint he was indeed a Jew, but he lived his life for all the world as a mortal of the realm.
\vs p141 7:8 To ensure the recognition of his Father in the unfolding of the plan of the kingdom, Jesus explained that he had purposely ignored the “great men of earth.” He began his work with the poor, the very class which had been so neglected by most of the evolutionary religions of preceding times. He despised no man; his plan was world\hyp{}wide, even universal. He was so bold and emphatic in these announcements that even Peter, James, and John were tempted to think he might possibly be beside himself.
\vs p141 7:9 He sought mildly to impart to these apostles the truth that he had come on this bestowal mission, not to set an example for a few earth creatures, but to establish and demonstrate a standard of human life for all peoples upon all worlds throughout his entire universe. And this standard approached the highest perfection, even the final goodness of the Universal Father. But the apostles could not grasp the meaning of his words.
\vs p141 7:10 He announced that he had come to function as a teacher, a teacher sent from heaven to present spiritual truth to the material mind. And this is exactly what he did; he was a teacher, not a preacher. From the human viewpoint Peter was a much more effective preacher than Jesus. Jesus’ preaching was so effective because of his unique personality, not so much because of compelling oratory or emotional appeal. Jesus spoke directly to men’s souls. He was a teacher of man’s spirit, but through the mind. He lived with men.
\vs p141 7:11 It was on this occasion that Jesus intimated to Peter, James, and John that his work on earth was in some respects to be limited by the commission of his “associate on high,” referring to the prebestowal instructions of his Paradise brother, Immanuel. He told them that he had come to do his Father’s will and only his Father’s will. Being thus motivated by a wholehearted singleness of purpose, he was not anxiously bothered by the evil in the world.
\vs p141 7:12 The apostles were beginning to recognize the unaffected friendliness of Jesus. Though the Master was easy of approach, he always lived independent of, and above, all human beings. Not for one moment was he ever dominated by any purely mortal influence or subject to frail human judgment. He paid no attention to public opinion, and he was uninfluenced by praise. He seldom paused to correct misunderstandings or to resent misrepresentation. He never asked any man for advice; he never made requests for prayers.
\vs p141 7:13 James was astonished at how Jesus seemed to see the end from the beginning. The Master rarely appeared to be surprised. He was never excited, vexed, or disconcerted. He never apologized to any man. He was at times saddened, but never discouraged.
\vs p141 7:14 More clearly John recognized that, notwithstanding all of his divine endowments, after all, he was human. Jesus lived as a man among men and understood, loved, and knew how to manage men. In his personal life he was so human, and yet so faultless. And he was always unselfish.
\vs p141 7:15 Although Peter, James, and John could not understand very much of what Jesus said on this occasion, his gracious words lingered in their hearts, and after the crucifixion and resurrection they came forth greatly to enrich and gladden their subsequent ministry. No wonder these apostles did not fully comprehend the Master’s words, for he was projecting to them the plan of a new age.
\usection{8.\bibnobreakspace Working in Jericho}
\vs p141 8:1 Throughout the four weeks’ sojourn at Bethany beyond Jordan, several times each week Andrew would assign apostolic couples to go up to Jericho for a day or two. John had many believers in Jericho, and the majority of them welcomed the more advanced teachings of Jesus and his apostles. On these Jericho visits the apostles began more specifically to carry out Jesus’ instructions to minister to the sick; they visited every house in the city and sought to comfort every afflicted person.
\vs p141 8:2 The apostles did some public work in Jericho, but their efforts were chiefly of a more quiet and personal nature. They now made the discovery that the good news of the kingdom was very comforting to the sick; that their message carried healing for the afflicted. And it was in Jericho that Jesus’ commission to the 12 to preach the glad tidings of the kingdom and minister to the afflicted was first fully carried into effect.
\vs p141 8:3 They stopped in Jericho on the way up to Jerusalem and were overtaken by a delegation from Mesopotamia that had come to confer with Jesus. The apostles had planned to spend but a day here, but when these truth seekers from the East arrived, Jesus spent three days with them, and they returned to their various homes along the Euphrates happy in the knowledge of the new truths of the kingdom of heaven.
\usection{9.\bibnobreakspace Departing for Jerusalem}
\vs p141 9:1 On Monday, the last day of March, Jesus and the apostles began their journey up the hills toward Jerusalem. Lazarus of Bethany had been down to the Jordan twice to see Jesus, and every arrangement had been made for the Master and his apostles to make their headquarters with Lazarus and his sisters at Bethany as long as they might desire to stay in Jerusalem.
\vs p141 9:2 The disciples of John remained at Bethany beyond the Jordan, teaching and baptizing the multitudes, so that Jesus was accompanied only by the 12 when he arrived at Lazarus’s home. Here Jesus and the apostles tarried for five days, resting and refreshing themselves before going on to Jerusalem for the Passover. It was a great event in the lives of Martha and Mary to have the Master and his apostles in the home of their brother, where they could minister to their needs.
\vs p141 9:3 On Sunday morning, April 6, Jesus and the apostles went down to Jerusalem; and this was the first time the Master and all of the 12 had been there together.
\quizlink
