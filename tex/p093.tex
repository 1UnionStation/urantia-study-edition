\upaper{93}{Machiventa Melchizedek}
\uminitoc{The Machiventa Incarnation}
\uminitoc{The Sage of Salem}
\uminitoc{Melchizedek’s Teachings}
\uminitoc{The Salem Religion}
\uminitoc{The Selection of Abraham}
\uminitoc{Melchizedek’s Covenant with Abraham}
\uminitoc{The Melchizedek Missionaries}
\uminitoc{Departure of Melchizedek}
\uminitoc{After Melchizedek’s Departure}
\uminitoc{Present Status of Machiventa Melchizedek}
\author{Melchizedek}
\vs p093 0:1 The Melchizedeks are widely known as emergency Sons, for they engage in an amazing range of activities on the worlds of a local universe. When any extraordinary problem arises, or when something unusual is to be attempted, it is quite often a Melchizedek who accepts the assignment. The ability of the Melchizedek Sons to function in emergencies and on widely divergent levels of the universe, even on the physical level of personality manifestation, is peculiar to their order. Only the Life Carriers share to any degree this metamorphic range of personality function.
\vs p093 0:2 \pc The Melchizedek order of universe sonship has been exceedingly active on Urantia. A corps of twelve served in conjunction with the Life Carriers. A later corps of twelve became receivers for your world shortly after the Caligastia secession and continued in authority until the time of Adam and Eve. These twelve Melchizedeks returned to Urantia upon the default of Adam and Eve, and they continued thereafter as planetary receivers on down to the day when Jesus of Nazareth, as the Son of Man, became the titular Planetary Prince of Urantia.
\usection{The Machiventa Incarnation}
\vs p093 1:1 Revealed truth was threatened with extinction during the millenniums which followed the miscarriage of the Adamic mission on Urantia. Though making progress intellectually, the human races were slowly losing ground spiritually. About 3000\,B.C. the concept of God had grown very hazy in the minds of men.
\vs p093 1:2 The twelve Melchizedek receivers knew of Michael’s impending bestowal on their planet, but they did not know how soon it would occur; therefore they convened in solemn council and petitioned the Most Highs of Edentia that some provision be made for maintaining the light of truth on Urantia. This plea was dismissed with the mandate that “the conduct of affairs on 606 of Satania is fully in the hands of the Melchizedek custodians.” The receivers then appealed to the Father Melchizedek for help but only received word that they should continue to uphold truth in the manner of their own election “until the arrival of a bestowal Son,” who “would rescue the planetary titles from forfeiture and uncertainty.”
\vs p093 1:3 And it was in consequence of having been thrown so completely on their own resources that Machiventa Melchizedek, one of the twelve planetary receivers, volunteered to do that which had been done only six times in all the history of Nebadon: to personalize on earth as a temporary man of the realm, to bestow himself as an emergency Son of world ministry. Permission was granted for this adventure by the Salvington authorities, and the actual incarnation of Machiventa Melchizedek was consummated near what was to become the city of Salem, in Palestine. The entire transaction of the materialization of this Melchizedek Son was completed by the planetary receivers with the co\hyp{}operation of the Life Carriers, certain of the Master Physical Controllers, and other celestial personalities resident on Urantia.
\usection{The Sage of Salem}
\vs p093 2:1 It was 1,973 years before the birth of Jesus that Machiventa was bestowed upon the human races of Urantia. His coming was unspectacular; his materialization was not witnessed by human eyes. He was first observed by mortal man on that eventful day when he entered the tent of Amdon, a Chaldean herder of Sumerian extraction. And the proclamation of his mission was embodied in the simple statement which he made to this shepherd, “I am Melchizedek, priest of El Elyon, the Most High, the one and only God.”
\vs p093 2:2 When the herder had recovered from his astonishment, and after he had plied this stranger with many questions, he asked Melchizedek to sup with him, and this was the first time in his long universe career that Machiventa had partaken of material food, the nourishment which was to sustain him throughout his 94 years of life as a material being.
\vs p093 2:3 And that night, as they talked out under the stars, Melchizedek began his mission of the revelation of the truth of the reality of God when, with a sweep of his arm, he turned to Amdon, saying, “El Elyon, the Most High, is the divine creator of the stars of the firmament and even of this very earth on which we live, and he is also the supreme God of heaven.”
\vs p093 2:4 \pc Within a few years Melchizedek had gathered around himself a group of pupils, disciples, and believers who formed the nucleus of the later community of Salem. He was soon known throughout Palestine as the priest of El Elyon, the Most High, and as the sage of Salem. Among some of the surrounding tribes he was often referred to as the sheik, or king, of Salem. Salem was the site which after the disappearance of Melchizedek became the city of Jebus, subsequently being called Jerusalem.
\vs p093 2:5 \pc In personal appearance, Melchizedek resembled the then blended Nodite and Sumerian peoples, being almost 1.8\,m in height and possessing a commanding presence. He spoke Chaldean and a half dozen other languages. He dressed much as did the Canaanite priests except that on his breast he wore an emblem of three concentric circles, the Satania symbol of the Paradise Trinity. In the course of his ministry this insignia of three concentric circles became regarded as so sacred by his followers that they never dared to use it, and it was soon forgotten with the passing of a few generations.
\vs p093 2:6 Though Machiventa lived after the manner of the men of the realm, he never married, nor could he have left offspring on earth. His physical body, while resembling that of the human male, was in reality on the order of those especially constructed bodies used by the 100 materialized members of Prince Caligastia’s staff except that it did not carry the life plasm of any human race. Nor was there available on Urantia the tree of life. Had Machiventa remained for any long period on earth, his physical mechanism would have gradually deteriorated; as it was, he terminated his bestowal mission in 94 years long before his material body had begun to disintegrate.
\vs p093 2:7 \pc This incarnated Melchizedek received a Thought Adjuster, who indwelt his superhuman personality as the monitor of time and the mentor of the flesh, thus gaining that experience and practical introduction to Urantian problems and to the technique of indwelling an incarnated Son which enabled this spirit of the Father to function so valiantly in the human mind of the later Son of God, Michael, when he appeared on earth in the likeness of mortal flesh. And this is the only Thought Adjuster who ever functioned in two minds on Urantia, but both minds were divine as well as human.
\vs p093 2:8 During the incarnation in the flesh, Machiventa was in full contact with his eleven fellows of the corps of planetary custodians, but he could not communicate with other orders of celestial personalities. Aside from the Melchizedek receivers, he had no more contact with superhuman intelligences than a human being.
\usection{Melchizedek’s Teachings}
\vs p093 3:1 With the passing of a decade, Melchizedek organized his schools at Salem, patterning them on the olden system which had been developed by the early Sethite priests of the second Eden. Even the idea of a tithing system, which was introduced by his later convert Abraham, was also derived from the lingering traditions of the methods of the ancient Sethites.
\vs p093 3:2 Melchizedek taught the concept of one God, a universal Deity, but he allowed the people to associate this teaching with the Constellation Father of Norlatiadek, whom he termed El Elyon --- the Most High. Melchizedek remained all but silent as to the status of Lucifer and the state of affairs on Jerusem. Lanaforge, the System Sovereign, had little to do with Urantia until after the completion of Michael’s bestowal. To a majority of the Salem students Edentia was heaven and the Most High was God.
\vs p093 3:3 The symbol of the three concentric circles, which Melchizedek adopted as the insignia of his bestowal, a majority of the people interpreted as standing for the three kingdoms of men, angels, and God. And they were allowed to continue in that belief; very few of his followers ever knew that these three circles were emblematic of the infinity, eternity, and universality of the Paradise Trinity of divine maintenance and direction; even Abraham rather regarded this symbol as standing for the three Most Highs of Edentia, as he had been instructed that the three Most Highs functioned as one. To the extent that Melchizedek taught the Trinity concept symbolized in his insignia, he usually associated it with the three Vorondadek rulers of the constellation of Norlatiadek.
\vs p093 3:4 To the rank and file of his followers he made no effort to present teaching beyond the fact of the rulership of the Most Highs of Edentia --- Gods of Urantia. But to some, Melchizedek taught advanced truth, embracing the conduct and organization of the local universe, while to his brilliant disciple Nordan the Kenite and his band of earnest students he taught the truths of the superuniverse and even of Havona.
\vs p093 3:5 The members of the family of Katro, with whom Melchizedek lived for more than 30 years, knew many of these higher truths and long perpetuated them in their family, even to the days of their illustrious descendant Moses, who thus had a compelling tradition of the days of Melchizedek handed down to him on this, his father’s side, as well as through other sources on his mother’s side.
\vs p093 3:6 Melchizedek taught his followers all they had capacity to receive and assimilate. Even many modern religious ideas about heaven and earth, of man, God, and angels, are not far removed from these teachings of Melchizedek. But this great teacher subordinated everything to the doctrine of one God, a universe Deity, a heavenly Creator, a divine Father. Emphasis was placed upon this teaching for the purpose of appealing to man’s adoration and of preparing the way for the subsequent appearance of Michael as the Son of this same Universal Father.
\vs p093 3:7 Melchizedek taught that at some future time another Son of God would come in the flesh as he had come, but that he would be born of a woman; and that is why numerous later teachers held that Jesus was a priest, or minister, “forever after the order of Melchizedek.”
\vs p093 3:8 And thus did Melchizedek prepare the way and set the monotheistic stage of world tendency for the bestowal of an actual Paradise Son of the one God, whom he so vividly portrayed as the Father of all, and whom he represented to Abraham as a God who would accept man on the simple terms of personal faith. And Michael, when he appeared on earth, confirmed all that Melchizedek had taught concerning the Paradise Father.
\usection{The Salem Religion}
\vs p093 4:1 The ceremonies of the Salem worship were very simple. Every person who signed or marked the clay\hyp{}tablet rolls of the Melchizedek church committed to memory, and subscribed to, the following belief:
\vs p093 4:2 \ublistelem{1.}\bibnobreakspace I believe in El Elyon, the Most High God, the only Universal Father and Creator of all things.
\vs p093 4:3 \ublistelem{2.}\bibnobreakspace I accept the Melchizedek covenant with the Most High, which bestows the favour of God on my faith, not on sacrifices and burnt offerings.
\vs p093 4:4 \ublistelem{3.}\bibnobreakspace I promise to obey the seven commandments of Melchizedek and to tell the good news of this covenant with the Most High to all men.
\vs p093 4:5 \pc And that was the whole of the creed of the Salem colony. But even such a short and simple declaration of faith was altogether too much and too advanced for the men of those days. They simply could not grasp the idea of getting divine favour for nothing --- by faith. They were too deeply confirmed in the belief that man was born under forfeit to the gods. Too long and too earnestly had they sacrificed and made gifts to the priests to be able to comprehend the good news that salvation, divine favour, was a free gift to all who would believe in the Melchizedek covenant. But Abraham did believe half\hyp{}heartedly, and even that was “counted for righteousness.”
\vs p093 4:6 \pc The seven commandments promulgated by Melchizedek were patterned along the lines of the ancient Dalamatian supreme law and very much resembled the seven commands taught in the first and second Edens. These commands of the Salem religion were:
\vs p093 4:7 \ublistelem{1.}\bibnobreakspace You shall not serve any God but the Most High Creator of heaven and earth.
\vs p093 4:8 \ublistelem{2.}\bibnobreakspace You shall not doubt that faith is the only requirement for eternal salvation.
\vs p093 4:9 \ublistelem{3.}\bibnobreakspace You shall not bear false witness.
\vs p093 4:10 \ublistelem{4.}\bibnobreakspace You shall not kill.
\vs p093 4:11 \ublistelem{5.}\bibnobreakspace You shall not steal.
\vs p093 4:12 \ublistelem{6.}\bibnobreakspace You shall not commit adultery.
\vs p093 4:13 \ublistelem{7.}\bibnobreakspace You shall not show disrespect for your parents and elders.
\vs p093 4:14 \pc While no sacrifices were permitted within the colony, Melchizedek well knew how difficult it is to suddenly uproot long\hyp{}established customs and accordingly had wisely offered these people the substitute of a sacrament of bread and wine for the older sacrifice of flesh and blood. It is of record, “Melchizedek, king of Salem, brought forth bread and wine.” But even this cautious innovation was not altogether successful; the various tribes all maintained auxiliary centres on the outskirts of Salem where they offered sacrifices and burnt offerings. Even Abraham resorted to this barbarous practice after his victory over Chedorlaomer; he simply did not feel quite at ease until he had offered a conventional sacrifice. And Melchizedek never did succeed in fully eradicating this proclivity to sacrifice from the religious practices of his followers, even of Abraham.
\vs p093 4:15 Like Jesus, Melchizedek attended strictly to the fulfilment of the mission of his bestowal. He did not attempt to reform the mores, to change the habits of the world, nor to promulgate even advanced sanitary practices or scientific truths. He came to achieve two tasks: to keep alive on earth the truth of the one God and to prepare the way for the subsequent mortal bestowal of a Paradise Son of that Universal Father.
\vs p093 4:16 \pc Melchizedek taught elementary revealed truth at Salem for 94 years, and during this time Abraham attended the Salem school three different times. He finally became a convert to the Salem teachings, becoming one of Melchizedek’s most brilliant pupils and chief supporters.
\usection{The Selection of Abraham}
\vs p093 5:1 Although it may be an error to speak of “chosen people,” it is not a mistake to refer to Abraham as a chosen individual. Melchizedek did lay upon Abraham the responsibility of keeping alive the truth of one God as distinguished from the prevailing belief in plural deities.
\vs p093 5:2 The choice of Palestine as the site for Machiventa’s activities was in part predicated upon the desire to establish contact with some human family embodying the potentials of leadership. At the time of the incarnation of Melchizedek there were many families on earth just as well prepared to receive the doctrine of Salem as was that of Abraham. There were equally endowed families among the red men, the yellow men, and the descendants of the Andites to the west and north. But, again, none of these localities were so favourably situated for Michael’s subsequent appearance on earth as was the eastern shore of the Mediterranean Sea. The Melchizedek mission in Palestine and the subsequent appearance of Michael among the Hebrew people were in no small measure determined by geography, by the fact that Palestine was centrally located with reference to the then existent trade, travel, and civilization of the world.
\vs p093 5:3 For some time the Melchizedek receivers had been observing the ancestors of Abraham, and they confidently expected offspring in a certain generation who would be characterized by intelligence, initiative, sagacity, and sincerity. The children of Terah, the father of Abraham, in every way met these expectations. It was this possibility of contact with these versatile children of Terah that had considerable to do with the appearance of Machiventa at Salem, rather than in Egypt, China, India, or among the northern tribes.
\vs p093 5:4 Terah and his whole family were half\hyp{}hearted converts to the Salem religion, which had been preached in Chaldea; they learned of Melchizedek through the preaching of Ovid, a Phoenician teacher who proclaimed the Salem doctrines in Ur. They left Ur intending to go directly through to Salem, but Nahor, Abraham’s brother, not having seen Melchizedek, was lukewarm and persuaded them to tarry at Haran. And it was a long time after they arrived in Palestine before they were willing to destroy \bibemph{all} of the household gods they had brought with them; they were slow to give up the many gods of Mesopotamia for the one God of Salem.
\vs p093 5:5 A few weeks after the death of Abraham’s father, Terah, Melchizedek sent one of his students, Jaram the Hittite, to extend this invitation to both Abraham and Nahor: “Come to Salem, where you shall hear our teachings of the truth of the eternal Creator, and in the enlightened offspring of you two brothers shall all the world be blessed.” Now Nahor had not wholly accepted the Melchizedek gospel; he remained behind and built up a strong city\hyp{}state which bore his name; but Lot, Abraham’s nephew, decided to go with his uncle to Salem.
\vs p093 5:6 Upon arriving at Salem, Abraham and Lot chose a hilly fastness near the city where they could defend themselves against the many surprise attacks of northern raiders. At this time the Hittites, Assyrians, Philistines, and other groups were constantly raiding the tribes of central and southern Palestine. From their stronghold in the hills Abraham and Lot made frequent pilgrimages to Salem.
\vs p093 5:7 \pc Not long after they had established themselves near Salem, Abraham and Lot journeyed to the valley of the Nile to obtain food supplies as there was then a drought in Palestine. During his brief sojourn in Egypt Abraham found a distant relative on the Egyptian throne, and he served as the commander of two very successful military expeditions for this king. During the latter part of his sojourn on the Nile he and his wife, Sarah, lived at court, and when leaving Egypt, he was given a share of the spoils of his military campaigns.
\vs p093 5:8 It required great determination for Abraham to forgo the honours of the Egyptian court and return to the more spiritual work sponsored by Machiventa. But Melchizedek was revered even in Egypt, and when the full story was laid before Pharaoh, he strongly urged Abraham to return to the execution of his vows to the cause of Salem.
\vs p093 5:9 \pc Abraham had kingly ambitions, and on the way back from Egypt he laid before Lot his plan to subdue all Canaan and bring its people under the rule of Salem. Lot was more bent on business; so, after a later disagreement, he went to Sodom to engage in trade and animal husbandry. Lot liked neither a military nor a herder’s life.
\vs p093 5:10 Upon returning with his family to Salem, Abraham began to mature his military projects. He was soon recognized as the civil ruler of the Salem territory and had confederated under his leadership seven near\hyp{}by tribes. Indeed, it was with great difficulty that Melchizedek restrained Abraham, who was fired with a zeal to go forth and round up the neighbouring tribes with the sword that they might thus more quickly be brought to a knowledge of the Salem truths.
\vs p093 5:11 Melchizedek maintained peaceful relations with all the surrounding tribes; he was not militaristic and was never attacked by any of the armies as they moved back and forth. He was entirely willing that Abraham should formulate a defensive policy for Salem such as was subsequently put into effect, but he would not approve of his pupil’s ambitious schemes for conquest; so there occurred a friendly severance of relationship, Abraham going over to Hebron to establish his military capital.
\vs p093 5:12 Abraham, because of his close connection with the illustrious Melchizedek, possessed great advantage over the surrounding petty kings; they all revered Melchizedek and unduly feared Abraham. Abraham knew of this fear and only awaited an opportune occasion to attack his neighbours, and this excuse came when some of these rulers presumed to raid the property of his nephew Lot, who dwelt in Sodom. Upon hearing of this, Abraham, at the head of his seven confederated tribes, moved on the enemy. His own bodyguard of 318 officered the army, numbering more than 4,000, which struck at this time.
\vs p093 5:13 When Melchizedek heard of Abraham’s declaration of war, he went forth to dissuade him but only caught up with his former disciple as he returned victorious from the battle. Abraham insisted that the God of Salem had given him victory over his enemies and persisted in giving a tenth of his spoils to the Salem treasury. The other 90\% he removed to his capital at Hebron.
\vs p093 5:14 After this battle of Siddim, Abraham became leader of a second confederation of eleven tribes and not only paid tithes to Melchizedek but saw to it that all others in that vicinity did the same. His diplomatic dealings with the king of Sodom, together with the fear in which he was so generally held, resulted in the king of Sodom and others joining the Hebron military confederation; Abraham was really well on the way to establishing a powerful state in Palestine.
\usection{Melchizedek’s Covenant with Abraham}
\vs p093 6:1 Abraham envisaged the conquest of all Canaan. His determination was only weakened by the fact that Melchizedek would not sanction the undertaking. But Abraham had about decided to embark upon the enterprise when the thought that he had no son to succeed him as ruler of this proposed kingdom began to worry him. He arranged another conference with Melchizedek; and it was in the course of this interview that the priest of Salem, the visible Son of God, persuaded Abraham to abandon his scheme of material conquest and temporal rule in favour of the spiritual concept of the kingdom of heaven.
\vs p093 6:2 Melchizedek explained to Abraham the futility of contending with the Amorite confederation but made it equally clear that these backward clans were certainly committing suicide by their foolish practices so that in a few generations they would be so weakened that the descendants of Abraham, meanwhile greatly increased, could easily overcome them.
\vs p093 6:3 And Melchizedek made a formal covenant with Abraham at Salem. Said he to Abraham: “Look now up to the heavens and number the stars if you are able; so numerous shall your seed be.” And Abraham believed Melchizedek, “and it was counted to him for righteousness.” And then Melchizedek told Abraham the story of the future occupation of Canaan by his offspring after their sojourn in Egypt.
\vs p093 6:4 \pc This covenant of Melchizedek with Abraham represents the great Urantian agreement between divinity and humanity whereby God agrees to do \bibemph{everything;} man only agrees to \bibemph{believe} God’s promises and follow his instructions. Heretofore it had been believed that salvation could be secured only by works --- sacrifices and offerings; now, Melchizedek again brought to Urantia the good news that salvation, favour with God, is to be had by \bibemph{faith.} But this gospel of simple faith in God was too advanced; the Semitic tribesmen subsequently preferred to go back to the older sacrifices and atonement for sin by the shedding of blood.
\vs p093 6:5 It was not long after the establishment of this covenant that Isaac, the son of Abraham, was born in accordance with the promise of Melchizedek. After the birth of Isaac, Abraham took a very solemn attitude toward his covenant with Melchizedek, going over to Salem to have it stated in writing. It was at this public and formal acceptance of the covenant that he changed his name from Abram to Abraham.
\vs p093 6:6 Most of the Salem believers had practised circumcision, though it had never been made obligatory by Melchizedek. Now Abraham had always so opposed circumcision that on this occasion he decided to solemnize the event by formally accepting this rite in token of the ratification of the Salem covenant.
\vs p093 6:7 It was following this real and public surrender of his personal ambitions in behalf of the larger plans of Melchizedek that the three celestial beings appeared to him on the plains of Mamre. This was an appearance of fact, notwithstanding its association with the subsequently fabricated narratives relating to the natural destruction of Sodom and Gomorrah. And these legends of the happenings of those days indicate how retarded were the morals and ethics of even so recent a time.
\vs p093 6:8 Upon the consummation of the solemn covenant, the reconciliation between Abraham and Melchizedek was complete. Abraham again assumed the civil and military leadership of the Salem colony, which at its height carried over 100,000 regular tithe payers on the rolls of the Melchizedek brotherhood. Abraham greatly improved the Salem temple and provided new tents for the entire school. He not only extended the tithing system but also instituted many improved methods of conducting the business of the school, besides contributing greatly to the better handling of the department of missionary propaganda. He also did much to effect improvement of the herds and the reorganization of the Salem dairying projects. Abraham was a shrewd and efficient business man, a wealthy man for his day; he was not overly pious, but he was thoroughly sincere, and he did believe in Machiventa Melchizedek.
\usection{The Melchizedek Missionaries}
\vs p093 7:1 Melchizedek continued for some years to instruct his students and to train the Salem missionaries, who penetrated to all the surrounding tribes, especially to Egypt, Mesopotamia, and Asia Minor. And as the decades passed, these teachers journeyed farther and farther from Salem, carrying with them Machiventa’s gospel of belief and faith in God.
\vs p093 7:2 The descendants of Adamson, clustered about the shores of the lake of Van, were willing listeners to the Hittite teachers of the Salem cult. From this onetime Andite centre, teachers were dispatched to the remote regions of both Europe and Asia. Salem missionaries penetrated all Europe, even to the British Isles. One group went by way of the Faroes to the Andonites of Iceland, while another traversed China and reached the Japanese of the eastern islands. The lives and experiences of the men and women who ventured forth from Salem, Mesopotamia, and Lake Van to enlighten the tribes of the Eastern Hemisphere present a heroic chapter in the annals of the human race.
\vs p093 7:3 But the task was so great and the tribes were so backward that the results were vague and indefinite. From one generation to another the Salem gospel found lodgement here and there, but except in Palestine, never was the idea of one God able to claim the continued allegiance of a whole tribe or race. Long before the coming of Jesus the teachings of the early Salem missionaries had become generally submerged in the older and more universal superstitions and beliefs. The original Melchizedek gospel had been almost wholly absorbed in the beliefs in the Great Mother, the Sun, and other ancient cults.
\vs p093 7:4 \pc You who today enjoy the advantages of the art of printing little understand how difficult it was to perpetuate truth during these earlier times; how easy it was to lose sight of a new doctrine from one generation to another. There was always a tendency for the new doctrine to become absorbed into the older body of religious teaching and magical practice. A new revelation is always contaminated by the older evolutionary beliefs.
\usection{Departure of Melchizedek}
\vs p093 8:1 It was shortly after the destruction of Sodom and Gomorrah that Machiventa decided to end his emergency bestowal on Urantia. Melchizedek’s decision to terminate his sojourn in the flesh was influenced by numerous conditions, chief of which was the growing tendency of the surrounding tribes, and even of his immediate associates, to regard him as a demigod, to look upon him as a supernatural being, which indeed he was; but they were beginning to reverence him unduly and with a highly superstitious fear. In addition to these reasons, Melchizedek wanted to leave the scene of his earthly activities a sufficient length of time before Abraham’s death to ensure that the truth of the one and only God would become strongly established in the minds of his followers. Accordingly Machiventa retired one night to his tent at Salem, having said good night to his human companions, and when they went to call him in the morning, he was not there, for his fellows had taken him.
\usection{After Melchizedek’s Departure}
\vs p093 9:1 It was a great trial for Abraham when Melchizedek so suddenly disappeared. Although he had fully warned his followers that he must sometime go as he had come, they were not reconciled to the loss of their wonderful leader. The great organization built up at Salem nearly disappeared, though the traditions of these days were what Moses built upon when he led the Hebrew slaves out of Egypt.
\vs p093 9:2 \pc The loss of Melchizedek produced a sadness in the heart of Abraham that he never fully overcame. Hebron he had abandoned when he gave up the ambition of building a material kingdom; and now, upon the loss of his associate in the building of the spiritual kingdom, he departed from Salem, going south to live near his interests at Gerar.
\vs p093 9:3 Abraham became fearful and timid immediately after the disappearance of Melchizedek. He withheld his identity upon arrival at Gerar, so that Abimelech appropriated his wife. (Shortly after his marriage to Sarah, Abraham one night had overheard a plot to murder him in order to get his brilliant wife. This dread became a terror to the otherwise brave and daring leader; all his life he feared that someone would kill him secretly in order to get Sarah. And this explains why, on three separate occasions, this brave man exhibited real cowardice.)
\vs p093 9:4 But Abraham was not long to be deterred in his mission as the successor of Melchizedek. Soon he made converts among the Philistines and of Abimelech’s people, made a treaty with them, and, in turn, became contaminated with many of their superstitions, particularly with their practice of sacrificing first\hyp{}born sons. Thus did Abraham again become a great leader in Palestine. He was held in reverence by all groups and honoured by all kings. He was the spiritual leader of all the surrounding tribes, and his influence continued for some time after his death. During the closing years of his life he once more returned to Hebron, the scene of his earlier activities and the place where he had worked in association with Melchizedek. Abraham’s last act was to send trusty servants to the city of his brother, Nahor, on the border of Mesopotamia, to secure a woman of his own people as a wife for his son Isaac. It had long been the custom of Abraham’s people to marry their cousins. And Abraham died confident in that faith in God which he had learned from Melchizedek in the vanished schools of Salem.
\vs p093 9:5 \pc It was hard for the next generation to comprehend the story of Melchizedek; within 500 years many regarded the whole narrative as a myth. Isaac held fairly well to the teachings of his father and nourished the gospel of the Salem colony, but it was harder for Jacob to grasp the significance of these traditions. Joseph was a firm believer in Melchizedek and was, largely because of this, regarded by his brothers as a dreamer. Joseph’s honour in Egypt was chiefly due to the memory of his great\hyp{}grandfather Abraham. Joseph was offered military command of the Egyptian armies, but being such a firm believer in the traditions of Melchizedek and the later teachings of Abraham and Isaac, he elected to serve as a civil administrator, believing that he could thus better labour for the advancement of the kingdom of heaven.
\vs p093 9:6 The teaching of Melchizedek was full and replete, but the records of these days seemed impossible and fantastic to the later Hebrew priests, although many had some understanding of these transactions, at least up to the times of the en masse editing of the Old Testament records in Babylon.
\vs p093 9:7 What the Old Testament records describe as conversations between Abraham and God were in reality conferences between Abraham and Melchizedek. Later scribes regarded the term Melchizedek as synonymous with God. The record of so many contacts of Abraham and Sarah with “the angel of the Lord” refers to their numerous visits with Melchizedek.
\vs p093 9:8 The Hebrew narratives of Isaac, Jacob, and Joseph are far more reliable than those about Abraham, although they also contain many diversions from the facts, alterations made intentionally and unintentionally at the time of the compilation of these records by the Hebrew priests during the Babylonian captivity. Keturah was not a wife of Abraham; like Hagar, she was merely a concubine. All of Abraham’s property went to Isaac, the son of Sarah, the status wife. Abraham was not so old as the records indicate, and his wife was much younger. These ages were deliberately altered in order to provide for the subsequent alleged miraculous birth of Isaac.
\vs p093 9:9 \pc The national ego of the Jews was tremendously depressed by the Babylonian captivity. In their reaction against national inferiority they swung to the other extreme of national and racial egotism, in which they distorted and perverted their traditions with the view of exalting themselves above all races as the chosen people of God; and hence they carefully edited all their records for the purpose of raising Abraham and their other national leaders high up above all other persons, not excepting Melchizedek himself. The Hebrew scribes therefore destroyed every record of these momentous times which they could find, preserving only the narrative of the meeting of Abraham and Melchizedek after the battle of Siddim, which they deemed reflected great honour upon Abraham.
\vs p093 9:10 And thus, in losing sight of Melchizedek, they also lost sight of the teaching of this emergency Son regarding the spiritual mission of the promised bestowal Son; lost sight of the nature of this mission so fully and completely that very few of their progeny were able or willing to recognize and receive Michael when he appeared on earth and in the flesh as Machiventa had foretold.
\vs p093 9:11 But one of the writers of the Book of Hebrews understood the mission of Melchizedek, for it is written: “This Melchizedek, priest of the Most High, was also king of peace; without father, without mother, without pedigree, having neither beginning of days nor end of life but made like a Son of God, he abides a priest continually.” This writer designated Melchizedek as a type of the later bestowal of Michael, affirming that Jesus was “a minister forever on the order of Melchizedek.” While this comparison was not altogether fortunate, it was literally true that Christ did receive provisional title to Urantia “upon the orders of the twelve Melchizedek receivers” on duty at the time of his world bestowal.
\usection{Present Status of Machiventa Melchizedek}
\vs p093 10:1 During the years of Machiventa’s incarnation the Urantia Melchizedek receivers functioned as eleven. When Machiventa considered that his mission as an emergency Son was finished, he signalized this fact to his eleven associates, and they immediately made ready the technique whereby he was to be released from the flesh and safely restored to his original Melchizedek status. And on the third day after his disappearance from Salem he appeared among his eleven fellows of the Urantia assignment and resumed his interrupted career as one of the planetary receivers of 606 of Satania.
\vs p093 10:2 Machiventa terminated his bestowal as a creature of flesh and blood just as suddenly and unceremoniously as he had begun it. Neither his appearance nor departure were accompanied by any unusual announcement or demonstration; neither resurrection roll call nor ending of planetary dispensation marked his appearance on Urantia; his was an emergency bestowal. But Machiventa did not end his sojourn in the flesh of human beings until he had been duly released by the Father Melchizedek and had been informed that his emergency bestowal had received the approval of the chief executive of Nebadon, Gabriel of Salvington.
\vs p093 10:3 \pc Machiventa Melchizedek continued to take a great interest in the affairs of the descendants of those men who had believed in his teachings when he was in the flesh. But the progeny of Abraham through Isaac as intermarried with the Kenites were the only line which long continued to nourish any clear concept of the Salem teachings.
\vs p093 10:4 This same Melchizedek continued to collaborate throughout the nineteen succeeding centuries with the many prophets and seers, thus endeavouring to keep alive the truths of Salem until the fullness of the time for Michael’s appearance on earth.
\vs p093 10:5 Machiventa continued as a planetary receiver up to the times of the triumph of Michael on Urantia. Subsequently, he was attached to the Urantia service on Jerusem as one of the four and twenty directors, only just recently having been elevated to the position of personal ambassador on Jerusem of the Creator Son, bearing the title Vicegerent Planetary Prince of Urantia. It is our belief that, as long as Urantia remains an inhabited planet, Machiventa Melchizedek will not be fully returned to the duties of his order of sonship but will remain, speaking in the terms of time, forever a planetary minister representing Christ Michael.
\vs p093 10:6 As his was an emergency bestowal on Urantia, it does not appear from the records what Machiventa’s future may be. It may develop that the Melchizedek corps of Nebadon have sustained the permanent loss of one of their number. Recent rulings handed down from the Most Highs of Edentia, and later confirmed by the Ancients of Days of Uversa, strongly suggest that this bestowal Melchizedek is destined to take the place of the fallen Planetary Prince, Caligastia. If our conjectures in this respect are correct, it is altogether possible that Machiventa Melchizedek may again appear in person on Urantia and in some modified manner resume the role of the dethroned Planetary Prince, or else appear on earth to function as vicegerent Planetary Prince representing Christ Michael, who now actually holds the title of Planetary Prince of Urantia. While it is far from clear to us as to what Machiventa’s destiny may be, nevertheless, events which have so recently taken place strongly suggest that the foregoing conjectures are probably not far from the truth.
\vs p093 10:7 We well understand how, by his triumph on Urantia, Michael became the successor of both Caligastia and Adam; how he became the planetary Prince of Peace and the second Adam. And now we behold the conferring upon this Melchizedek of the title Vicegerent Planetary Prince of Urantia. Will he also be constituted Vicegerent Material Son of Urantia? Or is there a possibility that an unexpected and unprecedented event is to take place, the sometime return to the planet of Adam and Eve or certain of their progeny as representatives of Michael with the titles vicegerents of the second Adam of Urantia?
\vs p093 10:8 And all these speculations associated with the certainty of future appearances of both Magisterial and Trinity Teacher Sons, in conjunction with the explicit promise of the Creator Son to return sometime, make Urantia a planet of future uncertainty and render it one of the most interesting and intriguing spheres in all the universe of Nebadon. It is altogether possible that, in some future age when Urantia is approaching the era of light and life, after the affairs of the Lucifer rebellion and the Caligastia secession have been finally adjudicated, we may witness the presence on Urantia, simultaneously, of Machiventa, Adam, Eve, and Christ Michael, as well as either a Magisterial Son or even Trinity Teacher Sons.
\vs p093 10:9 It has long been the opinion of our order that Machiventa’s presence on the Jerusem corps of Urantia directors, the four and twenty counsellors, is sufficient evidence to warrant the belief that he is destined to follow the mortals of Urantia on through the universe scheme of progression and ascension even to the Paradise Corps of the Finality. We know that Adam and Eve are thus destined to accompany their earth fellows on the Paradise adventure when Urantia has become settled in light and life.
\vs p093 10:10 Less than a thousand years ago this same Machiventa Melchizedek, the onetime sage of Salem, was invisibly present on Urantia for a period of 100 years, acting as resident governor general of the planet; and if the present system of directing planetary affairs should continue, he will be due to return in the same capacity in a little over one thousand years.
\vs p093 10:11 \pc This is the story of Machiventa Melchizedek, one of the most unique of all characters ever to become connected with the history of Urantia and a personality who may be destined to play an important role in the future experience of your irregular and unusual world.
\vsetoff
\vs p093 10:12 [Presented by a Melchizedek of Nebadon.]
\quizlink
