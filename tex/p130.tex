\upaper{130}{On the Way to Rome}
\uminitoc{At Joppa --- Discourse on Jonah}
\uminitoc{At Caesarea}
\uminitoc{At Alexandria}
\uminitoc{Discourse on Reality}
\uminitoc{On the Island of Crete}
\uminitoc{The Young Man Who Was Afraid}
\uminitoc{At Carthage --- Discourse on Time and Space}
\uminitoc{On the Way to Naples and Rome}
\author{Midwayer Commission}
\vs p130 0:1 The tour of the Roman world consumed most of the 28\ts{th} and the entire 29\ts{th} year of Jesus’ life on earth. Jesus and the two natives from India --- Gonod and his son Ganid --- left Jerusalem on a Sunday morning, April 26, A.D.\,22. They made their journey according to schedule, and Jesus said good\hyp{}bye to the father and son in the city of Charax on the Persian Gulf on the 10\ts{th} day of December the following year, A.D.\,23.
\vs p130 0:2 \pc From Jerusalem they went to Caesarea by way of Joppa. At Caesarea they took a boat for Alexandria. From Alexandria they sailed for Lasea in Crete. From Crete they sailed for Carthage, touching at Cyrene. At Carthage they took a boat for Naples, stopping at Malta, Syracuse, and Messina. From Naples they went to Capua, whence they travelled by the Appian Way to Rome.
\vs p130 0:3 After their stay in Rome they went overland to Tarentum, where they set sail for Athens in Greece, stopping at Nicopolis and Corinth. From Athens they went to Ephesus by way of Troas. From Ephesus they sailed for Cyprus, putting in at Rhodes on the way. They spent considerable time visiting and resting on Cyprus and then sailed for Antioch in Syria. From Antioch they journeyed south to Sidon and then went over to Damascus. From there they travelled by caravan to Mesopotamia, passing through Thapsacus and Larissa. They spent some time in Babylon, visited Ur and other places, and then went to Susa. From Susa they journeyed to Charax, from which place Gonod and Ganid embarked for India.
\vs p130 0:4 \pc It was while working four months at Damascus that Jesus had picked up the rudiments of the language spoken by Gonod and Ganid. While there he had laboured much of the time on translations from Greek into one of the languages of India, being assisted by a native of Gonod’s home district.
\vs p130 0:5 \pc On this Mediterranean tour Jesus spent about half of each day teaching Ganid and acting as interpreter during Gonod’s business conferences and social contacts. The remainder of each day, which was at his disposal, he devoted to making those close personal contacts with his fellow men, those intimate associations with the mortals of the realm, which so characterized his activities during these years that just preceded his public ministry.
\vs p130 0:6 From firsthand observation and actual contact Jesus acquainted himself with the higher material and intellectual civilization of the Occident and the Levant; from Gonod and his brilliant son he learned a great deal about the civilization and culture of India and China, for Gonod, himself a citizen of India, had made three extensive trips to the empire of the yellow race.
\vs p130 0:7 Ganid, the young man, learned much from Jesus during this long and intimate association. They developed a great affection for each other, and the lad’s father many times tried to persuade Jesus to return with them to India, but Jesus always declined, pleading the necessity for returning to his family in Palestine.
\usection{At Joppa --- Discourse on Jonah}
\vs p130 1:1 During their stay in Joppa, Jesus met Gadiah, a Philistine interpreter who worked for one Simon a tanner. Gonod’s agents in Mesopotamia had transacted much business with this Simon; so Gonod and his son desired to pay him a visit on their way to Caesarea. While they tarried at Joppa, Jesus and Gadiah became warm friends. This young Philistine was a truth seeker. Jesus was a truth giver; he \bibemph{was} the truth for that generation on Urantia. When a great truth seeker and a great truth giver meet, the result is a great and liberating enlightenment born of the experience of new truth.
\vs p130 1:2 One day after the evening meal Jesus and the young Philistine strolled down by the sea, and Gadiah, not knowing that this “scribe of Damascus” was so well versed in the Hebrew traditions, pointed out to Jesus the ship landing from which it was reputed that Jonah had embarked on his ill\hyp{}fated voyage to Tarshish. And when he had concluded his remarks, he asked Jesus this question: “But do you suppose the big fish really did swallow Jonah?” Jesus perceived that this young man’s life had been tremendously influenced by this tradition, and that its contemplation had impressed upon him the folly of trying to run away from duty; Jesus therefore said nothing that would suddenly destroy the foundations of Gadiah’s present motivation for practical living. In answering this question, Jesus said: \textcolour{ubdarkred}{“My friend, we are all Jonahs with lives to live in accordance with the will of God, and at all times when we seek to escape the present duty of living by running away to far\hyp{}off enticements, we thereby put ourselves in the immediate control of those influences which are not directed by the powers of truth and the forces of righteousness. The flight from duty is the sacrifice of truth. The escape from the service of light and life can only result in those distressing conflicts with the difficult whales of selfishness which lead eventually to darkness and death unless such God\hyp{}forsaking Jonahs shall turn their hearts, even when in the very depths of despair, to seek after God and his goodness. And when such disheartened souls sincerely seek for God --- hunger for truth and thirst for righteousness --- there is nothing that can hold them in further captivity. No matter into what great depths they may have fallen, when they seek the light with a whole heart, the spirit of the Lord God of heaven will deliver them from their captivity; the evil circumstances of life will spew them out upon the dry land of fresh opportunities for renewed service and wiser living.”}
\vs p130 1:3 Gadiah was mightily moved by Jesus’ teaching, and they talked long into the night by the seaside, and before they went to their lodgings, they prayed together and for each other. This was the same Gadiah who listened to the later preaching of Peter, became a profound believer in Jesus of Nazareth, and held a memorable argument with Peter one evening at the home of Dorcas. And Gadiah had very much to do with the final decision of Simon, the wealthy leather merchant, to embrace Christianity.
\vs p130 1:4 \pc (In this narrative of the personal work of Jesus with his fellow mortals on this tour of the Mediterranean, we shall, in accordance with our permission, freely translate his words into modern phraseology current on Urantia at the time of this presentation.)
\vs p130 1:5 \pc Jesus’ last visit with Gadiah had to do with a discussion of good and evil. This young Philistine was much troubled by a feeling of injustice because of the presence of evil in the world alongside the good. He said: “How can God, if he is infinitely good, permit us to suffer the sorrows of evil; after all, who creates evil?” It was still believed by many in those days that God creates both good and evil, but Jesus never taught such error. In answering this question, Jesus said: \textcolour{ubdarkred}{“My brother, God is love; therefore he must be good, and his goodness is so great and real that it cannot contain the small and unreal things of evil. God is so positively good that there is absolutely no place in him for negative evil. Evil is the immature choosing and the unthinking misstep of those who are resistant to goodness, rejectful of beauty, and disloyal to truth. Evil is only the misadaptation of immaturity or the disruptive and distorting influence of ignorance. Evil is the inevitable darkness which follows upon the heels of the unwise rejection of light. Evil is that which is dark and untrue, and which, when consciously embraced and wilfully endorsed, becomes sin.}
\vs p130 1:6 \textcolour{ubdarkred}{“Your Father in heaven, by endowing you with the power to choose between truth and error, created the potential negative of the positive way of light and life; but such errors of evil are really nonexistent until such a time as an intelligent creature wills their existence by mischoosing the way of life. And then are such evils later exalted into sin by the knowing and deliberate choice of such a wilful and rebellious creature. This is why our Father in heaven permits the good and the evil to go along together until the end of life, just as nature allows the wheat and the tares to grow side by side until the harvest.”} Gadiah was fully satisfied with Jesus’ answer to his question after their subsequent discussion had made clear to his mind the real meaning of these momentous statements.
\usection{At Caesarea}
\vs p130 2:1 Jesus and his friends tarried in Caesarea beyond the time expected because one of the huge steering paddles of the vessel on which they intended to embark was discovered to be in danger of cleaving. The captain decided to remain in port while a new one was being made. There was a shortage of skilled woodworkers for this task, so Jesus volunteered to assist. During the evenings Jesus and his friends strolled about on the beautiful wall which served as a promenade around the port. Ganid greatly enjoyed Jesus’ explanation of the water system of the city and the technique whereby the tides were utilized to flush the city’s streets and sewers. This youth of India was much impressed with the temple of Augustus, situated upon an elevation and surmounted by a colossal statue of the Roman emperor. The second afternoon of their stay the three of them attended a performance in the enormous amphitheatre which could seat 20,000 persons, and that night they went to a Greek play at the theatre. These were the first exhibitions of this sort Ganid had ever witnessed, and he asked Jesus many questions about them. On the morning of the third day they paid a formal visit to the governor’s palace, for Caesarea was the capital of Palestine and the residence of the Roman procurator.
\vs p130 2:2 \pc At their inn there also lodged a merchant from Mongolia, and since this Far\hyp{}Easterner talked Greek fairly well, Jesus had several long visits with him. This man was much impressed with Jesus’ philosophy of life and never forgot his words of wisdom regarding “the living of the heavenly life while on earth by means of daily submission to the will of the heavenly Father.” This merchant was a Taoist, and he had thereby become a strong believer in the doctrine of a universal Deity. When he returned to Mongolia, he began to teach these advanced truths to his neighbours and to his business associates, and as a direct result of such activities, his eldest son decided to become a Taoist priest. This young man exerted a great influence in behalf of advanced truth throughout his lifetime and was followed by a son and a grandson who likewise were devotedly loyal to the doctrine of the One God --- the Supreme Ruler of Heaven.
\vs p130 2:3 While the eastern branch of the early Christian church, having its headquarters at Philadelphia, held more faithfully to the teachings of Jesus than did the Jerusalem brethren, it was regrettable that there was no one like Peter to go into China, or like Paul to enter India, where the spiritual soil was then so favourable for planting the seed of the new gospel of the kingdom. These very teachings of Jesus, as they were held by the Philadelphians, would have made just such an immediate and effective appeal to the minds of the spiritually hungry Asiatic peoples as did the preaching of Peter and Paul in the West.
\vs p130 2:4 \pc One of the young men who worked with Jesus one day on the steering paddle became much interested in the words which he dropped from hour to hour as they toiled in the shipyard. When Jesus intimated that the Father in heaven was interested in the welfare of his children on earth, this young Greek, Anaxand, said: “If the Gods are interested in me, then why do they not remove the cruel and unjust foreman of this workshop?” He was startled when Jesus replied, \textcolour{ubdarkred}{“Since you know the ways of kindness and value justice, perhaps the Gods have brought this erring man near that you may lead him into this better way. Maybe you are the salt which is to make this brother more agreeable to all other men; that is, if you have not lost your savour. As it is, this man is your master in that his evil ways unfavourably influence you. Why not assert your mastery of evil by virtue of the power of goodness and thus become the master of all relations between the two of you? I predict that the good in you could overcome the evil in him if you gave it a fair and living chance. There is no adventure in the course of mortal existence more enthralling than to enjoy the exhilaration of becoming the material life partner with spiritual energy and divine truth in one of their triumphant struggles with error and evil. It is a marvellous and transforming experience to become the living channel of spiritual light to the mortal who sits in spiritual darkness. If you are more blessed with truth than is this man, his need should challenge you. Surely you are not the coward who could stand by on the seashore and watch a fellow man who could not swim perish! How much more of value is this man’s soul floundering in darkness compared to his body drowning in water!”}
\vs p130 2:5 Anaxand was mightily moved by Jesus’ words. Presently he told his superior what Jesus had said, and that night they both sought Jesus’ advice as to the welfare of their souls. And later on, after the Christian message had been proclaimed in Caesarea, both of these men, one a Greek and the other a Roman, believed Philip’s preaching and became prominent members of the church which he founded. Later this young Greek was appointed the steward of a Roman centurion, Cornelius, who became a believer through Peter’s ministry. Anaxand continued to minister light to those who sat in darkness until the days of Paul’s imprisonment at Caesarea, when he perished, by accident, in the great slaughter of 20,000 Jews while he ministered to the suffering and dying.
\vs p130 2:6 \pc Ganid was, by this time, beginning to learn how his tutor spent his leisure in this unusual personal ministry to his fellow men, and the young Indian set about to find out the motive for these incessant activities. He asked, “Why do you occupy yourself so continuously with these visits with strangers?” And Jesus answered: \textcolour{ubdarkred}{“Ganid, no man is a stranger to one who knows God. In the experience of finding the Father in heaven you discover that all men are your brothers, and does it seem strange that one should enjoy the exhilaration of meeting a newly discovered brother? To become acquainted with one’s brothers and sisters, to know their problems and to learn to love them, is the supreme experience of living.”}
\vs p130 2:7 This was a conference which lasted well into the night, in the course of which the young man requested Jesus to tell him the difference between the will of God and that human mind act of choosing which is also called will. In substance Jesus said: The will of God is the way of God, partnership with the choice of God in the face of any potential alternative. To do the will of God, therefore, is the progressive experience of becoming more and more like God, and God is the source and destiny of all that is good and beautiful and true. The will of man is the way of man, the sum and substance of that which the mortal chooses to be and do. Will is the deliberate choice of a self\hyp{}conscious being which leads to decision\hyp{}conduct based on intelligent reflection.\tunemarkup{private}{\begin{figure}[H]\centering\includegraphics[width=\columnwidth]{images/Shepherd-Dog.jpg}\caption{Playing with the Shepherd Dog by Slawa Radziszewska}\end{figure}}
\vs p130 2:8 That afternoon Jesus and Ganid had both enjoyed playing with a very intelligent shepherd dog, and Ganid wanted to know whether the dog had a soul, whether it had a will, and in response to his questions Jesus said: \textcolour{ubdarkred}{“The dog has a mind which can know material man, his master, but cannot know God, who is spirit; therefore the dog does not possess a spiritual nature and cannot enjoy a spiritual experience. The dog may have a will derived from nature and augmented by training, but such a power of mind is not a spiritual force, neither is it comparable to the human will, inasmuch as it is not \bibemph{reflective ---} it is not the result of discriminating higher and moral meanings or choosing spiritual and eternal values. It is the possession of such powers of spiritual discrimination and truth choosing that makes mortal man a moral being, a creature endowed with the attributes of spiritual responsibility and the potential of eternal survival.”} Jesus went on to explain that it is the absence of such mental powers in the animal which makes it forever impossible for the animal world to develop language in time or to experience anything equivalent to personality survival in eternity. As a result of this day’s instruction Ganid never again entertained belief in the transmigration of the souls of men into the bodies of animals.
\vs p130 2:9 \pc The next day Ganid talked all this over with his father, and it was in answer to Gonod’s question that Jesus explained that \textcolour{ubdarkred}{“human wills which are fully occupied with passing only upon temporal decisions having to do with the material problems of animal existence are doomed to perish in time. Those who make wholehearted moral decisions and unqualified spiritual choices are thus progressively identified with the indwelling and divine spirit, and thereby are they increasingly transformed into the values of eternal survival --- unending progression of divine service.”}
\vs p130 2:10 \pc It was on this same day that we first heard that momentous truth which, stated in modern terms, would signify: “Will is that manifestation of the human mind which enables the subjective consciousness to express itself objectively and to experience the phenomenon of aspiring to be Godlike.” And it is in this same sense that every reflective and spiritually minded human being can become \bibemph{creative.}
\usection{At Alexandria}
\vs p130 3:1 It had been an eventful visit at Caesarea, and when the boat was ready, Jesus and his two friends departed at noon one day for Alexandria in Egypt.
\vs p130 3:2 The three enjoyed a most pleasant passage to Alexandria. Ganid was delighted with the voyage and kept Jesus busy answering questions. As they approached the city’s harbour, the young man was thrilled by the great lighthouse of Pharos, located on the island which Alexander had joined by a mole to the mainland, thus creating two magnificent harbours and thereby making Alexandria the maritime commercial crossroads of Africa, Asia, and Europe. This great lighthouse was one of the seven wonders of the world and was the forerunner of all subsequent lighthouses. They arose early in the morning to view this splendid lifesaving device of man, and amidst the exclamations of Ganid Jesus said: \textcolour{ubdarkred}{“And you, my son, will be like this lighthouse when you return to India, even after your father is laid to rest; you will become like the light of life to those who sit about you in darkness, showing all who so desire the way to reach the harbour of salvation in safety.”} And as Ganid squeezed Jesus’ hand, he said, “I will.”
\vs p130 3:3 \pc And again we remark that the early teachers of the Christian religion made a great mistake when they so exclusively turned their attention to the western civilization of the Roman world. The teachings of Jesus, as they were held by the Mesopotamian believers of the first century, would have been readily received by the various groups of Asiatic religionists.\tunemarkup{private}{\begin{figure}[H]\centering\includegraphics[width=\columnwidth]{images/Jesus-Ganid-lighthouse.jpg}\caption{Jesus and Ganid at the Great Lighthouse by Slawa Radziszewska}\end{figure}}
\vs p130 3:4 \pc By the fourth hour after landing they were settled near the eastern end of the long and broad avenue, 30\,m wide and 8\,km long, which stretched on out to the western limits of this city of 1,000,000 people. After the first survey of the city’s chief attractions --- university (museum), library, the royal mausoleum of Alexander, the palace, temple of Neptune, theatre, and gymnasium --- Gonod addressed himself to business while Jesus and Ganid went to the library, the greatest in the world. Here were assembled nearly 1,000,000 manuscripts from all the civilized world: Greece, Rome, Palestine, Parthia, India, China, and even Japan. In this library Ganid saw the largest collection of Indian literature in all the world; and they spent some time here each day throughout their stay in Alexandria. Jesus told Ganid about the translation of the Hebrew scriptures into Greek at this place. And they discussed again and again all the religions of the world, Jesus endeavouring to point out to this young mind the truth in each, always adding: \textcolour{ubdarkred}{“But Yahweh is the God developed from the revelations of Melchizedek and the covenant of Abraham. The Jews were the offspring of Abraham and subsequently occupied the very land wherein Melchizedek had lived and taught, and from which he sent teachers to all the world; and their religion eventually portrayed a clearer recognition of the Lord God of Israel as the Universal Father in heaven than any other world religion.”}
\vs p130 3:5 \pc Under Jesus’ direction Ganid made a collection of the teachings of all those religions of the world which recognized a Universal Deity, even though they might also give more or less recognition to subordinate deities. After much discussion Jesus and Ganid decided that the Romans had no real God in their religion, that their religion was hardly more than emperor worship. The Greeks, they concluded, had a philosophy but hardly a religion with a personal God. The mystery cults they discarded because of the confusion of their multiplicity, and because their varied concepts of Deity seemed to be derived from other and older religions.\tunemarkup{private}{\begin{figure}[H]\centering\includegraphics[width=\tunemarkup{pgkoboaurahd}{0.7}\columnwidth]{images/Jesus-Ganid-Alexandria.jpg}\caption{Studying at the Alexandrian Library by Slawa Radziszewska}\end{figure}}
\vs p130 3:6 Although these translations were made at Alexandria, Ganid did not finally arrange these selections and add his own personal conclusions until near the end of their sojourn in Rome. He was much surprised to discover that the best of the authors of the world’s sacred literature all more or less clearly recognized the existence of an eternal God and were much in agreement with regard to his character and his relationship with mortal man.
\vs p130 3:7 \pc Jesus and Ganid spent much time in the museum during their stay in Alexandria. This museum was not a collection of rare objects but rather a university of fine art, science, and literature. Learned professors here gave daily lectures, and in those times this was the intellectual centre of the Occidental world. Day by day Jesus interpreted the lectures to Ganid; one day during the second week the young man exclaimed: “Teacher Joshua, you know more than these professors; you should stand up and tell them the great things you have told me; they are befogged by much thinking. I shall speak to my father and have him arrange it.” Jesus smiled, saying: \textcolour{ubdarkred}{“You are an admiring pupil, but these teachers are not minded that you and I should instruct them. The pride of unspiritualized learning is a treacherous thing in human experience. The true teacher maintains his intellectual integrity by ever remaining a learner.”}
\vs p130 3:8 Alexandria was the city of the blended culture of the Occident and next to Rome the largest and most magnificent in the world. Here was located the largest Jewish synagogue in the world, the seat of government of the Alexandria Sanhedrin, the seventy ruling elders.
\vs p130 3:9 Among the many men with whom Gonod transacted business was a certain Jewish banker, Alexander, whose brother, Philo, was a famous religious philosopher of that time. Philo was engaged in the laudable but exceedingly difficult task of harmonizing Greek philosophy and Hebrew theology. Ganid and Jesus talked much about Philo’s teachings and expected to attend some of his lectures, but throughout their stay at Alexandria this famous Hellenistic Jew lay sick abed.
\vs p130 3:10 Jesus commended to Ganid much in the Greek philosophy and the Stoic doctrines, but he impressed upon the lad the truth that these systems of belief, like the indefinite teachings of some of his own people, were religions only in the sense that they led men to find God and enjoy a living experience in knowing the Eternal.
\usection{Discourse on Reality}
\vs p130 4:1 The night before they left Alexandria Ganid and Jesus had a long visit with one of the government professors at the university who lectured on the teachings of Plato. Jesus interpreted for the learned Greek teacher but injected no teaching of his own in refutation of the Greek philosophy. Gonod was away on business that evening; so, after the professor had departed, the teacher and his pupil had a long and heart\hyp{}to\hyp{}heart talk about Plato’s doctrines. While Jesus gave qualified approval of some of the Greek teachings which had to do with the theory that the material things of the world are shadowy reflections of invisible but more substantial spiritual realities, he sought to lay a more trustworthy foundation for the lad’s thinking; so he began a long dissertation concerning the nature of reality in the universe. In substance and in modern phraseology Jesus said to Ganid:
\vs p130 4:2 \pc The source of universe reality is the Infinite. The material things of finite creation are the time\hyp{}space repercussions of the Paradise Pattern and the Universal Mind of the eternal God. Causation in the physical world, self\hyp{}consciousness in the intellectual world, and progressing selfhood in the spirit world --- these realities, projected on a universal scale, combined in eternal relatedness, and experienced with perfection of quality and divinity of value --- constitute the \bibemph{reality of the Supreme.} But in an ever\hyp{}changing universe the Original Personality of causation, intelligence, and spirit experience is changeless, absolute. All things, even in an eternal universe of limitless values and divine qualities, may, and oftentimes do, change except the Absolutes and that which has attained the physical status, intellectual embrace, or spiritual identity which is absolute.
\vs p130 4:3 The highest level to which a finite creature can progress is the recognition of the Universal Father and the knowing of the Supreme. And even then such beings of finality destiny go on experiencing change in the motions of the physical world and in its material phenomena. Likewise do they remain aware of selfhood progression in their continuing ascension of the spiritual universe and of growing consciousness in their deepening appreciation of, and response to, the intellectual cosmos. Only in the perfection, harmony, and unanimity of will can the creature become as one with the Creator; and such a state of divinity is attained and maintained only by the creature’s continuing to live in time and eternity by consistently conforming his finite personal will to the divine will of the Creator. Always must the desire to do the Father’s will be supreme in the soul and dominant over the mind of an ascending son of God.
\vs p130 4:4 A one\hyp{}eyed person can never hope to visualize depth of perspective. Neither can single\hyp{}eyed material scientists nor single\hyp{}eyed spiritual mystics and allegorists correctly visualize and adequately comprehend the true depths of universe reality. All true values of creature experience are concealed in depth of recognition.
\vs p130 4:5 Mindless causation cannot evolve the refined and complex from the crude and the simple, neither can spiritless experience evolve the divine characters of eternal survival from the material minds of the mortals of time. The one attribute of the universe which so exclusively characterizes the infinite Deity is this unending creative bestowal of personality which can survive in progressive Deity attainment.
\vs p130 4:6 Personality is that cosmic endowment, that phase of universal reality, which can coexist with unlimited change and at the same time retain its identity in the very presence of all such changes, and forever afterwards.
\vs p130 4:7 Life is an adaptation of the original cosmic causation to the demands and possibilities of universe situations, and it comes into being by the action of the Universal Mind and the activation of the spirit spark of the God who is spirit. The meaning of life is its adaptability; the value of life is its progressability --- even to the heights of God\hyp{}consciousness.
\vs p130 4:8 Misadaptation of self\hyp{}conscious life to the universe results in cosmic disharmony. Final divergence of personality will from the trend of the universes terminates in intellectual isolation, personality segregation. Loss of the indwelling spirit pilot supervenes in spiritual cessation of existence. Intelligent and progressing life becomes then, in and of itself, an incontrovertible proof of the existence of a purposeful universe expressing the will of a divine Creator. And this life, in the aggregate, struggles toward higher values, having for its final goal the Universal Father.
\vs p130 4:9 Only in degree does man possess mind above the animal level aside from the higher and quasi\hyp{}spiritual ministrations of intellect. Therefore animals (not having worship and wisdom) cannot experience superconsciousness, consciousness of consciousness. The animal mind is only conscious of the objective universe.
\vs p130 4:10 Knowledge is the sphere of the material or fact\hyp{}discerning mind. Truth is the domain of the spiritually endowed intellect that is conscious of knowing God. Knowledge is demonstrable; truth is experienced. Knowledge is a possession of the mind; truth an experience of the soul, the progressing self. Knowledge is a function of the nonspiritual level; truth is a phase of the mind\hyp{}spirit level of the universes. The eye of the material mind perceives a world of factual knowledge; the eye of the spiritualized intellect discerns a world of true values. These two views, synchronized and harmonized, reveal the world of reality, wherein wisdom interprets the phenomena of the universe in terms of progressive personal experience.
\vs p130 4:11 Error (evil) is the penalty of imperfection. The qualities of imperfection or facts of misadaptation are disclosed on the material level by critical observation and by scientific analysis; on the moral level, by human experience. The presence of evil constitutes proof of the inaccuracies of mind and the immaturity of the evolving self. Evil is, therefore, also a measure of imperfection in universe interpretation. The possibility of making mistakes is inherent in the acquisition of wisdom, the scheme of progressing from the partial and temporal to the complete and eternal, from the relative and imperfect to the final and perfected. Error is the shadow of relative incompleteness which must of necessity fall across man’s ascending universe path to Paradise perfection. Error (evil) is not an actual universe quality; it is simply the observation of a relativity in the relatedness of the imperfection of the incomplete finite to the ascending levels of the Supreme and Ultimate.
\vs p130 4:12 \pc Although Jesus told all this to the lad in language best suited to his comprehension, at the end of the discussion Ganid was heavy of eye and was soon lost in slumber. They rose early the next morning to go aboard the boat bound for Lasea on the island of Crete. But before they embarked, the lad had still further questions to ask about evil, to which Jesus replied:
\vs p130 4:13 \pc Evil is a relativity concept. It arises out of the observation of the imperfections which appear in the shadow cast by a finite universe of things and beings as such a cosmos obscures the living light of the universal expression of the eternal realities of the Infinite One.
\vs p130 4:14 Potential evil is inherent in the necessary incompleteness of the revelation of God as a time\hyp{}space\hyp{}limited expression of infinity and eternity. The fact of the partial in the presence of the complete constitutes relativity of reality, creates necessity for intellectual choosing, and establishes value levels of spirit recognition and response. The incomplete and finite concept of the Infinite which is held by the temporal and limited creature mind is, in and of itself, \bibemph{potential evil.} But the augmenting error of unjustified deficiency in reasonable spiritual rectification of these originally inherent intellectual disharmonies and spiritual insufficiencies, is equivalent to the realization of \bibemph{actual evil.}
\vs p130 4:15 All static, dead, concepts are potentially evil. The finite shadow of relative and living truth is continually moving. Static concepts invariably retard science, politics, society, and religion. Static concepts may represent a certain knowledge, but they are deficient in wisdom and devoid of truth. But do not permit the concept of relativity so to mislead you that you fail to recognize the co\hyp{}ordination of the universe under the guidance of the cosmic mind, and its stabilized control by the energy and spirit of the Supreme.
\usection{On the Island of Crete}
\vs p130 5:1 The travellers had but one purpose in going to Crete, and that was to play, to walk about over the island, and to climb the mountains. The Cretans of that time did not enjoy an enviable reputation among the surrounding peoples. Nevertheless, Jesus and Ganid won many souls to higher levels of thinking and living and thus laid the foundation for the quick reception of the later gospel teachings when the first preachers from Jerusalem arrived. Jesus loved these Cretans, notwithstanding the harsh words which Paul later spoke concerning them when he subsequently sent Titus to the island to reorganize their churches.
\vs p130 5:2 On the mountainside in Crete Jesus had his first long talk with Gonod regarding religion. And the father was much impressed, saying: “No wonder the boy believes everything you tell him, but I never knew they had such a religion even in Jerusalem, much less in Damascus.” It was during the island sojourn that Gonod first proposed to Jesus that he go back to India with them, and Ganid was delighted with the thought that Jesus might consent to such an arrangement.
\vs p130 5:3 One day when Ganid asked Jesus why he had not devoted himself to the work of a public teacher, he said: \textcolour{ubdarkred}{“My son, everything must await the coming of its time. You are born into the world, but no amount of anxiety and no manifestation of impatience will help you to grow up. You must, in all such matters, wait upon time. Time alone will ripen the green fruit upon the tree. Season follows season and sundown follows sunrise only with the passing of time. I am now on the way to Rome with you and your father, and that is sufficient for today. My tomorrow is wholly in the hands of my Father in heaven.”} And then he told Ganid the story of Moses and the 40 years of watchful waiting and continued preparation.
\vs p130 5:4 One thing happened on a visit to Fair Havens which Ganid never forgot; the memory of this episode always caused him to wish he might do something to change the caste system of his native India. A drunken degenerate was attacking a slave girl on the public highway. When Jesus saw the plight of the girl, he rushed forward and drew the maiden away from the assault of the madman. While the frightened child clung to him, he held the infuriated man at a safe distance by his powerful extended right arm until the poor fellow had exhausted himself beating the air with his angry blows. Ganid felt a strong impulse to help Jesus handle the affair, but his father forbade him. Though they could not speak the girl’s language, she could understand their act of mercy and gave token of her heartfelt appreciation as they all three escorted her home. This was probably as near a personal encounter with his fellows as Jesus ever had throughout his entire life in the flesh. But he had a difficult task that evening trying to explain to Ganid why he did not smite the drunken man. Ganid thought this man should have been struck at least as many times as he had struck the girl.
\usection{The Young Man Who Was Afraid}
\vs p130 6:1 While they were up in the mountains, Jesus had a long talk with a young man who was fearful and downcast. Failing to derive comfort and courage from association with his fellows, this youth had sought the solitude of the hills; he had grown up with a feeling of helplessness and inferiority. These natural tendencies had been augmented by numerous difficult circumstances which the lad had encountered as he grew up, notably, the loss of his father when he was 12 years of age. As they met, Jesus said: \textcolour{ubdarkred}{“Greetings, my friend! why so downcast on such a beautiful day? If something has happened to distress you, perhaps I can in some manner assist you. At any rate it affords me real pleasure to proffer my services.”}\tunemarkup{private}{\begin{figure}[H]\centering\includegraphics[width=\tunemarkup{pgkoboaurahd}{0.9}\columnwidth]{images/Young-man.jpg}\caption{The young man who was afraid by Russ Docken}\end{figure}}
\vs p130 6:2 The young man was disinclined to talk, and so Jesus made a second approach to his soul, saying: \textcolour{ubdarkred}{“I understand you come up in these hills to get away from folks; so, of course, you do not want to talk with me, but I would like to know whether you are familiar with these hills; do you know the direction of the trails? and, perchance, could you inform me as to the best route to Phenix?”} Now this youth was very familiar with these mountains, and he really became much interested in telling Jesus the way to Phenix, so much so that he marked out all the trails on the ground and fully explained every detail. But he was startled and made curious when Jesus, after saying good\hyp{}bye and making as if he were taking leave, suddenly turned to him, saying: \textcolour{ubdarkred}{“I well know you wish to be left alone with your disconsolation; but it would be neither kind nor fair for me to receive such generous help from you as to how best to find my way to Phenix and then unthinkingly to go away from you without making the least effort to answer your appealing request for help and guidance regarding the best route to the goal of destiny which you seek in your heart while you tarry here on the mountainside. As you so well know the trails to Phenix, having traversed them many times, so do I well know the way to the city of your disappointed hopes and thwarted ambitions. And since you have asked me for help, I will not disappoint you.”} The youth was almost overcome, but he managed to stammer out, “But --- I did not ask you for anything --- ” And Jesus, laying a gentle hand on his shoulder, said: \textcolour{ubdarkred}{“No, son, not with words but with longing looks did you appeal to my heart. My boy, to one who loves his fellows there is an eloquent appeal for help in your countenance of discouragement and despair. Sit down with me while I tell you of the service trails and happiness highways which lead from the sorrows of self to the joys of loving activities in the brotherhood of men and in the service of the God of heaven.”}
\vs p130 6:3 By this time the young man very much desired to talk with Jesus, and he knelt at his feet imploring Jesus to help him, to show him the way of escape from his world of personal sorrow and defeat. Said Jesus: “My friend, arise! Stand up like a man! You may be surrounded with small enemies and be retarded by many obstacles, but the big things and the real things of this world and the universe are on your side. The sun rises every morning to salute you just as it does the most powerful and prosperous man on earth. Look --- you have a strong body and powerful muscles --- your physical equipment is better than the average. Of course, it is just about useless while you sit out here on the mountainside and grieve over your misfortunes, real and fancied. But you could do great things with your body if you would hasten off to where great things are waiting to be done. You are trying to run away from your unhappy self, but it cannot be done. You and your problems of living are real; you cannot escape them as long as you live. But look again, your mind is clear and capable. Your strong body has an intelligent mind to direct it. Set your mind at work to solve its problems; teach your intellect to work for you; refuse longer to be dominated by fear like an unthinking animal. Your mind should be your courageous ally in the solution of your life problems rather than your being, as you have been, its abject fear\hyp{}slave and the bond servant of depression and defeat. But most valuable of all, your potential of real achievement is the spirit which lives within you, and which will stimulate and inspire your mind to control itself and activate the body if you will release it from the fetters of fear and thus enable your spiritual nature to begin your deliverance from the evils of inaction by the power\hyp{}presence of living faith. And then, forthwith, will this faith vanquish fear of men by the compelling presence of that new and all\hyp{}dominating \bibemph{love of your fellows} which will so soon fill your soul to overflowing because of the consciousness which has been born in your heart that you are a child of God.
\vs p130 6:4 \textcolour{ubdarkred}{“This day, my son, you are to be reborn, re\hyp{}established as a man of faith, courage, and devoted service to man, for God’s sake. And when you become so readjusted to life within yourself, you become likewise readjusted to the universe; you have been born again --- born of the spirit --- and henceforth will your whole life become one of victorious accomplishment. Trouble will invigorate you; disappointment will spur you on; difficulties will challenge you; and obstacles will stimulate you. Arise, young man! Say farewell to the life of cringing fear and fleeing cowardice. Hasten back to duty and live your life in the flesh as a son of God, a mortal dedicated to the ennobling service of man on earth and destined to the superb and eternal service of God in eternity.”}
\vs p130 6:5 And this youth, Fortune, subsequently became the leader of the Christians in Crete and the close associate of Titus in his labours for the uplift of the Cretan believers.
\vs p130 6:6 \pc The travellers were truly rested and refreshed when they made ready about noon one day to sail for Carthage in northern Africa, stopping for two days at Cyrene. It was here that Jesus and Ganid gave first aid to a lad named Rufus, who had been injured by the breakdown of a loaded oxcart. They carried him home to his mother, and his father, Simon, little dreamed that the man whose cross he subsequently bore by orders of a Roman soldier was the stranger who once befriended his son.
\usection{At Carthage --- Discourse on Time and Space}
\vs p130 7:1 Most of the time en route to Carthage Jesus talked with his fellow travellers about things social, political, and commercial; hardly a word was said about religion. For the first time Gonod and Ganid discovered that Jesus was a good storyteller, and they kept him busy telling tales about his early life in Galilee. They also learned that he was reared in Galilee and not in either Jerusalem or Damascus.
\vs p130 7:2 When Ganid inquired what one could do to make friends, having noticed that the majority of persons whom they chanced to meet were attracted to Jesus, his teacher said: \textcolour{ubdarkred}{“Become interested in your fellows; learn how to love them and watch for the opportunity to do something for them which you are sure they want done,”} and then he quoted the olden Jewish proverb --- \textcolour{ubdarkred}{“A man who would have friends must show himself friendly.”}
\vs p130 7:3 At Carthage Jesus had a long and memorable talk with a Mithraic priest about immortality, about time and eternity. This Persian had been educated at Alexandria, and he really desired to learn from Jesus. Put into the words of today, in substance Jesus said in answer to his many questions:
\vs p130 7:4 \pc Time is the stream of flowing temporal events perceived by creature consciousness. Time is a name given to the succession\hyp{}arrangement whereby events are recognized and segregated. The universe of space is a time\hyp{}related phenomenon as it is viewed from any interior position outside of the fixed abode of Paradise. The motion of time is only revealed in relation to something which does not move in space as a time phenomenon. In the universe of universes Paradise and its Deities transcend both time and space. On the inhabited worlds, human personality (indwelt and oriented by the Paradise Father’s spirit) is the only physically related reality which can transcend the material sequence of temporal events.
\vs p130 7:5 Animals do not sense time as does man, and even to man, because of his sectional and circumscribed view, time appears as a succession of events; but as man ascends, as he progresses inward, the enlarging view of this event procession is such that it is discerned more and more in its wholeness. That which formerly appeared as a succession of events then will be viewed as a whole and perfectly related cycle; in this way will circular simultaneity increasingly displace the onetime consciousness of the linear sequence of events.
\vs p130 7:6 There are seven different conceptions of space as it is conditioned by time\fnst{\textbf{seven different conceptions of space}, These seven dimensions of space are the same as the seven dimensions of human type of personality as explained in \bibref[112:1.9]{p112 1:9}. The perception of the seven dimensions of space is in principle achievable even on the material level of existence, despite the claim of human mind being ``rigidly space\hyp{}bound'' in \bibref[12:5.5]{p012 5:5}. There is no real contradiction between what the Perfector of Wisdom is teaching us in Paper 12 and what Jesus is teaching the Mithraic mystic at Carthage, as long as we understand that the angelic teachings are usually very basic, whereas Jesus is here addressing a highly advanced individual.}. Space is measured by time, not time by space. The confusion of the scientist grows out of failure to recognize the reality of space. Space is not merely an intellectual concept of the variation in relatedness of universe objects. Space is not empty, and the only thing man knows which can even partially transcend space is mind\fnst{\textbf{the only thing man knows which can even partially transcend space is mind}, This would directly contradict the statement at \bibref[12:5.5]{p012 5:5}, if we do not take into account what was said in the previous note.}. Mind can function independently of the concept of the space\hyp{}relatedness of material objects. Space is relatively and comparatively finite to all beings of creature status. The nearer consciousness approaches the awareness of seven cosmic dimensions, the more does the concept of potential space approach ultimacy. But the space potential is truly ultimate only on the absolute level.
\vs p130 7:7 It must be apparent that universal reality has an expanding and always relative meaning on the ascending and perfecting levels of the cosmos. Ultimately, surviving mortals achieve identity in a seven\hyp{}dimensional universe.
\vs p130 7:8 \pc The time\hyp{}space concept of a mind of material origin is destined to undergo successive enlargements as the conscious and conceiving personality ascends the levels of the universes. When man attains the mind intervening between the material and the spiritual planes of existence, his ideas of time\hyp{}space will be enormously expanded both as to quality of perception and quantity of experience. The enlarging cosmic conceptions of an advancing spirit personality are due to augmentations of both depth of insight and scope of consciousness. And as personality passes on, upward and inward, to the transcendental levels of Deity\hyp{}likeness, the time\hyp{}space concept will increasingly approximate the timeless and spaceless concepts of the Absolutes. Relatively, and in accordance with transcendental attainment, these concepts of the absolute level are to be envisioned by the children of ultimate destiny.
\usection{On the Way to Naples and Rome}
\vs p130 8:1 The first stop on the way to Italy was at the island of Malta. Here Jesus had a long talk with a downhearted and discouraged young man named Claudus. This fellow had contemplated taking his life, but when he had finished talking with the scribe of Damascus, he said: “I will face life like a man; I am through playing the coward. I will go back to my people and begin all over again.” Shortly he became an enthusiastic preacher of the Cynics, and still later on he joined hands with Peter in proclaiming Christianity in Rome and Naples, and after the death of Peter he went on to Spain preaching the gospel. But he never knew that the man who inspired him in Malta was the Jesus whom he subsequently proclaimed the world’s Deliverer.
\vs p130 8:2 \pc At Syracuse they spent a full week. The notable event of their stop here was the rehabilitation of Ezra, the backslidden Jew, who kept the tavern where Jesus and his companions stopped. Ezra was charmed by Jesus’ approach and asked him to help him come back to the faith of Israel. He expressed his hopelessness by saying, “I want to be a true son of Abraham, but I cannot find God.” Said Jesus: \textcolour{ubdarkred}{“If you truly want to find God, that desire is in itself evidence that you have already found him. Your trouble is not that you cannot find God, for the Father has already found you; your trouble is simply that you do not know God. Have you not read in the Prophet Jeremiah, ‘You shall seek me and find me when you shall search for me with all your heart’? And again, does not this same prophet say: ‘And I will give you a heart to know me, that I am the Lord, and you shall belong to my people, and I will be your God’? And have you not also read in the Scriptures where it says: ‘He looks down upon men, and if any will say: I have sinned and perverted that which was right, and it profited me not, then will God deliver that man’s soul from darkness, and he shall see the light’?”} And Ezra found God and to the satisfaction of his soul. Later, this Jew, in association with a well\hyp{}to\hyp{}do Greek proselyte, built the first Christian church in Syracuse.
\vs p130 8:3 \pc At Messina they stopped for only one day, but that was long enough to change the life of a small boy, a fruit vendor, of whom Jesus bought fruit and in turn fed with the bread of life. The lad never forgot the words of Jesus and the kindly look which went with them when, placing his hand on the boy’s shoulder, he said: \textcolour{ubdarkred}{“Farewell, my lad, be of good courage as you grow up to manhood and after you have fed the body learn how also to feed the soul. And my Father in heaven will be with you and go before you.”} The lad became a devotee of the Mithraic religion and later on turned to the Christian faith.
\vs p130 8:4 \pc At last they reached Naples and felt they were not far from their destination, Rome. Gonod had much business to transact in Naples, and aside from the time Jesus was required as interpreter, he and Ganid spent their leisure visiting and exploring the city. Ganid was becoming adept at sighting those who appeared to be in need. They found much poverty in this city and distributed many alms. But Ganid never understood the meaning of Jesus’ words when, after he had given a coin to a street beggar, he refused to pause and speak comfortingly to the man. Said Jesus: \textcolour{ubdarkred}{“Why waste words upon one who cannot perceive the meaning of what you say? The spirit of the Father cannot teach and save one who has no capacity for sonship.”} What Jesus meant was that the man was not of normal mind; that he lacked the ability to respond to spirit leading.
\vs p130 8:5 There was no outstanding experience in Naples; Jesus and the young man thoroughly canvassed the city and spread good cheer with many smiles upon hundreds of men, women, and children.
\vs p130 8:6 From here they went by way of Capua to Rome, making a stop of three days at Capua. By the Appian Way they journeyed on beside their pack animals toward Rome, all three being anxious to see this mistress of empire and the greatest city in all the world.
\quizlink
