\upaper{193}{Final Appearances and Ascension}
\uminitoc{The Appearance at Sychar}
\uminitoc{The Phoenician Appearance}
\uminitoc{Last Appearance in Jerusalem}
\uminitoc{Causes of Judas’s Downfall}
\uminitoc{The Master’s Ascension}
\uminitoc{Peter Calls a Meeting}
\author{Midwayer Commission}
\vs p193 0:1 The 16\ts{th} morontia manifestation of Jesus occurred on Friday, May 5, in the courtyard of Nicodemus, about 21:00. On this evening the Jerusalem believers had made their first attempt to get together since the resurrection. Assembled here at this time were the 11 apostles, the women’s corps and their associates, and about 50 other leading disciples of the Master, including a number of the Greeks. This company of believers had been visiting informally for more than half an hour when, suddenly, the morontia Master appeared in full view and immediately began to instruct them. Said Jesus:
\vs p193 0:2 \pc \textcolour{ubdarkred}{“Peace be upon you. This is the most representative group of believers --- apostles and disciples, both men and women --- to which I have appeared since the time of my deliverance from the flesh. I now call you to witness that I told you beforehand that my sojourn among you must come to an end; I told you that presently I must return to the Father. And then I plainly told you how the chief priests and the rulers of the Jews would deliver me up to be put to death, and that I would rise from the grave. Why, then, did you allow yourselves to become so disconcerted by all this when it came to pass? and why were you so surprised when I rose from the tomb on the third day? You failed to believe me because you heard my words without comprehending the meaning thereof.}
\vs p193 0:3 \textcolour{ubdarkred}{“And now you should give ear to my words lest you again make the mistake of hearing my teaching with the mind while in your hearts you fail to comprehend the meaning. From the beginning of my sojourn as one of you, I taught you that my one purpose was to reveal my Father in heaven to his children on earth. I have lived the God\hyp{}revealing bestowal that you might experience the God\hyp{}knowing career. I have revealed God as your Father in heaven; I have revealed you as the sons of God on earth. It is a fact that God loves you, his sons. By faith in my word this fact becomes an eternal and living truth in your hearts. When, by living faith, you become divinely God\hyp{}conscious, you are then born of the spirit as children of light and life, even the eternal life wherewith you shall ascend the universe of universes and attain the experience of finding God the Father on Paradise.}
\vs p193 0:4 \textcolour{ubdarkred}{“I admonish you ever to remember that your mission among men is to proclaim the gospel of the kingdom --- the reality of the fatherhood of God and the truth of the sonship of man. Proclaim the whole truth of the good news, not just a part of the saving gospel. Your message is not changed by my resurrection experience. Sonship with God, by faith, is still the saving truth of the gospel of the kingdom. You are to go forth preaching the love of God and the service of man. That which the world needs most to know is: Men are the sons of God, and through faith they can actually realize, and daily experience, this ennobling truth. My bestowal should help all men to know that they are the children of God, but such knowledge will not suffice if they fail personally to faith\hyp{}grasp the saving truth that they are the living spirit sons of the eternal Father. The gospel of the kingdom is concerned with the love of the Father and the service of his children on earth.}
\vs p193 0:5 \textcolour{ubdarkred}{“Among yourselves, here, you share the knowledge that I have risen from the dead, but that is not strange. I have the power to lay down my life and to take it up again; the Father gives such power to his Paradise Sons. You should the rather be stirred in your hearts by the knowledge that the dead of an age entered upon the eternal ascent soon after I left Joseph’s new tomb. I lived my life in the flesh to show how you can, through loving service, become God\hyp{}revealing to your fellow men even as, by loving you and serving you, I have become God\hyp{}revealing to you. I have lived among you as the Son of Man that you, and all other men, might know that you are all indeed the sons of God. Therefore, go you now into all the world preaching this gospel of the kingdom of heaven to all men. Love all men as I have loved you; serve your fellow mortals as I have served you. Freely you have received, freely give. Only tarry here in Jerusalem while I go to the Father, and until I send you the Spirit of Truth. He shall lead you into the enlarged truth, and I will go with you into all the world. I am with you always, and my peace I leave with you.”}
\vs p193 0:6 \pc When the Master had spoken to them, he vanished from their sight. It was near daybreak before these believers dispersed; all night they remained together, earnestly discussing the Master’s admonitions and contemplating all that had befallen them. James Zebedee and others of the apostles also told them of their experiences with the morontia Master in Galilee and recited how he had three times appeared to them.
\usection{The Appearance at Sychar}
\vs p193 1:1 About 16:00 on Sabbath afternoon, May 13, the Master appeared to Nalda and about 75 Samaritan believers near Jacob’s well, at Sychar. The believers were in the habit of meeting at this place, near where Jesus had spoken to Nalda concerning the water of life. On this day, just as they had finished their discussions of the reported resurrection, Jesus suddenly appeared before them, saying:
\vs p193 1:2 \pc \textcolour{ubdarkred}{“Peace be upon you. You rejoice to know that I am the resurrection and the life, but this will avail you nothing unless you are first born of the eternal spirit, thereby coming to possess, by faith, the gift of eternal life. If you are the faith sons of my Father, you shall never die; you shall not perish. The gospel of the kingdom has taught you that all men are the sons of God. And this good news concerning the love of the heavenly Father for his children on earth must be carried to all the world. The time has come when you worship God neither on Gerizim nor at Jerusalem, but where you are, as you are, in spirit and in truth. It is your faith that saves your souls. Salvation is the gift of God to all who believe they are his sons. But be not deceived; while salvation is the free gift of God and is bestowed upon all who accept it by faith, there follows the experience of bearing the fruits of this spirit life as it is lived in the flesh. The acceptance of the doctrine of the fatherhood of God implies that you also freely accept the associated truth of the brotherhood of man. And if man is your brother, he is even more than your neighbour, whom the Father requires you to love as yourself. Your brother, being of your own family, you will not only love with a family affection, but you will also serve as you would serve yourself. And you will thus love and serve your brother because you, being my brethren, have been thus loved and served by me. Go, then, into all the world telling this good news to all creatures of every race, tribe, and nation. My spirit shall go before you, and I will be with you always.”}
\vs p193 1:3 \pc These Samaritans were greatly astonished at this appearance of the Master, and they hastened off to the near\hyp{}by towns and villages, where they published abroad the news that they had seen Jesus, and that he had talked to them. And this was the 17\ts{th} morontia appearance of the Master.
\usection{The Phoenician Appearance}
\vs p193 2:1 The Master’s 18\ts{th} morontia appearance was at Tyre, on Tuesday, May 16, at a little before 21:00. Again he appeared at the close of a meeting of believers, as they were about to disperse, saying:
\vs p193 2:2 \pc \textcolour{ubdarkred}{“Peace be upon you. You rejoice to know that the Son of Man has risen from the dead because you thereby know that you and your brethren shall also survive mortal death. But such survival is dependent on your having been previously born of the spirit of truth\hyp{}seeking and God\hyp{}finding. The bread of life and the water thereof are given only to those who hunger for truth and thirst for righteousness --- for God. The fact that the dead rise is not the gospel of the kingdom. These great truths and these universe facts are all related to this gospel in that they are a part of the result of believing the good news and are embraced in the subsequent experience of those who, by faith, become, in deed and in truth, the everlasting sons of the eternal God. My Father sent me into the world to proclaim this salvation of sonship to all men. And so send I you abroad to preach this salvation of sonship. Salvation is the free gift of God, but those who are born of the spirit will immediately begin to show forth the fruits of the spirit in loving service to their fellow creatures. And the fruits of the divine spirit which are yielded in the lives of spirit\hyp{}born and God\hyp{}knowing mortals are: loving service, unselfish devotion, courageous loyalty, sincere fairness, enlightened honesty, undying hope, confiding trust, merciful ministry, unfailing goodness, forgiving tolerance, and enduring peace. If professed believers bear not these fruits of the divine spirit in their lives, they are dead; the Spirit of Truth is not in them; they are useless branches on the living vine, and they soon will be taken away. My Father requires of the children of faith that they bear much spirit fruit. If, therefore, you are not fruitful, he will dig about your roots and cut away your unfruitful branches. Increasingly, must you yield the fruits of the spirit as you progress heavenward in the kingdom of God. You may enter the kingdom as a child, but the Father requires that you grow up, by grace, to the full stature of spiritual adulthood. And when you go abroad to tell all nations the good news of this gospel, I will go before you, and my Spirit of Truth shall abide in your hearts. My peace I leave with you.”}
\vs p193 2:3 \pc And then the Master disappeared from their sight. The next day there went out from Tyre those who carried this story to Sidon and even to Antioch and Damascus. Jesus had been with these believers when he was in the flesh, and they were quick to recognize him when he began to teach them. While his friends could not readily recognize his morontia form when made visible, they were never slow to identify his personality when he spoke to them.
\usection{Last Appearance in Jerusalem}
\vs p193 3:1 Early Thursday morning, May 18, Jesus made his last appearance on earth as a morontia personality. As the 11 apostles were about to sit down to breakfast in the upper chamber of Mary Mark’s home, Jesus appeared to them and said:
\vs p193 3:2 \pc \textcolour{ubdarkred}{“Peace be upon you. I have asked you to tarry here in Jerusalem until I ascend to the Father, even until I send you the Spirit of Truth, who shall soon be poured out upon all flesh, and who shall endow you with power from on high.”} Simon Zelotes interrupted Jesus, asking, “Then, Master, will you restore the kingdom, and will we see the glory of God manifested on earth?” When Jesus had listened to Simon’s question, he answered: \textcolour{ubdarkred}{“Simon, you still cling to your old ideas about the Jewish Messiah and the material kingdom. But you will receive spiritual power after the spirit has descended upon you, and you will presently go into all the world preaching this gospel of the kingdom. As the Father sent me into the world, so do I send you. And I wish that you would love and trust one another. Judas is no more with you because his love grew cold, and because he refused to trust you, his loyal brethren. Have you not read in the Scripture where it is written: ‘It is not good for man to be alone. No man lives to himself’? And also where it says: ‘He who would have friends must show himself friendly’? And did I not even send you out to teach, two and two, that you might not become lonely and fall into the mischief and miseries of isolation? You also well know that, when I was in the flesh, I did not permit myself to be alone for long periods. From the very beginning of our associations I always had two or three of you constantly by my side or else very near at hand even when I communed with the Father. Trust, therefore, and confide in one another. And this is all the more needful since I am this day going to leave you alone in the world. The hour has come; I am about to go to the Father.”}
\vs p193 3:3 \pc When he had spoken, he beckoned for them to come with him, and he led them out on the Mount of Olives, where he bade them farewell preparatory to departing from Urantia. This was a solemn journey to Olivet. Not a word was spoken by any of them from the time they left the upper chamber until Jesus paused with them on the Mount of Olives.
\usection{Causes of Judas’s Downfall}
\vs p193 4:1 It was in the first part of the Master’s farewell message to his apostles that he alluded to the loss of Judas and held up the tragic fate of their traitorous fellow worker as a solemn warning against the dangers of social and fraternal isolation. It may be helpful to believers, in this and in future ages, briefly to review the causes of Judas’s downfall in the light of the Master’s remarks and in view of the accumulated enlightenment of succeeding centuries.
\vs p193 4:2 As we look back upon this tragedy, we conceive that Judas went wrong, primarily, because he was very markedly an isolated personality, a personality shut in and away from ordinary social contacts. He persistently refused to confide in, or freely fraternize with, his fellow apostles. But his being an isolated type of personality would not, in and of itself, have wrought such mischief for Judas had it not been that he also failed to increase in love and grow in spiritual grace. And then, as if to make a bad matter worse, he persistently harboured grudges and fostered such psychologic enemies as revenge and the generalized craving to “get even” with somebody for all his disappointments.
\vs p193 4:3 This unfortunate combination of individual peculiarities and mental tendencies conspired to destroy a well\hyp{}intentioned man who failed to subdue these evils by love, faith, and trust. That Judas need not have gone wrong is well proved by the cases of Thomas and Nathaniel, both of whom were cursed with this same sort of suspicion and overdevelopment of the individualistic tendency. Even Andrew and Matthew had many leanings in this direction; but all these men grew to love Jesus and their fellow apostles more, and not less, as time passed. They grew in grace and in a knowledge of the truth. They became increasingly more trustful of their brethren and slowly developed the ability to confide in their fellows. Judas persistently refused to confide in his brethren. When he was impelled, by the accumulation of his emotional conflicts, to seek relief in self\hyp{}expression, he invariably sought the advice and received the unwise consolation of his unspiritual relatives or those chance acquaintances who were either indifferent, or actually hostile, to the welfare and progress of the spiritual realities of the heavenly kingdom, of which he was one of the twelve consecrated ambassadors on earth.
\vs p193 4:4 Judas met defeat in his battles of the earth struggle because of the following factors of personal tendencies and character weakness:
\vs p193 4:5 \ublistelem{1.}\bibnobreakspace He was an isolated type of human being. He was highly individualistic and chose to grow into a confirmed “shut\hyp{}in” and unsociable sort of person.
\vs p193 4:6 \ublistelem{2.}\bibnobreakspace As a child, life had been made too easy for him. He bitterly resented thwarting. He always expected to win; he was a very poor loser.
\vs p193 4:7 \ublistelem{3.}\bibnobreakspace He never acquired a philosophic technique for meeting disappointment. Instead of accepting disappointments as a regular and commonplace feature of human existence, he unfailingly resorted to the practice of blaming someone in particular, or his associates as a group, for all his personal difficulties and disappointments.
\vs p193 4:8 \ublistelem{4.}\bibnobreakspace He was given to holding grudges; he was always entertaining the idea of revenge.
\vs p193 4:9 \ublistelem{5.}\bibnobreakspace He did not like to face facts frankly; he was dishonest in his attitude toward life situations.
\vs p193 4:10 \ublistelem{6.}\bibnobreakspace He disliked to discuss his personal problems with his immediate associates; he refused to talk over his difficulties with his real friends and those who truly loved him. In all the years of their association he never once went to the Master with a purely personal problem.
\vs p193 4:11 \ublistelem{7.}\bibnobreakspace He never learned that the real rewards for noble living are, after all, spiritual prizes, which are not always distributed during this one short life in the flesh.
\vs p193 4:12 \pc As a result of his persistent isolation of personality, his griefs multiplied, his sorrows increased, his anxieties augmented, and his despair deepened almost beyond endurance.
\vs p193 4:13 While this self\hyp{}centred and ultraindividualistic apostle had many psychic, emotional, and spiritual troubles, his main difficulties were: In personality, he was isolated. In mind, he was suspicious and vengeful. In temperament, he was surly and vindictive. Emotionally, he was loveless and unforgiving. Socially, he was unconfiding and almost wholly self\hyp{}contained. In spirit, he became arrogant and selfishly ambitious. In life, he ignored those who loved him, and in death, he was friendless.
\vs p193 4:14 These, then, are the factors of mind and influences of evil which, taken altogether, explain why a well\hyp{}meaning and otherwise onetime sincere believer in Jesus, even after several years of intimate association with his transforming personality, forsook his fellows, repudiated a sacred cause, renounced his holy calling, and betrayed his divine Master.
\usection{The Master’s Ascension}
\vs p193 5:1 It was almost 7:30 this Thursday morning, May 18, when Jesus arrived on the western slope of Mount Olivet with his 11 silent and somewhat bewildered apostles. From this location, about \bibfrac{2}{3}\ts{rds} the way up the mountain, they could look out over Jerusalem and down upon Gethsemane. Jesus now prepared to say his last farewell to the apostles before he took leave of Urantia. As he stood there before them, without being directed they knelt about him in a circle, and the Master said:
\vs p193 5:2 \pc \textcolour{ubdarkred}{“I bade you tarry in Jerusalem until you were endowed with power from on high. I am now about to take leave of you; I am about to ascend to my Father, and soon, very soon, will we send into this world of my sojourn the Spirit of Truth; and when he has come, you shall begin the new proclamation of the gospel of the kingdom, first in Jerusalem and then to the uttermost parts of the world. Love men with the love wherewith I have loved you and serve your fellow mortals even as I have served you. By the spirit fruits of your lives impel souls to believe the truth that man is a son of God, and that all men are brethren. Remember all I have taught you and the life I have lived among you. My love overshadows you, my spirit will dwell with you, and my peace shall abide upon you. Farewell.”}
\vs p193 5:3 \pc When the morontia Master had thus spoken, he vanished from their sight. This so\hyp{}called ascension of Jesus was in no way different from his other disappearances from mortal vision during the 40 days of his morontia career on Urantia.
\vs p193 5:4 The Master went to Edentia by way of Jerusem, where the Most Highs, under the observation of the Paradise Son, released Jesus of Nazareth from the morontia state and, through the spirit channels of ascension, returned him to the status of Paradise sonship and supreme sovereignty on Salvington.
\vs p193 5:5 It was about 7:45 this morning when the morontia Jesus disappeared from the observation of his 11 apostles to begin the ascent to the right hand of his Father, there to receive formal confirmation of his completed sovereignty of the universe of Nebadon.
\usection{Peter Calls a Meeting}
\vs p193 6:1 Acting upon the instruction of Peter, John Mark and others went forth to call the leading disciples together at the home of Mary Mark. By 10:30\fnst{\textbf{10:30}, The previous section ends at 7:45, 18\ts{th} May and this section begins 10 days later, at 10:30, 28\ts{th} May. That this is so can be inferred from \bibref[194:1.1]{p194 1:1}, where it is stated that this is the day of Pentecost, which occurs 50 days after the Passover.}, 120 of the foremost disciples of Jesus living in Jerusalem had forgathered to hear the report of the farewell message of the Master and to learn of his ascension. Among this company was Mary the mother of Jesus. She had returned to Jerusalem with John Zebedee when the apostles came back from their recent sojourn in Galilee. Soon after Pentecost she returned to the home of Salome at Bethsaida. James the brother of Jesus was also present at this meeting, the first conference of the Master’s disciples to be called after the termination of his planetary career.
\vs p193 6:2 Simon Peter took it upon himself to speak for his fellow apostles and made a thrilling report of the last meeting of the 11 with their Master and most touchingly portrayed the Master’s final farewell and his ascension disappearance. It was a meeting the like of which had never before occurred on this world. This part of the meeting lasted not quite one hour. Peter then explained that they had decided to choose a successor to Judas Iscariot, and that a recess would be granted to enable the apostles to decide between the two men who had been suggested for this position, Matthias and Justus.
\vs p193 6:3 The 11 apostles then went downstairs, where they agreed to cast lots in order to determine which of these men should become an apostle to serve in Judas’s place. The lot fell on Matthias, and he was declared to be the new apostle. He was duly inducted into his office and then appointed treasurer. But Matthias had little part in the subsequent activities of the apostles.
\vs p193 6:4 \pc Soon after Pentecost the twins returned to their homes in Galilee. Simon Zelotes was in retirement for some time before he went forth preaching the gospel. Thomas worried for a shorter period and then resumed his teaching. Nathaniel differed increasingly with Peter regarding preaching about Jesus in the place of proclaiming the former gospel of the kingdom. This disagreement became so acute by the middle of the following month that Nathaniel withdrew, going to Philadelphia to visit Abner and Lazarus; and after tarrying there for more than a year, he went on into the lands beyond Mesopotamia preaching the gospel as he understood it.
\vs p193 6:5 This left but 6 of the original twelve apostles to become actors on the stage of the early proclamation of the gospel in Jerusalem: Peter, Andrew, James, John, Philip, and Matthew.
\vs p193 6:6 \pc Just about noon the apostles returned to their brethren in the upper chamber and announced that Matthias had been chosen as the new apostle. And then Peter called all of the believers to engage in prayer, prayer that they might be prepared to receive the gift of the spirit which the Master had promised to send.
\quizlink
