\upaper{103}{The Reality of Religious Experience}
\author{Melchizedek}
\vs p103 0:1 All of man’s truly religious reactions are sponsored by the early ministry of the adjutant of worship and are censored by the adjutant of wisdom. Man’s first supermind endowment is that of personality encircuitment in the Holy Spirit of the Universe Creative Spirit; and long before either the bestowals of the divine Sons or the universal bestowal of the Adjusters, this influence functions to enlarge man’s viewpoint of ethics, religion, and spirituality. Subsequent to the bestowals of the Paradise Sons the liberated Spirit of Truth makes mighty contributions to the enlargement of the human capacity to perceive religious truths. As evolution advances on an inhabited world, the Thought Adjusters increasingly participate in the development of the higher types of human religious insight. The Thought Adjuster is the cosmic window through which the finite creature may faith\hyp{}glimpse the certainties and divinities of limitless Deity, the Universal Father.
\vs p103 0:2 The religious tendencies of the human races are innate; they are universally manifested and have an apparently natural origin; primitive religions are always evolutionary in their genesis. As natural religious experience continues to progress, periodic revelations of truth punctuate the otherwise slow\hyp{}moving course of planetary evolution.
\vs p103 0:3 \pc On Urantia, today, there are four kinds of religion:
\vs p103 0:4 \ublistelem{1.}\bibnobreakspace Natural or evolutionary religion.
\vs p103 0:5 \ublistelem{2.}\bibnobreakspace Supernatural or revelatory religion.
\vs p103 0:6 \ublistelem{3.}\bibnobreakspace Practical or current religion, varying degrees of the admixture of natural and supernatural religions.
\vs p103 0:7 \ublistelem{4.}\bibnobreakspace Philosophic religions, man\hyp{}made or philosophically thought\hyp{}out theologic doctrines and reason\hyp{}created religions.
\usection{1.\bibnobreakspace Philosophy of Religion}
\vs p103 1:1 The unity of religious experience among a social or racial group derives from the identical nature of the God fragment indwelling the individual. It is this divine in man that gives origin to his unselfish interest in the welfare of other men. But since personality is unique --- no two mortals being alike --- it inevitably follows that no two human beings can similarly interpret the leadings and urges of the spirit of divinity which lives within their minds. A group of mortals can experience spiritual unity, but they can never attain philosophic uniformity. And this diversity of the interpretation of religious thought and experience is shown by the fact that XX century theologians and philosophers have formulated upward of 500 different definitions of religion. In reality, every human being defines religion in the terms of his own experiential interpretation of the divine impulses emanating from the God spirit that indwells him, and therefore must such an interpretation be unique and wholly different from the religious philosophy of all other human beings.
\vs p103 1:2 When one mortal is in full agreement with the religious philosophy of a fellow mortal, that phenomenon indicates that these two beings have had a similar \bibemph{religious experience} touching the matters concerned in their similarity of philosophic religious interpretation.
\vs p103 1:3 While your religion is a matter of personal experience, it is most important that you should be exposed to the knowledge of a vast number of other religious experiences (the diverse interpretations of other and diverse mortals) to the end that you may prevent your religious life from becoming egocentric --- circumscribed, selfish, and unsocial.
\vs p103 1:4 Rationalism is wrong when it assumes that religion is at first a primitive belief in something which is then followed by the pursuit of values. Religion is primarily a pursuit of values, and then there formulates a system of interpretative beliefs. It is much easier for men to agree on religious values --- goals --- than on beliefs --- interpretations. And this explains how religion can agree on values and goals while exhibiting the confusing phenomenon of maintaining a belief in hundreds of conflicting beliefs --- creeds. This also explains why a given person can maintain his religious experience in the face of giving up or changing many of his religious beliefs. Religion persists in spite of revolutionary changes in religious beliefs. Theology does not produce religion; it is religion that produces theologic philosophy.
\vs p103 1:5 That religionists have believed so much that was false does not invalidate religion because religion is founded on the recognition of values and is validated by the faith of personal religious experience. Religion, then, is based on experience and religious thought; theology, the philosophy of religion, is an honest attempt to interpret that experience. Such interpretative beliefs may be right or wrong, or a mixture of truth and error.
\vs p103 1:6 The realization of the recognition of spiritual values is an experience which is superideational. There is no word in any human language which can be employed to designate this “sense,” “feeling,” “intuition,” or “experience” which we have elected to call God\hyp{}consciousness. The spirit of God that dwells in man is not personal --- the Adjuster is prepersonal --- but this Monitor presents a value, exudes a flavour of divinity, which is personal in the highest and infinite sense. If God were not at least personal, he could not be conscious, and if not conscious, then would he be infrahuman.
\usection{2.\bibnobreakspace Religion and the Individual}
\vs p103 2:1 Religion is functional in the human mind and has been realized in experience prior to its appearance in human consciousness. A child has been in existence about nine months before it experiences \bibemph{birth.} But the “birth” of religion is not sudden; it is rather a gradual emergence. Nevertheless, sooner or later there is a “birth day.” You do not enter the kingdom of heaven unless you have been “born again” --- born of the Spirit. Many spiritual births are accompanied by much anguish of spirit and marked psychological perturbations, as many physical births are characterized by a “stormy labour” and other abnormalities of “delivery.” Other spiritual births are a natural and normal growth of the recognition of supreme values with an enhancement of spiritual experience, albeit no religious development occurs without conscious effort and positive and individual determinations. Religion is never a passive experience, a negative attitude. What is termed the “birth of religion” is not directly associated with so\hyp{}called conversion experiences which usually characterize religious episodes occurring later in life as a result of mental conflict, emotional repression, and temperamental upheavals.
\vs p103 2:2 But those persons who were so reared by their parents that they grew up in the consciousness of being children of a loving heavenly Father, should not look askance at their fellow mortals who could only attain such consciousness of fellowship with God through a psychological crisis, an emotional upheaval.
\vs p103 2:3 The evolutionary soil in the mind of man in which the seed of revealed religion germinates is the moral nature that so early gives origin to a social consciousness. The first promptings of a child’s moral nature have not to do with sex, guilt, or personal pride, but rather with impulses of justice, fairness, and urges to kindness --- helpful ministry to one’s fellows. And when such early moral awakenings are nurtured, there occurs a gradual development of the religious life which is comparatively free from conflicts, upheavals, and crises.
\vs p103 2:4 Every human being very early experiences something of a conflict between his self\hyp{}seeking and his altruistic impulses, and many times the first experience of God\hyp{}consciousness may be attained as the result of seeking for superhuman help in the task of resolving such moral conflicts.
\vs p103 2:5 The psychology of a child is naturally positive, not negative. So many mortals are negative because they were so trained. When it is said that the child is positive, reference is made to his moral impulses, those powers of mind whose emergence signals the arrival of the Thought Adjuster.
\vs p103 2:6 In the absence of wrong teaching, the mind of the normal child moves positively, in the emergence of religious consciousness, toward moral righteousness and social ministry, rather than negatively, away from sin and guilt. There may or may not be conflict in the development of religious experience, but there are always present the inevitable decisions, effort, and function of the human will.
\vs p103 2:7 Moral choosing is usually accompanied by more or less moral conflict. And this very first conflict in the child mind is between the urges of egoism and the impulses of altruism. The Thought Adjuster does not disregard the personality values of the egoistic motive but does operate to place a slight preference upon the altruistic impulse as leading to the goal of human happiness and to the joys of the kingdom of heaven.
\vs p103 2:8 When a moral being chooses to be unselfish when confronted by the urge to be selfish, that is primitive religious experience. No animal can make such a choice; such a decision is both human and religious. It embraces the fact of God\hyp{}consciousness and exhibits the impulse of social service, the basis of the brotherhood of man. When mind chooses a right moral judgment by an act of the free will, such a decision constitutes a religious experience.
\vs p103 2:9 But before a child has developed sufficiently to acquire moral capacity and therefore to be able to choose altruistic service, he has already developed a strong and well\hyp{}unified egoistic nature. And it is this factual situation that gives rise to the theory of the struggle between the “higher” and the “lower” natures, between the “old man of sin” and the “new nature” of grace. Very early in life the normal child begins to learn that it is “more blessed to give than to receive.”
\vs p103 2:10 Man tends to identify the urge to be self\hyp{}serving with his ego --- himself. In contrast he is inclined to identify the will to be altruistic with some influence outside himself --- God. And indeed is such a judgment right, for all such nonself desires do actually have their origin in the leadings of the indwelling Thought Adjuster, and this Adjuster is a fragment of God. The impulse of the spirit Monitor is realized in human consciousness as the urge to be altruistic, fellow\hyp{}creature minded. At least this is the early and fundamental experience of the child mind. When the growing child fails of personality unification, the altruistic drive may become so overdeveloped as to work serious injury to the welfare of the self. A misguided conscience can become responsible for much conflict, worry, sorrow, and no end of human unhappiness.
\usection{3.\bibnobreakspace Religion and the Human Race}
\vs p103 3:1 While the belief in spirits, dreams, and diverse other superstitions all played a part in the evolutionary origin of primitive religions, you should not overlook the influence of the clan or tribal spirit of solidarity. In the group relationship there was presented the exact social situation which provided the challenge to the egoistic\hyp{}altruistic conflict in the moral nature of the early human mind. In spite of their belief in spirits, primitive Australians still focus their religion upon the clan. In time, such religious concepts tend to personalize, first, as animals, and later, as a superman or as a God. Even such inferior races as the African Bushmen, who are not even totemic in their beliefs, do have a recognition of the difference between the self\hyp{}interest and the group\hyp{}interest, a primitive distinction between the values of the secular and the sacred. But the social group is not the source of religious experience. Regardless of the influence of all these primitive contributions to man’s early religion, the fact remains that the true religious impulse has its origin in genuine spirit presences activating the will to be unselfish.
\vs p103 3:2 \pc Later religion is foreshadowed in the primitive belief in natural wonders and mysteries, the impersonal mana. But sooner or later the evolving religion requires that the individual should make some personal sacrifice for the good of his social group, should do something to make other people happier and better. Ultimately, religion is destined to become the service of God and of man.
\vs p103 3:3 Religion is designed to change man’s environment, but much of the religion found among mortals today has become helpless to do this. Environment has all too often mastered religion.
\vs p103 3:4 \pc Remember that in the religion of all ages the experience which is paramount is the feeling regarding moral values and social meanings, not the thinking regarding theologic dogmas or philosophic theories. Religion evolves favourably as the element of magic is replaced by the concept of morals.
\vs p103 3:5 Man evolved through the superstitions of mana, magic, nature worship, spirit fear, and animal worship to the various ceremonials whereby the religious attitude of the individual became the group reactions of the clan. And then these ceremonies became focalized and crystallized into tribal beliefs, and eventually these fears and faiths became personalized into gods. But in all of this religious evolution the moral element was never wholly absent. The impulse of the God within man was always potent. And these powerful influences --- one human and the other divine --- ensured the survival of religion throughout the vicissitudes of the ages and that notwithstanding it was so often threatened with extinction by 1,000 subversive tendencies and hostile antagonisms.
\usection{4.\bibnobreakspace Spiritual Communion}
\vs p103 4:1 The characteristic difference between a social occasion and a religious gathering is that in contrast with the secular the religious is pervaded by the atmosphere of \bibemph{communion.} In this way human association generates a feeling of fellowship with the divine, and this is the beginning of group worship. Partaking of a common meal was the earliest type of social communion, and so did early religions provide that some portion of the ceremonial sacrifice should be eaten by the worshippers. Even in Christianity the Lord’s Supper retains this mode of communion. The atmosphere of the communion provides a refreshing and comforting period of truce in the conflict of the self\hyp{}seeking ego with the altruistic urge of the indwelling spirit Monitor. And this is the prelude to true worship --- the practice of the presence of God which eventuates in the emergence of the brotherhood of man.
\vs p103 4:2 When primitive man felt that his communion with God had been interrupted, he resorted to sacrifice of some kind in an effort to make atonement, to restore friendly relationship. The hunger and thirst for righteousness leads to the discovery of truth, and truth augments ideals, and this creates new problems for the individual religionists, for our ideals tend to grow by geometrical progression, while our ability to live up to them is enhanced only by arithmetical progression.
\vs p103 4:3 The sense of guilt (not the consciousness of sin) comes either from interrupted spiritual communion or from the lowering of one’s moral ideals. Deliverance from such a predicament can only come through the realization that one’s highest moral ideals are not necessarily synonymous with the will of God. Man cannot hope to live up to his highest ideals, but he can be true to his purpose of finding God and becoming more and more like him.
\vs p103 4:4 Jesus swept away all of the ceremonials of sacrifice and atonement. He destroyed the basis of all this fictitious guilt and sense of isolation in the universe by declaring that man is a child of God; the creature\hyp{}Creator relationship was placed on a child\hyp{}parent basis. God becomes a loving Father to his mortal sons and daughters. All ceremonials not a legitimate part of such an intimate family relationship are forever abrogated.
\vs p103 4:5 God the Father deals with man his child on the basis, not of actual virtue or worthiness, but in recognition of the child’s motivation --- the creature purpose and intent. The relationship is one of parent\hyp{}child association and is actuated by divine love.
\usection{5.\bibnobreakspace The Origin of Ideals}
\vs p103 5:1 The early evolutionary mind gives origin to a feeling of social duty and moral obligation derived chiefly from emotional fear. The more positive urge of social service and the idealism of altruism are derived from the direct impulse of the divine spirit indwelling the human mind.
\vs p103 5:2 This idea\hyp{}ideal of doing good to others --- the impulse to deny the ego something for the benefit of one’s neighbour --- is very circumscribed at first. Primitive man regards as neighbour only those very close to him, those who treat him neighbourly; as religious civilization advances, one’s neighbour expands in concept to embrace the clan, the tribe, the nation. And then Jesus enlarged the neighbour scope to embrace the whole of humanity, even that we should love our enemies. And there is something inside of every normal human being that tells him this teaching is moral --- right. Even those who practise this ideal least, admit that it is right in theory.
\vs p103 5:3 All men recognize the morality of this universal human urge to be unselfish and altruistic. The humanist ascribes the origin of this urge to the natural working of the material mind; the religionist more correctly recognizes that the truly unselfish drive of mortal mind is in response to the inner spirit leadings of the Thought Adjuster.
\vs p103 5:4 But man’s interpretation of these early conflicts between the ego\hyp{}will and the other\hyp{}than\hyp{}self\hyp{}will is not always dependable. Only a fairly well unified personality can arbitrate the multiform contentions of the ego cravings and the budding social consciousness. The self has rights as well as one’s neighbours. Neither has exclusive claims upon the attention and service of the individual. Failure to resolve this problem gives origin to the earliest type of human guilt feelings.
\vs p103 5:5 Human happiness is achieved only when the ego desire of the self and the altruistic urge of the higher self (divine spirit) are co\hyp{}ordinated and reconciled by the unified will of the integrating and supervising personality. The mind of evolutionary man is ever confronted with the intricate problem of refereeing the contest between the natural expansion of emotional impulses and the moral growth of unselfish urges predicated on spiritual insight --- genuine religious reflection.
\vs p103 5:6 The attempt to secure equal good for the self and for the greatest number of other selves presents a problem which cannot always be satisfactorily resolved in a time\hyp{}space frame. Given an eternal life, such antagonisms can be worked out, but in one short human life they are incapable of solution. Jesus referred to such a paradox when he said: “Whosoever shall save his life shall lose it, but whosoever shall lose his life for the sake of the kingdom, shall find it.”
\vs p103 5:7 \pc The pursuit of the ideal --- the striving to be Godlike --- is a continuous effort before death and after. The life after death is no different in the essentials than the mortal existence. Everything we do in this life which is good contributes directly to the enhancement of the future life. Real religion does not foster moral indolence and spiritual laziness by encouraging the vain hope of having all the virtues of a noble character bestowed upon one as a result of passing through the portals of natural death. True religion does not belittle man’s efforts to progress during the mortal lease on life. Every mortal gain is a direct contribution to the enrichment of the first stages of the immortal survival experience.
\vs p103 5:8 \pc It is fatal to man’s idealism when he is taught that all of his altruistic impulses are merely the development of his natural herd instincts. But he is ennobled and mightily energized when he learns that these higher urges of his soul emanate from the spiritual forces that indwell his mortal mind.
\vs p103 5:9 It lifts man out of himself and beyond himself when he once fully realizes that there lives and strives within him something which is eternal and divine. And so it is that a living faith in the superhuman origin of our ideals validates our belief that we are the sons of God and makes real our altruistic convictions, the feelings of the brotherhood of man.
\vs p103 5:10 Man, in his spiritual domain, does have a free will. Mortal man is neither a helpless slave of the inflexible sovereignty of an all\hyp{}powerful God nor the victim of the hopeless fatality of a mechanistic cosmic determinism. Man is most truly the architect of his own eternal destiny.
\vs p103 5:11 \pc But man is not saved or ennobled by pressure. Spirit growth springs from within the evolving soul. Pressure may deform the personality, but it never stimulates growth. Even educational pressure is only negatively helpful in that it may aid in the prevention of disastrous experiences. Spiritual growth is greatest where all external pressures are at a minimum. “Where the spirit of the Lord is, there is freedom.” Man develops best when the pressures of home, community, church, and state are least. But this must not be construed as meaning that there is no place in a progressive society for home, social institutions, church, and state.
\vs p103 5:12 When a member of a social religious group has complied with the requirements of such a group, he should be encouraged to enjoy religious liberty in the full expression of his own personal interpretation of the truths of religious belief and the facts of religious experience. The security of a religious group depends on spiritual unity, not on theological uniformity. A religious group should be able to enjoy the liberty of freethinking without having to become “freethinkers.” There is great hope for any church that worships the living God, validates the brotherhood of man, and dares to remove all creedal pressure from its members.
\usection{6.\bibnobreakspace Philosophic Co\hyp{}ordination}
\vs p103 6:1 Theology is the study of the actions and reactions of the human spirit; it can never become a science since it must always be combined more or less with psychology in its personal expression and with philosophy in its systematic portrayal. Theology is always the study of \bibemph{your} religion; the study of another’s religion is psychology.
\vs p103 6:2 \pc When man approaches the study and examination of his universe from the \bibemph{outside,} he brings into being the various physical sciences; when he approaches the research of himself and the universe from the \bibemph{inside,} he gives origin to theology and metaphysics. The later art of philosophy develops in an effort to harmonize the many discrepancies which are destined at first to appear between the findings and teachings of these two diametrically opposite avenues of approaching the universe of things and beings.
\vs p103 6:3 Religion has to do with the spiritual viewpoint, the awareness of the \bibemph{insideness} of human experience. Man’s spiritual nature affords him the opportunity of turning the universe outside in. It is therefore true that, viewed exclusively from the insideness of personality experience, all creation appears to be spiritual in nature.
\vs p103 6:4 When man analytically inspects the universe through the material endowments of his physical senses and associated mind perception, the cosmos appears to be mechanical and energy\hyp{}material. Such a technique of studying reality consists in turning the universe inside out.
\vs p103 6:5 \pc A logical and consistent philosophic concept of the universe cannot be built up on the postulations of either materialism or spiritism, for both of these systems of thinking, when universally applied, are compelled to view the cosmos in distortion, the former contacting with a universe turned inside out, the latter realizing the nature of a universe turned outside in. Never, then, can either science or religion, in and of themselves, standing alone, hope to gain an adequate understanding of universal truths and relationships without the guidance of human philosophy and the illumination of divine revelation.
\vs p103 6:6 Always must man’s inner spirit depend for its expression and self\hyp{}realization upon the mechanism and technique of the mind. Likewise must man’s outer experience of material reality be predicated on the mind consciousness of the experiencing personality. Therefore are the spiritual and the material, the inner and the outer, human experiences always correlated with the mind function and conditioned, as to their conscious realization, by the mind activity. Man experiences matter in his mind; he experiences spiritual reality in the soul but becomes conscious of this experience in his mind. The intellect is the harmonizer and the ever\hyp{}present conditioner and qualifier of the sum total of mortal experience. Both energy\hyp{}things and spirit values are coloured by their interpretation through the mind media of consciousness.
\vs p103 6:7 Your difficulty in arriving at a more harmonious co\hyp{}ordination between science and religion is due to your utter ignorance of the intervening domain of the morontia world of things and beings. The local universe consists of three degrees, or stages, of reality manifestation: matter, morontia, and spirit. The morontia angle of approach erases all divergence between the findings of the physical sciences and the functioning of the spirit of religion. Reason is the understanding technique of the sciences; faith is the insight technique of religion; mota is the technique of the morontia level. Mota is a supermaterial reality sensitivity which is beginning to compensate incomplete growth, having for its substance knowledge\hyp{}reason and for its essence faith\hyp{}insight. Mota is a superphilosophical reconciliation of divergent reality perception which is nonattainable by material personalities; it is predicated, in part, on the experience of having survived the material life of the flesh. But many mortals have recognized the desirability of having some method of reconciling the interplay between the widely separated domains of science and religion; and metaphysics is the result of man’s unavailing attempt to span this well\hyp{}recognized chasm. But human metaphysics has proved more confusing than illuminating. Metaphysics stands for man’s well\hyp{}meant but futile effort to compensate for the absence of the mota of morontia.
\vs p103 6:8 \pc Metaphysics has proved a failure; mota, man cannot perceive. Revelation is the only technique which can compensate for the absence of the truth sensitivity of mota in a material world. Revelation authoritatively clarifies the muddle of reason\hyp{}developed metaphysics on an evolutionary sphere.
\vs p103 6:9 Science is man’s attempted study of his physical environment, the world of energy\hyp{}matter; religion is man’s experience with the cosmos of spirit values; philosophy has been developed by man’s mind effort to organize and correlate the findings of these widely separated concepts into something like a reasonable and unified attitude toward the cosmos. Philosophy, clarified by revelation, functions acceptably in the absence of mota and in the presence of the breakdown and failure of man’s reason substitute for mota --- metaphysics.
\vs p103 6:10 \pc Early man did not differentiate between the energy level and the spirit level. It was the violet race and their Andite successors who first attempted to divorce the mathematical from the volitional. Increasingly has civilized man followed in the footsteps of the earliest Greeks and the Sumerians who distinguished between the inanimate and the animate. And as civilization progresses, philosophy will have to bridge ever\hyp{}widening gulfs between the spirit concept and the energy concept. But in the time of space these divergencies are at one in the Supreme.
\vs p103 6:11 \pc Science must always be grounded in reason, although imagination and conjecture are helpful in the extension of its borders. Religion is forever dependent on faith, albeit reason is a stabilizing influence and a helpful handmaid. And always there have been, and ever will be, misleading interpretations of the phenomena of both the natural and the spiritual worlds, sciences and religions falsely so called.
\vs p103 6:12 Out of his incomplete grasp of science, his faint hold upon religion, and his abortive attempts at metaphysics, man has attempted to construct his formulations of philosophy. And modern man would indeed build a worthy and engaging philosophy of himself and his universe were it not for the breakdown of his all\hyp{}important and indispensable metaphysical connection between the worlds of matter and spirit, the failure of metaphysics to bridge the morontia gulf between the physical and the spiritual. Mortal man lacks the concept of morontia mind and material; and \bibemph{revelation} is the only technique for atoning for this deficiency in the conceptual data which man so urgently needs in order to construct a logical philosophy of the universe and to arrive at a satisfying understanding of his sure and settled place in that universe.
\vs p103 6:13 Revelation is evolutionary man’s only hope of bridging the morontia gulf. Faith and reason, unaided by mota, cannot conceive and construct a logical universe. Without the insight of mota, mortal man cannot discern goodness, love, and truth in the phenomena of the material world.
\vs p103 6:14 When the philosophy of man leans heavily toward the world of matter, it becomes rationalistic or \bibemph{naturalistic.} When philosophy inclines particularly toward the spiritual level, it becomes \bibemph{idealistic} or even mystical. When philosophy is so unfortunate as to lean upon metaphysics, it unfailingly becomes \bibemph{sceptical,} confused. In past ages, most of man’s knowledge and intellectual evaluations have fallen into one of these three distortions of perception. Philosophy dare not project its interpretations of reality in the linear fashion of logic; it must never fail to reckon with the elliptic symmetry of reality and with the essential curvature of all relation concepts.
\vs p103 6:15 The highest attainable philosophy of mortal man must be logically based on the reason of science, the faith of religion, and the truth insight afforded by revelation. By this union man can compensate somewhat for his failure to develop an adequate metaphysics and for his inability to comprehend the mota of the morontia.
\usection{7.\bibnobreakspace Science and Religion}
\vs p103 7:1 Science is sustained by reason, religion by faith. Faith, though not predicated on reason, is reasonable; though independent of logic, it is nonetheless encouraged by sound logic. Faith cannot be nourished even by an ideal philosophy; indeed, it is, with science, the very source of such a philosophy. Faith, human religious insight, can be surely instructed only by revelation, can be surely elevated only by personal mortal experience with the spiritual Adjuster presence of the God who is spirit.
\vs p103 7:2 \pc True salvation is the technique of the divine evolution of the mortal mind from matter identification through the realms of morontia liaison to the high universe status of spiritual correlation. And as material intuitive instinct precedes the appearance of reasoned knowledge in terrestrial evolution, so does the manifestation of spiritual intuitive insight presage the later appearance of morontia and spirit reason and experience in the supernal program of celestial evolution, the business of transmuting the potentials of man the temporal into the actuality and divinity of man the eternal, a Paradise finaliter.
\vs p103 7:3 But as ascending man reaches inward and Paradiseward for the God experience, he will likewise be reaching outward and spaceward for an energy understanding of the material cosmos. The progression of science is not limited to the terrestrial life of man; his universe and superuniverse ascension experience will to no small degree be the study of energy transmutation and material metamorphosis. God is spirit, but Deity is unity, and the unity of Deity not only embraces the spiritual values of the Universal Father and the Eternal Son but is also cognizant of the energy facts of the Universal Controller and the Isle of Paradise, while these two phases of universal reality are perfectly correlated in the mind relationships of the Conjoint Actor and unified on the finite level in the emerging Deity of the Supreme Being.
\vs p103 7:4 \pc The union of the scientific attitude and the religious insight by the mediation of experiential philosophy is part of man’s long Paradise\hyp{}ascension experience. The approximations of mathematics and the certainties of insight will always require the harmonizing function of mind logic on all levels of experience short of the maximum attainment of the Supreme.
\vs p103 7:5 But logic can never succeed in harmonizing the findings of science and the insights of religion unless both the scientific and the religious aspects of a personality are truth dominated, sincerely desirous of following the truth wherever it may lead regardless of the conclusions which it may reach.
\vs p103 7:6 Logic is the technique of philosophy, its method of expression. Within the domain of true science, reason is always amenable to genuine logic; within the domain of true religion, faith is always logical from the basis of an inner viewpoint, even though such faith may appear to be quite unfounded from the inlooking viewpoint of the scientific approach. From outward, looking within, the universe may appear to be material; from within, looking out, the same universe appears to be wholly spiritual. Reason grows out of material awareness, faith out of spiritual awareness, but through the mediation of a philosophy strengthened by revelation, logic may confirm both the inward and the outward view, thereby effecting the stabilization of both science and religion. Thus, through common contact with the logic of philosophy, may both science and religion become increasingly tolerant of each other, less and less sceptical.
\vs p103 7:7 What both developing science and religion need is more searching and fearless self\hyp{}criticism, a greater awareness of incompleteness in evolutionary status. The teachers of both science and religion are often altogether too self\hyp{}confident and dogmatic. Science and religion can only be self\hyp{}critical of their \bibemph{facts.} The moment departure is made from the stage of facts, reason abdicates or else rapidly degenerates into a consort of false logic.
\vs p103 7:8 \pc The truth --- an understanding of cosmic relationships, universe facts, and spiritual values --- can best be had through the ministry of the Spirit of Truth and can best be criticized by \bibemph{revelation.} But revelation originates neither a science nor a religion; its function is to co\hyp{}ordinate both science and religion with the truth of reality. Always, in the absence of revelation or in the failure to accept or grasp it, has mortal man resorted to his futile gesture of metaphysics, that being the only human substitute for the revelation of truth or for the mota of morontia personality.
\vs p103 7:9 The science of the material world enables man to control, and to some extent dominate, his physical environment. The religion of the spiritual experience is the source of the fraternity impulse which enables men to live together in the complexities of the civilization of a scientific age. Metaphysics, but more certainly revelation, affords a common meeting ground for the discoveries of both science and religion and makes possible the human attempt logically to correlate these separate but interdependent domains of thought into a well\hyp{}balanced philosophy of scientific stability and religious certainty.
\vs p103 7:10 \pc In the mortal state, nothing can be absolutely proved; both science and religion are predicated on assumptions. On the morontia level, the postulates of both science and religion are capable of partial proof by mota logic. On the spiritual level of maximum status, the need for finite proof gradually vanishes before the actual experience of and with reality; but even then there is much beyond the finite that remains unproved.
\vs p103 7:11 All divisions of human thought are predicated on certain assumptions which are accepted, though unproved, by the constitutive reality sensitivity of the mind endowment of man. Science starts out on its vaunted career of reasoning by \bibemph{assuming} the reality of three things: matter, motion, and life. Religion starts out with the assumption of the validity of three things: mind, spirit, and the universe --- the Supreme Being.
\vs p103 7:12 Science becomes the thought domain of mathematics, of the energy and material of time in space. Religion assumes to deal not only with finite and temporal spirit but also with the spirit of eternity and supremacy. Only through a long experience in mota can these two extremes of universe perception be made to yield analogous interpretations of origins, functions, relations, realities, and destinies. The maximum harmonization of the energy\hyp{}spirit divergence is in the encircuitment of the Seven Master Spirits; the first unification thereof, in the Deity of the Supreme; the finality unity thereof, in the infinity of the First Source and Centre, the I AM.
\vs p103 7:13 \pc \bibemph{Reason} is the act of recognizing the conclusions of consciousness with regard to the experience in and with the physical world of energy and matter. \bibemph{Faith} is the act of recognizing the validity of spiritual consciousness --- something which is incapable of other mortal proof. \bibemph{Logic} is the synthetic truth\hyp{}seeking progression of the unity of faith and reason and is founded on the constitutive mind endowments of mortal beings, the innate recognition of things, meanings, and values.
\vs p103 7:14 \pc There is a real proof of spiritual reality in the presence of the Thought Adjuster, but the validity of this presence is not demonstrable to the external world, only to the one who thus experiences the indwelling of God. The consciousness of the Adjuster is based on the intellectual reception of truth, the supermind perception of goodness, and the personality motivation to love.
\vs p103 7:15 Science discovers the material world, religion evaluates it, and philosophy endeavours to interpret its meanings while co\hyp{}ordinating the scientific material viewpoint with the religious spiritual concept. But history is a realm in which science and religion may never fully agree.
\usection{8.\bibnobreakspace Philosophy and Religion}
\vs p103 8:1 Although both science and philosophy may assume the probability of God by their reason and logic, only the personal religious experience of a spirit\hyp{}led man can affirm the certainty of such a supreme and personal Deity. By the technique of such an incarnation of living truth the philosophic hypothesis of the probability of God becomes a religious reality.
\vs p103 8:2 The confusion about the experience of the certainty of God arises out of the dissimilar interpretations and relations of that experience by separate individuals and by different races of men. The experiencing of God may be wholly valid, but the discourse \bibemph{about} God, being intellectual and philosophical, is divergent and oftentimes confusingly fallacious.
\vs p103 8:3 A good and noble man may be consummately in love with his wife but utterly unable to pass a satisfactory written examination on the psychology of marital love. Another man, having little or no love for his spouse, might pass such an examination most acceptably. The imperfection of the lover’s insight into the true nature of the beloved does not in the least invalidate either the reality or sincerity of his love.
\vs p103 8:4 \pc If you truly believe in God --- by faith know him and love him --- do not permit the reality of such an experience to be in any way lessened or detracted from by the doubting insinuations of science, the cavilling of logic, the postulates of philosophy, or the clever suggestions of well\hyp{}meaning souls who would create a religion without God.
\vs p103 8:5 The certainty of the God\hyp{}knowing religionist should not be disturbed by the uncertainty of the doubting materialist; rather should the uncertainty of the unbeliever be mightily challenged by the profound faith and unshakable certainty of the experiential believer.
\vs p103 8:6 \pc Philosophy, to be of the greatest service to both science and religion, should avoid the extremes of both materialism and pantheism. Only a philosophy which recognizes the reality of personality --- permanence in the presence of change --- can be of moral value to man, can serve as a liaison between the theories of material science and spiritual religion. Revelation is a compensation for the frailties of evolving philosophy.
\usection{9.\bibnobreakspace The Essence of Religion}
\vs p103 9:1 Theology deals with the intellectual content of religion, metaphysics (revelation) with the philosophic aspects. Religious experience \bibemph{is} the spiritual content of religion. Notwithstanding the mythologic vagaries and the psychologic illusions of the intellectual content of religion, the metaphysical assumptions of error and the techniques of self\hyp{}deception, the political distortions and the socio\hyp{}economic perversions of the philosophic content of religion, the spiritual experience of personal religion remains genuine and valid.
\vs p103 9:2 Religion has to do with feeling, acting, and living, not merely with thinking. Thinking is more closely related to the material life and should be in the main, but not altogether, dominated by reason and the facts of science and, in its nonmaterial reaches toward the spirit realms, by truth. No matter how illusory and erroneous one’s theology, one’s religion may be wholly genuine and everlastingly true.
\vs p103 9:3 Buddhism in its original form is one of the best religions without a God which has arisen throughout all the evolutionary history of Urantia, although, as this faith developed, it did not remain godless. Religion without faith is a contradiction; without God, a philosophic inconsistency and an intellectual absurdity.
\vs p103 9:4 The magical and mythological parentage of natural religion does not invalidate the reality and truth of the later revelational religions and the consummate saving gospel of the religion of Jesus. Jesus’ life and teachings finally divested religion of the superstitions of magic, the illusions of mythology, and the bondage of traditional dogmatism. But this early magic and mythology very effectively prepared the way for later and superior religion by assuming the existence and reality of supermaterial values and beings.
\vs p103 9:5 Although religious experience is a purely spiritual subjective phenomenon, such an experience embraces a positive and living faith attitude toward the highest realms of universe objective reality. The ideal of religious philosophy is such a faith\hyp{}trust as would lead man unqualifiedly to depend upon the absolute love of the infinite Father of the universe of universes. Such a genuine religious experience far transcends the philosophic objectification of idealistic desire; it actually takes salvation for granted and concerns itself only with learning and doing the will of the Father in Paradise. The earmarks of such a religion are: faith in a supreme Deity, hope of eternal survival, and love, especially of one’s fellows.
\vs p103 9:6 \pc When theology masters religion, religion dies; it becomes a doctrine instead of a life. The mission of theology is merely to facilitate the self\hyp{}consciousness of personal spiritual experience. Theology constitutes the religious effort to define, clarify, expound, and justify the experiential claims of religion, which, in the last analysis, can be validated only by living faith. In the higher philosophy of the universe, wisdom, like reason, becomes allied to faith. Reason, wisdom, and faith are man’s highest human attainments. Reason introduces man to the world of facts, to things; wisdom introduces him to a world of truth, to relationships; faith initiates him into a world of divinity, spiritual experience.
\vs p103 9:7 Faith most willingly carries reason along as far as reason can go and then goes on with wisdom to the full philosophic limit; and then it dares to launch out upon the limitless and never\hyp{}ending universe journey in the sole company of TRUTH.
\vs p103 9:8 \pc Science (knowledge) is founded on the inherent (adjutant spirit) assumption that reason is valid, that the universe can be comprehended. Philosophy (co\hyp{}ordinate comprehension) is founded on the inherent (spirit of wisdom) assumption that wisdom is valid, that the material universe can be co\hyp{}ordinated with the spiritual. Religion (the truth of personal spiritual experience) is founded on the inherent (Thought Adjuster) assumption that faith is valid, that God can be known and attained.
\vs p103 9:9 The full realization of the reality of mortal life consists in a progressive willingness to believe these assumptions of reason, wisdom, and faith. Such a life is one motivated by truth and dominated by love; and these are the ideals of objective cosmic reality whose existence cannot be materially demonstrated.
\vs p103 9:10 When reason once recognizes right and wrong, it exhibits wisdom; when wisdom chooses between right and wrong, truth and error, it demonstrates spirit leading. And thus are the functions of mind, soul, and spirit ever closely united and functionally interassociated. Reason deals with factual knowledge; wisdom, with philosophy and revelation; faith, with living spiritual experience. Through truth man attains beauty and by spiritual love ascends to goodness.
\vs p103 9:11 Faith leads to knowing God, not merely to a mystical feeling of the divine presence. Faith must not be overmuch influenced by its emotional consequences. True religion is an experience of believing and knowing as well as a satisfaction of feeling.
\vs p103 9:12 \pc There is a reality in religious experience that is proportional to the spiritual content, and such a reality is transcendent to reason, science, philosophy, wisdom, and all other human achievements. The convictions of such an experience are unassailable; the logic of religious living is incontrovertible; the certainty of such knowledge is superhuman; the satisfactions are superbly divine, the courage indomitable, the devotions unquestioning, the loyalties supreme, and the destinies final --- eternal, ultimate, and universal.
\vsetoff
\vs p103 9:13 [Presented by a Melchizedek of Nebadon.]
\quizlink
