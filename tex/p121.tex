\upaper{121}{The Times of Michael’s Bestowal}
\uminitoc{The Occident of the First Century after Christ}
\uminitoc{The Jewish People}
\uminitoc{Among the Gentiles}
\uminitoc{Gentile Philosophy}
\uminitoc{The Gentile Religions}
\uminitoc{The Hebrew Religion}
\uminitoc{Jews and Gentiles}
\uminitoc{Previous Written Records}
\author{Midwayer Commission}
\vs p121 0:1 Acting under the supervision of a commission of 12 members of the United Brotherhood of Urantia Midwayers, conjointly sponsored by the presiding head of our order and the Melchizedek of record, I am the secondary midwayer of onetime attachment to the Apostle Andrew, and I am authorized to place on record the narrative of the life transactions of Jesus of Nazareth as they were observed by my order of earth creatures, and as they were subsequently partially recorded by the human subject of my temporal guardianship. Knowing how his Master so scrupulously avoided leaving written records behind him, Andrew steadfastly refused to multiply copies of his written narrative. A similar attitude on the part of the other apostles of Jesus greatly delayed the writing of the Gospels.
\usection{The Occident of the First Century after Christ}
\vs p121 1:1 Jesus did not come to this world during an age of spiritual decadence; at the time of his birth Urantia was experiencing such a revival of spiritual thinking and religious living as it had not known in all its previous post\hyp{}Adamic history nor has experienced in any era since. When Michael incarnated on Urantia, the world presented the most favourable condition for the Creator Son’s bestowal that had ever previously prevailed or has since obtained. In the centuries just prior to these times Greek culture and the Greek language had spread over Occident and near Orient, and the Jews, being a Levantine race, in nature part Occidental and part Oriental, were eminently fitted to utilize such cultural and linguistic settings for the effective spread of a new religion to both East and West. These most favourable circumstances were further enhanced by the tolerant political rule of the Mediterranean world by the Romans.
\vs p121 1:2 This entire combination of world influences is well illustrated by the activities of Paul, who, being in religious culture a Hebrew of the Hebrews, proclaimed the gospel of a Jewish Messiah in the Greek tongue, while he himself was a Roman citizen.
\vs p121 1:3 Nothing like the civilization of the times of Jesus has been seen in the Occident before or since those days. European civilization was unified and co\hyp{}ordinated under an extraordinary threefold influence:
\vs p121 1:4 \ublistelem{1.}\bibnobreakspace The Roman political and social systems.
\vs p121 1:5 \ublistelem{2.}\bibnobreakspace The Grecian language and culture --- and philosophy to a certain extent.
\vs p121 1:6 \ublistelem{3.}\bibnobreakspace The rapidly spreading influence of Jewish religious and moral teachings.
\vs p121 1:7 \pc When Jesus was born, the entire Mediterranean world was a unified empire. Good roads, for the first time in the world’s history, interconnected many major centres. The seas were cleared of pirates, and a great era of trade and travel was rapidly advancing. Europe did not again enjoy another such period of travel and trade until the XIX century after Christ.
\vs p121 1:8 Notwithstanding the internal peace and superficial prosperity of the Gr\ae co\hyp{}Roman world, a majority of the inhabitants of the empire languished in squalor and poverty. The small upper class was rich; a miserable and impoverished lower class embraced the rank and file of humanity. There was no happy and prosperous middle class in those days; it had just begun to make its appearance in Roman society.
\vs p121 1:9 The first struggles between the expanding Roman and Parthian states had been concluded in the then recent past, leaving Syria in the hands of the Romans. In the times of Jesus, Palestine and Syria were enjoying a period of prosperity, relative peace, and extensive commercial intercourse with the lands to both the East and the West.
\usection{The Jewish People}
\vs p121 2:1 The Jews were a part of the older Semitic race, which also included the Babylonians, the Phoenicians, and the more recent enemies of Rome, the Carthaginians. During the fore part of the first century after Christ, the Jews were the most influential group of the Semitic peoples, and they happened to occupy a peculiarly strategic geographic position in the world as it was at that time ruled and organized for trade.
\vs p121 2:2 Many of the great highways joining the nations of antiquity passed through Palestine, which thus became the meeting place, or crossroads, of three continents. The travel, trade, and armies of Babylonia, Assyria, Egypt, Syria, Greece, Parthia, and Rome successively swept over Palestine. From time immemorial, many caravan routes from the Orient passed through some part of this region to the few good seaports of the eastern end of the Mediterranean, whence ships carried their cargoes to all the maritime Occident. And more than half of this caravan traffic passed through or near the little town of Nazareth in Galilee.
\vs p121 2:3 Although Palestine was the home of Jewish religious culture and the birthplace of Christianity, the Jews were abroad in the world, dwelling in many nations and trading in every province of the Roman and Parthian states.
\vs p121 2:4 Greece provided a language and a culture, Rome built the roads and unified an empire, but the dispersion of the Jews, with their more than 200 synagogues and well\hyp{}organized religious communities scattered hither and yon throughout the Roman world, provided the cultural centres in which the new gospel of the kingdom of heaven found initial reception, and from which it subsequently spread to the uttermost parts of the world.
\vs p121 2:5 Each Jewish synagogue tolerated a fringe of gentile believers, “devout” or “God\hyp{}fearing” men, and it was among this fringe of proselytes that Paul made the bulk of his early converts to Christianity. Even the temple at Jerusalem possessed its ornate court of the gentiles. There was very close connection between the culture, commerce, and worship of Jerusalem and Antioch. In Antioch Paul’s disciples were first called “Christians.”
\vs p121 2:6 The centralization of the Jewish temple worship at Jerusalem constituted alike the secret of the survival of their monotheism and the promise of the nurture and sending forth to the world of a new and enlarged concept of that one God of all nations and Father of all mortals. The temple service at Jerusalem represented the survival of a religious cultural concept in the face of the downfall of a succession of gentile national overlords and racial persecutors.
\vs p121 2:7 \pc The Jewish people of this time, although under Roman suzerainty, enjoyed a considerable degree of self\hyp{}government and, remembering the then only recent heroic exploits of deliverance executed by Judas Maccabee and his immediate successors, were vibrant with the expectation of the immediate appearance of a still greater deliverer, the long\hyp{}expected Messiah.
\vs p121 2:8 The secret of the survival of Palestine, the kingdom of the Jews, as a semi\hyp{}independent state was wrapped up in the foreign policy of the Roman government, which desired to maintain control of the Palestinian highway of travel between Syria and Egypt as well as the western terminals of the caravan routes between the Orient and the Occident. Rome did not wish any power to arise in the Levant which might curb her future expansion in these regions. The policy of intrigue which had for its object the pitting of Seleucid Syria and Ptolemaic Egypt against each other necessitated fostering Palestine as a separate and independent state. Roman policy, the degeneration of Egypt, and the progressive weakening of the Seleucids before the rising power of Parthia, explain why it was that for several generations a small and unpowerful group of Jews was able to maintain its independence against both Seleucidae to the north and Ptolemies to the south. This fortuitous liberty and independence of the political rule of surrounding and more powerful peoples the Jews attributed to the fact that they were the “chosen people,” to the direct interposition of Yahweh. Such an attitude of racial superiority made it all the harder for them to endure Roman suzerainty when it finally fell upon their land. But even in that sad hour the Jews refused to learn that their world mission was spiritual, not political.
\vs p121 2:9 \pc The Jews were unusually apprehensive and suspicious during the times of Jesus because they were then ruled by an outsider, Herod the Idumean, who had seized the overlordship of Judea by cleverly ingratiating himself with the Roman rulers. And though Herod professed loyalty to the Hebrew ceremonial observances, he proceeded to build temples for many strange gods.
\vs p121 2:10 The friendly relations of Herod with the Roman rulers made the world safe for Jewish travel and thus opened the way for increased Jewish penetration even of distant portions of the Roman Empire and of foreign treaty nations with the new gospel of the kingdom of heaven. Herod’s reign also contributed much toward the further blending of Hebrew and Hellenistic philosophies.
\vs p121 2:11 Herod built the harbour of Caesarea, which further aided in making Palestine the crossroads of the civilized world. He died in 4\,B.C., and his son Herod Antipas governed Galilee and Perea during Jesus’ youth and ministry to A.D.\,39. Antipas, like his father, was a great builder. He rebuilt many of the cities of Galilee, including the important trade centre of Sepphoris.
\vs p121 2:12 The Galileans were not regarded with full favour by the Jerusalem religious leaders and rabbinical teachers. Galilee was more gentile than Jewish when Jesus was born.
\usection{Among the Gentiles}
\vs p121 3:1 Although the social and economic condition of the Roman state was not of the highest order, the widespread domestic peace and prosperity was propitious for the bestowal of Michael. In the first century after Christ the society of the Mediterranean world consisted of five well\hyp{}defined strata:
\vs p121 3:2 \ublistelem{1.}\bibnobreakspace \bibemph{The aristocracy.} The upper classes with money and official power, the privileged and ruling groups.
\vs p121 3:3 \ublistelem{2.}\bibnobreakspace \bibemph{The business groups.} The merchant princes and the bankers, the traders --- the big importers and exporters --- the international merchants.
\vs p121 3:4 \ublistelem{3.}\bibnobreakspace \bibemph{The small middle class.} Although this group was indeed small, it was very influential and provided the moral backbone of the early Christian church, which encouraged these groups to continue in their various crafts and trades. Among the Jews many of the Pharisees belonged to this class of tradesmen.
\vs p121 3:5 \ublistelem{4.}\bibnobreakspace \bibemph{The free proletariat.} This group had little or no social standing. Though proud of their freedom, they were placed at great disadvantage because they were forced to compete with slave labour. The upper classes regarded them disdainfully, allowing that they were useless except for “breeding purposes.”
\vs p121 3:6 \ublistelem{5.}\bibnobreakspace \bibemph{The slaves.} Half the population of the Roman state were slaves; many were superior individuals and quickly made their way up among the free proletariat and even among the tradesmen. The majority were either mediocre or very inferior.
\vs p121 3:7 Slavery, even of superior peoples, was a feature of Roman military conquest. The power of the master over his slave was unqualified. The early Christian church was largely composed of the lower classes and these slaves.
\vs p121 3:8 Superior slaves often received wages and by saving their earnings were able to purchase their freedom. Many such emancipated slaves rose to high positions in state, church, and the business world. And it was just such possibilities that made the early Christian church so tolerant of this modified form of slavery.
\vs p121 3:9 \pc There was no widespread social problem in the Roman Empire in the first century after Christ. The major portion of the populace regarded themselves as belonging in that group into which they chanced to be born. There was always the open door through which talented and able individuals could ascend from the lower to the higher strata of Roman society, but the people were generally content with their social rank. They were not class conscious, neither did they look upon these class distinctions as being unjust or wrong. Christianity was in no sense an economic movement having for its purpose the amelioration of the miseries of the depressed classes.
\vs p121 3:10 Although woman enjoyed more freedom throughout the Roman Empire than in her restricted position in Palestine, the family devotion and natural affection of the Jews far transcended that of the gentile world.
\usection{Gentile Philosophy}
\vs p121 4:1 The gentiles were, from a moral standpoint, somewhat inferior to the Jews, but there was present in the hearts of the nobler gentiles abundant soil of natural goodness and potential human affection in which it was possible for the seed of Christianity to sprout and bring forth an abundant harvest of moral character and spiritual achievement. The gentile world was then dominated by four great philosophies, all more or less derived from the earlier Platonism of the Greeks. These schools of philosophy were:
\vs p121 4:2 \ublistelem{1.}\bibnobreakspace \bibemph{The Epicurean.} This school of thought was dedicated to the pursuit of happiness. The better Epicureans were not given to sensual excesses. At least this doctrine helped to deliver the Romans from a more deadly form of fatalism; it taught that men could do something to improve their terrestrial status. It did effectually combat ignorant superstition.
\vs p121 4:3 \ublistelem{2.}\bibnobreakspace \bibemph{The Stoic.} Stoicism was the superior philosophy of the better classes. The Stoics believed that a controlling Reason\hyp{}Fate dominated all nature. They taught that the soul of man was divine; that it was imprisoned in the evil body of physical nature. Man’s soul achieved liberty by living in harmony with nature, with God; thus virtue came to be its own reward. Stoicism ascended to a sublime morality, ideals never since transcended by any purely human system of philosophy. While the Stoics professed to be the “offspring of God,” they failed to know him and therefore failed to find him. Stoicism remained a philosophy; it never became a religion. Its followers sought to attune their minds to the harmony of the Universal Mind, but they failed to envisage themselves as the children of a loving Father. Paul leaned heavily toward Stoicism when he wrote, “I have learned in whatsoever state I am, therewith to be content.”
\vs p121 4:4 \ublistelem{3.}\bibnobreakspace \bibemph{The Cynic.} Although the Cynics traced their philosophy to Diogenes of Athens, they derived much of their doctrine from the remnants of the teachings of Machiventa Melchizedek. Cynicism had formerly been more of a religion than a philosophy. At least the Cynics made their religio\hyp{}philosophy democratic. In the fields and in the market places they continually preached their doctrine that “man could save himself if he would.” They preached simplicity and virtue and urged men to meet death fearlessly. These wandering Cynic preachers did much to prepare the spiritually hungry populace for the later Christian missionaries. Their plan of popular preaching was much after the pattern, and in accordance with the style, of Paul’s Epistles.
\vs p121 4:5 \ublistelem{4.}\bibnobreakspace \bibemph{The Sceptic.} Scepticism asserted that knowledge was fallacious, and that conviction and assurance were impossible. It was a purely negative attitude and never became widespread.
\vs p121 4:6 \pc These philosophies were semireligious; they were often invigorating, ethical, and ennobling but were usually above the common people. With the possible exception of Cynicism, they were philosophies for the strong and the wise, not religions of salvation for even the poor and the weak.
\usection{The Gentile Religions}
\vs p121 5:1 Throughout preceding ages religion had chiefly been an affair of the tribe or nation; it had not often been a matter of concern to the individual. Gods were tribal or national, not personal. Such religious systems afforded little satisfaction for the individual spiritual longings of the average person.
\vs p121 5:2 In the times of Jesus the religions of the Occident included:
\vs p121 5:3 \ublistelem{1.}\bibnobreakspace \bibemph{The pagan cults.} These were a combination of Hellenic and Latin mythology, patriotism, and tradition.
\vs p121 5:4 \ublistelem{2.}\bibnobreakspace \bibemph{Emperor worship.} This deification of man as the symbol of the state was very seriously resented by the Jews and the early Christians and led directly to the bitter persecutions of both churches by the Roman government.
\vs p121 5:5 \ublistelem{3.}\bibnobreakspace \bibemph{Astrology.} This pseudo science of Babylon developed into a religion throughout the Gr\ae co\hyp{}Roman Empire. Even in the XX century man has not been fully delivered from this superstitious belief.
\vs p121 5:6 \ublistelem{4.}\bibnobreakspace \bibemph{The mystery religions.} Upon such a spiritually hungry world a flood of mystery cults had broken, new and strange religions from the Levant, which had enamoured the common people and had promised them \bibemph{individual} salvation. These religions rapidly became the accepted belief of the lower classes of the Gr\ae co\hyp{}Roman world. And they did much to prepare the way for the rapid spread of the vastly superior Christian teachings, which presented a majestic concept of Deity, associated with an intriguing theology for the intelligent and a profound proffer of salvation for all, including the ignorant but spiritually hungry average man of those days.
\vs p121 5:7 \pc The mystery religions spelled the end of national beliefs and resulted in the birth of the numerous personal cults. The mysteries were many but were all characterized by:
\vs p121 5:8 \ublistelem{1.}\bibnobreakspace Some mythical legend, a mystery --- whence their name. As a rule this mystery pertained to the story of some god’s life and death and return to life, as illustrated by the teachings of Mithraism, which, for a time, were contemporary with, and a competitor of, Paul’s rising cult of Christianity.
\vs p121 5:9 \ublistelem{2.}\bibnobreakspace The mysteries were nonnational and interracial. They were personal and fraternal, giving rise to religious brotherhoods and numerous sectarian societies.
\vs p121 5:10 \ublistelem{3.}\bibnobreakspace They were, in their services, characterized by elaborate ceremonies of initiation and impressive sacraments of worship. Their secret rites and rituals were sometimes gruesome and revolting.
\vs p121 5:11 \ublistelem{4.}\bibnobreakspace But no matter what the nature of their ceremonies or the degree of their excesses, these mysteries invariably promised their devotees \bibemph{salvation,} “deliverance from evil, survival after death, and enduring life in blissful realms beyond this world of sorrow and slavery.”
\vs p121 5:12 \pc But do not make the mistake of confusing the teachings of Jesus with the mysteries. The popularity of the mysteries reveals man’s quest for survival, thus portraying a real hunger and thirst for personal religion and individual righteousness. Although the mysteries failed adequately to satisfy this longing, they did prepare the way for the subsequent appearance of Jesus, who truly brought to this world the bread of life and the water thereof.
\vs p121 5:13 Paul, in an effort to utilize the widespread adherence to the better types of the mystery religions, made certain adaptations of the teachings of Jesus so as to render them more acceptable to a larger number of prospective converts. But even Paul’s compromise of Jesus’ teachings (Christianity) was superior to the best in the mysteries in that:
\vs p121 5:14 \ublistelem{1.}\bibnobreakspace Paul taught a moral redemption, an ethical salvation. Christianity pointed to a new life and proclaimed a new ideal. Paul forsook magic rites and ceremonial enchantments.
\vs p121 5:15 \ublistelem{2.}\bibnobreakspace Christianity presented a religion which grappled with final solutions of the human problem, for it not only offered salvation from sorrow and even from death, but it also promised deliverance from sin followed by the endowment of a righteous character of eternal survival qualities.
\vs p121 5:16 \ublistelem{3.}\bibnobreakspace The mysteries were built upon myths. Christianity, as Paul preached it, was founded upon a historic fact: the bestowal of Michael, the Son of God, upon mankind.
\vs p121 5:17 \pc Morality among the gentiles was not necessarily related to either philosophy or religion. Outside of Palestine it not always occurred to people that a priest of religion was supposed to lead a moral life. Jewish religion and subsequently the teachings of Jesus and later the evolving Christianity of Paul were the first European religions to lay one hand upon morals and the other upon ethics, insisting that religionists pay some attention to both.
\vs p121 5:18 Into such a generation of men, dominated by such incomplete systems of philosophy and perplexed by such complex cults of religion, Jesus was born in Palestine. And to this same generation he subsequently gave his gospel of personal religion --- sonship with God.
\usection{The Hebrew Religion}
\vs p121 6:1 By the close of the first century before Christ the religious thought of Jerusalem had been tremendously influenced and somewhat modified by Greek cultural teachings and even by Greek philosophy. In the long contest between the views of the Eastern and Western schools of Hebrew thought, Jerusalem and the rest of the Occident and the Levant in general adopted the Western Jewish or modified Hellenistic viewpoint.
\vs p121 6:2 In the days of Jesus three languages prevailed in Palestine: The common people spoke some dialect of Aramaic; the priests and rabbis spoke Hebrew; the educated classes and the better strata of Jews in general spoke Greek. The early translation of the Hebrew scriptures into Greek at Alexandria was responsible in no small measure for the subsequent predominance of the Greek wing of Jewish culture and theology. And the writings of the Christian teachers were soon to appear in the same language. The renaissance of Judaism dates from the Greek translation of the Hebrew scriptures. This was a vital influence which later determined the drift of Paul’s Christian cult toward the West instead of toward the East.
\vs p121 6:3 Though the Hellenized Jewish beliefs were very little influenced by the teachings of the Epicureans, they were very materially affected by the philosophy of Plato and the self\hyp{}abnegation doctrines of the Stoics. The great inroad of Stoicism is exemplified by the Fourth Book of the Maccabees; the penetration of both Platonic philosophy and Stoic doctrines is exhibited in the Wisdom of Solomon. The Hellenized Jews brought to the Hebrew scriptures such an allegorical interpretation that they found no difficulty in conforming Hebrew theology with their revered Aristotelian philosophy. But this all led to disastrous confusion until these problems were taken in hand by Philo of Alexandria, who proceeded to harmonize and systemize Greek philosophy and Hebrew theology into a compact and fairly consistent system of religious belief and practice. And it was this later teaching of combined Greek philosophy and Hebrew theology that prevailed in Palestine when Jesus lived and taught, and which Paul utilized as the foundation on which to build his more advanced and enlightening cult of Christianity.
\vs p121 6:4 Philo was a great teacher; not since Moses had there lived a man who exerted such a profound influence on the ethical and religious thought of the Occidental world. In the matter of the combination of the better elements in contemporaneous systems of ethical and religious teachings, there have been seven outstanding human teachers: Sethard, Moses, Zoroaster, Lao\hyp{}tse, Buddha, Philo, and Paul.
\vs p121 6:5 Many, but not all, of Philo’s inconsistencies resulting from an effort to combine Greek mystical philosophy and Roman Stoic doctrines with the legalistic theology of the Hebrews, Paul recognized and wisely eliminated from his pre\hyp{}Christian basic theology. Philo led the way for Paul more fully to restore the concept of the Paradise Trinity, which had long been dormant in Jewish theology. In only one matter did Paul fail to keep pace with Philo or to transcend the teachings of this wealthy and educated Jew of Alexandria, and that was the doctrine of the atonement; Philo taught deliverance from the doctrine of forgiveness only by the shedding of blood. He also possibly glimpsed the reality and presence of the Thought Adjusters more clearly than did Paul. But Paul’s theory of original sin, the doctrines of hereditary guilt and innate evil and redemption therefrom, was partially Mithraic in origin, having little in common with Hebrew theology, Philo’s philosophy, or Jesus’ teachings. Some phases of Paul’s teachings regarding original sin and the atonement were original with himself.
\vs p121 6:6 The Gospel of John, the last of the narratives of Jesus’ earth life, was addressed to the Western peoples and presents its story much in the light of the viewpoint of the later Alexandrian Christians, who were also disciples of the teachings of Philo.
\vs p121 6:7 \pc At about the time of Christ a strange reversion of feeling toward the Jews occurred in Alexandria, and from this former Jewish stronghold there went forth a virulent wave of persecution, extending even to Rome, from which many thousands were banished. But such a campaign of misrepresentation was short\hyp{}lived; very soon the imperial government fully restored the curtailed liberties of the Jews throughout the empire.
\vs p121 6:8 Throughout the whole wide world, no matter where the Jews found themselves dispersed by commerce or oppression, all with one accord kept their hearts centred on the holy temple at Jerusalem. Jewish theology did survive as it was interpreted and practised at Jerusalem, notwithstanding that it was several times saved from oblivion by the timely intervention of certain Babylonian teachers.
\vs p121 6:9 As many as 2,500,000 of these dispersed Jews used to come to Jerusalem for the celebration of their national religious festivals. And no matter what the theologic or philosophic differences of the Eastern (Babylonian) and the Western (Hellenic) Jews, they were all agreed on Jerusalem as the centre of their worship and in ever looking forward to the coming of the Messiah.
\usection{Jews and Gentiles}
\vs p121 7:1 By the times of Jesus the Jews had arrived at a settled concept of their origin, history, and destiny. They had built up a rigid wall of separation between themselves and the gentile world; they looked upon all gentile ways with utter contempt. They worshipped the letter of the law and indulged a form of self\hyp{}righteousness based upon the false pride of descent. They had formed preconceived notions regarding the promised Messiah, and most of these expectations envisaged a Messiah who would come as a part of their national and racial history. To the Hebrews of those days Jewish theology was irrevocably settled, forever fixed.
\vs p121 7:2 The teachings and practices of Jesus regarding tolerance and kindness ran counter to the long\hyp{}standing attitude of the Jews toward other peoples whom they considered heathen. For generations the Jews had nourished an attitude toward the outside world which made it impossible for them to accept the Master’s teachings about the spiritual brotherhood of man. They were unwilling to share Yahweh on equal terms with the gentiles and were likewise unwilling to accept as the Son of God one who taught such new and strange doctrines.
\vs p121 7:3 The scribes, the Pharisees, and the priesthood held the Jews in a terrible bondage of ritualism and legalism, a bondage far more real than that of the Roman political rule. The Jews of Jesus’ time were not only held in subjugation to the \bibemph{law} but were equally bound by the slavish demands of the \bibemph{traditions,} which involved and invaded every domain of personal and social life. These minute regulations of conduct pursued and dominated every loyal Jew, and it is not strange that they promptly rejected one of their number who presumed to ignore their sacred traditions, and who dared to flout their long\hyp{}honoured regulations of social conduct. They could hardly regard with favour the teachings of one who did not hesitate to clash with dogmas which they regarded as having been ordained by Father Abraham himself. Moses had given them their law and they would not compromise.
\vs p121 7:4 By the time of the first century after Christ the spoken interpretation of the law by the recognized teachers, the scribes, had become a higher authority than the written law itself. And all this made it easier for certain religious leaders of the Jews to array the people against the acceptance of a new gospel.
\vs p121 7:5 These circumstances rendered it impossible for the Jews to fulfil their divine destiny as messengers of the new gospel of religious freedom and spiritual liberty. They could not break the fetters of tradition. Jeremiah had told of the “law to be written in men’s hearts,” Ezekiel had spoken of a “new spirit to live in man’s soul,” and the Psalmist had prayed that God would “create a clean heart within and renew a right spirit.” But when the Jewish religion of good works and slavery to law fell victim to the stagnation of traditionalistic inertia, the motion of religious evolution passed westward to the European peoples.
\vs p121 7:6 And so a different people were called upon to carry an advancing theology to the world, a system of teaching embodying the philosophy of the Greeks, the law of the Romans, the morality of the Hebrews, and the gospel of personality sanctity and spiritual liberty formulated by Paul and based on the teachings of Jesus.
\vs p121 7:7 \pc Paul’s cult of Christianity exhibited its morality as a Jewish birthmark. The Jews viewed history as the providence of God --- Yahweh at work. The Greeks brought to the new teaching clearer concepts of the eternal life. Paul’s doctrines were influenced in theology and philosophy not only by Jesus’ teachings but also by Plato and Philo. In ethics he was inspired not only by Christ but also by the Stoics.
\vs p121 7:8 The gospel of Jesus, as it was embodied in Paul’s cult of Antioch Christianity, became blended with the following teachings:
\vs p121 7:9 \ublistelem{1.}\bibnobreakspace The philosophic reasoning of the Greek proselytes to Judaism, including some of their concepts of the eternal life.
\vs p121 7:10 \ublistelem{2.}\bibnobreakspace The appealing teachings of the prevailing mystery cults, especially the Mithraic doctrines of redemption, atonement, and salvation by the sacrifice made by some god.
\vs p121 7:11 \ublistelem{3.}\bibnobreakspace The sturdy morality of the established Jewish religion.
\vs p121 7:12 \pc The Mediterranean Roman Empire, the Parthian kingdom, and the adjacent peoples of Jesus’ time all held crude and primitive ideas regarding the geography of the world, astronomy, health, and disease; and naturally they were amazed by the new and startling pronouncements of the carpenter of Nazareth. The ideas of spirit possession, good and bad, applied not merely to human beings, but every rock and tree was viewed by many as being spirit possessed. This was an enchanted age, and everybody believed in miracles as commonplace occurrences.
\usection{Previous Written Records}
\vs p121 8:1 As far as possible, consistent with our mandate, we have endeavoured to utilize and to some extent co\hyp{}ordinate the existing records having to do with the life of Jesus on Urantia. Although we have enjoyed access to the lost record of the Apostle Andrew and have benefited from the collaboration of a vast host of celestial beings who were on earth during the times of Michael’s bestowal (notably his now Personalized Adjuster), it has been our purpose also to make use of the so\hyp{}called Gospels of Matthew, Mark, Luke, and John.
\vs p121 8:2 These New Testament records had their origin in the following circumstances:
\vs p121 8:3 \ublistelem{1.}\bibnobreakspace \bibemph{The Gospel by Mark.} John Mark wrote the earliest (excepting the notes of Andrew), briefest, and most simple record of Jesus’ life. He presented the Master as a minister, as man among men. Although Mark was a lad lingering about many of the scenes which he depicts, his record is in reality the Gospel according to Simon Peter. He was early associated with Peter; later with Paul. Mark wrote this record at the instigation of Peter and on the earnest petition of the church at Rome. Knowing how consistently the Master refused to write out his teachings when on earth and in the flesh, Mark, like the apostles and other leading disciples, was hesitant to put them in writing. But Peter felt the church at Rome required the assistance of such a written narrative, and Mark consented to undertake its preparation. He made many notes before Peter died in A.D.\,67, and in accordance with the outline approved by Peter and for the church at Rome, he began his writing soon after Peter’s death. The Gospel was completed near the end of A.D.\,68. Mark wrote entirely from his own memory and Peter’s memory. The record has since been considerably changed, numerous passages having been taken out and some later matter added at the end to replace the latter \bibfrac{1}{5}\ts{th} of the original Gospel, which was lost from the first manuscript before it was ever copied. This record by Mark, in conjunction with Andrew’s and Matthew’s notes, was the written basis of all subsequent Gospel narratives which sought to portray the life and teachings of Jesus.
\vs p121 8:4 \ublistelem{2.}\bibnobreakspace \bibemph{The Gospel of Matthew.} The so\hyp{}called Gospel according to Matthew is the record of the Master’s life which was written for the edification of Jewish Christians. The author of this record constantly seeks to show in Jesus’ life that much which he did was that “it might be fulfilled which was spoken by the prophet.” Matthew’s Gospel portrays Jesus as a son of David, picturing him as showing great respect for the law and the prophets.
\vs p121 8:5 The Apostle Matthew did not write this Gospel. It was written by Isador, one of his disciples, who had as a help in his work not only Matthew’s personal remembrance of these events but also a certain record which the latter had made of the sayings of Jesus directly after the crucifixion. This record by Matthew was written in Aramaic; Isador wrote in Greek. There was no intent to deceive in accrediting the production to Matthew. It was the custom in those days for pupils thus to honour their teachers.
\vs p121 8:6 Matthew’s original record was edited and added to in A.D.\,40 just before he left Jerusalem to engage in evangelistic preaching. It was a private record, the last copy having been destroyed in the burning of a Syrian monastery in A.D.\,416\fnst{\textbf{burning of a Syrian monastery in A.D.\,416}, This event is likely to be associated with the life of St~Eustochia, as narrated by Filaret of Chernigov (Gumilevsky) in his ``Lives of the Saints of the Eastern Church''. He describes the burning of the monastery by the Pelagians in A.D.\,416, two years prior to the horrible earthquake in Palestine which was likened to ``the end of the world''.}.
\vs p121 8:7 Isador escaped from Jerusalem in A.D.\,70 after the investment of the city by the armies of Titus, taking with him to Pella a copy of Matthew’s notes. In the year 71, while living at Pella, Isador wrote the Gospel according to Matthew. He also had with him the first \bibfrac{4}{5}\ts{ths} of Mark’s narrative.
\vs p121 8:8 \ublistelem{3.}\bibnobreakspace \bibemph{The Gospel by Luke.} Luke, the physician of Antioch in Pisidia, was a gentile convert of Paul, and he wrote quite a different story of the Master’s life. He began to follow Paul and learn of the life and teachings of Jesus in A.D.\,47. Luke preserves much of the “grace of the Lord Jesus Christ” in his record as he gathered up these facts from Paul and others. Luke presents the Master as “the friend of publicans and sinners.” He did not formulate his many notes into the Gospel until after Paul’s death. Luke wrote in the year 82 in Achaia. He planned three books dealing with the history of Christ and Christianity but died in A.D.\,90 just before he finished the second of these works, the “Acts of the Apostles.”
\vs p121 8:9 As material for the compilation of his Gospel, Luke first depended upon the story of Jesus’ life as Paul had related it to him. Luke’s Gospel is, therefore, in some ways the Gospel according to Paul. But Luke had other sources of information. He not only interviewed scores of eyewitnesses to the numerous episodes of Jesus’ life which he records, but he also had with him a copy of Mark’s Gospel, that is, the first \bibfrac{4}{5}\ts{ths}, Isador’s narrative, and a brief record made in the year A.D.\,78 at Antioch by a believer named Cedes. Luke also had a mutilated and much\hyp{}edited copy of some notes purported to have been made by the Apostle Andrew.
\vs p121 8:10 \ublistelem{4.}\bibnobreakspace \bibemph{The Gospel of John.} The Gospel according to John relates much of Jesus’ work in Judea and around Jerusalem which is not contained in the other records. This is the so\hyp{}called Gospel according to John the son of Zebedee, and though John did not write it, he did inspire it. Since its first writing it has several times been edited to make it appear to have been written by John himself. When this record was made, John had the other Gospels, and he saw that much had been omitted; accordingly, in the year A.D.\,101 he encouraged his associate, Nathan, a Greek Jew from Caesarea, to begin the writing. John supplied his material from memory and by reference to the three records already in existence. He had no written records of his own. The Epistle known as “First John” was written by John himself as a covering letter for the work which Nathan executed under his direction.
\vs p121 8:11 \pc All these writers presented honest pictures of Jesus as they saw, remembered, or had learned of him, and as their concepts of these distant events were affected by their subsequent espousal of Paul’s theology of Christianity. And these records, imperfect as they are, have been sufficient to change the course of the history of Urantia for almost 2,000 years.
\vsetoff
\vs p121 8:12 [\bibemph{Acknowledgement:} In carrying out my commission to restate the teachings and retell the doings of Jesus of Nazareth, I have drawn freely upon all sources of record and planetary information. My ruling motive has been to prepare a record which will not only be enlightening to the generation of men now living, but which may also be helpful to all future generations. From the vast store of information made available to me, I have chosen that which is best suited to the accomplishment of this purpose. As far as possible I have derived my information from purely human sources. Only when such sources failed, have I resorted to those records which are superhuman. When ideas and concepts of Jesus’ life and teachings have been acceptably expressed by a human mind, I invariably gave preference to such apparently human thought patterns. Although I have sought to adjust the verbal expression the better to conform to our concept of the real meaning and the true import of the Master’s life and teachings, as far as possible, I have adhered to the actual human concept and thought pattern in all my narratives. I well know that those concepts which have had origin in the human mind will prove more acceptable and helpful to all other human minds. When unable to find the necessary concepts in the human records or in human expressions, I have next resorted to the memory resources of my own order of earth creatures, the midwayers. And when that secondary source of information proved inadequate, I have unhesitatingly resorted to the superplanetary sources of information.
\vs p121 8:13 The memoranda which I have collected, and from which I have prepared this narrative of the life and teachings of Jesus --- aside from the memory of the record of the Apostle Andrew --- embrace thought gems and superior concepts of Jesus’ teachings assembled from more than 2,000 human beings who have lived on earth from the days of Jesus down to the time of the inditing of these revelations, more correctly restatements. The revelatory permission has been utilized only when the human record and human concepts failed to supply an adequate thought pattern. My revelatory commission forbade me to resort to extrahuman sources of either information or expression until such a time as I could testify that I had failed in my efforts to find the required conceptual expression in purely human sources.
\vs p121 8:14 While I, with the collaboration of my 11 associate fellow midwayers and under the supervision of the Melchizedek of record, have portrayed this narrative in accordance with my concept of its effective arrangement and in response to my choice of immediate expression, nevertheless, the majority of the ideas and even some of the effective expressions which I have thus utilized had their origin in the minds of the men of many races who have lived on earth during the intervening generations, right on down to those who are still alive at the time of this undertaking. In many ways I have served more as a collector and editor than as an original narrator. I have unhesitatingly appropriated those ideas and concepts, preferably human, which would enable me to create the most effective portraiture of Jesus’ life, and which would qualify me to restate his matchless teachings in the most strikingly helpful and universally uplifting phraseology. In behalf of the Brotherhood of the United Midwayers of Urantia, I most gratefully acknowledge our indebtedness to all sources of record and concept which have been hereinafter utilized in the further elaboration of our restatement of Jesus’ life on earth.]
\quizlink
