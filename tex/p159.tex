\upaper{159}{The Decapolis Tour}
\uminitoc{The Sermon on Forgiveness}
\uminitoc{The Strange Preacher}
\uminitoc{Instruction for Teachers and Believers}
\uminitoc{The Talk with Nathaniel}
\uminitoc{The Positive Nature of Jesus’ Religion}
\uminitoc{The Return to Magadan}
\author{Midwayer Commission}
\vs p159 0:1 When Jesus and the twelve arrived at Magadan Park, they found awaiting them a group of almost 100 evangelists and disciples, including the women’s corps, and they were ready immediately to begin the teaching and preaching tour of the cities of the Decapolis.
\vs p159 0:2 On this Thursday morning, August 18, the Master called his followers together and directed that each of the apostles should associate himself with one of the twelve evangelists, and that with others of the evangelists they should go out in twelve groups to labour in the cities and villages of the Decapolis. The women’s corps and others of the disciples he directed to remain with him. Jesus allotted four weeks to this tour, instructing his followers to return to Magadan not later than Friday, September 16. He promised to visit them often during this time. In the course of this month these twelve groups laboured in Gerasa, Gamala, Hippos, Zaphon, Gadara, Abila, Edrei, Philadelphia, Heshbon, Dium, Scythopolis, and many other cities. Throughout this tour no miracles of healing or other extraordinary events occurred.
\usection{The Sermon on Forgiveness}
\vs p159 1:1 One evening at Hippos, in answer to a disciple’s question, Jesus taught the lesson on forgiveness. Said the Master:
\vs p159 1:2 \pc \textcolour{ubdarkred}{“If a kindhearted man has 100 sheep and 1 of them goes astray, does he not immediately leave the 99 and go out in search of the 1 that has gone astray? And if he is a good shepherd, will he not keep up his quest for the lost sheep until he finds it? And then, when the shepherd has found his lost sheep, he lays it over his shoulder and, going home rejoicing, calls to his friends and neighbours, ‘Rejoice with me, for I have found my sheep that was lost.’ I declare that there is more joy in heaven over one sinner who repents than over 99 righteous persons who need no repentance. Even so, it is not the will of my Father in heaven that one of these little ones should go astray, much less that they should perish. In your religion God may receive repentant sinners; in the gospel of the kingdom the Father goes forth to find them even before they have seriously thought of repentance.}
\vs p159 1:3 \textcolour{ubdarkred}{“The Father in heaven loves his children, and therefore should you learn to love one another; the Father in heaven forgives you your sins; therefore should you learn to forgive one another. If your brother sins against you, go to him and with tact and patience show him his fault. And do all this between you and him alone. If he will listen to you, then have you won your brother. But if your brother will not hear you, if he persists in the error of his way, go again to him, taking with you one or two mutual friends that you may thus have two or even three witnesses to confirm your testimony and establish the fact that you have dealt justly and mercifully with your offending brother. Now if he refuses to hear your brethren, you may tell the whole story to the congregation, and then, if he refuses to hear the brotherhood, let them take such action as they deem wise; let such an unruly member become an outcast from the kingdom. While you cannot pretend to sit in judgment on the souls of your fellows, and while you may not forgive sins or otherwise presume to usurp the prerogatives of the supervisors of the heavenly hosts, at the same time, it has been committed to your hands that you should maintain temporal order in the kingdom on earth. While you may not meddle with the divine decrees concerning eternal life, you shall determine the issues of conduct as they concern the temporal welfare of the brotherhood on earth. And so, in all these matters connected with the discipline of the brotherhood, whatsoever you shall decree on earth, shall be recognized in heaven. Although you cannot determine the eternal fate of the individual, you may legislate regarding the conduct of the group, for, where two or three of you agree concerning any of these things and ask of me, it shall be done for you if your petition is not inconsistent with the will of my Father in heaven. And all this is ever true, for, where two or three believers are gathered together, there am I in the midst of them.”}
\vs p159 1:4 \pc Simon Peter was the apostle in charge of the workers at Hippos, and when he heard Jesus thus speak, he asked: “Lord, how often shall my brother sin against me, and I forgive him? Until 7 times?” And Jesus answered Peter: \textcolour{ubdarkred}{“Not only 7 times but even to 70 times and 7. Therefore may the kingdom of heaven be likened to a certain king who ordered a financial reckoning with his stewards. And when they had begun to conduct this examination of accounts, one of his chief retainers was brought before him confessing that he owed his king 10,000 talents. Now this officer of the king’s court pleaded that hard times had come upon him, and that he did not have wherewith to pay this obligation. And so the king commanded that his property be confiscated, and that his children be sold to pay his debt. When this chief steward heard this stern decree, he fell down on his face before the king and implored him to have mercy and grant him more time, saying, ‘Lord, have a little more patience with me, and I will pay you all.’ And when the king looked upon this negligent servant and his family, he was moved with compassion. He ordered that he should be released, and that the loan should be wholly forgiven.}
\vs p159 1:5 \textcolour{ubdarkred}{“And this chief steward, having thus received mercy and forgiveness at the hands of the king, went about his business, and finding one of his subordinate stewards who owed him a mere 100 denarii, he laid hold upon him and, taking him by the throat, said, ‘Pay me all you owe.’ And then did this fellow steward fall down before the chief steward and, beseeching him, said: ‘Only have patience with me, and I will presently be able to pay you.’ But the chief steward would not show mercy to his fellow steward but rather had him cast in prison until he should pay his debt. When his fellow servants saw what had happened, they were so distressed that they went and told their lord and master, the king. When the king heard of the doings of his chief steward, he called this ungrateful and unforgiving man before him and said: ‘You are a wicked and unworthy steward. When you sought for compassion, I freely forgave you your entire debt. Why did you not also show mercy to your fellow steward, even as I showed mercy to you?’ And the king was so very angry that he delivered his ungrateful chief steward to the jailers that they might hold him until he had paid all that was due. And even so shall my heavenly Father show the more abundant mercy to those who freely show mercy to their fellows. How can you come to God asking consideration for your shortcomings when you are wont to chastise your brethren for being guilty of these same human frailties? I say to all of you: Freely you have received the good things of the kingdom; therefore freely give to your fellows on earth.”}
\vs p159 1:6 \pc Thus did Jesus teach the dangers and illustrate the unfairness of sitting in personal judgment upon one’s fellows. Discipline must be maintained, justice must be administered, but in all these matters the wisdom of the brotherhood should prevail. Jesus invested legislative and judicial authority in the \bibemph{group,} not in the \bibemph{individual.} Even this investment of authority in the group must not be exercised as personal authority. There is always danger that the verdict of an individual may be warped by prejudice or distorted by passion. Group judgment is more likely to remove the dangers and eliminate the unfairness of personal bias. Jesus sought always to minimize the elements of unfairness, retaliation, and vengeance.
\vs p159 1:7 \pc [The use of the term 77 as an illustration of mercy and forbearance was derived from the Scriptures referring to Lamech’s exultation because of the metal weapons of his son Tubal\hyp{}Cain, who, comparing these superior instruments with those of his enemies, exclaimed: “If Cain, with no weapon in his hand, was avenged 7 times, I shall now be avenged 77.”]
\usection{The Strange Preacher}
\vs p159 2:1 Jesus went over to Gamala to visit John and those who worked with him at that place. That evening, after the session of questions and answers, John said to Jesus: “Master, yesterday I went over to Ashtaroth to see a man who was teaching in your name and even claiming to be able to cast out devils. Now this fellow had never been with us, neither does he follow after us; therefore I forbade him to do such things.” Then said Jesus: \textcolour{ubdarkred}{“Forbid him not. Do you not perceive that this gospel of the kingdom shall presently be proclaimed in all the world? How can you expect that all who will believe the gospel shall be subject to your direction? Rejoice that already our teaching has begun to manifest itself beyond the bounds of our personal influence. Do you not see, John, that those who profess to do great works in my name must eventually support our cause? They certainly will not be quick to speak evil of me. My son, in matters of this sort it would be better for you to reckon that he who is not against us is for us. In the generations to come many who are not wholly worthy will do many strange things in my name, but I will not forbid them. I tell you that, even when a cup of cold water is given to a thirsty soul, the Father’s messengers shall ever make record of such a service of love.”}
\vs p159 2:2 This instruction greatly perplexed John. Had he not heard the Master say, \textcolour{ubdarkred}{“He who is not with me is against me”?} And he did not perceive that in this case Jesus was referring to man’s personal relation to the spiritual teachings of the kingdom, while in the other case reference was made to the outward and far\hyp{}flung social relations of believers regarding the questions of administrative control and the jurisdiction of one group of believers over the work of other groups which would eventually compose the forthcoming world\hyp{}wide brotherhood.
\vs p159 2:3 But John oftentimes recounted this experience in connection with his subsequent labours in behalf of the kingdom. Nevertheless, many times did the apostles take offence at those who made bold to teach in the Master’s name. To them it always seemed inappropriate that those who had never sat at Jesus’ feet should dare to teach in his name.
\vs p159 2:4 This man whom John forbade to teach and work in Jesus’ name did not heed the apostle’s injunction. He went right on with his efforts and raised up a considerable company of believers at Kanata before going on into Mesopotamia. This man, Aden, had been led to believe in Jesus through the testimony of the demented man whom Jesus healed near Kheresa, and who so confidently believed that the supposed evil spirits which the Master cast out of him entered the herd of swine and rushed them headlong over the cliff to their destruction.
\usection{Instruction for Teachers and Believers}
\vs p159 3:1 At Edrei, where Thomas and his associates laboured, Jesus spent a day and a night and, in the course of the evening’s discussion, gave expression to the principles which should guide those who preach truth, and which should activate all who teach the gospel of the kingdom. Summarized and restated in modern phraseology, Jesus taught:
\vs p159 3:2 \pc Always respect the personality of man. Never should a righteous cause be promoted by force; spiritual victories can be won only by spiritual power. This injunction against the employment of material influences refers to psychic force as well as to physical force. Overpowering arguments and mental superiority are not to be employed to coerce men and women into the kingdom. Man’s mind is not to be crushed by the mere weight of logic or overawed by shrewd eloquence. While emotion as a factor in human decisions cannot be wholly eliminated, it should not be directly appealed to in the teachings of those who would advance the cause of the kingdom. Make your appeals directly to the divine spirit that dwells within the minds of men. Do not appeal to fear, pity, or mere sentiment. In appealing to men, be fair; exercise self\hyp{}control and exhibit due restraint; show proper respect for the personalities of your pupils. Remember that I have said: \textcolour{ubdarkred}{“Behold, I stand at the door and knock, and if any man will open, I will come in.”}
\vs p159 3:3 In bringing men into the kingdom, do not lessen or destroy their self\hyp{}respect. While overmuch self\hyp{}respect may destroy proper humility and end in pride, conceit, and arrogance, the loss of self\hyp{}respect often ends in paralysis of the will. It is the purpose of this gospel to restore self\hyp{}respect to those who have lost it and to restrain it in those who have it. Make not the mistake of only condemning the wrongs in the lives of your pupils; remember also to accord generous recognition for the most praiseworthy things in their lives. Forget not that I will stop at nothing to restore self\hyp{}respect to those who have lost it, and who really desire to regain it.
\vs p159 3:4 Take care that you do not wound the self\hyp{}respect of timid and fearful souls. Do not indulge in sarcasm at the expense of my simple\hyp{}minded brethren. Be not cynical with my fear\hyp{}ridden children. Idleness is destructive of self\hyp{}respect; therefore, admonish your brethren ever to keep busy at their chosen tasks, and put forth every effort to secure work for those who find themselves without employment.
\vs p159 3:5 Never be guilty of such unworthy tactics as endeavouring to frighten men and women into the kingdom. A loving father does not frighten his children into yielding obedience to his just requirements.
\vs p159 3:6 Sometime the children of the kingdom will realize that strong feelings of emotion are not equivalent to the leadings of the divine spirit. To be strongly and strangely impressed to do something or to go to a certain place, does not necessarily mean that such impulses are the leadings of the indwelling spirit.
\vs p159 3:7 Forewarn all believers regarding the fringe of conflict which must be traversed by all who pass from the life as it is lived in the flesh to the higher life as it is lived in the spirit. To those who live quite wholly within either realm, there is little conflict or confusion, but all are doomed to experience more or less uncertainty during the times of transition between the two levels of living. In entering the kingdom, you cannot escape its responsibilities or avoid its obligations, but remember: The gospel yoke is easy and the burden of truth is light.
\vs p159 3:8 The world is filled with hungry souls who famish in the very presence of the bread of life; men die searching for the very God who lives within them. Men seek for the treasures of the kingdom with yearning hearts and weary feet when they are all within the immediate grasp of living faith. Faith is to religion what sails are to a ship; it is an addition of power, not an added burden of life. There is but one struggle for those who enter the kingdom, and that is to fight the good fight of faith. The believer has only one battle, and that is against doubt --- unbelief.
\vs p159 3:9 In preaching the gospel of the kingdom, you are simply teaching friendship with God. And this fellowship will appeal alike to men and women in that both will find that which most truly satisfies their characteristic longings and ideals. Tell my children that I am not only tender of their feelings and patient with their frailties, but that I am also ruthless with sin and intolerant of iniquity. I am indeed meek and humble in the presence of my Father, but I am equally and relentlessly inexorable where there is deliberate evil\hyp{}doing and sinful rebellion against the will of my Father in heaven.
\vs p159 3:10 You shall not portray your teacher as a man of sorrows. Future generations shall know also the radiance of our joy, the buoyance of our good will, and the inspiration of our good humour. We proclaim a message of good news which is infectious in its transforming power. Our religion is throbbing with new life and new meanings. Those who accept this teaching are filled with joy and in their hearts are constrained to rejoice evermore. Increasing happiness is always the experience of all who are certain about God.
\vs p159 3:11 Teach all believers to avoid leaning upon the insecure props of false sympathy. You cannot develop strong characters out of the indulgence of self\hyp{}pity; honestly endeavour to avoid the deceptive influence of mere fellowship in misery. Extend sympathy to the brave and courageous while you withhold overmuch pity from those cowardly souls who only half\hyp{}heartedly stand up before the trials of living. Offer not consolation to those who lie down before their troubles without a struggle. Sympathize not with your fellows merely that they may sympathize with you in return.
\vs p159 3:12 \pc When my children once become self\hyp{}conscious of the assurance of the divine presence, such a faith will expand the mind, ennoble the soul, reinforce the personality, augment the happiness, deepen the spirit perception, and enhance the power to love and be loved.
\vs p159 3:13 Teach all believers that those who enter the kingdom are not thereby rendered immune to the accidents of time or to the ordinary catastrophes of nature. Believing the gospel will not prevent getting into trouble, but it will ensure that you shall be \bibemph{unafraid} when trouble does overtake you. If you dare to believe in me and wholeheartedly proceed to follow after me, you shall most certainly by so doing enter upon the sure pathway to trouble. I do not promise to deliver you from the waters of adversity, but I do promise to go with you through all of them.
\vs p159 3:14 \pc And much more did Jesus teach this group of believers before they made ready for the night’s sleep. And they who heard these sayings treasured them in their hearts and did often recite them for the edification of the apostles and disciples who were not present when they were spoken.
\usection{The Talk with Nathaniel}
\vs p159 4:1 And then went Jesus over to Abila, where Nathaniel and his associates laboured. Nathaniel was much bothered by some of Jesus’ pronouncements which seemed to detract from the authority of the recognized Hebrew scriptures. Accordingly, on this night, after the usual period of questions and answers, Nathaniel took Jesus away from the others and asked: “Master, could you trust me to know the truth about the Scriptures? I observe that you teach us only a portion of the sacred writings --- the best as I view it --- and I infer that you reject the teachings of the rabbis to the effect that the words of the law are the very words of God, having been with God in heaven even before the times of Abraham and Moses. What is the truth about the Scriptures?” When Jesus heard the question of his bewildered apostle, he answered:
\vs p159 4:2 \pc \textcolour{ubdarkred}{“Nathaniel, you have rightly judged; I do not regard the Scriptures as do the rabbis. I will talk with you about this matter on condition that you do not relate these things to your brethren, who are not all prepared to receive this teaching. The words of the law of Moses and the teachings of the Scriptures were not in existence before Abraham. Only in recent times have the Scriptures been gathered together as we now have them. While they contain the best of the higher thoughts and longings of the Jewish people, they also contain much that is far from being representative of the character and teachings of the Father in heaven; wherefore must I choose from among the better teachings those truths which are to be gleaned for the gospel of the kingdom.}
\vs p159 4:3 \textcolour{ubdarkred}{“These writings are the work of men, some of them holy men, others not so holy. The teachings of these books represent the views and extent of enlightenment of the times in which they had their origin. As a revelation of truth, the last are more dependable than the first. The Scriptures are faulty and altogether human in origin, but mistake not, they do constitute the best collection of religious wisdom and spiritual truth to be found in all the world at this time.}
\vs p159 4:4 \textcolour{ubdarkred}{“Many of these books were not written by the persons whose names they bear, but that in no way detracts from the value of the truths which they contain. If the story of Jonah should not be a fact, even if Jonah had never lived, still would the profound truth of this narrative, the love of God for Nineveh and the so\hyp{}called heathen, be none the less precious in the eyes of all those who love their fellow men. The Scriptures are sacred because they present the thoughts and acts of men who were searching for God, and who in these writings left on record their highest concepts of righteousness, truth, and holiness. The Scriptures contain much that is true, very much, but in the light of your present teaching, you know that these writings also contain much that is misrepresentative of the Father in heaven, the loving God I have come to reveal to all the worlds.}
\vs p159 4:5 \textcolour{ubdarkred}{“Nathaniel, never permit yourself for one moment to believe the Scripture records which tell you that the God of love directed your forefathers to go forth in battle to slay all their enemies --- men, women, and children. Such records are the words of men, not very holy men, and they are not the word of God. The Scriptures always have, and always will, reflect the intellectual, moral, and spiritual status of those who create them. Have you not noted that the concepts of Yahweh grow in beauty and glory as the prophets make their records from Samuel to Isaiah? And you should remember that the Scriptures are intended for religious instruction and spiritual guidance. They are not the works of either historians or philosophers.}
\vs p159 4:6 \textcolour{ubdarkred}{“The thing most deplorable is not merely this erroneous idea of the absolute perfection of the Scripture record and the infallibility of its teachings, but rather the confusing misinterpretation of these sacred writings by the tradition\hyp{}enslaved scribes and Pharisees at Jerusalem. And now will they employ both the doctrine of the inspiration of the Scriptures and their misinterpretations thereof in their determined effort to withstand these newer teachings of the gospel of the kingdom. Nathaniel, never forget, the Father does not limit the revelation of truth to any one generation or to any one people. Many earnest seekers after the truth have been, and will continue to be, confused and disheartened by these doctrines of the perfection of the Scriptures.}
\vs p159 4:7 \textcolour{ubdarkred}{“The authority of truth is the very spirit that indwells its living manifestations, and not the dead words of the less illuminated and supposedly inspired men of another generation. And even if these holy men of old lived inspired and spirit\hyp{}filled lives, that does not mean that their \bibemph{words} were similarly spiritually inspired. Today we make no record of the teachings of this gospel of the kingdom lest, when I have gone, you speedily become divided up into sundry groups of truth contenders as a result of the diversity of your interpretation of my teachings. For this generation it is best that we \bibemph{live} these truths while we shun the making of records.}
\vs p159 4:8 \textcolour{ubdarkred}{“Mark you well my words, Nathaniel, nothing which human nature has touched can be regarded as infallible. Through the mind of man divine truth may indeed shine forth, but always of relative purity and partial divinity. The creature may crave infallibility, but only the Creators possess it.}
\vs p159 4:9 \textcolour{ubdarkred}{“But the greatest error of the teaching about the Scriptures is the doctrine of their being sealed books of mystery and wisdom which only the wise minds of the nation dare to interpret. The revelations of divine truth are not sealed except by human ignorance, bigotry, and narrow\hyp{}minded intolerance. The light of the Scriptures is only dimmed by prejudice and darkened by superstition. A false fear of sacredness has prevented religion from being safeguarded by common sense. The fear of the authority of the sacred writings of the past effectively prevents the honest souls of today from accepting the new light of the gospel, the light which these very God\hyp{}knowing men of another generation so intensely longed to see.}
\vs p159 4:10 \textcolour{ubdarkred}{“But the saddest feature of all is the fact that some of the teachers of the sanctity of this traditionalism know this very truth. They more or less fully understand these limitations of Scripture, but they are moral cowards, intellectually dishonest. They know the truth regarding the sacred writings, but they prefer to withhold such disturbing facts from the people. And thus do they pervert and distort the Scriptures, making them the guide to slavish details of the daily life and an authority in things nonspiritual instead of appealing to the sacred writings as the repository of the moral wisdom, religious inspiration, and the spiritual teaching of the God\hyp{}knowing men of other generations.”}
\vs p159 4:11 \pc Nathaniel was enlightened, and shocked, by the Master’s pronouncement. He long pondered this talk in the depths of his soul, but he told no man concerning this conference until after Jesus’ ascension; and even then he feared to impart the full story of the Master’s instruction.
\usection{The Positive Nature of Jesus’ Religion}
\vs p159 5:1 At Philadelphia, where James was working, Jesus taught the disciples about the positive nature of the gospel of the kingdom. When, in the course of his remarks, he intimated that some parts of the Scripture were more truth\hyp{}containing than others and admonished his hearers to feed their souls upon the best of the spiritual food, James interrupted the Master, asking: “Would you be good enough, Master, to suggest to us how we may choose the better passages from the Scriptures for our personal edification?” And Jesus replied: \textcolour{ubdarkred}{“Yes, James, when you read the Scriptures look for those eternally true and divinely beautiful teachings, such as:}
\vs p159 5:2 \textcolour{ubdarkred}{“Create in me a clean heart, O Lord.}
\vs p159 5:3 \pc \textcolour{ubdarkred}{“The Lord is my shepherd; I shall not want.}
\vs p159 5:4 \pc \textcolour{ubdarkred}{“You should love your neighbour as yourself.}
\vs p159 5:5 \pc \textcolour{ubdarkred}{“For I, the Lord your God, will hold your right hand, saying, fear not; I will help you.}
\vs p159 5:6 \pc \textcolour{ubdarkred}{“Neither shall the nations learn war any more.”}
\vs p159 5:7 \pc And this is illustrative of the way Jesus, day by day, appropriated the cream of the Hebrew scriptures for the instruction of his followers and for inclusion in the teachings of the new gospel of the kingdom. Other religions had suggested the thought of the nearness of God to man, but Jesus made the care of God for man like the solicitude of a loving father for the welfare of his dependent children and then made this teaching the cornerstone of his religion. And thus did the doctrine of the fatherhood of God make imperative the practice of the brotherhood of man. The worship of God and the service of man became the sum and substance of his religion. Jesus took the best of the Jewish religion and translated it to a worthy setting in the new teachings of the gospel of the kingdom.
\vs p159 5:8 Jesus put the spirit of positive action into the passive doctrines of the Jewish religion. In the place of negative compliance with ceremonial requirements, Jesus enjoined the positive doing of that which his new religion required of those who accepted it. Jesus’ religion consisted not merely in \bibemph{believing,} but in actually \bibemph{doing,} those things which the gospel required. He did not teach that the essence of his religion consisted in social service, but rather that social service was one of the certain effects of the possession of the spirit of true religion.
\vs p159 5:9 Jesus did not hesitate to appropriate the better half of a Scripture while he repudiated the lesser portion. His great exhortation, \textcolour{ubdarkred}{“Love your neighbour as yourself,”} he took from the Scripture which reads: “You shall not take vengeance against the children of your people, but you shall love your neighbour as yourself.” Jesus appropriated the positive portion of this Scripture while rejecting the negative part. He even opposed negative or purely passive nonresistance. Said he: \textcolour{ubdarkred}{“When an enemy smites you on one cheek, do not stand there dumb and passive but in positive attitude turn the other; that is, do the best thing possible actively to lead your brother in error away from the evil paths into the better ways of righteous living.”} Jesus required his followers to react positively and aggressively to every life situation. The turning of the other cheek, or whatever act that may typify, demands initiative, necessitates vigorous, active, and courageous expression of the believer’s personality.
\vs p159 5:10 Jesus did not advocate the practice of negative submission to the indignities of those who might purposely seek to impose upon the practitioners of nonresistance to evil, but rather that his followers should be wise and alert in the quick and positive reaction of good to evil to the end that they might effectively overcome evil with good. Forget not, the truly good is invariably more powerful than the most malignant evil. The Master taught a positive standard of righteousness: \textcolour{ubdarkred}{“Whosoever wishes to be my disciple, let him disregard himself and take up the full measure of his responsibilities daily to follow me.”} And he so lived himself in that “he went about doing good.” And this aspect of the gospel was well illustrated by many parables which he later spoke to his followers. He never exhorted his followers patiently to bear their obligations but rather with energy and enthusiasm to live up to the full measure of their human responsibilities and divine privileges in the kingdom of God.
\vs p159 5:11 When Jesus instructed his apostles that they should, when one unjustly took away the coat, offer the other garment, he referred not so much to a literal second coat as to the idea of doing something \bibemph{positive} to save the wrongdoer in the place of the olden advice to retaliate --- “an eye for an eye” and so on. Jesus abhorred the idea either of retaliation or of becoming just a passive sufferer or victim of injustice. On this occasion he taught them the three ways of contending with, and resisting, evil:
\vs p159 5:12 \ublistelem{1.}\bibnobreakspace To return evil for evil --- the positive but unrighteous method.
\vs p159 5:13 \ublistelem{2.}\bibnobreakspace To suffer evil without complaint and without resistance --- the purely negative method.
\vs p159 5:14 \ublistelem{3.}\bibnobreakspace To return good for evil, to assert the will so as to become master of the situation, to overcome evil with good --- the positive and righteous method.
\vs p159 5:15 \pc One of the apostles once asked: “Master, what should I do if a stranger forced me to carry his pack for a mile?” Jesus answered: \textcolour{ubdarkred}{“Do not sit down and sigh for relief while you berate the stranger under your breath. Righteousness comes not from such passive attitudes. If you can think of nothing more effectively positive to do, you can at least carry the pack a second mile. That will of a certainty challenge the unrighteous and ungodly stranger.”}
\vs p159 5:16 The Jews had heard of a God who would forgive repentant sinners and try to forget their misdeeds, but not until Jesus came, did men hear about a God who went in search of lost sheep, who took the initiative in looking for sinners, and who rejoiced when he found them willing to return to the Father’s house. This positive note in religion Jesus extended even to his prayers. And he converted the negative golden rule into a positive admonition of human fairness.
\vs p159 5:17 In all his teaching Jesus unfailingly avoided distracting details. He shunned flowery language and avoided the mere poetic imagery of a play upon words. He habitually put large meanings into small expressions. For purposes of illustration Jesus reversed the current meanings of many terms, such as salt, leaven, fishing, and little children. He most effectively employed the antithesis, comparing the minute to the infinite and so on. His pictures were striking, such as, \textcolour{ubdarkred}{“The blind leading the blind.”} But the greatest strength to be found in his illustrative teaching was its naturalness. Jesus brought the philosophy of religion from heaven down to earth. He portrayed the elemental needs of the soul with a new insight and a new bestowal of affection.
\usection{The Return to Magadan}
\vs p159 6:1 The mission of four weeks in the Decapolis was moderately successful. Hundreds of souls were received into the kingdom, and the apostles and evangelists had a valuable experience in carrying on their work without the inspiration of the immediate personal presence of Jesus.
\vs p159 6:2 On Friday, September 16, the entire corps of workers assembled by prearrangement at Magadan Park. On the Sabbath day a council of more than 100 believers was held at which the future plans for extending the work of the kingdom were fully considered. The messengers of David were present and made reports concerning the welfare of the believers throughout Judea, Samaria, Galilee, and adjoining districts.
\vs p159 6:3 Few of Jesus’ followers at this time fully appreciated the great value of the services of the messenger corps. Not only did the messengers keep the believers throughout Palestine in touch with each other and with Jesus and the apostles, but during these dark days they also served as collectors of funds, not only for the sustenance of Jesus and his associates, but also for the support of the families of the twelve apostles and the twelve evangelists.
\vs p159 6:4 About this time Abner moved his base of operations from Hebron to Bethlehem, and this latter place was also the headquarters in Judea for David’s messengers. David maintained an overnight relay messenger service between Jerusalem and Bethsaida. These runners left Jerusalem each evening, relaying at Sychar and Scythopolis, arriving in Bethsaida by breakfast time the next morning.
\vs p159 6:5 Jesus and his associates now prepared to take a week’s rest before they made ready to start upon the last epoch of their labours in behalf of the kingdom. This was their last rest, for the Perean mission developed into a campaign of preaching and teaching which extended right on down to the time of their arrival at Jerusalem and of the enactment of the closing episodes of Jesus’ earth career.
\quizlink
