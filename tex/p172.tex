\upaper{172}{Going into Jerusalem}
\uminitoc{Sabbath at Bethany}
\uminitoc{Sunday Morning with the Apostles}
\uminitoc{The Start for Jerusalem}
\uminitoc{Visiting about the Temple}
\uminitoc{The Apostles’ Attitude}
\author{Midwayer Commission}
\vs p172 0:1 Jesus and the apostles arrived at Bethany shortly after 16:00 on Friday afternoon, March 31, A.D.\,30. Lazarus, his sisters, and their friends were expecting them; and since so many people came every day to talk with Lazarus about his resurrection, Jesus was informed that arrangements had been made for him to stay with a neighbouring believer, one Simon, the leading citizen of the little village since the death of Lazarus’s father.
\vs p172 0:2 That evening, Jesus received many visitors, and the common folks of Bethany and Bethpage did their best to make him feel welcome. Although many thought Jesus was now going into Jerusalem, in utter defiance of the Sanhedrin’s decree of death, to proclaim himself king of the Jews, the Bethany family --- Lazarus, Martha, and Mary --- more fully realized that the Master was not that kind of a king; they dimly felt that this might be his last visit to Jerusalem and Bethany.
\vs p172 0:3 The chief priests were informed that Jesus lodged at Bethany, but they thought best not to attempt to seize him among his friends; they decided to await his coming on into Jerusalem. Jesus knew about all this, but he was majestically calm; his friends had never seen him more composed and congenial; even the apostles were astounded that he should be so unconcerned when the Sanhedrin had called upon all Jewry to deliver him into their hands. While the Master slept that night, the apostles watched over him by twos, and many of them were girded with swords. Early the next morning they were awakened by hundreds of pilgrims who came out from Jerusalem, even on the Sabbath day, to see Jesus and Lazarus, whom he had raised from the dead.
\usection{Sabbath at Bethany}
\vs p172 1:1 Pilgrims from outside of Judea, as well as the Jewish authorities, had all been asking: “What do you think? will Jesus come up to the feast?” Therefore, when the people heard that Jesus was at Bethany, they were glad, but the chief priests and Pharisees were somewhat perplexed. They were pleased to have him under their jurisdiction, but they were a trifle disconcerted by his boldness; they remembered that on his previous visit to Bethany, Lazarus had been raised from the dead, and Lazarus was becoming a big problem to the enemies of Jesus.
\vs p172 1:2 Six days before the Passover, on the evening after the Sabbath, all Bethany and Bethpage joined in celebrating the arrival of Jesus by a public banquet at the home of Simon. This supper was in honour of both Jesus and Lazarus; it was tendered in defiance of the Sanhedrin. Martha directed the serving of the food; her sister Mary was among the women onlookers as it was against the custom of the Jews for a woman to sit at a public banquet. The agents of the Sanhedrin were present, but they feared to apprehend Jesus in the midst of his friends.
\vs p172 1:3 Jesus talked with Simon about Joshua of old, whose namesake he was, and recited how Joshua and the Israelites had come up to Jerusalem through Jericho. In commenting on the legend of the walls of Jericho falling down, Jesus said: \textcolour{ubdarkred}{“I am not concerned with such walls of brick and stone; but I would cause the walls of prejudice, self\hyp{}righteousness, and hate to crumble before this preaching of the Father’s love for all men.”}
\vs p172 1:4 The banquet went along in a very cheerful and normal manner except that all the apostles were unusually sober. Jesus was exceptionally cheerful and had been playing with the children up to the time of coming to the table.
\vs p172 1:5 \pc Nothing out of the ordinary happened until near the close of the feasting when Mary the sister of Lazarus stepped forward from among the group of women onlookers and, going up to where Jesus reclined as the guest of honour, proceeded to open a large alabaster cruse of very rare and costly ointment; and after anointing the Master’s head, she began to pour it upon his feet as she took down her hair and wiped them with it. The whole house became filled with the odour of the ointment, and everybody present was amazed at what Mary had done. Lazarus said nothing, but when some of the people murmured, showing indignation that so costly an ointment should be thus used, Judas Iscariot stepped over to where Andrew reclined and said: “Why was this ointment not sold and the money bestowed to feed the poor? You should speak to the Master that he rebuke such waste.”
\vs p172 1:6 Jesus, knowing what they thought and hearing what they said, put his hand upon Mary’s head as she knelt by his side and, with a kindly expression upon his face, said: \textcolour{ubdarkred}{“Let her alone, every one of you. Why do you trouble her about this, seeing that she has done a good thing in her heart? To you who murmur and say that this ointment should have been sold and the money given to the poor, let me say that you have the poor always with you so that you may minister to them at any time it seems good to you; but I shall not always be with you; I go soon to my Father. This woman has long saved this ointment for my body at its burial, and now that it has seemed good to her to make this anointing in anticipation of my death, she shall not be denied such satisfaction. In the doing of this, Mary has reproved all of you in that by this act she evinces faith in what I have said about my death and ascension to my Father in heaven. This woman shall not be reproved for that which she has this night done; rather do I say to you that in the ages to come, wherever this gospel shall be preached throughout the whole world, what she has done will be spoken of in memory of her.”}\tunemarkup{private}{\begin{figure}[H]\centering\includegraphics[width=\columnwidth]{images/Turning-Point.jpg}\caption{The Turning Point by Stephen Sawyer and Slawa Radziszewska}\end{figure}}
\vs p172 1:7 It was because of this rebuke, which he took as a personal reproof, that Judas Iscariot finally made up his mind to seek revenge for his hurt feelings. Many times had he entertained such ideas subconsciously, but now he dared to think such wicked thoughts in his open and conscious mind. And many others encouraged him in this attitude since the cost of this ointment was a sum equal to the earnings of one man for one year --- enough to provide bread for 5,000 persons. But Mary loved Jesus; she had provided this precious ointment with which to embalm his body in death, for she believed his words when he forewarned them that he must die, and it was not to be denied her if she changed her mind and chose to bestow this offering upon the Master while he yet lived.
\vs p172 1:8 Both Lazarus and Martha knew that Mary had long saved the money wherewith to buy this cruse of spikenard, and they heartily approved of her doing as her heart desired in such a matter, for they were well\hyp{}to\hyp{}do and could easily afford to make such an offering.
\vs p172 1:9 When the chief priests heard of this dinner in Bethany for Jesus and Lazarus, they began to take counsel among themselves as to what should be done with Lazarus. And presently they decided that Lazarus must also die. They rightly concluded that it would be useless to put Jesus to death if they permitted Lazarus, whom he had raised from the dead, to live.
\usection{Sunday Morning with the Apostles}
\vs p172 2:1 On this Sunday morning, in Simon’s beautiful garden, the Master called his 12 apostles around him and gave them their final instructions preparatory to entering Jerusalem. He told them that he would probably deliver many addresses and teach many lessons before returning to the Father but advised the apostles to refrain from doing any public work during this Passover sojourn in Jerusalem. He instructed them to remain near him and to \textcolour{ubdarkred}{“watch and pray.”} Jesus knew that many of his apostles and immediate followers even then carried swords concealed on their persons, but he made no reference to this fact.
\vs p172 2:2 This morning’s instructions embraced a brief review of their ministry from the day of their ordination near Capernaum down to this day when they were preparing to enter Jerusalem. The apostles listened in silence; they asked no questions.
\vs p172 2:3 Early that morning David Zebedee had turned over to Judas the funds realized from the sale of the equipment of the Pella encampment, and Judas, in turn, had placed the greater part of this money in the hands of Simon, their host, for safekeeping in anticipation of the exigencies of their entry into Jerusalem.
\vs p172 2:4 After the conference with the apostles Jesus held converse with Lazarus and instructed him to avoid the sacrifice of his life to the vengefulness of the Sanhedrin. It was in obedience to this admonition that Lazarus, a few days later, fled to Philadelphia when the officers of the Sanhedrin sent men to arrest him.
\vs p172 2:5 In a way, all of Jesus’ followers sensed the impending crisis, but they were prevented from fully realizing its seriousness by the unusual cheerfulness and exceptional good humour of the Master.
\usection{The Start for Jerusalem}
\vs p172 3:1 Bethany was about 3.2\,km from the temple, and it was 13:30 that Sunday afternoon when Jesus made ready to start for Jerusalem. He had feelings of profound affection for Bethany and its simple people. Nazareth, Capernaum, and Jerusalem had rejected him, but Bethany had accepted him, had believed in him. And it was in this small village, where almost every man, woman, and child were believers, that he chose to perform the mightiest work of his earth bestowal, the resurrection of Lazarus. He did not raise Lazarus that the villagers might believe, but rather because they already believed.
\vs p172 3:2 All morning Jesus had thought about his entry into Jerusalem. Heretofore he had always endeavoured to suppress all public acclaim of him as the Messiah, but it was different now; he was nearing the end of his career in the flesh, his death had been decreed by the Sanhedrin, and no harm could come from allowing his disciples to give free expression to their feelings, just as might occur if he elected to make a formal and public entry into the city.
\vs p172 3:3 Jesus did not decide to make this public entrance into Jerusalem as a last bid for popular favour nor as a final grasp for power. Neither did he do it altogether to satisfy the human longings of his disciples and apostles. Jesus entertained none of the illusions of a fantastic dreamer; he well knew what was to be the outcome of this visit.
\vs p172 3:4 Having decided upon making a public entrance into Jerusalem, the Master was confronted with the necessity of choosing a proper method of executing such a resolve. Jesus thought over all of the many more or less contradictory so\hyp{}called Messianic prophesies, but there seemed to be only one which was at all appropriate for him to follow. Most of these prophetic utterances depicted a king, the son and successor of David, a bold and aggressive temporal deliverer of all Israel from the yoke of foreign domination. But there was one Scripture that had sometimes been associated with the Messiah by those who held more to the spiritual concept of his mission, which Jesus thought might consistently be taken as a guide for his projected entry into Jerusalem. This Scripture was found in Zechariah, and it said: “Rejoice greatly, O daughter of Zion; shout, O daughter of Jerusalem. Behold, your king comes to you. He is just and he brings salvation. He comes as the lowly one, riding upon an ass, upon a colt, the foal of an ass.”
\vs p172 3:5 \pc A warrior king always entered a city riding upon a horse; a king on a mission of peace and friendship always entered riding upon an ass. Jesus would not enter Jerusalem as a man on horseback, but he was willing to enter peacefully and with good will as the Son of Man on a donkey.
\vs p172 3:6 \pc Jesus had long tried by direct teaching to impress upon his apostles and his disciples that his kingdom was not of this world, that it was a purely spiritual matter; but he had not succeeded in this effort. Now, what he had failed to do by plain and personal teaching, he would attempt to accomplish by a symbolic appeal. Accordingly, right after the noon lunch, Jesus called Peter and John, and after directing them to go over to Bethpage, a neighbouring village a little off the main road and a short distance north\hyp{}west of Bethany, he further said: “Go to Bethpage, and when you come to the junction of the roads, you will find the colt of an ass tied there. Loose the colt and bring it back with you. If anyone asks you why you do this, merely say, ‘The Master has need of him.’” And when the two apostles had gone into Bethpage as the Master had directed, they found the colt tied near his mother in the open street and close to a house on the corner. As Peter began to untie the colt, the owner came over and asked why they did this, and when Peter answered him as Jesus had directed, the man said: “If your Master is Jesus from Galilee, let him have the colt.” And so they returned bringing the colt with them.
\vs p172 3:7 By this time several hundred pilgrims had gathered around Jesus and his apostles. Since midforenoon the visitors passing by on their way to the Passover had tarried. Meanwhile, David Zebedee and some of his former messenger associates took it upon themselves to hasten on down to Jerusalem, where they effectively spread the report among the throngs of visiting pilgrims about the temple that Jesus of Nazareth was making a triumphal entry into the city. Accordingly, several thousand of these visitors flocked forth to greet this much\hyp{}talked\hyp{}of prophet and wonder\hyp{}worker, whom some believed to be the Messiah. This multitude, coming out from Jerusalem, met Jesus and the crowd going into the city just after they had passed over the brow of Olivet and had begun the descent into the city.
\vs p172 3:8 As the procession started out from Bethany, there was great enthusiasm among the festive crowd of disciples, believers, and visiting pilgrims, many hailing from Galilee and Perea. Just before they started, the 12 women of the original women’s corps, accompanied by some of their associates, arrived on the scene and joined this unique procession as it moved on joyously toward the city.
\vs p172 3:9 Before they started, the Alpheus twins put their cloaks on the donkey and held him while the Master got on. As the procession moved toward the summit of Olivet, the festive crowd threw their garments on the ground and brought branches from the near\hyp{}by trees in order to make a carpet of honour for the donkey bearing the royal Son, the promised Messiah. As the merry crowd moved on toward Jerusalem, they began to sing, or rather to shout in unison, the Psalm, “Hosanna to the son of David; blessed is he who comes in the name of the Lord. Hosanna in the highest. Blessed be the kingdom that comes down from heaven.”
\vs p172 3:10 Jesus was light\hyp{}hearted and cheerful as they moved along until he came to the brow of Olivet, where the city and the temple towers came into full view; there the Master stopped the procession, and a great silence came upon all as they beheld him weeping. Looking down upon the vast multitude coming forth from the city to greet him, the Master, with much emotion and with tearful voice, said: \textcolour{ubdarkred}{“O Jerusalem, if you had only known, even you, at least in this your day, the things which belong to your peace, and which you could so freely have had! But now are these glories about to be hid from your eyes. You are about to reject the Son of Peace and turn your backs upon the gospel of salvation. The days will soon come upon you wherein your enemies will cast a trench around about you and lay siege to you on every side; they shall utterly destroy you, insomuch that not one stone shall be left upon another. And all this shall befall you because you knew not the time of your divine visitation. You are about to reject the gift of God, and all men will reject you.”}
\vs p172 3:11 When he had finished speaking, they began the descent of Olivet and presently were joined by the multitude of visitors who had come from Jerusalem waving palm branches, shouting hosannas, and otherwise expressing gleefulness and good fellowship. The Master had not planned that these crowds should come out from Jerusalem to meet them; that was the work of others. He never premeditated anything which was dramatic.
\vs p172 3:12 Along with the multitude which poured out to welcome the Master, there came also many of the Pharisees and his other enemies. They were so much perturbed by this sudden and unexpected outburst of popular acclaim that they feared to arrest him lest such action precipitate an open revolt of the populace. They greatly feared the attitude of the large numbers of visitors, who had heard much of Jesus, and who, many of them, believed in him.
\vs p172 3:13 As they neared Jerusalem, the crowd became more demonstrative, so much so that some of the Pharisees made their way up alongside Jesus and said: “Teacher, you should rebuke your disciples and exhort them to behave more seemly.” Jesus answered: \textcolour{ubdarkred}{“It is only fitting that these children should welcome the Son of Peace, whom the chief priests have rejected. It would be useless to stop them lest in their stead these stones by the roadside cry out.”}
\vs p172 3:14 The Pharisees hastened on ahead of the procession to rejoin the Sanhedrin, which was then in session at the temple, and they reported to their associates: “Behold, all that we do is of no avail; we are confounded by this Galilean. The people have gone mad over him; if we do not stop these ignorant ones, all the world will go after him.”
\vs p172 3:15 There really was no deep significance to be attached to this superficial and spontaneous outburst of popular enthusiasm. This welcome, although it was joyous and sincere, did not betoken any real or deep\hyp{}seated conviction in the hearts of this festive multitude. These same crowds were equally as willing quickly to reject Jesus later on this week when the Sanhedrin once took a firm and decided stand against him, and when they became disillusioned --- when they realized that Jesus was not going to establish the kingdom in accordance with their long\hyp{}cherished expectations.
\vs p172 3:16 But the whole city was mightily stirred up, insomuch that everyone asked, “Who is this man?” And the multitude answered, “This is the prophet of Galilee, Jesus of Nazareth.”
\usection{Visiting about the Temple}
\vs p172 4:1 While the Alpheus twins returned the donkey to its owner, Jesus and the ten apostles detached themselves from their immediate associates and strolled about the temple, viewing the preparations for the Passover. No attempt was made to molest Jesus as the Sanhedrin greatly feared the people, and that was, after all, one of the reasons Jesus had for allowing the multitude thus to acclaim him. The apostles little understood that this was the only human procedure which could have been effective in preventing Jesus’ immediate arrest upon entering the city. The Master desired to give the inhabitants of Jerusalem, high and low, as well as the tens of thousands of Passover visitors, this one more and last chance to hear the gospel and receive, if they would, the Son of Peace.
\vs p172 4:2 And now, as the evening drew on and the crowds went in quest of nourishment, Jesus and his immediate followers were left alone. What a strange day it had been! The apostles were thoughtful, but speechless. Never, in their years of association with Jesus, had they seen such a day. For a moment they sat down by the treasury, watching the people drop in their contributions: the rich putting much in the receiving box and all giving something in accordance with the extent of their possessions. At last there came along a poor widow, scantily attired, and they observed as she cast two mites (small coppers) into the trumpet. And then said Jesus, calling the attention of the apostles to the widow: \textcolour{ubdarkred}{“Heed well what you have just seen. This poor widow cast in more than all the others, for all these others, from their superfluity, cast in some trifle as a gift, but this poor woman, even though she is in want, gave all that she had, even her living.”}
\vs p172 4:3 As the evening drew on, they walked about the temple courts in silence, and after Jesus had surveyed these familiar scenes once more, recalling his emotions in connection with previous visits, not excepting the earlier ones, he said, \textcolour{ubdarkred}{“Let us go up to Bethany for our rest.”} Jesus, with Peter and John, went to the home of Simon, while the other apostles lodged among their friends in Bethany and Bethpage.
\usection{The Apostles’ Attitude}
\vs p172 5:1 This Sunday evening as they returned to Bethany, Jesus walked in front of the apostles. Not a word was spoken until they separated after arriving at Simon’s house. No 12 human beings ever experienced such diverse and inexplicable emotions as now surged through the minds and souls of these ambassadors of the kingdom. These sturdy Galileans were confused and disconcerted; they did not know what to expect next; they were too surprised to be much afraid. They knew nothing of the Master’s plans for the next day, and they asked no questions. They went to their lodgings, though they did not sleep much, save the twins. But they did not keep armed watch over Jesus at Simon’s house.
\vs p172 5:2 Andrew was thoroughly bewildered, well\hyp{}nigh confused. He was the one apostle who did not seriously undertake to evaluate the popular outburst of acclaim. He was too preoccupied with the thought of his responsibility as chief of the apostolic corps to give serious consideration to the meaning or significance of the loud hosannas of the multitude. Andrew was busy watching some of his associates who he feared might be led away by their emotions during the excitement, particularly Peter, James, John, and Simon Zelotes. Throughout this day and those which immediately followed, Andrew was troubled with serious doubts, but he never expressed any of these misgivings to his apostolic associates. He was concerned about the attitude of some of the 12 who he knew were armed with swords; but he did not know that his own brother, Peter, was carrying such a weapon. And so the procession into Jerusalem made a comparatively superficial impression upon Andrew; he was too busy with the responsibilities of his office to be otherwise affected.\fnc{Andrew was busy watching some of his associates \bibtextul{whom} he feared might be led away by their emotions\ldots{} \bibexpl{The pronoun here is the subject of the verb phrase “might be led away;” not the object of “feared.” To clarify, Andrew feared they might be led away by their emotions; he was not watching his associates, whom he feared. --- He did not fear them, but he was afraid they might be led astray.Also:First printing: He was concerned about the attitude of some of the twelve \bibtextul{whom} he knew were armed with swords\ldots{}Changed to: He was concerned about the attitude of some of the twelve \bibtextul{who} he knew were armed with swords\ldots{} --- The pronoun is the subject of the verb “were armed,” not the object of “knew” nor of “were armed;” therefore “who” is the correct form. To illustrate: \ldots{}some of the twelve whom he knew Peter had armed\ldots{}[he knew Peter had armed them] \ldots{}some of the twelve who he knew were armed\ldots{} [he knew they were armed] The sentence might have been written “He was concerned about the attitude of the twelve, some of whom he knew were armed with swords.” In which case, “whom” would be the object of the prepositional phrase “some of whom,” while the phrase itself would be the subject of “were armed,” but it was not.}}
\vs p172 5:3 Simon Peter was at first almost swept off his feet by this popular manifestation of enthusiasm; but he was considerably sobered by the time they returned to Bethany that night. Peter simply could not figure out what the Master was about. He was terribly disappointed that Jesus did not follow up this wave of popular favour with some kind of a pronouncement. Peter could not understand why Jesus did not speak to the multitude when they arrived at the temple, or at least permit one of the apostles to address the crowd. Peter was a great preacher, and he disliked to see such a large, receptive, and enthusiastic audience go to waste. He would so much have liked to preach the gospel of the kingdom to that throng right there in the temple; but the Master had specifically charged them that they were to do no teaching or preaching while in Jerusalem this Passover week. The reaction from the spectacular procession into the city was disastrous to Simon Peter; by night he was sobered and inexpressibly saddened.
\vs p172 5:4 To James Zebedee, this Sunday was a day of perplexity and profound confusion; he could not grasp the purport of what was going on; he could not comprehend the Master’s purpose in permitting this wild acclaim and then in refusing to say a word to the people when they arrived at the temple. As the procession moved down Olivet toward Jerusalem, more especially when they were met by the thousands of pilgrims who poured forth to welcome the Master, James was cruelly torn by his conflicting emotions of elation and gratification at what he saw and by his profound feeling of fear as to what would happen when they reached the temple. And then was he downcast and overcome by disappointment when Jesus climbed off the donkey and proceeded to walk leisurely about the temple courts. James could not understand the reason for throwing away such a magnificent opportunity to proclaim the kingdom. By night, his mind was held firmly in the grip of a distressing and dreadful uncertainty.
\vs p172 5:5 John Zebedee came somewhere near understanding why Jesus did this; at least he grasped in part the spiritual significance of this so\hyp{}called triumphal entry into Jerusalem. As the multitude moved on toward the temple, and as John beheld his Master sitting there astride the colt, he recalled hearing Jesus onetime quote the passage of Scripture, the utterance of Zechariah, which described the coming of the Messiah as a man of peace and riding into Jerusalem on an ass. As John turned this Scripture over in his mind, he began to comprehend the symbolic significance of this Sunday\hyp{}afternoon pageant. At least, he grasped enough of the meaning of this Scripture to enable him somewhat to enjoy the episode and to prevent his becoming overmuch depressed by the apparent purposeless ending of the triumphal procession. John had a type of mind which naturally tended to think and feel in symbols.
\vs p172 5:6 \pc Philip was entirely unsettled by the suddenness and spontaneity of the outburst. He could not collect his thoughts sufficiently while on the way down Olivet to arrive at any settled notion as to what all the demonstration was about. In a way, he enjoyed the performance because his Master was being honoured. By the time they reached the temple, he was perturbed by the thought that Jesus might possibly ask him to feed the multitude, so that the conduct of Jesus in turning leisurely away from the crowds, which so sorely disappointed the majority of the apostles, was a great relief to Philip. Multitudes had sometimes been a great trial to the steward of the 12. After he was relieved of these personal fears regarding the material needs of the crowds, Philip joined with Peter in the expression of disappointment that nothing was done to teach the multitude. That night Philip got to thinking over these experiences and was tempted to doubt the whole idea of the kingdom; he honestly wondered what all these things could mean, but he expressed his doubts to no one; he loved Jesus too much. He had great personal faith in the Master.
\vs p172 5:7 \pc Nathaniel, aside from the symbolic and prophetic aspects, came the nearest to understanding the Master’s reason for enlisting the popular support of the Passover pilgrims. He reasoned it out, before they reached the temple, that without such a demonstrative entry into Jerusalem Jesus would have been arrested by the Sanhedrin officials and cast into prison the moment he presumed to enter the city. He was not, therefore, in the least surprised that the Master made no further use of the cheering crowds when he had once got inside the walls of the city and had thus so forcibly impressed the Jewish leaders that they would refrain from placing him under immediate arrest. Understanding the real reason for the Master’s entering the city in this manner, Nathaniel naturally followed along with more poise and was less perturbed and disappointed by Jesus’ subsequent conduct than were the other apostles. Nathaniel had great confidence in Jesus’ understanding of men as well as in his sagacity and cleverness in handling difficult situations.
\vs p172 5:8 \pc Matthew was at first nonplussed by this pageant performance. He did not grasp the meaning of what his eyes were seeing until he also recalled the Scripture in Zechariah where the prophet had alluded to the rejoicing of Jerusalem because her king had come bringing salvation and riding upon the colt of an ass. As the procession moved in the direction of the city and then drew on toward the temple, Matthew became ecstatic; he was certain that something extraordinary would happen when the Master arrived at the temple at the head of this shouting multitude. When one of the Pharisees mocked Jesus, saying, “Look, everybody, see who comes here, the king of the Jews riding on an ass!” Matthew kept his hands off of him only by exercising great restraint. None of the 12 was more depressed on the way back to Bethany that evening. Next to Simon Peter and Simon Zelotes, he experienced the highest nervous tension and was in a state of exhaustion by night. But by morning Matthew was much cheered; he was, after all, a cheerful loser.
\vs p172 5:9 \pc Thomas was the most bewildered and puzzled man of all the 12. Most of the time he just followed along, gazing at the spectacle and honestly wondering what could be the Master’s motive for participating in such a peculiar demonstration. Down deep in his heart he regarded the whole performance as a little childish, if not downright foolish. He had never seen Jesus do anything like this and was at a loss to account for his strange conduct on this Sunday afternoon. By the time they reached the temple, Thomas had deduced that the purpose of this popular demonstration was so to frighten the Sanhedrin that they would not dare immediately to arrest the Master. On the way back to Bethany Thomas thought much but said nothing. By bedtime the Master’s cleverness in staging the tumultuous entry into Jerusalem had begun to make a somewhat humorous appeal, and he was much cheered up by this reaction.
\vs p172 5:10 This Sunday started off as a great day for Simon Zelotes. He saw visions of wonderful doings in Jerusalem the next few days, and in that he was right, but Simon dreamed of the establishment of the new national rule of the Jews, with Jesus on the throne of David. Simon saw the nationalists springing into action as soon as the kingdom was announced, and himself in supreme command of the assembling military forces of the new kingdom. On the way down Olivet he even envisaged the Sanhedrin and all of their sympathizers dead before sunset of that day. He really believed something great was going to happen. He was the noisiest man in the whole multitude. By 17:00 he was a silent, crushed, and disillusioned apostle. He never fully recovered from the depression which settled down on him as a result of this day’s shock; at least not until long after the Master’s resurrection.
\vs p172 5:11 To the Alpheus twins this was a perfect day. They really enjoyed it all the way through, and not being present during the time of quiet visitation about the temple, they escaped much of the anticlimax of the popular upheaval. They could not possibly understand the downcast behaviour of the apostles when they came back to Bethany that evening. In the memory of the twins this was always their day of being nearest heaven on earth. This day was the satisfying climax of their whole career as apostles. And the memory of the elation of this Sunday afternoon carried them on through all of the tragedy of this eventful week, right up to the hour of the crucifixion. It was the most befitting entry of the king the twins could conceive; they enjoyed every moment of the whole pageant. They fully approved of all they saw and long cherished the memory.
\vs p172 5:12 Of all the apostles, Judas Iscariot was the most adversely affected by this processional entry into Jerusalem. His mind was in a disagreeable ferment because of the Master’s rebuke the preceding day in connection with Mary’s anointing at the feast in Simon’s house. Judas was disgusted with the whole spectacle. To him it seemed childish, if not indeed ridiculous. As this vengeful apostle looked upon the proceedings of this Sunday afternoon, Jesus seemed to him more to resemble a clown than a king. He heartily resented the whole performance. He shared the views of the Greeks and Romans, who looked down upon anyone who would consent to ride upon an ass or the colt of an ass. By the time the triumphal procession had entered the city, Judas had about made up his mind to abandon the whole idea of such a kingdom; he was almost resolved to forsake all such farcical attempts to establish the kingdom of heaven. And then he thought of the resurrection of Lazarus, and many other things, and decided to stay on with the 12, at least for another day. Besides, he carried the bag, and he would not desert with the apostolic funds in his possession. On the way back to Bethany that night his conduct did not seem strange since all of the apostles were equally downcast and silent.
\vs p172 5:13 Judas was tremendously influenced by the ridicule of his Sadducean friends. No other single factor exerted such a powerful influence on him, in his final determination to forsake Jesus and his fellow apostles, as a certain episode which occurred just as Jesus reached the gate of the city: A prominent Sadducee (a friend of Judas’s family) rushed up to him in a spirit of gleeful ridicule and, slapping him on the back, said: “Why so troubled of countenance, my good friend; cheer up and join us all while we acclaim this Jesus of Nazareth the king of the Jews as he rides through the gates of Jerusalem seated on an ass.” Judas had never shrunk from persecution, but he could not stand this sort of ridicule. With the long\hyp{}nourished emotion of revenge there was now blended this fatal fear of ridicule, that terrible and fearful feeling of being ashamed of his Master and his fellow apostles. At heart, this ordained ambassador of the kingdom was already a deserter; it only remained for him to find some plausible excuse for an open break with the Master.
\quizlink
