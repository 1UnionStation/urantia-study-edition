\upaper{170}{The Kingdom of Heaven}
\uminitoc{Concepts of the Kingdom of Heaven}
\uminitoc{Jesus’ Concept of the Kingdom}
\uminitoc{In Relation to Righteousness}
\uminitoc{Jesus’ Teaching about the Kingdom}
\uminitoc{Later Ideas of the Kingdom}
\author{Midwayer Commission}
\vs p170 0:1 Saturday afternoon, March 11, Jesus preached his last sermon at Pella. This was among the notable addresses of his public ministry, embracing a full and complete discussion of the kingdom of heaven. He was aware of the confusion which existed in the minds of his apostles and disciples regarding the meaning and significance of the terms “kingdom of heaven” and “kingdom of God,” which he used as interchangeable designations of his bestowal mission. Although the very term kingdom of \bibemph{heaven} should have been enough to separate what it stood for from all connection with \bibemph{earthly} kingdoms and temporal governments, it was not. The idea of a temporal king was too deep\hyp{}rooted in the Jewish mind thus to be dislodged in a single generation. Therefore Jesus did not at first openly oppose this long\hyp{}nourished concept of the kingdom.
\vs p170 0:2 This Sabbath afternoon the Master sought to clarify the teaching about the kingdom of heaven; he discussed the subject from every viewpoint and endeavoured to make clear the many different senses in which the term had been used. In this narrative we will amplify the address by adding numerous statements made by Jesus on previous occasions and by including some remarks made only to the apostles during the evening discussions of this same day. We will also make certain comments dealing with the subsequent outworking of the kingdom idea as it is related to the later Christian church.
\usection{Concepts of the Kingdom of Heaven}
\vs p170 1:1 In connection with the recital of Jesus’ sermon it should be noted that throughout the Hebrew scriptures there was a dual concept of the kingdom of heaven. The prophets presented the kingdom of God as:
\vs p170 1:2 \ublistelem{1.}\bibnobreakspace A present reality; and as
\vs p170 1:3 \ublistelem{2.}\bibnobreakspace A future hope --- when the kingdom would be realized in fullness upon the appearance of the Messiah. This is the kingdom concept which John the Baptist taught.
\vs p170 1:4 From the very first Jesus and the apostles taught both of these concepts. There were two other ideas of the kingdom which should be borne in mind:
\vs p170 1:5 \ublistelem{3.}\bibnobreakspace The later Jewish concept of a world\hyp{}wide and transcendental kingdom of supernatural origin and miraculous inauguration.
\vs p170 1:6 \ublistelem{4.}\bibnobreakspace The Persian teachings portraying the establishment of a divine kingdom as the achievement of the triumph of good over evil at the end of the world.
\vs p170 1:7 \pc Just before the advent of Jesus on earth, the Jews combined and confused all of these ideas of the kingdom into their apocalyptic concept of the Messiah’s coming to establish the age of the Jewish triumph, the eternal age of God’s supreme rule on earth, the new world, the era in which all mankind would worship Yahweh. In choosing to utilize this concept of the kingdom of heaven, Jesus elected to appropriate the most vital and culminating heritage of both the Jewish and Persian religions.
\vs p170 1:8 The kingdom of heaven, as it has been understood and misunderstood down through the centuries of the Christian era, embraced four distinct groups of ideas:
\vs p170 1:9 \ublistelem{1.}\bibnobreakspace The concept of the Jews.
\vs p170 1:10 \ublistelem{2.}\bibnobreakspace The concept of the Persians.
\vs p170 1:11 \ublistelem{3.}\bibnobreakspace The personal\hyp{}experience concept of Jesus --- \textcolour{ubdarkred}{“the kingdom of heaven within you.”}
\vs p170 1:12 \ublistelem{4.}\bibnobreakspace The composite and confused concepts which the founders and promulgators of Christianity have sought to impress upon the world.
\vs p170 1:13 \pc At different times and in varying circumstances it appears that Jesus may have presented numerous concepts of the “kingdom” in his public teachings, but to his apostles he always taught the kingdom as embracing man’s personal experience in relation to his fellows on earth and to the Father in heaven. Concerning the kingdom, his last word always was, \textcolour{ubdarkred}{“The kingdom is within you.”}
\vs p170 1:14 Centuries of confusion regarding the meaning of the term “kingdom of heaven” have been due to three factors:
\vs p170 1:15 \ublistelem{1.}\bibnobreakspace The confusion occasioned by observing the idea of the “kingdom” as it passed through the various progressive phases of its recasting by Jesus and his apostles.
\vs p170 1:16 \ublistelem{2.}\bibnobreakspace The confusion which was inevitably associated with the transplantation of early Christianity from a Jewish to a gentile soil.
\vs p170 1:17 \ublistelem{3.}\bibnobreakspace The confusion which was inherent in the fact that Christianity became a religion which was organized about the central idea of Jesus’ person; the gospel of the kingdom became more and more a religion \bibemph{about} him.
\usection{Jesus’ Concept of the Kingdom}
\vs p170 2:1 The Master made it clear that the kingdom of heaven must begin with, and be centred in, the dual concept of the truth of the fatherhood of God and the correlated fact of the brotherhood of man. The acceptance of such a teaching, Jesus declared, would liberate man from the age\hyp{}long bondage of animal fear and at the same time enrich human living with the following endowments of the new life of spiritual liberty:
\vs p170 2:2 \ublistelem{1.}\bibnobreakspace The possession of new courage and augmented spiritual power. The gospel of the kingdom was to set man free and inspire him to dare to hope for eternal life.
\vs p170 2:3 \ublistelem{2.}\bibnobreakspace The gospel carried a message of new confidence and true consolation for all men, even for the poor.
\vs p170 2:4 \ublistelem{3.}\bibnobreakspace It was in itself a new standard of moral values, a new ethical yardstick wherewith to measure human conduct. It portrayed the ideal of a resultant new order of human society.
\vs p170 2:5 \ublistelem{4.}\bibnobreakspace It taught the pre\hyp{}eminence of the spiritual compared with the material; it glorified spiritual realities and exalted superhuman ideals.
\vs p170 2:6 \ublistelem{5.}\bibnobreakspace This new gospel held up spiritual attainment as the true goal of living. Human life received a new endowment of moral value and divine dignity.
\vs p170 2:7 \ublistelem{6.}\bibnobreakspace Jesus taught that eternal realities were the result (reward) of righteous earthly striving. Man’s mortal sojourn on earth acquired new meanings consequent upon the recognition of a noble destiny.
\vs p170 2:8 \ublistelem{7.}\bibnobreakspace The new gospel affirmed that human salvation is the revelation of a far\hyp{}reaching divine purpose to be fulfilled and realized in the future destiny of the endless service of the salvaged sons of God.
\vs p170 2:9 \pc These teachings cover the expanded idea of the kingdom which was taught by Jesus. This great concept was hardly embraced in the elementary and confused kingdom teachings of John the Baptist.
\vs p170 2:10 The apostles were unable to grasp the real meaning of the Master’s utterances regarding the kingdom. The subsequent distortion of Jesus’ teachings, as they are recorded in the New Testament, is because the concept of the gospel writers was coloured by the belief that Jesus was then absent from the world for only a short time; that he would soon return to establish the kingdom in power and glory --- just such an idea as they held while he was with them in the flesh. But Jesus did not connect the establishment of the kingdom with the idea of his return to this world. That centuries have passed with no signs of the appearance of the “New Age” is in no way out of harmony with Jesus’ teaching.
\vs p170 2:11 The great effort embodied in this sermon was the attempt to translate the concept of the kingdom of heaven into the ideal of the idea of doing the will of God. Long had the Master taught his followers to pray: \textcolour{ubdarkred}{“Your kingdom come; your will be done”;} and at this time he earnestly sought to induce them to abandon the use of the term \bibemph{kingdom of God} in favour of the more practical equivalent, \bibemph{the will of God.} But he did not succeed.
\vs p170 2:12 Jesus desired to substitute for the idea of the kingdom, king, and subjects, the concept of the heavenly family, the heavenly Father, and the liberated sons of God engaged in joyful and voluntary service for their fellow men and in the sublime and intelligent worship of God the Father.
\vs p170 2:13 Up to this time the apostles had acquired a double viewpoint of the kingdom; they regarded it as:
\vs p170 2:14 \ublistelem{1.}\bibnobreakspace A matter of personal experience then present in the hearts of true believers, and
\vs p170 2:15 \ublistelem{2.}\bibnobreakspace A question of racial or world phenomena; that the kingdom was in the future, something to look forward to.
\vs p170 2:16 \pc They looked upon the coming of the kingdom in the hearts of men as a gradual development, like the leaven in the dough or like the growing of the mustard seed. They believed that the coming of the kingdom in the racial or world sense would be both sudden and spectacular. Jesus never tired of telling them that the kingdom of heaven was their personal experience of realizing the higher qualities of spiritual living; that these realities of the spirit experience are progressively translated to new and higher levels of divine certainty and eternal grandeur.
\vs p170 2:17 On this afternoon the Master distinctly taught a new concept of the double nature of the kingdom in that he portrayed the following two phases:
\vs p170 2:18 \textcolour{ubdarkred}{“First. The kingdom of God in this world, the supreme desire to do the will of God, the unselfish love of man which yields the good fruits of improved ethical and moral conduct.}
\vs p170 2:19 \textcolour{ubdarkred}{“Second. The kingdom of God in heaven, the goal of mortal believers, the estate wherein the love for God is perfected, and wherein the will of God is done more divinely.”}
\vs p170 2:20 Jesus taught that, by faith, the believer enters the kingdom \bibemph{now.} In the various discourses he taught that two things are essential to faith\hyp{}entrance into the kingdom:
\vs p170 2:21 \ublistelem{1.}\bibnobreakspace \bibemph{Faith, sincerity.} To come as a little child, to receive the bestowal of sonship as a gift; to submit to the doing of the Father’s will without questioning and in the full confidence and genuine trustfulness of the Father’s wisdom; to come into the kingdom free from prejudice and preconception; to be open\hyp{}minded and teachable like an unspoiled child.
\vs p170 2:22 \ublistelem{2.}\bibnobreakspace \bibemph{Truth hunger.} The thirst for righteousness, a change of mind, the acquirement of the motive to be like God and to find God.
\vs p170 2:23 Jesus taught that sin is not the child of a defective nature but rather the offspring of a knowing mind dominated by an unsubmissive will. Regarding sin, he taught that God \bibemph{has} forgiven; that we make such forgiveness personally available by the act of forgiving our fellows. When you forgive your brother in the flesh, you thereby create the capacity in your own soul for the reception of the reality of God’s forgiveness of your own misdeeds.
\vs p170 2:24 By the time the Apostle John began to write the story of Jesus’ life and teachings, the early Christians had experienced so much trouble with the kingdom\hyp{}of\hyp{}God idea as a breeder of persecution that they had largely abandoned the use of the term. John talks much about the “eternal life.” Jesus often spoke of it as the \textcolour{ubdarkred}{“kingdom of life.”} He also frequently referred to \textcolour{ubdarkred}{“the kingdom of God within you.”} He once spoke of such an experience as \textcolour{ubdarkred}{“family fellowship with God the Father.”} Jesus sought to substitute many terms for the kingdom but always without success. Among others, he used: the family of God, the Father’s will, the friends of God, the fellowship of believers, the brotherhood of man, the Father’s fold, the children of God, the fellowship of the faithful, the Father’s service, and the liberated sons of God.
\vs p170 2:25 But he could not escape the use of the kingdom idea. It was more than 50 years later, not until after the destruction of Jerusalem by the Roman armies, that this concept of the kingdom began to change into the cult of eternal life as its social and institutional aspects were taken over by the rapidly expanding and crystallizing Christian church.
\usection{In Relation to Righteousness}
\vs p170 3:1 Jesus was always trying to impress upon his apostles and disciples that they must acquire, by faith, a righteousness which would exceed the righteousness of slavish works which some of the scribes and Pharisees paraded so vaingloriously before the world.
\vs p170 3:2 Though Jesus taught that faith, simple childlike belief, is the key to the door of the kingdom, he also taught that, having entered the door, there are the progressive steps of righteousness which every believing child must ascend in order to grow up to the full stature of the robust sons of God.
\vs p170 3:3 It is in the consideration of the technique of \bibemph{receiving} God’s forgiveness that the attainment of the righteousness of the kingdom is revealed. Faith is the price you pay for entrance into the family of God; but forgiveness is the act of God which accepts your faith as the price of admission. And the reception of the forgiveness of God by a kingdom believer involves a definite and actual experience and consists in the following four steps, the kingdom steps of inner righteousness:
\vs p170 3:4 \ublistelem{1.}\bibnobreakspace God’s forgiveness is made actually available and is personally experienced by man just in so far as he forgives his fellows.
\vs p170 3:5 \ublistelem{2.}\bibnobreakspace Man will not truly forgive his fellows unless he loves them as himself.
\vs p170 3:6 \ublistelem{3.}\bibnobreakspace To thus love your neighbour as yourself \bibemph{is} the highest ethics.
\vs p170 3:7 \ublistelem{4.}\bibnobreakspace Moral conduct, true righteousness, becomes, then, the natural result of such love.
\vs p170 3:8 \pc It therefore is evident that the true and inner religion of the kingdom unfailingly and increasingly tends to manifest itself in practical avenues of social service. Jesus taught a living religion that impelled its believers to engage in the doing of loving service. But Jesus did not put ethics in the place of religion. He taught religion as a cause and ethics as a result.
\vs p170 3:9 The righteousness of any act must be measured by the motive; the highest forms of good are therefore unconscious. Jesus was never concerned with morals or ethics as such. He was wholly concerned with that inward and spiritual fellowship with God the Father which so certainly and directly manifests itself as outward and loving service for man. He taught that the religion of the kingdom is a genuine personal experience which no man can contain within himself; that the consciousness of being a member of the family of believers leads inevitably to the practice of the precepts of the family conduct, the service of one’s brothers and sisters in the effort to enhance and enlarge the brotherhood.
\vs p170 3:10 The religion of the kingdom is personal, individual; the fruits, the results, are familial, social. Jesus never failed to exalt the sacredness of the individual as contrasted with the community. But he also recognized that man develops his character by unselfish service; that he unfolds his moral nature in loving relations with his fellows.
\vs p170 3:11 By teaching that the kingdom is within, by exalting the individual, Jesus struck the deathblow of the old society in that he ushered in the new dispensation of true social righteousness. This new order of society the world has little known because it has refused to practise the principles of the gospel of the kingdom of heaven. And when this kingdom of spiritual pre\hyp{}eminence does come upon the earth, it will not be manifested in mere improved social and material conditions, but rather in the glories of those enhanced and enriched spiritual values which are characteristic of the approaching age of improved human relations and advancing spiritual attainments.
\usection{Jesus’ Teaching about the Kingdom}
\vs p170 4:1 Jesus never gave a precise definition of the kingdom. At one time he would discourse on one phase of the kingdom, and at another time he would discuss a different aspect of the brotherhood of God’s reign in the hearts of men. In the course of this Sabbath afternoon’s sermon Jesus noted no less than five phases, or epochs, of the kingdom, and they were:
\vs p170 4:2 \ublistelem{1.}\bibnobreakspace The personal and inward experience of the spiritual life of the fellowship of the individual believer with God the Father.
\vs p170 4:3 \ublistelem{2.}\bibnobreakspace The enlarging brotherhood of gospel believers, the social aspects of the enhanced morals and quickened ethics resulting from the reign of God’s spirit in the hearts of individual believers.
\vs p170 4:4 \ublistelem{3.}\bibnobreakspace The supermortal brotherhood of invisible spiritual beings which prevails on earth and in heaven, the superhuman kingdom of God.
\vs p170 4:5 \ublistelem{4.}\bibnobreakspace The prospect of the more perfect fulfilment of the will of God, the advance toward the dawn of a new social order in connection with improved spiritual living --- the next age of man.
\vs p170 4:6 \ublistelem{5.}\bibnobreakspace The kingdom in its fullness, the future spiritual age of light and life on earth.
\vs p170 4:7 \pc Wherefore must we always examine the Master’s teaching to ascertain which of these five phases he may have reference to when he makes use of the term kingdom of heaven. By this process of gradually changing man’s will and thus affecting human decisions, Michael and his associates are likewise gradually but certainly changing the entire course of human evolution, social and otherwise.
\vs p170 4:8 The Master on this occasion placed emphasis on the following five points as representing the cardinal features of the gospel of the kingdom:
\vs p170 4:9 \ublistelem{1.}\bibnobreakspace The pre\hyp{}eminence of the individual.
\vs p170 4:10 \ublistelem{2.}\bibnobreakspace The will as the determining factor in man’s experience.
\vs p170 4:11 \ublistelem{3.}\bibnobreakspace Spiritual fellowship with God the Father.
\vs p170 4:12 \ublistelem{4.}\bibnobreakspace The supreme satisfactions of the loving service of man.
\vs p170 4:13 \ublistelem{5.}\bibnobreakspace The transcendency of the spiritual over the material in human personality.
\vs p170 4:14 \pc This world has never seriously or sincerely or honestly tried out these dynamic ideas and divine ideals of Jesus’ doctrine of the kingdom of heaven. But you should not become discouraged by the apparently slow progress of the kingdom idea on Urantia. Remember that the order of progressive evolution is subjected to sudden and unexpected periodical changes in both the material and the spiritual worlds. The bestowal of Jesus as an incarnated Son was just such a strange and unexpected event in the spiritual life of the world. Neither make the fatal mistake, in looking for the age manifestation of the kingdom, of failing to effect its establishment within your own souls.
\vs p170 4:15 Although Jesus referred one phase of the kingdom to the future and did, on numerous occasions, intimate that such an event might appear as a part of a world crisis; and though he did likewise most certainly, on several occasions, definitely promise sometime to return to Urantia, it should be recorded that he never positively linked these two ideas together. He promised a new revelation of the kingdom on earth and at some future time; he also promised sometime to come back to this world in person; but he did not say that these two events were synonymous. From all we know these promises may, or may not, refer to the same event.
\vs p170 4:16 His apostles and disciples most certainly linked these two teachings together. When the kingdom failed to materialize as they had expected, recalling the Master’s teaching concerning a future kingdom and remembering his promise to come again, they jumped to the conclusion that these promises referred to an identical event; and therefore they lived in hope of his immediate second coming to establish the kingdom in its fullness and with power and glory. And so have successive believing generations lived on earth entertaining the same inspiring but disappointing hope.
\usection{Later Ideas of the Kingdom}
\vs p170 5:1 Having summarized the teachings of Jesus about the kingdom of heaven, we are permitted to narrate certain later ideas which became attached to the concept of the kingdom and to engage in a prophetic forecast of the kingdom as it may evolve in the age to come.
\vs p170 5:2 Throughout the first centuries of the Christian propaganda, the idea of the kingdom of heaven was tremendously influenced by the then rapidly spreading notions of Greek idealism, the idea of the natural as the shadow of the spiritual --- the temporal as the time shadow of the eternal.
\vs p170 5:3 But the great step which marked the transplantation of the teachings of Jesus from a Jewish to a gentile soil was taken when the Messiah of the kingdom became the Redeemer of the church, a religious and social organization growing out of the activities of Paul and his successors and based on the teachings of Jesus as they were supplemented by the ideas of Philo and the Persian doctrines of good and evil.
\vs p170 5:4 The ideas and ideals of Jesus, embodied in the teaching of the gospel of the kingdom, nearly failed of realization as his followers progressively distorted his pronouncements. The Master’s concept of the kingdom was notably modified by two great tendencies:
\vs p170 5:5 \ublistelem{1.}\bibnobreakspace The Jewish believers persisted in regarding him as the \bibemph{Messiah.} They believed that Jesus would very soon return actually to establish the world\hyp{}wide and more or less material kingdom.
\vs p170 5:6 \ublistelem{2.}\bibnobreakspace The gentile Christians began very early to accept the doctrines of Paul, which led increasingly to the general belief that Jesus was the \bibemph{Redeemer} of the children of the church, the new and institutional successor of the earlier concept of the purely spiritual brotherhood of the kingdom.
\vs p170 5:7 \pc The church, as a social outgrowth of the kingdom, would have been wholly natural and even desirable. The evil of the church was not its existence, but rather that it almost completely supplanted the Jesus concept of the kingdom. Paul’s institutionalized church became a virtual substitute for the kingdom of heaven which Jesus had proclaimed.
\vs p170 5:8 But doubt not, this same kingdom of heaven which the Master taught exists within the heart of the believer, will yet be proclaimed to this Christian church, even as to all other religions, races, and nations on earth --- even to every individual.
\vs p170 5:9 The kingdom of Jesus’ teaching, the spiritual ideal of individual righteousness and the concept of man’s divine fellowship with God, became gradually submerged into the mystic conception of the person of Jesus as the Redeemer\hyp{}Creator and spiritual head of a socialized religious community. In this way a formal and institutional church became the substitute for the individually spirit\hyp{}led brotherhood of the kingdom.
\vs p170 5:10 The church was an inevitable and useful \bibemph{social} result of Jesus’ life and teachings; the tragedy consisted in the fact that this social reaction to the teachings of the kingdom so fully displaced the spiritual concept of the real kingdom as Jesus taught and lived it.
\vs p170 5:11 The kingdom, to the Jews, was the Israelite \bibemph{community;} to the gentiles it became the Christian \bibemph{church.} To Jesus the kingdom was the sum of those \bibemph{individuals} who had confessed their faith in the fatherhood of God, thereby declaring their wholehearted dedication to the doing of the will of God, thus becoming members of the spiritual brotherhood of man.
\vs p170 5:12 The Master fully realized that certain social results would appear in the world as a consequence of the spread of the gospel of the kingdom; but he intended that all such desirable social manifestations should appear as unconscious and inevitable outgrowths, or natural fruits, of this inner personal experience of individual believers, this purely spiritual fellowship and communion with the divine spirit which indwells and activates all such believers.
\vs p170 5:13 Jesus foresaw that a social organization, or church, would follow the progress of the true spiritual kingdom, and that is why he never opposed the apostles’ practising the rite of John’s baptism. He taught that the truth\hyp{}loving soul, the one who hungers and thirsts for righteousness, for God, is admitted by faith to the spiritual kingdom; at the same time the apostles taught that such a believer is admitted to the social organization of disciples by the outward rite of baptism.
\vs p170 5:14 When Jesus’ immediate followers recognized their partial failure to realize his ideal of the establishment of the kingdom in the hearts of men by the spirit’s domination and guidance of the individual believer, they set about to save his teaching from being wholly lost by substituting for the Master’s ideal of the kingdom the gradual creation of a visible social organization, the Christian church. And when they had accomplished this program of substitution, in order to maintain consistency and to provide for the recognition of the Master’s teaching regarding the fact of the kingdom, they proceeded to set the kingdom off into the future. The church, just as soon as it was well established, began to teach that the kingdom was in reality to appear at the culmination of the Christian age, at the second coming of Christ.
\vs p170 5:15 In this manner the kingdom became the concept of an age, the idea of a future visitation, and the ideal of the final redemption of the saints of the Most High. The early Christians (and all too many of the later ones) generally lost sight of the Father\hyp{}and\hyp{}son idea embodied in Jesus’ teaching of the kingdom, while they substituted therefor the well\hyp{}organized social fellowship of the church. The church thus became in the main a \bibemph{social} brotherhood which effectively displaced Jesus’ concept and ideal of a \bibemph{spiritual} brotherhood.
\vs p170 5:16 Jesus’ ideal concept largely failed, but upon the foundation of the Master’s personal life and teachings, supplemented by the Greek and Persian concepts of eternal life and augmented by Philo’s doctrine of the temporal contrasted with the spiritual, Paul went forth to build up one of the most progressive human societies which has ever existed on Urantia.
\vs p170 5:17 The concept of Jesus is still alive in the advanced religions of the world. Paul’s Christian church is the socialized and humanized shadow of what Jesus intended the kingdom of heaven to be --- and what it most certainly will yet become. Paul and his successors partly transferred the issues of eternal life from the individual to the church. Christ thus became the head of the church rather than the elder brother of each individual believer in the Father’s family of the kingdom. Paul and his contemporaries applied all of Jesus’ spiritual implications regarding himself and the individual believer to the \bibemph{church} as a group of believers; and in doing this, they struck a deathblow to Jesus’ concept of the divine kingdom in the heart of the individual believer.
\vs p170 5:18 And so, for centuries, the Christian church has laboured under great embarrassment because it dared to lay claim to those mysterious powers and privileges of the kingdom, powers and privileges which can be exercised and experienced only between Jesus and his spiritual believer brothers. And thus it becomes apparent that membership in the church does not necessarily mean fellowship in the kingdom; one is spiritual, the other mainly social.
\vs p170 5:19 Sooner or later another and greater John the Baptist is due to arise proclaiming “the kingdom of God is at hand” --- meaning a return to the high spiritual concept of Jesus, who proclaimed that the kingdom is the will of his heavenly Father dominant and transcendent in the heart of the believer --- and doing all this without in any way referring either to the visible church on earth or to the anticipated second coming of Christ. There must come a revival of the \bibemph{actual} teachings of Jesus, such a restatement as will undo the work of his early followers who went about to create a sociophilosophical system of belief regarding the \bibemph{fact} of Michael’s sojourn on earth. In a short time the teaching of this story \bibemph{about} Jesus nearly supplanted the preaching of Jesus’ gospel of the kingdom. In this way a historical religion displaced that teaching in which Jesus had blended man’s highest moral ideas and spiritual ideals with man’s most sublime hope for the future --- eternal life. And that was the gospel of the kingdom.
\vs p170 5:20 It is just because the gospel of Jesus was so many\hyp{}sided that within a few centuries students of the records of his teachings became divided up into so many cults and sects. This pitiful subdivision of Christian believers results from failure to discern in the Master’s manifold teachings the divine oneness of his matchless life. But someday the true believers in Jesus will not be thus spiritually divided in their attitude before unbelievers. Always we may have diversity of intellectual comprehension and interpretation, even varying degrees of socialization, but lack of spiritual brotherhood is both inexcusable and reprehensible.
\vs p170 5:21 Mistake not! there is in the teachings of Jesus an eternal nature which will not permit them forever to remain unfruitful in the hearts of thinking men. The kingdom as Jesus conceived it has to a large extent failed on earth; for the time being, an outward church has taken its place; but you should comprehend that this church is only the larval stage of the thwarted spiritual kingdom, which will carry it through this material age and over into a more spiritual dispensation where the Master’s teachings may enjoy a fuller opportunity for development. Thus does the so\hyp{}called Christian church become the cocoon in which the kingdom of Jesus’ concept now slumbers. The kingdom of the divine brotherhood is still alive and will eventually and certainly come forth from this long submergence, just as surely as the butterfly eventually emerges as the beautiful unfolding of its less attractive creature of metamorphic development.
\quizlink
