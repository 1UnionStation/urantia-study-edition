\upaper{165}{The Perean Mission Begins}
\uminitoc{At the Pella Camp}
\uminitoc{Sermon on the Good Shepherd}
\uminitoc{Sabbath Sermon at Pella}
\uminitoc{Dividing the Inheritance}
\uminitoc{Talks to the Apostles on Wealth}
\uminitoc{Answer to Peter’s Question}
\author{Midwayer Commission}
\vs p165 0:1 On Tuesday, January 3, A.D.\,30, Abner, the former chief of the twelve apostles of John the Baptist, a Nazarite and onetime head of the Nazarite school at Engedi, now chief of the seventy messengers of the kingdom, called his associates together and gave them final instructions before sending them on a mission to all of the cities and villages of Perea. This Perean mission continued for almost three months and was the last ministry of the Master. From these labours Jesus went directly to Jerusalem to pass through his final experiences in the flesh. The seventy, supplemented by the periodic labours of Jesus and the twelve apostles, worked in the following cities and towns and some 50 additional villages: Zaphon, Gadara, Macad, Arbela, Ramath, Edrei, Bosora, Caspin, Mispeh, Gerasa, Ragaba, Succoth, Amathus, Adam, Penuel, Capitolias, Dion, Hatita, Gadda, Philadelphia, Jogbehah, Gilead, Beth\hyp{}Nimrah, Tyrus, Elealah, Livias, Heshbon, Callirrhoe, Beth\hyp{}Peor, Shittim, Sibmah, Medeba, Beth\hyp{}Meon, Areopolis, and Aroer.
\vs p165 0:2 Throughout this tour of Perea the women’s corps, now numbering 62, took over most of the work of ministration to the sick. This was the final period of the development of the higher spiritual aspects of the gospel of the kingdom, and there was, accordingly, an absence of miracle working. No other part of Palestine was so thoroughly worked by the apostles and disciples of Jesus, and in no other region did the better classes of citizens so generally accept the Master’s teaching.
\vs p165 0:3 Perea at this time was about equally gentile and Jewish, the Jews having been generally removed from these regions during the times of Judas Maccabee. Perea was the most beautiful and picturesque province of all Palestine. It was generally referred to by the Jews as “the land beyond the Jordan.”\fnc{\ldots{}from these regions during the times of Judas \bibtextul{Maccabeus.} \bibexpl{Although Maccabeus is a more accurate transliteration of the Greek, Maccabee is very common in English works and is used in all other occurrences of the word in the Urantia papers. Therefore, the committee decided to standardize on “Maccabee.”}}
\vs p165 0:4 Throughout this period Jesus divided his time between the camp at Pella and trips with the twelve to assist the seventy in the various cities where they taught and preached. Under Abner’s instructions the seventy baptized all believers, although Jesus had not so charged them.
\usection{At the Pella Camp}
\vs p165 1:1 By the middle of January more than 1200 persons were gathered together at Pella, and Jesus taught this multitude at least once each day when he was in residence at the camp, usually speaking at 9:00 if not prevented by rain. Peter and the other apostles taught each afternoon. The evenings Jesus reserved for the usual sessions of questions and answers with the twelve and other advanced disciples. The evening groups averaged about 50.
\vs p165 1:2 By the middle of March, the time when Jesus began his journey toward Jerusalem, over 4,000 persons composed the large audience which heard Jesus or Peter preach each morning. The Master chose to terminate his work on earth when the interest in his message had reached a high point, the highest point attained under this second or nonmiraculous phase of the progress of the kingdom. While \bibfrac{3}{4}\ts{ers} of the multitude were truth seekers, there were also present a large number of Pharisees from Jerusalem and elsewhere, together with many doubters and cavillers.
\vs p165 1:3 Jesus and the twelve apostles devoted much of their time to the multitude assembled at the Pella camp. The twelve paid little or no attention to the field work, only going out with Jesus to visit Abner’s associates from time to time. Abner was very familiar with the Perean district since this was the field in which his former master, John the Baptist, had done most of his work. After beginning the Perean mission, Abner and the seventy never returned to the Pella camp.
\usection{Sermon on the Good Shepherd}
\vs p165 2:1 A company of over 300 Jerusalemites, Pharisees and others, followed Jesus north to Pella when he hastened away from the jurisdiction of the Jewish rulers at the ending of the feast of the dedication; and it was in the presence of these Jewish teachers and leaders, as well as in the hearing of the twelve apostles, that Jesus preached the sermon on the “Good Shepherd.” After half an hour of informal discussion, speaking to a group of about 100, Jesus said:
\vs p165 2:2 \pc \textcolour{ubdarkred}{“On this night I have much to tell you, and since many of you are my disciples and some of you my bitter enemies, I will present my teaching in a parable, so that you may each take for yourself that which finds a reception in your heart.}
\vs p165 2:3 \textcolour{ubdarkred}{“Tonight, here before me are men who would be willing to die for me and for this gospel of the kingdom, and some of them will so offer themselves in the years to come; and here also are some of you, slaves of tradition, who have followed me down from Jerusalem, and who, with your darkened and deluded leaders, seek to kill the Son of Man. The life which I now live in the flesh shall judge both of you, the true shepherds and the false shepherds. If the false shepherd were blind, he would have no sin, but you claim that you see; you profess to be teachers in Israel; therefore does your sin remain upon you.}
\vs p165 2:4 \textcolour{ubdarkred}{“The true shepherd gathers his flock into the fold for the night in times of danger. And when the morning has come, he enters into the fold by the door, and when he calls, the sheep know his voice. Every shepherd who gains entrance to the sheepfold by any other means than by the door is a thief and a robber. The true shepherd enters the fold after the porter has opened the door for him, and his sheep, knowing his voice, come out at his word; and when they that are his are thus brought forth, the true shepherd goes before them; he leads the way and the sheep follow him. His sheep follow him because they know his voice; they will not follow a stranger. They will flee from the stranger because they know not his voice. This multitude which is gathered about us here are like sheep without a shepherd, but when we speak to them, they know the shepherd’s voice, and they follow after us; at least, those who hunger for truth and thirst for righteousness do. Some of you are not of my fold; you know not my voice, and you do not follow me. And because you are false shepherds, the sheep know not your voice and will not follow you.”}
\vs p165 2:5 \pc And when Jesus had spoken this parable, no one asked him a question. After a time he began again to speak and went on to discuss the parable:
\vs p165 2:6 \textcolour{ubdarkred}{“You who would be the undershepherds of my Father’s flocks must not only be worthy leaders, but you must also \bibemph{feed} the flock with good food; you are not true shepherds unless you lead your flocks into green pastures and beside still waters.}
\vs p165 2:7 \textcolour{ubdarkred}{“And now, lest some of you too easily comprehend this parable, I will declare that I am both the door to the Father’s sheepfold and at the same time the true shepherd of my Father’s flocks. Every shepherd who seeks to enter the fold without me shall fail, and the sheep will not hear his voice. I, with those who minister with me, am the door. Every soul who enters upon the eternal way by the means I have created and ordained shall be saved and will be able to go on to the attainment of the eternal pastures of Paradise.}
\vs p165 2:8 \textcolour{ubdarkred}{“But I also am the true shepherd who is willing even to lay down his life for the sheep. The thief breaks into the fold only to steal, and to kill, and to destroy; but I have come that you all may have life and have it more abundantly. He who is a hireling, when danger arises, will flee and allow the sheep to be scattered and destroyed; but the true shepherd will not flee when the wolf comes; he will protect his flock and, if necessary, lay down his life for his sheep. Verily, verily, I say to you, friends and enemies, I am the true shepherd; I know my own and my own know me. I will not flee in the face of danger. I will finish this service of the completion of my Father’s will, and I will not forsake the flock which the Father has entrusted to my keeping.}
\vs p165 2:9 \textcolour{ubdarkred}{“But I have many other sheep not of this fold, and these words are true not only of this world. These other sheep also hear and know my voice, and I have promised the Father that they shall all be brought into one fold, one brotherhood of the sons of God. And then shall you all know the voice of one shepherd, the true shepherd, and shall all acknowledge the fatherhood of God.}
\vs p165 2:10 \textcolour{ubdarkred}{“And so shall you know why the Father loves me and has put all of his flocks in this domain in my hands for keeping; it is because the Father knows that I will not falter in the safeguarding of the sheepfold, that I will not desert my sheep, and that, if it shall be required, I will not hesitate to lay down my life in the service of his manifold flocks. But, mind you, if I lay down my life, I will take it up again. No man nor any other creature can take away my life. I have the right and the power to lay down my life, and I have the same power and right to take it up again. You cannot understand this, but I received such authority from my Father even before this world was.”}
\vs p165 2:11 \pc When they heard these words, his apostles were confused, his disciples were amazed, while the Pharisees from Jerusalem and around about went out into the night, saying, “He is either mad or has a devil.” But even some of the Jerusalem teachers said: “He speaks like one having authority; besides, who ever saw one having a devil open the eyes of a man born blind and do all of the wonderful things which this man has done?”
\vs p165 2:12 On the morrow about half of these Jewish teachers professed belief in Jesus, and the other half in dismay returned to Jerusalem and their homes.
\usection{Sabbath Sermon at Pella}
\vs p165 3:1 By the end of January the Sabbath\hyp{}afternoon multitudes numbered almost 3,000. On Saturday, January 28, Jesus preached the memorable sermon on “Trust and Spiritual Preparedness.” After preliminary remarks by Simon Peter, the Master said:
\vs p165 3:2 \pc \textcolour{ubdarkred}{“What I have many times said to my apostles and to my disciples, I now declare to this multitude: Beware of the leaven of the Pharisees which is hypocrisy, born of prejudice and nurtured in traditional bondage, albeit many of these Pharisees are honest of heart and some of them abide here as my disciples. Presently all of you shall understand my teaching, for there is nothing now covered that shall not be revealed. That which is now hid from you shall all be made known when the Son of Man has completed his mission on earth and in the flesh.}
\vs p165 3:3 \textcolour{ubdarkred}{“Soon, very soon, will the things which our enemies now plan in secrecy and in darkness be brought out into the light and be proclaimed from the housetops. But I say to you, my friends, when they seek to destroy the Son of Man, be not afraid of them. Fear not those who, although they may be able to kill the body, after that have no more power over you. I admonish you to fear none, in heaven or on earth, but to rejoice in the knowledge of Him who has power to deliver you from all unrighteousness and to present you blameless before the judgment seat of a universe.}
\vs p165 3:4 \textcolour{ubdarkred}{“Are not five sparrows sold for two pennies? And yet, when these birds flit about in quest of their sustenance, not one of them exists without the knowledge of the Father, the source of all life. To the seraphic guardians the very hairs of your head are numbered. And if all of this is true, why should you live in fear of the many trifles which come up in your daily lives? I say to you: Fear not; you are of much more value than many sparrows.}
\vs p165 3:5 \textcolour{ubdarkred}{“All of you who have had the courage to confess faith in my gospel before men I will presently acknowledge before the angels of heaven; but he who shall knowingly deny the truth of my teachings before men shall be denied by his guardian of destiny even before the angels of heaven.}
\vs p165 3:6 \textcolour{ubdarkred}{“Say what you will about the Son of Man, and it shall be forgiven you; but he who presumes to blaspheme against God shall hardly find forgiveness. When men go so far as knowingly to ascribe the doings of God to the forces of evil, such deliberate rebels will hardly seek forgiveness for their sins.}
\vs p165 3:7 \textcolour{ubdarkred}{“And when our enemies bring you before the rulers of the synagogues and before other high authorities, be not concerned about what you should say and be not anxious as to how you should answer their questions, for the spirit that dwells within you shall certainly teach you in that very hour what you should say in honour of the gospel of the kingdom.}
\vs p165 3:8 \textcolour{ubdarkred}{“How long will you tarry in the valley of decision? Why do you halt between two opinions? Why should Jew or gentile hesitate to accept the good news that he is a son of the eternal God? How long will it take us to persuade you to enter joyfully into your spiritual inheritance? I came into this world to reveal the Father to you and to lead you to the Father. The first I have done, but the last I may not do without your consent; the Father never compels any man to enter the kingdom. The invitation ever has been and always will be: Whosoever will, let him come and freely partake of the water of life.”}
\vs p165 3:9 \pc When Jesus had finished speaking, many went forth to be baptized by the apostles in the Jordan while he listened to the questions of those who remained.
\usection{Dividing the Inheritance}
\vs p165 4:1 As the apostles baptized believers, the Master talked with those who tarried. And a certain young man said to him: “Master, my father died leaving much property to me and my brother, but my brother refuses to give me that which is my own. Will you, then, bid my brother divide this inheritance with me?” Jesus was mildly indignant that this material\hyp{}minded youth should bring up for discussion such a question of business; but he proceeded to use the occasion for the impartation of further instruction. Said Jesus: \textcolour{ubdarkred}{“Man, who made me a divider over you? Where did you get the idea that I give attention to the material affairs of this world?”} And then, turning to all who were about him, he said: \textcolour{ubdarkred}{“Take heed and keep yourselves free from covetousness; a man’s life consists not in the abundance of the things which he may possess. Happiness comes not from the power of wealth, and joy springs not from riches. Wealth, in itself, is not a curse, but the love of riches many times leads to such devotion to the things of this world that the soul becomes blinded to the beautiful attractions of the spiritual realities of the kingdom of God on earth and to the joys of eternal life in heaven.}
\vs p165 4:2 \pc \textcolour{ubdarkred}{“Let me tell you a story of a certain rich man whose ground brought forth plentifully; and when he had become very rich, he began to reason with himself, saying: ‘What shall I do with all my riches? I now have so much that I have no place to store my wealth.’ And when he had meditated on his problem, he said: ‘This I will do; I will pull down my barns and build greater ones, and thus will I have abundant room in which to store my fruits and my goods. Then can I say to my soul, soul, you have much wealth laid up for many years; take now your ease; eat, drink, and be merry, for you are rich and increased in goods.’}
\vs p165 4:3 \textcolour{ubdarkred}{“But this rich man was also foolish. In providing for the material requirements of his mind and body, he had failed to lay up treasures in heaven for the satisfaction of the spirit and for the salvation of the soul. And even then he was not to enjoy the pleasure of consuming his hoarded wealth, for that very night was his soul required of him. That night there came the brigands who broke into his house to kill him, and after they had plundered his barns, they burned that which remained. And for the property which escaped the robbers his heirs fell to fighting among themselves. This man laid up treasures for himself on earth, but he was not rich toward God.”}
\vs p165 4:4 \pc Jesus thus dealt with the young man and his inheritance because he knew that his trouble was covetousness. Even if this had not been the case, the Master would not have interfered, for he never meddled with the temporal affairs of even his apostles, much less his disciples.
\vs p165 4:5 When Jesus had finished his story, another man rose up and asked him: “Master, I know that your apostles have sold all their earthly possessions to follow you, and that they have all things in common as do the Essenes, but would you have all of us who are your disciples do likewise? Is it a sin to possess honest wealth?” And Jesus replied to this question: \textcolour{ubdarkred}{“My friend, it is not a sin to have honourable wealth; but it is a sin if you convert the wealth of material possessions into \bibemph{treasures} which may absorb your interests and divert your affections from devotion to the spiritual pursuits of the kingdom. There is no sin in having honest possessions on earth provided your \bibemph{treasure} is in heaven, for where your treasure is there will your heart be also. There is a great difference between wealth which leads to covetousness and selfishness and that which is held and dispensed in the spirit of stewardship by those who have an abundance of this world’s goods, and who so bountifully contribute to the support of those who devote all their energies to the work of the kingdom. Many of you who are here and without money are fed and lodged in yonder tented city because liberal men and women of means have given funds to your host, David Zebedee, for such purposes.}
\vs p165 4:6 \textcolour{ubdarkred}{“But never forget that, after all, wealth is unenduring. The love of riches all too often obscures and even destroys the spiritual vision. Fail not to recognize the danger of wealth’s becoming, not your servant, but your master.”}
\vs p165 4:7 \pc Jesus did not teach nor countenance improvidence, idleness, indifference to providing the physical necessities for one’s family, or dependence upon alms. But he did teach that the material and temporal must be subordinated to the welfare of the soul and the progress of the spiritual nature in the kingdom of heaven.
\vs p165 4:8 \pc Then, as the people went down by the river to witness the baptizing, the first man came privately to Jesus about his inheritance inasmuch as he thought Jesus had dealt harshly with him; and when the Master had again heard him, he replied: “My son, why do you miss the opportunity to feed upon the bread of life on a day like this in order to indulge your covetous disposition? Do you not know that the Jewish laws of inheritance will be justly administered if you will go with your complaint to the court of the synagogue? Can you not see that my work has to do with making sure that you know about your heavenly inheritance? Have you not read the Scripture: ‘There is he who waxes rich by his wariness and much pinching, and this is the portion of his reward: Whereas he says, I have found rest and now shall be able to eat continually of my goods, yet he knows not what time shall bring upon him, and also that he must leave all these things to others when he dies.’ Have you not read the commandment: ‘You shall not covet.’ And again, ‘They have eaten and filled themselves and waxed fat, and then did they turn to other gods.’ Have you read in the Psalms that ‘the Lord abhors the covetous,’ and that ‘the little a righteous man has is better than the riches of many wicked.’ ‘If riches increase, set not your heart upon them.’ Have you read where Jeremiah said, ‘Let not the rich man glory in his riches’; and Ezekiel spoke truth when he said, ‘With their mouths they make a show of love, but their hearts are set upon their own selfish gain.’”
\vs p165 4:9 Jesus sent the young man away, saying to him, \textcolour{ubdarkred}{“My son, what shall it profit you if you gain the whole world and lose your own soul?”}
\vs p165 4:10 To another standing near by who asked Jesus how the wealthy would stand in the day of judgment, he replied: \textcolour{ubdarkred}{“I have come to judge neither the rich nor the poor, but the lives men live will sit in judgment on all. Whatever else may concern the wealthy in the judgment, at least three questions must be answered by all who acquire great wealth, and these questions are:}
\vs p165 4:11 \textcolour{ubdarkred}{“\ublistelem{1.}\bibnobreakspace How much wealth did you accumulate?}
\vs p165 4:12 \textcolour{ubdarkred}{“\ublistelem{2.}\bibnobreakspace How did you get this wealth?}
\vs p165 4:13 \textcolour{ubdarkred}{“\ublistelem{3.}\bibnobreakspace How did you use your wealth?”}
\vs p165 4:14 \pc Then Jesus went into his tent to rest for a while before the evening meal. When the apostles had finished with the baptizing, they came also and would have talked with him about wealth on earth and treasure in heaven, but he was asleep.
\usection{Talks to the Apostles on Wealth}
\vs p165 5:1 That evening after supper, when Jesus and the twelve gathered together for their daily conference, Andrew asked: “Master, while we were baptizing the believers, you spoke many words to the lingering multitude which we did not hear. Would you be willing to repeat these words for our benefit?” And in response to Andrew’s request, Jesus said:
\vs p165 5:2 \pc \textcolour{ubdarkred}{“Yes, Andrew, I will speak to you about these matters of wealth and self\hyp{}support, but my words to you, the apostles, must be somewhat different from those spoken to the disciples and the multitude since you have forsaken everything, not only to follow me, but to be ordained as ambassadors of the kingdom. Already have you had several years’ experience, and you know that the Father whose kingdom you proclaim will not forsake you. You have dedicated your lives to the ministry of the kingdom; therefore be not anxious or worried about the things of the temporal life, what you shall eat, nor yet for your body, what you shall wear. The welfare of the soul is more than food and drink; the progress in the spirit is far above the need of raiment. When you are tempted to doubt the sureness of your bread, consider the ravens; they sow not neither reap, they have no storehouses or barns, and yet the Father provides food for every one of them that seeks it. And of how much more value are you than many birds! Besides, all of your anxiety or fretting doubts can do nothing to supply your material needs. Which of you by anxiety can add a handbreadth to your stature or a day to your life? Since such matters are not in your hands, why do you give anxious thought to any of these problems?}
\vs p165 5:3 \textcolour{ubdarkred}{“Consider the lilies, how they grow; they toil not, neither do they spin; yet I say to you, even Solomon in all his glory was not arrayed like one of these. If God so clothes the grass of the field, which is alive today and tomorrow is cut down and cast into the fire, how much more shall he clothe you, the ambassadors of the heavenly kingdom. O you of little faith! When you wholeheartedly devote yourselves to the proclamation of the gospel of the kingdom, you should not be of doubtful minds concerning the support of yourselves or the families you have forsaken. If you give your lives truly to the gospel, you shall live by the gospel. If you are only believing disciples, you must earn your own bread and contribute to the sustenance of all who teach and preach and heal. If you are anxious about your bread and water, wherein are you different from the nations of the world who so diligently seek such necessities? Devote yourselves to your work, believing that both the Father and I know that you have need of all these things. Let me assure you, once and for all, that, if you dedicate your lives to the work of the kingdom, all your real needs shall be supplied. Seek the greater thing, and the lesser will be found therein; ask for the heavenly, and the earthly shall be included. The shadow is certain to follow the substance.}
\vs p165 5:4 \textcolour{ubdarkred}{“You are only a small group, but if you have faith, if you will not stumble in fear, I declare that it is my Father’s good pleasure to give you this kingdom. You have laid up your treasures where the purse waxes not old, where no thief can despoil, and where no moth can destroy. And as I told the people, where your treasure is, there will your heart be also.}
\vs p165 5:5 \textcolour{ubdarkred}{“But in the work which is just ahead of us, and in that which remains for you after I go to the Father, you will be grievously tried. You must all be on your watch against fear and doubts. Every one of you, gird up the loins of your minds and let your lamps be kept burning. Keep yourselves like men who are watching for their master to return from the marriage feast so that, when he comes and knocks, you may quickly open to him. Such watchful servants are blessed by the master who finds them faithful at such a great moment. Then will the master make his servants sit down while he himself serves them. Verily, verily, I say to you that a crisis is just ahead in your lives, and it behoves you to watch and be ready.}
\vs p165 5:6 \textcolour{ubdarkred}{“You well understand that no man would suffer his house to be broken into if he knew what hour the thief was to come. Be you also on watch for yourselves, for in an hour that you least suspect and in a manner you think not, shall the Son of Man depart.”}
\vs p165 5:7 \pc For some minutes the twelve sat in silence. Some of these warnings they had heard before but not in the setting presented to them at this time.
\usection{Answer to Peter’s Question}
\vs p165 6:1 As they sat thinking, Simon Peter asked: “Do you speak this parable to us, your apostles, or is it for all the disciples?” And Jesus answered:
\vs p165 6:2 \pc \textcolour{ubdarkred}{“In the time of testing, a man’s soul is revealed; trial discloses what really is in the heart. When the servant is tested and proved, then may the lord of the house set such a servant over his household and safely trust this faithful steward to see that his children are fed and nurtured. Likewise, will I soon know who can be trusted with the welfare of my children when I shall have returned to the Father. As the lord of the household shall set the true and tried servant over the affairs of his family, so will I exalt those who endure the trials of this hour in the affairs of my kingdom.}
\vs p165 6:3 \textcolour{ubdarkred}{“But if the servant is slothful and begins to say in his heart, ‘My master delays his coming,’ and begins to mistreat his fellow servants and to eat and drink with the drunken, then the lord of that servant will come at a time when he looks not for him and, finding him unfaithful, will cast him out in disgrace. Therefore you do well to prepare yourselves for that day when you will be visited suddenly and in an unexpected manner. Remember, much has been given to you; therefore will much be required of you. Fiery trials are drawing near you. I have a baptism to be baptized with, and I am on watch until this is accomplished. You preach peace on earth, but my mission will not bring peace in the material affairs of men --- not for a time, at least. Division can only be the result where two members of a family believe in me and three members reject this gospel. Friends, relatives, and loved ones are destined to be set against each other by the gospel you preach. True, each of these believers shall have great and lasting peace in his own heart, but peace on earth will not come until all are willing to believe and enter into their glorious inheritance of sonship with God. Nevertheless, go into all the world proclaiming this gospel to all nations, to every man, woman, and child.”}
\vs p165 6:4 \pc And this was the end of a full and busy Sabbath day. On the morrow Jesus and the twelve went into the cities of northern Perea to visit with the seventy, who were working in these regions under Abner’s supervision.
\quizlink
