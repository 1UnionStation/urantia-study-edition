\upaper{151}{Tarrying and Teaching by the Seaside}
\uminitoc{The Parable of the Sower}
\uminitoc{Interpretation of the Parable}
\uminitoc{More about Parables}
\uminitoc{More Parables by the Sea}
\uminitoc{The Visit to Kheresa}
\uminitoc{The Kheresa Lunatic}
\author{Midwayer Commission}
\vs p151 0:1 By March 10 all of the preaching and teaching groups had forgathered at Bethsaida. Thursday night and Friday many of them went out to fish, while on the Sabbath day they attended the synagogue to hear an aged Jew of Damascus discourse on the glory of father Abraham. Jesus spent most of this Sabbath day alone in the hills. That Saturday night the Master talked for more than an hour to the assembled groups on “The mission of adversity and the spiritual value of disappointment.” This was a memorable occasion, and his hearers never forgot the lesson he imparted.
\vs p151 0:2 Jesus had not fully recovered from the sorrow of his recent rejection at Nazareth; the apostles were aware of a peculiar sadness mingled with his usual cheerful demeanour. James and John were with him much of the time, Peter being more than occupied with the many responsibilities having to do with the welfare and direction of the new corps of evangelists. This time of waiting before starting for the Passover at Jerusalem, the women spent in visiting from house to house, teaching the gospel, and ministering to the sick in Capernaum and the surrounding cities and villages.
\usection{The Parable of the Sower}
\vs p151 1:1 About this time Jesus first began to employ the parable method of teaching the multitudes that so frequently gathered about him. Since Jesus had talked with the apostles and others long into the night, on this Sunday morning very few of the group were up for breakfast; so he went out by the seaside and sat alone in the boat, the old fishing boat of Andrew and Peter, which was always kept at his disposal, and meditated on the next move to be made in the work of extending the kingdom. But the Master was not to be alone for long. Very soon the people from Capernaum and near\hyp{}by villages began to arrive, and by 10:00 that morning almost 1,000 were assembled on shore near Jesus’ boat and were clamouring for attention. Peter was now up and, making his way to the boat, said to Jesus, “Master, shall I talk to them?” But Jesus answered, “No, Peter, I will tell them a story.” And then Jesus began the recital of the parable of the sower, one of the first of a long series of such parables which he taught the throngs that followed after him. This boat had an elevated seat on which he sat (for it was the custom to sit when teaching) while he talked to the crowd assembled along the shore. After Peter had spoken a few words, Jesus said:
\vs p151 1:2 \pc \textcolour{ubdarkred}{“A sower went forth to sow, and it came to pass as he sowed that some seed fell by the wayside to be trodden underfoot and devoured by the birds of heaven. Other seed fell upon the rocky places where there was little earth, and immediately it sprang up because there was no depth to the soil, but as soon as the sun shone, it withered because it had no root whereby to secure moisture. Other seed fell among the thorns, and as the thorns grew up, it was choked so that it yielded no grain. Still other seed fell upon good ground and, growing, yielded, some thirtyfold, some sixtyfold, and some a hundredfold.” And when he had finished speaking this parable, he said to the multitude, “He who has ears to hear, let him hear.”}
\vs p151 1:3 \pc The apostles and those who were with them, when they heard Jesus teach the people in this manner, were greatly perplexed; and after much talking among themselves, that evening in the Zebedee garden Matthew said to Jesus: “Master, what is the meaning of the dark sayings which you present to the multitude? Why do you speak in parables to those who seek the truth?” And Jesus answered:
\vs p151 1:4 \textcolour{ubdarkred}{“In patience have I instructed you all this time. To you it is given to know the mysteries of the kingdom of heaven, but to the undiscerning multitudes and to those who seek our destruction, from now on, the mysteries of the kingdom shall be presented in parables. And this we will do so that those who really desire to enter the kingdom may discern the meaning of the teaching and thus find salvation, while those who listen only to ensnare us may be the more confounded in that they will see without seeing and will hear without hearing. My children, do you not perceive the law of the spirit which decrees that to him who has shall be given so that he shall have an abundance; but from him who has not shall be taken away even that which he has. Therefore will I henceforth speak to the people much in parables to the end that our friends and those who desire to know the truth may find that which they seek, while our enemies and those who love not the truth may hear without understanding. Many of these people follow not in the way of the truth. The prophet did, indeed, describe all such undiscerning souls when he said: ‘For this people’s heart has waxed gross, and their ears are dull of hearing, and their eyes they have closed lest they should discern the truth and understand it in their hearts.’”}
\vs p151 1:5 The apostles did not fully comprehend the significance of the Master’s words. As Andrew and Thomas talked further with Jesus, Peter and the other apostles withdrew to another portion of the garden where they engaged in earnest and prolonged discussion.
\usection{Interpretation of the Parable}
\vs p151 2:1 Peter and the group about him came to the conclusion that the parable of the sower was an allegory, that each feature had some hidden meaning, and so they decided to go to Jesus and ask for an explanation. Accordingly, Peter approached the Master, saying: “We are not able to penetrate the meaning of this parable, and we desire that you explain it to us since you say it is given us to know the mysteries of the kingdom.” And when Jesus heard this, he said to Peter: \textcolour{ubdarkred}{“My son, I desire to withhold nothing from you, but first suppose you tell me what you have been talking about; what is your interpretation of the parable?”}
\vs p151 2:2 After a moment of silence, Peter said: “Master, we have talked much concerning the parable, and this is the interpretation I have decided upon: The sower is the gospel preacher; the seed is the word of God. The seed which fell by the wayside represents those who do not understand the gospel teaching. The birds which snatched away the seed that fell upon the hardened ground represent Satan, or the evil one, who steals away that which has been sown in the hearts of these ignorant ones. The seed which fell upon the rocky places, and which sprang up so suddenly, represents those superficial and unthinking persons who, when they hear the glad tidings, receive the message with joy; but because the truth has no real root in their deeper understanding, their devotion is short\hyp{}lived in the face of tribulation and persecution. When trouble comes, these believers stumble; they fall away when tempted. The seed which fell among thorns represents those who hear the word willingly, but who allow the cares of the world and the deceitfulness of riches to choke the word of truth so that it becomes unfruitful. Now the seed which fell on good ground and sprang up to bear, some thirty, some sixty, and some a hundredfold, represents those who, when they have heard the truth, receive it with varying degrees of appreciation --- owing to their differing intellectual endowments --- and hence manifest these varying degrees of religious experience.”
\vs p151 2:3 Jesus, after listening to Peter’s interpretation of the parable, asked the other apostles if they did not also have suggestions to offer. To this invitation only Nathaniel responded. Said he: “Master, while I recognize many good things about Simon Peter’s interpretation of the parable, I do not fully agree with him. My idea of this parable would be: The seed represents the gospel of the kingdom, while the sower stands for the messengers of the kingdom. The seed which fell by the wayside on hardened ground represents those who have heard but little of the gospel, along with those who are indifferent to the message, and who have hardened their hearts. The birds of the sky that snatched away the seed which fell by the wayside represent one’s habits of life, the temptation of evil, and the desires of the flesh. The seed which fell among the rocks stands for those emotional souls who are quick to receive new teaching and equally quick to give up the truth when confronted with the difficulties and realities of living up to this truth; they lack spiritual perception. The seed which fell among the thorns represents those who are attracted to the truths of the gospel; they are minded to follow its teachings, but they are prevented by the pride of life, jealousy, envy, and the anxieties of human existence. The seed which fell on good soil, springing up to bear, some thirty, some sixty, and some a hundredfold, represents the natural and varying degrees of ability to comprehend truth and respond to its spiritual teachings by men and women who possess diverse endowments of spirit illumination.”
\vs p151 2:4 When Nathaniel had finished speaking, the apostles and their associates fell into serious discussion and engaged in earnest debate, some contending for the correctness of Peter’s interpretation, while almost an equal number sought to defend Nathaniel’s explanation of the parable. Meanwhile Peter and Nathaniel had withdrawn to the house, where they were involved in a vigorous and determined effort the one to convince and change the mind of the other.
\vs p151 2:5 The Master permitted this confusion to pass the point of most intense expression; then he clapped his hands and called them about him. When they had all gathered around him once more, he said, \textcolour{ubdarkred}{“Before I tell you about this parable, do any of you have aught to say?”} Following a moment of silence, Thomas spoke up: “Yes, Master, I wish to say a few words. I remember that you once told us to beware of this very thing. You instructed us that, when using illustrations for our preaching, we should employ true stories, not fables, and that we should select a story best suited to the illustration of the one central and vital truth which we wished to teach the people, and that, having so used the story, we should not attempt to make a spiritual application of all the minor details involved in the telling of the story. I hold that Peter and Nathaniel are both wrong in their attempts to interpret this parable. I admire their ability to do these things, but I am equally sure that all such attempts to make a natural parable yield spiritual analogies in all its features can only result in confusion and serious misconception of the true purpose of such a parable. That I am right is fully proved by the fact that, whereas we were all of one mind an hour ago, now are we divided into two separate groups who hold different opinions concerning this parable and hold such opinions so earnestly as to interfere, in my opinion, with our ability fully to grasp the great truth which you had in mind when you presented this parable to the multitude and subsequently asked us to make comment upon it.”
\vs p151 2:6 The words which Thomas spoke had a quieting effect on all of them. He caused them to recall what Jesus had taught them on former occasions, and before Jesus resumed speaking, Andrew arose, saying: “I am persuaded that Thomas is right, and I would like to have him tell us what meaning he attaches to the parable of the sower.” After Jesus had beckoned Thomas to speak, he said: “My brethren, I did not wish to prolong this discussion, but if you so desire, I will say that I think this parable was spoken to teach us one great truth. And that is that our teaching of the gospel of the kingdom, no matter how faithfully and efficiently we execute our divine commissions, is going to be attended by varying degrees of success; and that all such differences in results are directly due to conditions inherent in the circumstances of our ministry, conditions over which we have little or no control.”
\vs p151 2:7 When Thomas had finished speaking, the majority of his fellow preachers were about ready to agree with him, even Peter and Nathaniel were on their way over to speak with him, when Jesus arose and said: \textcolour{ubdarkred}{“Well done, Thomas; you have discerned the true meaning of parables; but both Peter and Nathaniel have done you all equal good in that they have so fully shown the danger of undertaking to make an allegory out of my parables. In your own hearts you may often profitably engage in such flights of the speculative imagination, but you make a mistake when you seek to offer such conclusions as a part of your public teaching.”}
\vs p151 2:8 Now that the tension was over, Peter and Nathaniel congratulated each other on their interpretations, and with the exception of the Alpheus twins, each of the apostles ventured to make an interpretation of the parable of the sower before they retired for the night. Even Judas Iscariot offered a very plausible interpretation. The twelve would often, among themselves, attempt to figure out the Master’s parables as they would an allegory, but never again did they regard such speculations seriously. This was a very profitable session for the apostles and their associates, especially so since from this time on Jesus more and more employed parables in connection with his public teaching.
\usection{More about Parables}
\vs p151 3:1 The apostles were parable\hyp{}minded, so much so that the whole of the next evening was devoted to the further discussion of parables. Jesus introduced the evening’s conference by saying: \textcolour{ubdarkred}{“My beloved, you must always make a difference in teaching so as to suit your presentation of truth to the minds and hearts before you. When you stand before a multitude of varying intellects and temperaments, you cannot speak different words for each class of hearers, but you can tell a story to convey your teaching; and each group, even each individual, will be able to make his own interpretation of your parable in accordance with his own intellectual and spiritual endowments. You are to let your light shine but do so with wisdom and discretion. No man, when he lights a lamp, covers it up with a vessel or puts it under the bed; he puts his lamp on a stand where all can behold the light. Let me tell you that nothing is hid in the kingdom of heaven which shall not be made manifest; neither are there any secrets which shall not ultimately be made known. Eventually, all these things shall come to light. Think not only of the multitudes and how they hear the truth; take heed also to yourselves how you hear. Remember that I have many times told you: To him who has shall be given more, while from him who has not shall be taken away even that which he thinks he has.”}
\vs p151 3:2 \pc The continued discussion of parables and further instruction as to their interpretation may be summarized and expressed in modern phraseology as follows:
\vs p151 3:3 \ublistelem{1.}\bibnobreakspace Jesus advised against the use of either fables or allegories in teaching the truths of the gospel. He did recommend the free use of parables, especially nature parables. He emphasized the value of utilizing the \bibemph{analogy} existing between the natural and the spiritual worlds as a means of teaching truth. He frequently alluded to the natural as “the unreal and fleeting shadow of spirit realities.”
\vs p151 3:4 \ublistelem{2.}\bibnobreakspace Jesus narrated three or four parables from the Hebrew scriptures, calling attention to the fact that this method of teaching was not wholly new. However, it became almost a new method of teaching as he employed it from this time onward.
\vs p151 3:5 \ublistelem{3.}\bibnobreakspace In teaching the apostles the value of parables, Jesus called attention to the following points:
\vs p151 3:6 The parable provides for a simultaneous appeal to vastly different levels of mind and spirit. The parable stimulates the imagination, challenges the discrimination, and provokes critical thinking; it promotes sympathy without arousing antagonism.
\vs p151 3:7 The parable proceeds from the things which are known to the discernment of the unknown. The parable utilizes the material and natural as a means of introducing the spiritual and the supermaterial.
\vs p151 3:8 Parables favour the making of impartial moral decisions. The parable evades much prejudice and puts new truth gracefully into the mind and does all this with the arousal of a minimum of the self\hyp{}defence of personal resentment.
\vs p151 3:9 To reject the truth contained in parabolical analogy requires conscious intellectual action which is directly in contempt of one’s honest judgment and fair decision. The parable conduces to the forcing of thought through the sense of hearing.
\vs p151 3:10 The use of the parable form of teaching enables the teacher to present new and even startling truths while at the same time he largely avoids all controversy and outward clashing with tradition and established authority.
\vs p151 3:11 The parable also possesses the advantage of stimulating the memory of the truth taught when the same familiar scenes are subsequently encountered.
\vs p151 3:12 \pc In this way Jesus sought to acquaint his followers with many of the reasons underlying his practice of increasingly using parables in his public teaching.
\vs p151 3:13 \pc Toward the close of the evening’s lesson Jesus made his first comment on the parable of the sower. He said the parable referred to two things: First, it was a review of his own ministry up to that time and a forecast of what lay ahead of him for the remainder of his life on earth. And second, it was also a hint as to what the apostles and other messengers of the kingdom might expect in their ministry from generation to generation as time passed.
\vs p151 3:14 Jesus also resorted to the use of parables as the best possible refutation of the studied effort of the religious leaders at Jerusalem to teach that all of his work was done by the assistance of demons and the prince of devils. The appeal to nature was in contravention of such teaching since the people of that day looked upon all natural phenomena as the product of the direct act of spiritual beings and supernatural forces. He also determined upon this method of teaching because it enabled him to proclaim vital truths to those who desired to know the better way while at the same time affording his enemies less opportunity to find cause for offence and for accusations against him.
\vs p151 3:15 Before he dismissed the group for the night, Jesus said: \textcolour{ubdarkred}{“Now will I tell you the last of the parable of the sower. I would test you to know how you will receive this: The kingdom of heaven is also like a man who cast good seed upon the earth; and while he slept by night and went about his business by day, the seed sprang up and grew, and although he knew not how it came about, the plant came to fruit. First there was the blade, then the ear, then the full grain in the ear. And then when the grain was ripe, he put forth the sickle, and the harvest was finished. He who has an ear to hear, let him hear.”}
\vs p151 3:16 Many times did the apostles turn this saying over in their minds, but the Master never made further mention of this addition to the parable of the sower.
\usection{More Parables by the Sea}
\vs p151 4:1 The next day Jesus again taught the people from the boat, saying: \textcolour{ubdarkred}{“The kingdom of heaven is like a man who sowed good seed in his field; but while he slept, his enemy came and sowed weeds among the wheat and hastened away. And so when the young blades sprang up and later were about to bring forth fruit, there appeared also the weeds. Then the servants of this householder came and said to him: ‘Sir, did you not sow good seed in your field? Whence then come these weeds?’ And he replied to his servants, ‘An enemy has done this.’ The servants then asked their master, ‘Would you have us go out and pluck up these weeds?’ But he answered them and said: ‘No, lest while you are gathering them up, you uproot the wheat also. Rather let them both grow together until the time of the harvest, when I will say to the reapers, Gather up first the weeds and bind them in bundles to burn and then gather up the wheat to be stored in my barn.’”}
\vs p151 4:2 \pc After the people had asked a few questions, Jesus spoke another parable: \textcolour{ubdarkred}{“The kingdom of heaven is like a grain of mustard seed which a man sowed in his field. Now a mustard seed is the least of seeds, but when it is full grown, it becomes the greatest of all herbs and is like a tree so that the birds of heaven are able to come and rest in the branches thereof.”}
\vs p151 4:3 \pc \textcolour{ubdarkred}{“The kingdom of heaven is also like leaven which a woman took and hid in three measures of meal, and in this way it came about that all of the meal was leavened.”}
\vs p151 4:4 \pc \textcolour{ubdarkred}{“The kingdom of heaven is also like a treasure hidden in a field, which a man discovered. In his joy he went forth to sell all he had that he might have the money to buy the field.”}
\vs p151 4:5 \pc \textcolour{ubdarkred}{“The kingdom of heaven is also like a merchant seeking goodly pearls; and having found one pearl of great price, he went out and sold everything he possessed that he might be able to buy the extraordinary pearl.”}
\vs p151 4:6 \pc \textcolour{ubdarkred}{“Again, the kingdom of heaven is like a sweep net which was cast into the sea, and it gathered up every kind of fish. Now, when the net was filled, the fishermen drew it up on the beach, where they sat down and sorted out the fish, gathering the good into vessels while the bad they threw away.”}
\vs p151 4:7 \pc Many other parables spoke Jesus to the multitudes. In fact, from this time forward he seldom taught the masses except by this means. After speaking to a public audience in parables, he would, during the evening classes, more fully and explicitly expound his teachings to the apostles and the evangelists.
\usection{The Visit to Kheresa}
\vs p151 5:1 The multitude continued to increase throughout the week. On Sabbath Jesus hastened away to the hills, but when Sunday morning came, the crowds returned. Jesus spoke to them in the early afternoon after the preaching of Peter, and when he had finished, he said to his apostles: \textcolour{ubdarkred}{“I am weary of the throngs; let us cross over to the other side that we may rest for a day.”}
\vs p151 5:2 On the way across the lake they encountered one of those violent and sudden windstorms which are characteristic of the Sea of Galilee, especially at this season of the year. This body of water is almost 210\,m below the level of the sea and is surrounded by high banks, especially on the west. There are steep gorges leading up from the lake into the hills, and as the heated air rises in a pocket over the lake during the day, there is a tendency after sunset for the cooling air of the gorges to rush down upon the lake. These gales come on quickly and sometimes go away just as suddenly.
\vs p151 5:3 It was just such an evening gale that caught the boat carrying Jesus over to the other side on this Sunday evening. Three other boats containing some of the younger evangelists were trailing after. This tempest was severe, notwithstanding that it was confined to this region of the lake, there being no evidence of a storm on the western shore. The wind was so strong that the waves began to wash over the boat. The high wind had torn the sail away before the apostles could furl it, and they were now entirely dependent on their oars as they laboriously pulled for the shore, a little more than 2.4\,km distant.
\vs p151 5:4 Meanwhile Jesus lay asleep in the stern of the boat under a small overhead shelter. The Master was weary when they left Bethsaida, and it was to secure rest that he had directed them to sail him across to the other side. These ex\hyp{}fishermen were strong and experienced oarsmen, but this was one of the worst gales they had ever encountered. Although the wind and the waves tossed their boat about as though it were a toy ship, Jesus slumbered on undisturbed. Peter was at the right\hyp{}hand oar near the stern. When the boat began to fill with water, he dropped his oar and, rushing over to Jesus, shook him vigorously in order to awaken him, and when he was aroused, Peter said: “Master, don’t you know we are in a violent storm? If you do not save us, we will all perish.”
\vs p151 5:5 As Jesus came out in the rain, he looked first at Peter, and then peering into the darkness at the struggling oarsmen, he turned his glance back upon Simon Peter, who, in his agitation, had not yet returned to his oar, and said: \textcolour{ubdarkred}{“Why are all of you so filled with fear? Where is your faith? Peace, be quiet.”} Jesus had hardly uttered this rebuke to Peter and the other apostles, he had hardly bidden Peter seek peace wherewith to quiet his troubled soul, when the disturbed atmosphere, having established its equilibrium, settled down into a great calm. The angry waves almost immediately subsided, while the dark clouds, having spent themselves in a short shower, vanished, and the stars of heaven shone overhead. All this was purely coincidental as far as we can judge; but the apostles, particularly Simon Peter, never ceased to regard the episode as a nature miracle. It was especially easy for the men of that day to believe in nature miracles inasmuch as they firmly believed that all nature was a phenomenon directly under the control of spirit forces and supernatural beings.
\vs p151 5:6 Jesus plainly explained to the twelve that he had spoken to their troubled spirits and had addressed himself to their fear\hyp{}tossed minds, that he had not commanded the elements to obey his word, but it was of no avail. The Master’s followers always persisted in placing their own interpretation on all such coincidental occurrences. From this day on they insisted on regarding the Master as having absolute power over the natural elements. Peter never grew weary of reciting how “even the winds and the waves obey him.”
\vs p151 5:7 It was late in the evening when Jesus and his associates reached the shore, and since it was a calm and beautiful night, they all rested in the boats, not going ashore until shortly after sunrise the next morning. When they were gathered together, about 40 in all, Jesus said: \textcolour{ubdarkred}{“Let us go up into yonder hills and tarry for a few days while we ponder over the problems of the Father’s kingdom.”}
\usection{The Kheresa Lunatic}
\vs p151 6:1 Although most of the near\hyp{}by eastern shore of the lake sloped up gently to the highlands beyond, at this particular spot there was a steep hillside, the shore in some places dropping sheer down into the lake. Pointing up to the side of the near\hyp{}by hill, Jesus said: \textcolour{ubdarkred}{“Let us go up on this hillside for our breakfast and under some of the shelters rest and talk.”}
\vs p151 6:2 This entire hillside was covered with caverns which had been hewn out of the rock. Many of these niches were ancient sepulchres. About halfway up the hillside on a small, relatively level spot was the cemetery of the little village of Kheresa. As Jesus and his associates passed near this burial ground, a lunatic who lived in these hillside caverns rushed up to them. This demented man was well known about these parts, having onetime been bound with fetters and chains and confined in one of the grottoes. Long since he had broken his shackles and now roamed at will among the tombs and abandoned sepulchres.\fnc{\ldots{}with fetters and chains and confined in one of the \bibtextul{grottos}\ldots{} \bibexpl{Though both forms are correct, this word is found elsewhere in the text as grottoes. Therefore, text standardization was adopted on the latter.}}
\vs p151 6:3 This man, whose name was Amos, was afflicted with a periodic form of insanity. There were considerable spells when he would find some clothing and deport himself fairly well among his fellows. During one of these lucid intervals he had gone over to Bethsaida, where he heard the preaching of Jesus and the apostles, and at that time had become a half\hyp{}hearted believer in the gospel of the kingdom. But soon a stormy phase of his trouble appeared, and he fled to the tombs, where he moaned, cried out aloud, and so conducted himself as to terrorize all who chanced to meet him.
\vs p151 6:4 When Amos recognized Jesus, he fell down at his feet and exclaimed: “I know you, Jesus, but I am possessed of many devils, and I beseech that you will not torment me.” This man truly believed that his periodic mental affliction was due to the fact that, at such times, evil or unclean spirits entered into him and dominated his mind and body. His troubles were mostly emotional --- his brain was not grossly diseased.
\vs p151 6:5 Jesus, looking down upon the man crouching like an animal at his feet, reached down and, taking him by the hand, stood him up and said to him: \textcolour{ubdarkred}{“Amos, you are not possessed of a devil; you have already heard the good news that you are a son of God. I command you to come out of this spell.”} And when Amos heard Jesus speak these words, there occurred such a transformation in his intellect that he was immediately restored to his right mind and the normal control of his emotions. By this time a considerable crowd had assembled from the near\hyp{}by village, and these people, augmented by the swine herders from the highland above them, were astonished to see the lunatic sitting with Jesus and his followers, in possession of his right mind and freely conversing with them.
\vs p151 6:6 As the swine herders rushed into the village to spread the news of the taming of the lunatic, the dogs charged upon a small and untended herd of about 30 swine and drove most of them over a precipice into the sea. And it was this incidental occurrence, in connection with the presence of Jesus and the supposed miraculous curing of the lunatic, that gave origin to the legend that Jesus had cured Amos by casting a legion of devils out of him, and that these devils had entered into the herd of swine, causing them forthwith to rush headlong to their destruction in the sea below. Before the day was over, this episode was published abroad by the swine tenders, and the whole village believed it. Amos most certainly believed this story; he saw the swine tumbling over the brow of the hill shortly after his troubled mind had quieted down, and he always believed that they carried with them the very evil spirits which had so long tormented and afflicted him. And this had a good deal to do with the permanency of his cure. It is equally true that all of Jesus’ apostles (save Thomas) believed that the episode of the swine was directly connected with the cure of Amos.
\vs p151 6:7 \pc Jesus did not obtain the rest he was looking for. Most of that day he was thronged by those who came in response to the word that Amos had been cured, and who were attracted by the story that the demons had gone out of the lunatic into the herd of swine. And so, after only one night of rest, early Tuesday morning Jesus and his friends were awakened by a delegation of these swine\hyp{}raising gentiles who had come to urge that he depart from their midst. Said their spokesman to Peter and Andrew: “Fishermen of Galilee, depart from us and take your prophet with you. We know he is a holy man, but the gods of our country do not know him, and we stand in danger of losing many swine. The fear of you has descended upon us, so that we pray you to go hence.” And when Jesus heard them, he said to Andrew, \textcolour{ubdarkred}{“Let us return to our place.”}
\vs p151 6:8 As they were about to depart, Amos besought Jesus to permit him to go back with them, but the Master would not consent. Said Jesus to Amos: \textcolour{ubdarkred}{“Forget not that you are a son of God. Return to your own people and show them what great things God has done for you.”} And Amos went about publishing that Jesus had cast a legion of devils out of his troubled soul, and that these evil spirits had entered into a herd of swine, driving them to quick destruction. And he did not stop until he had gone into all the cities of the Decapolis, declaring what great things Jesus had done for him.
\quizlink
