\upaper{195}{After Pentecost}
\uminitoc{Influence of the Greeks}
\uminitoc{The Roman Influence}
\uminitoc{Under the Roman Empire}
\uminitoc{The European Dark Ages}
\uminitoc{The Modern Problem}
\uminitoc{Materialism}
\uminitoc{The Vulnerability of Materialism}
\uminitoc{Secular Totalitarianism}
\uminitoc{Christianity’s Problem}
\uminitoc{The Future}
\author{Midwayer Commission}
\vs p195 0:1 The results of Peter’s preaching on the day of Pentecost were such as to decide the future policies, and to determine the plans, of the majority of the apostles in their efforts to proclaim the gospel of the kingdom. Peter was the real founder of the Christian church; Paul carried the Christian message to the gentiles, and the Greek believers carried it to the whole Roman Empire.
\vs p195 0:2 Although the tradition\hyp{}bound and priest\hyp{}ridden Hebrews, as a people, refused to accept either Jesus’ gospel of the fatherhood of God and the brotherhood of man or Peter’s and Paul’s proclamation of the resurrection and ascension of Christ (subsequent Christianity), the rest of the Roman Empire was found to be receptive to the evolving Christian teachings. Western civilization was at this time intellectual, war weary, and thoroughly sceptical of all existing religions and universe philosophies. The peoples of the Western world, the beneficiaries of Greek culture, had a revered tradition of a great past. They could contemplate the inheritance of great accomplishments in philosophy, art, literature, and political progress. But with all these achievements they had no soul\hyp{}satisfying religion. Their spiritual longings remained unsatisfied.
\vs p195 0:3 Upon such a stage of human society the teachings of Jesus, embraced in the Christian message, were suddenly thrust. A new order of living was thus presented to the hungry hearts of these Western peoples. This situation meant immediate conflict between the older religious practices and the new Christianized version of Jesus’ message to the world. Such a conflict must result in either decided victory for the new or for the old or in some degree of \bibemph{compromise.} History shows that the struggle ended in compromise. Christianity presumed to embrace too much for any one people to assimilate in one or two generations. It was not a simple spiritual appeal, such as Jesus had presented to the souls of men; it early struck a decided attitude on religious rituals, education, magic, medicine, art, literature, law, government, morals, sex regulation, polygamy, and, in limited degree, even slavery. Christianity came not merely as a new religion --- something all the Roman Empire and all the Orient were waiting for --- but as a \bibemph{new order of human society.} And as such a pretension it quickly precipitated the social\hyp{}moral clash of the ages. The ideals of Jesus, as they were reinterpreted by Greek philosophy and socialized in Christianity, now boldly challenged the traditions of the human race embodied in the ethics, morality, and religions of Western civilization.
\vs p195 0:4 \pc At first, Christianity won as converts only the lower social and economic strata. But by the beginning of the second century the very best of Gr\ae co\hyp{}Roman culture was increasingly turning to this new order of Christian belief, this new concept of the purpose of living and the goal of existence.
\vs p195 0:5 How did this new message of Jewish origin, which had almost failed in the land of its birth, so quickly and effectively capture the very best minds of the Roman Empire? The triumph of Christianity over the philosophic religions and the mystery cults was due to:
\vs p195 0:6 \ublistelem{1.}\bibnobreakspace Organization. Paul was a great organizer and his successors kept up the pace he set.
\vs p195 0:7 \ublistelem{2.}\bibnobreakspace Christianity was thoroughly Hellenized. It embraced the best in Greek philosophy as well as the cream of Hebrew theology.
\vs p195 0:8 \ublistelem{3.}\bibnobreakspace But best of all, it contained a new and great \bibemph{ideal,} the echo of the life bestowal of Jesus and the reflection of his message of salvation for all mankind.
\vs p195 0:9 \ublistelem{4.}\bibnobreakspace The Christian leaders were willing to make such compromises with Mithraism that the better half of its adherents were won over to the Antioch cult.
\vs p195 0:10 \ublistelem{5.}\bibnobreakspace Likewise did the next and later generations of Christian leaders make such further compromises with paganism that even the Roman emperor Constantine was won to the new religion.
\vs p195 0:11 \pc But the Christians made a shrewd bargain with the pagans in that they adopted the ritualistic pageantry of the pagan while compelling the pagan to accept the Hellenized version of Pauline Christianity. They made a better bargain with the pagans than they did with the Mithraic cult, but even in that earlier compromise they came off more than conquerors in that they succeeded in eliminating the gross immoralities and also numerous other reprehensible practices of the Persian mystery.
\vs p195 0:12 Wisely or unwisely, these early leaders of Christianity deliberately compromised the \bibemph{ideals} of Jesus in an effort to save and further many of his \bibemph{ideas.} And they were eminently successful. But mistake not! these compromised ideals of the Master are still latent in his gospel, and they will eventually assert their full power upon the world.
\vs p195 0:13 By this paganization of Christianity the old order won many minor victories of a ritualistic nature, but the Christians gained the ascendancy in that:
\vs p195 0:14 \ublistelem{1.}\bibnobreakspace A new and enormously higher note in human morals was struck.
\vs p195 0:15 \ublistelem{2.}\bibnobreakspace A new and greatly enlarged concept of God was given to the world.
\vs p195 0:16 \ublistelem{3.}\bibnobreakspace The hope of immortality became a part of the assurance of a recognized religion.
\vs p195 0:17 \ublistelem{4.}\bibnobreakspace Jesus of Nazareth was given to man’s hungry soul.
\vs p195 0:18 \pc Many of the great truths taught by Jesus were almost lost in these early compromises, but they yet slumber in this religion of paganized Christianity, which was in turn the Pauline version of the life and teachings of the Son of Man. And Christianity, even before it was paganized, was first thoroughly Hellenized. Christianity owes much, very much, to the Greeks. It was a Greek, from Egypt, who so bravely stood up at Nicaea and so fearlessly challenged this assembly that it dared not so obscure the concept of the nature of Jesus that the real truth of his bestowal might have been in danger of being lost to the world. This Greek’s name was Athanasius, and but for the eloquence and the logic of this believer, the persuasions of Arius would have triumphed.
\usection{Influence of the Greeks}
\vs p195 1:1 The Hellenization of Christianity started in earnest on that eventful day when the Apostle Paul stood before the council of the Areopagus in Athens and told the Athenians about “the Unknown God.” There, under the shadow of the Acropolis, this Roman citizen proclaimed to these Greeks his version of the new religion which had taken origin in the Jewish land of Galilee. And there was something strangely alike in Greek philosophy and many of the teachings of Jesus. They had a common goal --- both aimed at the \bibemph{emergence of the individual.} The Greek, at social and political emergence; Jesus, at moral and spiritual emergence. The Greek taught intellectual liberalism leading to political freedom; Jesus taught spiritual liberalism leading to religious liberty. These two ideas put together constituted a new and mighty charter for human freedom; they presaged man’s social, political, and spiritual liberty.
\vs p195 1:2 Christianity came into existence and triumphed over all contending religions primarily because of two things:
\vs p195 1:3 \ublistelem{1.}\bibnobreakspace The Greek mind was willing to borrow new and good ideas even from the Jews.
\vs p195 1:4 \ublistelem{2.}\bibnobreakspace Paul and his successors were willing but shrewd and sagacious compromisers; they were keen theologic traders.
\vs p195 1:5 \pc At the time Paul stood up in Athens preaching “Christ and Him Crucified,” the Greeks were spiritually hungry; they were inquiring, interested, and actually looking for spiritual truth. Never forget that at first the Romans fought Christianity, while the Greeks embraced it, and that it was the Greeks who literally forced the Romans subsequently to accept this new religion, as then modified, as a part of Greek culture.
\vs p195 1:6 The Greek revered beauty, the Jew holiness, but both peoples loved truth. For centuries the Greek had seriously thought and earnestly debated about all human problems --- social, economic, political, and philosophic --- except religion. Few Greeks had paid much attention to religion; they did not take even their own religion very seriously. For centuries the Jews had neglected these other fields of thought while they devoted their minds to religion. They took their religion very seriously, too seriously. As illuminated by the content of Jesus’ message, the united product of the centuries of the thought of these two peoples now became the driving power of a new order of human society and, to a certain extent, of a new order of human religious belief and practice.
\vs p195 1:7 \pc The influence of Greek culture had already penetrated the lands of the western Mediterranean when Alexander spread Hellenistic civilization over the near\hyp{}Eastern world. The Greeks did very well with their religion and their politics as long as they lived in small city\hyp{}states, but when the Macedonian king dared to expand Greece into an empire, stretching from the Adriatic to the Indus, trouble began. The art and philosophy of Greece were fully equal to the task of imperial expansion, but not so with Greek political administration or religion. After the city\hyp{}states of Greece had expanded into empire, their rather parochial gods seemed a little queer. The Greeks were really searching for \bibemph{one God,} a greater and better God, when the Christianized version of the older Jewish religion came to them.
\vs p195 1:8 The Hellenistic Empire, as such, could not endure. Its cultural sway continued on, but it endured only after securing from the West the Roman political genius for empire administration and after obtaining from the East a religion whose one God possessed empire dignity.
\vs p195 1:9 In the first century after Christ, Hellenistic culture had already attained its highest levels; its retrogression had begun; learning was advancing but genius was declining. It was at this very time that the ideas and ideals of Jesus, which were partially embodied in Christianity, became a part of the salvage of Greek culture and learning.
\vs p195 1:10 Alexander had charged on the East with the cultural gift of the civilization of Greece; Paul assaulted the West with the Christian version of the gospel of Jesus. And wherever the Greek culture prevailed throughout the West, there Hellenized Christianity took root.
\vs p195 1:11 \pc The Eastern version of the message of Jesus, notwithstanding that it remained more true to his teachings, continued to follow the uncompromising attitude of Abner. It never progressed as did the Hellenized version and was eventually lost in the Islamic movement.
\usection{The Roman Influence}
\vs p195 2:1 The Romans bodily took over Greek culture, putting representative government in the place of government by lot. And presently this change favoured Christianity in that Rome brought into the whole Western world a new tolerance for strange languages, peoples, and even religions.
\vs p195 2:2 Much of the early persecution of Christians in Rome was due solely to their unfortunate use of the term “kingdom” in their preaching. The Romans were tolerant of any and all religions but very resentful of anything that savoured of political rivalry. And so, when these early persecutions, due so largely to misunderstanding, died out, the field for religious propaganda was wide open. The Roman was interested in political administration; he cared little for either art or religion, but he was unusually tolerant of both.
\vs p195 2:3 Oriental law was stern and arbitrary; Greek law was fluid and artistic; Roman law was dignified and respect\hyp{}breeding. Roman education bred an unheard\hyp{}of and stolid loyalty. The early Romans were politically devoted and sublimely consecrated individuals. They were honest, zealous, and dedicated to their ideals, but without a religion worthy of the name. Small wonder that their Greek teachers were able to persuade them to accept Paul’s Christianity.
\vs p195 2:4 And these Romans were a great people. They could govern the Occident because they did govern themselves. Such unparalleled honesty, devotion, and stalwart self\hyp{}control was ideal soil for the reception and growth of Christianity.
\vs p195 2:5 It was easy for these Gr\ae co\hyp{}Romans to become just as spiritually devoted to an institutional church as they were politically devoted to the state. The Romans fought the church only when they feared it as a competitor of the state. Rome, having little national philosophy or native culture, took over Greek culture for its own and boldly adopted Christ as its moral philosophy. Christianity became the moral culture of Rome but hardly its religion in the sense of being the individual experience in spiritual growth of those who embraced the new religion in such a wholesale manner. True, indeed, many individuals did penetrate beneath the surface of all this state religion and found for the nourishment of their souls the real values of the hidden meanings held within the latent truths of Hellenized and paganized Christianity.
\vs p195 2:6 \pc The Stoic and his sturdy appeal to “nature and conscience” had only the better prepared all Rome to receive Christ, at least in an intellectual sense. The Roman was by nature and training a lawyer; he revered even the laws of nature. And now, in Christianity, he discerned in the laws of nature the laws of God. A people that could produce Cicero and Vergil were ripe for Paul’s Hellenized Christianity.
\vs p195 2:7 And so did these Romanized Greeks force both Jews and Christians to philosophize their religion, to co\hyp{}ordinate its ideas and systematize its ideals, to adapt religious practices to the existing current of life. And all this was enormously helped by translation of the Hebrew scriptures into Greek and by the later recording of the New Testament in the Greek tongue.
\vs p195 2:8 The Greeks, in contrast with the Jews and many other peoples, had long provisionally believed in immortality, some sort of survival after death, and since this was the very heart of Jesus’ teaching, it was certain that Christianity would make a strong appeal to them.
\vs p195 2:9 A succession of Greek\hyp{}cultural and Roman\hyp{}political victories had consolidated the Mediterranean lands into one empire, with one language and one culture, and had made the Western world ready for one God. Judaism provided this God, but Judaism was not acceptable as a religion to these Romanized Greeks. Philo helped some to mitigate their objections, but Christianity revealed to them an even better concept of one God, and they embraced it readily.
\usection{Under the Roman Empire}
\vs p195 3:1 After the consolidation of Roman political rule and after the dissemination of Christianity, the Christians found themselves with one God, a great religious concept, but without empire. The Gr\ae co\hyp{}Romans found themselves with a great empire but without a God to serve as the suitable religious concept for empire worship and spiritual unification. The Christians accepted the empire; the empire adopted Christianity. The Roman provided a unity of political rule; the Greek, a unity of culture and learning; Christianity, a unity of religious thought and practice.
\vs p195 3:2 Rome overcame the tradition of nationalism by imperial universalism and for the first time in history made it possible for different races and nations at least nominally to accept one religion.
\vs p195 3:3 Christianity came into favour in Rome at a time when there was great contention between the vigorous teachings of the Stoics and the salvation promises of the mystery cults. Christianity came with refreshing comfort and liberating power to a spiritually hungry people whose language had no word for “unselfishness.”
\vs p195 3:4 \pc That which gave greatest power to Christianity was the way its believers lived lives of service and even the way they died for their faith during the earlier times of drastic persecution.
\vs p195 3:5 \pc The teaching regarding Christ’s love for children soon put an end to the widespread practice of exposing children to death when they were not wanted, particularly girl babies.
\vs p195 3:6 \pc The early plan of Christian worship was largely taken over from the Jewish synagogue, modified by the Mithraic ritual; later on, much pagan pageantry was added. The backbone of the early Christian church consisted of Christianized Greek proselytes to Judaism.
\vs p195 3:7 \pc The second century after Christ was the best time in all the world’s history for a good religion to make progress in the Western world. During the first century Christianity had prepared itself, by struggle and compromise, to take root and rapidly spread. Christianity adopted the emperor; later, he adopted Christianity. This was a great age for the spread of a new religion. There was religious liberty; travel was universal and thought was untrammelled.
\vs p195 3:8 The spiritual impetus of nominally accepting Hellenized Christianity came to Rome too late to prevent the well\hyp{}started moral decline or to compensate for the already well\hyp{}established and increasing racial deterioration. This new religion was a cultural necessity for imperial Rome, and it is exceedingly unfortunate that it did not become a means of spiritual salvation in a larger sense.
\vs p195 3:9 Even a good religion could not save a great empire from the sure results of lack of individual participation in the affairs of government, from overmuch paternalism, overtaxation and gross collection abuses, unbalanced trade with the Levant which drained away the gold, amusement madness, Roman standardization, the degradation of woman, slavery and race decadence, physical plagues, and a state church which became institutionalized nearly to the point of spiritual barrenness.
\vs p195 3:10 Conditions, however, were not so bad at Alexandria. The early schools continued to hold much of Jesus’ teachings free from compromise. Pantaenus taught Clement and then went on to follow Nathaniel in proclaiming Christ in India. While some of the ideals of Jesus were sacrificed in the building of Christianity, it should in all fairness be recorded that, by the end of the second century, practically all the great minds of the Gr\ae co\hyp{}Roman world had become Christian. The triumph was approaching completion.\fnc{\bibtextul{Poutaenus} taught Clement and then went on to follow Nathaniel\ldots{} \bibexpl{The correct spelling of this name is Pantaenus; Dr. Sadler, in a March 17, 1959 letter to the Rev. Benjamin Adams of San Francisco, suggested the possible source of the error: “I think the spelling of the name of the teacher in Alexandria is undoubtedly an error in transcribing the manuscript into typewriting. An “an” was undoubtedly transcribed as an “ou”. I remember when we were sometimes in doubt as to whether a letter was an “n” or a “u” in the manuscript. Of course, we who were preparing this matter, did not know the name of this teacher so could have easily made this mistake.”}}
\vs p195 3:11 And this Roman Empire lasted sufficiently long to ensure the survival of Christianity even after the empire collapsed. But we have often conjectured what would have happened in Rome and in the world if it had been the gospel of the kingdom which had been accepted in the place of Greek Christianity.
\usection{The European Dark Ages}
\vs p195 4:1 The church, being an adjunct to society and the ally of politics, was doomed to share in the intellectual and spiritual decline of the so\hyp{}called European “dark ages.” During this time, religion became more and more monasticized, asceticized, and legalized. In a spiritual sense, Christianity was hibernating. Throughout this period there existed, alongside this slumbering and secularized religion, a continuous stream of mysticism, a fantastic spiritual experience bordering on unreality and philosophically akin to pantheism.
\vs p195 4:2 During these dark and despairing centuries, religion became virtually secondhanded again. The individual was almost lost before the overshadowing authority, tradition, and dictation of the church. A new spiritual menace arose in the creation of a galaxy of “saints” who were assumed to have special influence at the divine courts, and who, therefore, if effectively appealed to, would be able to intercede in man’s behalf before the Gods.
\vs p195 4:3 But Christianity was sufficiently socialized and paganized that, while it was impotent to stay the oncoming dark ages, it was the better prepared to survive this long period of moral darkness and spiritual stagnation. And it did persist on through the long night of Western civilization and was still functioning as a moral influence in the world when the renaissance dawned. The rehabilitation of Christianity, following the passing of the dark ages, resulted in bringing into existence numerous sects of the Christian teachings, beliefs suited to special intellectual, emotional, and spiritual types of human personality. And many of these special Christian groups, or religious families, still persist at the time of the making of this presentation.
\vs p195 4:4 \pc Christianity exhibits a history of having originated out of the unintended transformation of the religion of Jesus into a religion about Jesus. It further presents the history of having experienced Hellenization, paganization, secularization, institutionalization, intellectual deterioration, spiritual decadence, moral hibernation, threatened extinction, later rejuvenation, fragmentation, and more recent relative rehabilitation. Such a pedigree is indicative of inherent vitality and the possession of vast recuperative resources. And this same Christianity is now present in the civilized world of Occidental peoples and stands face to face with a struggle for existence which is even more ominous than those eventful crises which have characterized its past battles for dominance.
\vs p195 4:5 \pc Religion is now confronted by the challenge of a new age of scientific minds and materialistic tendencies. In this gigantic struggle between the secular and the spiritual, the religion of Jesus will eventually triumph.
\usection{The Modern Problem}
\vs p195 5:1 The XX century has brought new problems for Christianity and all other religions to solve. The higher a civilization climbs, the more necessitous becomes the duty to “seek first the realities of heaven” in all of man’s efforts to stabilize society and facilitate the solution of its material problems.
\vs p195 5:2 Truth often becomes confusing and even misleading when it is dismembered, segregated, isolated, and too much analysed. Living truth teaches the truth seeker aright only when it is embraced in wholeness and as a living spiritual reality, not as a fact of material science or an inspiration of intervening art.
\vs p195 5:3 Religion is the revelation to man of his divine and eternal destiny. Religion is a purely personal and spiritual experience and must forever be distinguished from man’s other high forms of thought, such as:
\vs p195 5:4 \ublistelem{1.}\bibnobreakspace Man’s logical attitude toward the things of material reality.
\vs p195 5:5 \ublistelem{2.}\bibnobreakspace Man’s aesthetic appreciation of beauty contrasted with ugliness.
\vs p195 5:6 \ublistelem{3.}\bibnobreakspace Man’s ethical recognition of social obligations and political duty.
\vs p195 5:7 \ublistelem{4.}\bibnobreakspace Even man’s sense of human morality is not, in and of itself, religious.
\vs p195 5:8 \pc Religion is designed to find those values in the universe which call forth faith, trust, and assurance; religion culminates in worship. Religion discovers for the soul those supreme values which are in contrast with the relative values discovered by the mind. Such superhuman insight can be had only through genuine religious experience.
\vs p195 5:9 A lasting social system without a morality predicated on spiritual realities can no more be maintained than could the solar system without gravity.
\vs p195 5:10 Do not try to satisfy the curiosity or gratify all the latent adventure surging within the soul in one short life in the flesh. Be patient! be not tempted to indulge in a lawless plunge into cheap and sordid adventure. Harness your energies and bridle your passions; be calm while you await the majestic unfolding of an endless career of progressive adventure and thrilling discovery.
\vs p195 5:11 \pc In confusion over man’s origin, do not lose sight of his eternal destiny. Forget not that Jesus loved even little children, and that he forever made clear the great worth of human personality.
\vs p195 5:12 \pc As you view the world, remember that the black patches of evil which you see are shown against a white background of ultimate good. You do not view merely white patches of good which show up miserably against a black background of evil.
\vs p195 5:13 When there is so much good truth to publish and proclaim, why should men dwell so much upon the evil in the world just because it appears to be a fact? The beauties of the spiritual values of truth are more pleasurable and uplifting than is the phenomenon of evil.
\vs p195 5:14 \pc In religion, Jesus advocated and followed the method of experience, even as modern science pursues the technique of experiment. We find God through the leadings of spiritual insight, but we approach this insight of the soul through the love of the beautiful, the pursuit of truth, loyalty to duty, and the worship of divine goodness. But of all these values, love is the true guide to real insight.
\usection{Materialism}
\vs p195 6:1 Scientists have unintentionally precipitated mankind into a materialistic panic; they have started an unthinking run on the moral bank of the ages, but this bank of human experience has vast spiritual resources; it can stand the demands being made upon it. Only unthinking men become panicky about the spiritual assets of the human race. When the materialistic\hyp{}secular panic is over, the religion of Jesus will not be found bankrupt. The spiritual bank of the kingdom of heaven will be paying out faith, hope, and moral security to all who draw upon it “in His name.”
\vs p195 6:2 No matter what the apparent conflict between materialism and the teachings of Jesus may be, you can rest assured that, in the ages to come, the teachings of the Master will fully triumph. In reality, true religion cannot become involved in any controversy with science; it is in no way concerned with material things. Religion is simply indifferent to, but sympathetic with, science, while it supremely concerns itself with the \bibemph{scientist.}
\vs p195 6:3 The pursuit of mere knowledge, without the attendant interpretation of wisdom and the spiritual insight of religious experience, eventually leads to pessimism and human despair. A little knowledge is truly disconcerting.
\vs p195 6:4 At the time of this writing the worst of the materialistic age is over; the day of a better understanding is already beginning to dawn. The higher minds of the scientific world are no longer wholly materialistic in their philosophy, but the rank and file of the people still lean in that direction as a result of former teachings. But this age of physical realism is only a passing episode in man’s life on earth. Modern science has left true religion --- the teachings of Jesus as translated in the lives of his believers --- untouched. All science has done is to destroy the childlike illusions of the misinterpretations of life.
\vs p195 6:5 Science is a quantitative experience, religion a qualitative experience, as regards man’s life on earth. Science deals with phenomena; religion, with origins, values, and goals. To assign \bibemph{causes} as an explanation of physical phenomena is to confess ignorance of ultimates and in the end only leads the scientist straight back to the first great cause --- the Universal Father of Paradise.
\vs p195 6:6 The violent swing from an age of miracles to an age of machines has proved altogether upsetting to man. The cleverness and dexterity of the false philosophies of mechanism belie their very mechanistic contentions. The fatalistic agility of the mind of a materialist forever disproves his assertions that the universe is a blind and purposeless energy phenomenon.
\vs p195 6:7 The mechanistic naturalism of some supposedly educated men and the thoughtless secularism of the man in the street are both exclusively concerned with \bibemph{things;} they are barren of all real values, sanctions, and satisfactions of a spiritual nature, as well as being devoid of faith, hope, and eternal assurances. One of the great troubles with modern life is that man thinks he is too busy to find time for spiritual meditation and religious devotion.
\vs p195 6:8 Materialism reduces man to a soulless automaton and constitutes him merely an arithmetical symbol finding a helpless place in the mathematical formula of an unromantic and mechanistic universe. But whence comes all this vast universe of mathematics without a Master Mathematician? Science may expatiate on the conservation of matter, but religion validates the conservation of men’s souls --- it concerns their experience with spiritual realities and eternal values.
\vs p195 6:9 The materialistic sociologist of today surveys a community, makes a report thereon, and leaves the people as he found them. 1900 years ago, unlearned Galileans surveyed Jesus giving his life as a spiritual contribution to man’s inner experience and then went out and turned the whole Roman Empire upside down.
\vs p195 6:10 But religious leaders are making a great mistake when they try to call modern man to spiritual battle with the trumpet blasts of the Middle Ages. Religion must provide itself with new and up\hyp{}to\hyp{}date slogans. Neither democracy nor any other political panacea will take the place of spiritual progress. False religions may represent an evasion of reality, but Jesus in his gospel introduced mortal man to the very entrance upon an eternal reality of spiritual progression.
\vs p195 6:11 To say that mind “emerged” from matter explains nothing. If the universe were merely a mechanism and mind were unapart from matter, we would never have two differing interpretations of any observed phenomenon. The concepts of truth, beauty, and goodness are not inherent in either physics or chemistry. A machine cannot \bibemph{know,} much less know truth, hunger for righteousness, and cherish goodness.
\vs p195 6:12 Science may be physical, but the mind of the truth\hyp{}discerning scientist is at once supermaterial. Matter knows not truth, neither can it love mercy nor delight in spiritual realities. Moral convictions based on spiritual enlightenment and rooted in human experience are just as real and certain as mathematical deductions based on physical observations, but on another and higher level.
\vs p195 6:13 If men were only machines, they would react more or less uniformly to a material universe. Individuality, much less personality, would be nonexistent.
\vs p195 6:14 \pc The fact of the absolute mechanism of Paradise at the centre of the universe of universes, in the presence of the unqualified volition of the Second Source and Centre, makes forever certain that determiners are not the exclusive law of the cosmos. Materialism is there, but it is not exclusive; mechanism is there, but it is not unqualified; determinism is there, but it is not alone.
\vs p195 6:15 The finite universe of matter would eventually become uniform and deterministic but for the combined presence of mind and spirit. The influence of the cosmic mind constantly injects spontaneity into even the material worlds.
\vs p195 6:16 Freedom or initiative in any realm of existence is directly proportional to the degree of spiritual influence and cosmic\hyp{}mind control; that is, in human experience, the degree of the actuality of doing “the Father’s will.” And so, when you once start out to find God, that is the conclusive proof that God has already found you.
\vs p195 6:17 The sincere pursuit of goodness, beauty, and truth leads to God. And every scientific discovery demonstrates the existence of both freedom and uniformity in the universe. The discoverer was free to make the discovery. The thing discovered is real and apparently uniform, or else it could not have become known as a \bibemph{thing.}
\usection{The Vulnerability of Materialism}
\vs p195 7:1 How foolish it is for material\hyp{}minded man to allow such vulnerable theories as those of a mechanistic universe to deprive him of the vast spiritual resources of the personal experience of true religion. Facts never quarrel with real spiritual faith; theories may. Better that science should be devoted to the destruction of superstition rather than attempting the overthrow of religious faith --- human belief in spiritual realities and divine values.
\vs p195 7:2 Science should do for man materially what religion does for him spiritually: extend the horizon of life and enlarge his personality. True science can have no lasting quarrel with true religion. The “scientific method” is merely an intellectual yardstick wherewith to measure material adventures and physical achievements. But being material and wholly intellectual, it is utterly useless in the evaluation of spiritual realities and religious experiences.
\vs p195 7:3 The inconsistency of the modern mechanist is: If this were merely a material universe and man only a machine, such a man would be wholly unable to recognize himself as such a machine, and likewise would such a machine\hyp{}man be wholly unconscious of the fact of the existence of such a material universe. The materialistic dismay and despair of a mechanistic science has failed to recognize the fact of the spirit\hyp{}indwelt mind of the scientist whose very supermaterial insight formulates these mistaken and self\hyp{}contradictory \bibemph{concepts} of a materialistic universe.
\vs p195 7:4 Paradise values of eternity and infinity, of truth, beauty, and goodness, are concealed within the facts of the phenomena of the universes of time and space. But it requires the eye of faith in a spirit\hyp{}born mortal to detect and discern these spiritual values.
\vs p195 7:5 The realities and values of spiritual progress are not a “psychologic projection” --- a mere glorified daydream of the material mind. Such things are the spiritual forecasts of the indwelling Adjuster, the spirit of God living in the mind of man. And let not your dabblings with the faintly glimpsed findings of “relativity” disturb your concepts of the eternity and infinity of God. And in all your solicitation concerning the necessity for \bibemph{self\hyp{}expression} do not make the mistake of failing to provide for \bibemph{Adjuster\hyp{}expression,} the manifestation of your real and better self.
\vs p195 7:6 If this were only a material universe, material man would never be able to arrive at the concept of the mechanistic character of such an exclusively material existence. This very \bibemph{mechanistic concept} of the universe is in itself a nonmaterial phenomenon of mind, and all mind is of nonmaterial origin, no matter how thoroughly it may appear to be materially conditioned and mechanistically controlled.
\vs p195 7:7 The partially evolved mental mechanism of mortal man is not overendowed with consistency and wisdom. Man’s conceit often outruns his reason and eludes his logic.
\vs p195 7:8 The very pessimism of the most pessimistic materialist is, in and of itself, sufficient proof that the universe of the pessimist is not wholly material. Both optimism and pessimism are concept reactions in a mind conscious of \bibemph{values} as well as of \bibemph{facts.} If the universe were truly what the materialist regards it to be, man as a human machine would then be devoid of all conscious recognition of that very \bibemph{fact.} Without the consciousness of the concept of \bibemph{values} within the spirit\hyp{}born mind, the fact of universe materialism and the mechanistic phenomena of universe operation would be wholly unrecognized by man. One machine cannot be conscious of the nature or value of another machine.
\vs p195 7:9 A mechanistic philosophy of life and the universe cannot be scientific because science recognizes and deals only with materials and facts. Philosophy is inevitably superscientific. Man is a material fact of nature, but his \bibemph{life} is a phenomenon which transcends the material levels of nature in that it exhibits the control attributes of mind and the creative qualities of spirit.
\vs p195 7:10 The sincere effort of man to become a mechanist represents the tragic phenomenon of that man’s futile effort to commit intellectual and moral suicide. But he cannot do it.
\vs p195 7:11 If the universe were only material and man only a machine, there would be no science to embolden the scientist to postulate this mechanization of the universe. Machines cannot measure, classify, nor evaluate themselves. Such a scientific piece of work could be executed only by some entity of supermachine status.
\vs p195 7:12 If universe reality is only one vast machine, then man must be outside of the universe and apart from it in order to recognize such a \bibemph{fact} and become conscious of the \bibemph{insight} of such an \bibemph{evaluation.}
\vs p195 7:13 \pc If man is only a machine, by what technique does this man come to \bibemph{believe} or claim to \bibemph{know} that he is only a machine? The experience of self\hyp{}conscious evaluation of one’s self is never an attribute of a mere machine. A self\hyp{}conscious and avowed mechanist is the best possible answer to mechanism. If materialism were a fact, there could be no self\hyp{}conscious mechanist. It is also true that one must first be a moral person before one can perform immoral acts.
\vs p195 7:14 \pc The very claim of materialism implies a supermaterial consciousness of the mind which presumes to assert such dogmas. A mechanism might deteriorate, but it could never progress. Machines do not think, create, dream, aspire, idealize, hunger for truth, or thirst for righteousness. They do not motivate their lives with the passion to serve other machines and to choose as their goal of eternal progression the sublime task of finding God and striving to be like him. Machines are never intellectual, emotional, aesthetic, ethical, moral, or spiritual.
\vs p195 7:15 Art proves that man is not mechanistic, but it does not prove that he is spiritually immortal. Art is mortal morontia, the intervening field between man, the material, and man, the spiritual. Poetry is an effort to escape from material realities to spiritual values.
\vs p195 7:16 In a high civilization, art humanizes science, while in turn it is spiritualized by true religion --- insight into spiritual and eternal values. Art represents the human and time\hyp{}space evaluation of reality. Religion \bibemph{is} the divine embrace of cosmic values and connotes eternal progression in spiritual ascension and expansion. The art of time is dangerous only when it becomes blind to the spirit standards of the divine patterns which eternity reflects as the reality shadows of time. True art is the effective manipulation of the material things of life; religion is the ennobling transformation of the material facts of life, and it never ceases in its spiritual evaluation of art.
\vs p195 7:17 \pc How foolish to presume that an automaton could conceive a philosophy of automatism, and how ridiculous that it should presume to form such a concept of other and fellow automatons!
\vs p195 7:18 \pc Any scientific interpretation of the material universe is valueless unless it provides due recognition for the \bibemph{scientist.} No appreciation of art is genuine unless it accords recognition to the \bibemph{artist.} No evaluation of morals is worth while unless it includes the \bibemph{moralist.} No recognition of philosophy is edifying if it ignores the \bibemph{philosopher,} and religion cannot exist without the real experience of the \bibemph{religionist} who, in and through this very experience, is seeking to find God and to know him. Likewise is the universe of universes without significance apart from the I AM, the infinite God who made it and unceasingly manages it.
\vs p195 7:19 \pc Mechanists --- humanists --- tend to drift with the material currents. Idealists and spiritists \bibemph{dare} to use their oars with intelligence and vigour in order to modify the apparently purely material course of the energy streams.
\vs p195 7:20 \pc Science lives by the mathematics of the mind; music expresses the tempo of the emotions. Religion is the spiritual rhythm of the soul in time\hyp{}space harmony with the higher and eternal melody measurements of Infinity. Religious experience is something in human life which is truly supermathematical.
\vs p195 7:21 In language, an alphabet represents the mechanism of materialism, while the words expressive of the meaning of 1,000 thoughts, grand ideas, and noble ideals --- of love and hate, of cowardice and courage --- represent the performances of mind within the scope defined by both material and spiritual law, directed by the assertion of the will of personality, and limited by the inherent situational endowment.
\vs p195 7:22 The universe is not like the laws, mechanisms, and the uniformities which the scientist discovers, and which he comes to regard as science, but rather like the curious, thinking, choosing, creative, combining, and discriminating \bibemph{scientist} who thus observes universe phenomena and classifies the mathematical facts inherent in the mechanistic phases of the material side of creation. Neither is the universe like the art of the artist, but rather like the striving, dreaming, aspiring, and advancing \bibemph{artist} who seeks to transcend the world of material things in an effort to achieve a spiritual goal.
\vs p195 7:23 The scientist, not science, perceives the reality of an evolving and advancing universe of energy and matter. The artist, not art, demonstrates the existence of the transient morontia world intervening between material existence and spiritual liberty. The religionist, not religion, proves the existence of the spirit realities and divine values which are to be encountered in the progress of eternity.
\usection{Secular Totalitarianism}
\vs p195 8:1 But even after materialism and mechanism have been more or less vanquished, the devastating influence of XX century secularism will still blight the spiritual experience of millions of unsuspecting souls.
\vs p195 8:2 Modern secularism has been fostered by two world\hyp{}wide influences. The father of secularism was the narrow\hyp{}minded and godless attitude of XIX and XX century so\hyp{}called science --- atheistic science. The mother of modern secularism was the totalitarian medieval Christian church. Secularism had its inception as a rising protest against the almost complete domination of Western civilization by the institutionalized Christian church.
\vs p195 8:3 At the time of this revelation, the prevailing intellectual and philosophical climate of both European and American life is decidedly secular --- humanistic. For 300 years Western thinking has been progressively secularized. Religion has become more and more a nominal influence, largely a ritualistic exercise. The majority of professed Christians of Western civilization are unwittingly actual secularists.
\vs p195 8:4 It required a great power, a mighty influence, to free the thinking and living of the Western peoples from the withering grasp of a totalitarian ecclesiastical domination. Secularism did break the bonds of church control, and now in turn it threatens to establish a new and godless type of mastery over the hearts and minds of modern man. The tyrannical and dictatorial political state is the direct offspring of scientific materialism and philosophic secularism. Secularism no sooner frees man from the domination of the institutionalized church than it sells him into slavish bondage to the totalitarian state. Secularism frees man from ecclesiastical slavery only to betray him into the tyranny of political and economic slavery.
\vs p195 8:5 \pc Materialism denies God, secularism simply ignores him; at least that was the earlier attitude. More recently, secularism has assumed a more militant attitude, assuming to take the place of the religion whose totalitarian bondage it onetime resisted. XX century secularism tends to affirm that man does not need God. But beware! this godless philosophy of human society will lead only to unrest, animosity, unhappiness, war, and world\hyp{}wide disaster.
\vs p195 8:6 \pc Secularism can never bring peace to mankind. Nothing can take the place of God in human society. But mark you well! do not be quick to surrender the beneficent gains of the secular revolt from ecclesiastical totalitarianism. Western civilization today enjoys many liberties and satisfactions as a result of the secular revolt. The great mistake of secularism was this: In revolting against the almost total control of life by religious authority, and after attaining the liberation from such ecclesiastical tyranny, the secularists went on to institute a revolt against God himself, sometimes tacitly and sometimes openly.
\vs p195 8:7 To the secularistic revolt you owe the amazing creativity of American industrialism and the unprecedented material progress of Western civilization. And because the secularistic revolt went too far and lost sight of God and \bibemph{true} religion, there also followed the unlooked\hyp{}for harvest of world wars and international unsettledness.
\vs p195 8:8 It is not necessary to sacrifice faith in God in order to enjoy the blessings of the modern secularistic revolt: tolerance, social service, democratic government, and civil liberties. It was not necessary for the secularists to antagonize true religion in order to promote science and to advance education.
\vs p195 8:9 But secularism is not the sole parent of all these recent gains in the enlargement of living. Behind the gains of the XX century are not only science and secularism but also the unrecognized and unacknowledged spiritual workings of the life and teaching of Jesus of Nazareth.
\vs p195 8:10 Without God, without religion, scientific secularism can never co\hyp{}ordinate its forces, harmonize its divergent and rivalrous interests, races, and nationalisms. This secularistic human society, notwithstanding its unparalleled materialistic achievement, is slowly disintegrating. The chief cohesive force resisting this disintegration of antagonism is nationalism. And nationalism is the chief barrier to world peace.
\vs p195 8:11 The inherent weakness of secularism is that it discards ethics and religion for politics and power. You simply cannot establish the brotherhood of men while ignoring or denying the fatherhood of God.
\vs p195 8:12 Secular social and political optimism is an illusion. Without God, neither freedom and liberty, nor property and wealth will lead to peace.
\vs p195 8:13 The complete secularization of science, education, industry, and society can lead only to disaster. During the first third of the XX century Urantians killed more human beings than were killed during the whole of the Christian dispensation up to that time. And this is only the beginning of the dire harvest of materialism and secularism; still more terrible destruction is yet to come.
\usection{Christianity’s Problem}
\vs p195 9:1 Do not overlook the value of your spiritual heritage, the river of truth running down through the centuries, even to the barren times of a materialistic and secular age. In all your worthy efforts to rid yourselves of the superstitious creeds of past ages, make sure that you hold fast the eternal truth. But be patient! when the present superstition revolt is over, the truths of Jesus’ gospel will persist gloriously to illuminate a new and better way.
\vs p195 9:2 But paganized and socialized Christianity stands in need of new contact with the uncompromised teachings of Jesus; it languishes for lack of a new vision of the Master’s life on earth. A new and fuller revelation of the religion of Jesus is destined to conquer an empire of materialistic secularism and to overthrow a world sway of mechanistic naturalism. Urantia is now quivering on the very brink of one of its most amazing and enthralling epochs of social readjustment, moral quickening, and spiritual enlightenment.
\vs p195 9:3 The teachings of Jesus, even though greatly modified, survived the mystery cults of their birthtime, the ignorance and superstition of the dark ages, and are even now slowly triumphing over the materialism, mechanism, and secularism of the XX century. And such times of great testing and threatened defeat are always times of great revelation.
\vs p195 9:4 \pc Religion does need new leaders, spiritual men and women who will dare to depend solely on Jesus and his incomparable teachings. If Christianity persists in neglecting its spiritual mission while it continues to busy itself with social and material problems, the spiritual renaissance must await the coming of these new teachers of Jesus’ religion who will be exclusively devoted to the spiritual regeneration of men. And then will these spirit\hyp{}born souls quickly supply the leadership and inspiration requisite for the social, moral, economic, and political reorganization of the world.
\vs p195 9:5 The modern age will refuse to accept a religion which is inconsistent with facts and out of harmony with its highest conceptions of truth, beauty, and goodness. The hour is striking for a rediscovery of the true and original foundations of present\hyp{}day distorted and compromised Christianity --- the real life and teachings of Jesus.
\vs p195 9:6 \pc Primitive man lived a life of superstitious bondage to religious fear. Modern, civilized men dread the thought of falling under the dominance of strong religious convictions. Thinking man has always feared to be \bibemph{held} by a religion. When a strong and moving religion threatens to dominate him, he invariably tries to rationalize, traditionalize, and institutionalize it, thereby hoping to gain control of it. By such procedure, even a revealed religion becomes man\hyp{}made and man\hyp{}dominated. Modern men and women of intelligence evade the religion of Jesus because of their fears of what it will do \bibemph{to} them --- and \bibemph{with} them. And all such fears are well founded. The religion of Jesus does, indeed, dominate and transform its believers, demanding that men dedicate their lives to seeking for a knowledge of the will of the Father in heaven and requiring that the energies of living be consecrated to the unselfish service of the brotherhood of man.
\vs p195 9:7 Selfish men and women simply will not pay such a price for even the greatest spiritual treasure ever offered mortal man. Only when man has become sufficiently disillusioned by the sorrowful disappointments attendant upon the foolish and deceptive pursuits of selfishness, and subsequent to the discovery of the barrenness of formalized religion, will he be disposed to turn wholeheartedly to the gospel of the kingdom, the religion of Jesus of Nazareth.
\vs p195 9:8 The world needs more firsthand religion. Even Christianity --- the best of the religions of the XX century --- is not only a religion \bibemph{about} Jesus, but it is so largely one which men experience secondhand. They take their religion wholly as handed down by their accepted religious teachers. What an awakening the world would experience if it could only see Jesus as he really lived on earth and know, firsthand, his life\hyp{}giving teachings! Descriptive words of things beautiful cannot thrill like the sight thereof, neither can creedal words inspire men’s souls like the experience of knowing the presence of God. But expectant faith will ever keep the hope\hyp{}door of man’s soul open for the entrance of the eternal spiritual realities of the divine values of the worlds beyond.
\vs p195 9:9 \pc Christianity has dared to lower its ideals before the challenge of human greed, war\hyp{}madness, and the lust for power; but the religion of Jesus stands as the unsullied and transcendent spiritual summons, calling to the best there is in man to rise above all these legacies of animal evolution and, by grace, attain the moral heights of true human destiny.
\vs p195 9:10 Christianity is threatened by slow death from formalism, over\hyp{}organization, intellectualism, and other nonspiritual\tunemarkup{pgnexus10}{\linebreak} trends. The modern Christian church is not such a brotherhood of dynamic believers as Jesus commissioned continuously to effect the spiritual transformation of successive generations of mankind.
\vs p195 9:11 So\hyp{}called Christianity has become a social and cultural movement as well as a religious belief and practice. The stream of modern Christianity drains many an ancient pagan swamp and many a barbarian morass; many olden cultural watersheds drain into this present\hyp{}day cultural stream as well as the high Galilean tablelands which are supposed to be its exclusive source.
\usection{The Future}
\vs p195 10:1 Christianity has indeed done a great service for this world, but what is now most needed is Jesus. The world needs to see Jesus living again on earth in the experience of spirit\hyp{}born mortals who effectively reveal the Master to all men. It is futile to talk about a revival of primitive Christianity; you must go forward from where you find yourselves. Modern culture must become spiritually baptized with a new revelation of Jesus’ life and illuminated with a new understanding of his gospel of eternal salvation. And when Jesus becomes thus lifted up, he will draw all men to himself. Jesus’ disciples should be more than conquerors, even overflowing sources of inspiration and enhanced living to all men. Religion is only an exalted humanism until it is made divine by the discovery of the reality of the presence of God in personal experience.
\vs p195 10:2 The beauty and sublimity, the humanity and divinity, the simplicity and uniqueness, of Jesus’ life on earth present such a striking and appealing picture of man\hyp{}saving and God\hyp{}revealing that the theologians and philosophers of all time should be effectively restrained from daring to form creeds or create theological systems of spiritual bondage out of such a transcendental bestowal of God in the form of man. In Jesus the universe produced a mortal man in whom the spirit of love triumphed over the material handicaps of time and overcame the fact of physical origin.
\vs p195 10:3 \pc Ever bear in mind --- God and men need each other. They are mutually necessary to the full and final attainment of eternal personality experience in the divine destiny of universe finality.
\vs p195 10:4 \textcolour{ubdarkred}{“The kingdom of God is within you”} was probably the greatest pronouncement Jesus ever made, next to the declaration that his Father is a living and loving spirit.
\vs p195 10:5 \pc In winning souls for the Master, it is not the first mile of compulsion, duty, or convention that will transform man and his world, but rather the \bibemph{second} mile of free service and liberty\hyp{}loving devotion that betokens the Jesusonian reaching forth to grasp his brother in love and sweep him on under spiritual guidance toward the higher and divine goal of mortal existence. Christianity even now willingly goes the \bibemph{first} mile, but mankind languishes and stumbles along in moral darkness because there are so few genuine second\hyp{}milers --- so few professed followers of Jesus who really live and love as he taught his disciples to live and love and serve.
\vs p195 10:6 The call to the adventure of building a new and transformed human society by means of the spiritual rebirth of Jesus’ brotherhood of the kingdom should thrill all who believe in him as men have not been stirred since the days when they walked about on earth as his companions in the flesh.
\vs p195 10:7 No social system or political regime which denies the reality of God can contribute in any constructive and lasting manner to the advancement of human civilization. But Christianity, as it is subdivided and secularized today, presents the greatest single obstacle to its further advancement; especially is this true concerning the Orient.
\vs p195 10:8 \pc Ecclesiasticism is at once and forever incompatible with that living faith, growing spirit, and firsthand experience of the faith\hyp{}comrades of Jesus in the brotherhood of man in the spiritual association of the kingdom of heaven. The praiseworthy desire to preserve traditions of past achievement often leads to the defence of outgrown systems of worship. The well\hyp{}meant desire to foster ancient thought systems effectually prevents the sponsoring of new and adequate means and methods designed to satisfy the spiritual longings of the expanding and advancing minds of modern men. Likewise, the Christian churches of the XX century stand as great, but wholly unconscious, obstacles to the immediate advance of the real gospel --- the teachings of Jesus of Nazareth.
\vs p195 10:9 Many earnest persons who would gladly yield loyalty to the Christ of the gospel find it very difficult enthusiastically to support a church which exhibits so little of the spirit of his life and teachings, and which they have been erroneously taught he founded. Jesus did not found the so\hyp{}called Christian church, but he has, in every manner consistent with his nature, \bibemph{fostered} it as the best existent exponent of his lifework on earth.
\vs p195 10:10 If the Christian church would only dare to espouse the Master’s program, thousands of apparently indifferent youths would rush forward to enlist in such a spiritual undertaking, and they would not hesitate to go all the way through with this great adventure.
\vs p195 10:11 Christianity is seriously confronted with the doom embodied in one of its own slogans: “A house divided against itself cannot stand.” The non\hyp{}Christian world will hardly capitulate to a sect\hyp{}divided Christendom. The living Jesus is the only hope of a possible unification of Christianity. The true church --- the Jesus brotherhood --- is invisible, spiritual, and is characterized by \bibemph{unity,} not necessarily by \bibemph{uniformity.} Uniformity is the earmark of the physical world of mechanistic nature. Spiritual unity is the fruit of faith union with the living Jesus. The visible church should refuse longer to handicap the progress of the invisible and spiritual brotherhood of the kingdom of God. And this brotherhood is destined to become a \bibemph{living organism} in contrast to an institutionalized social organization. It may well utilize such social organizations, but it must not be supplanted by them.
\vs p195 10:12 But the Christianity of even the XX century must not be despised. It is the product of the combined moral genius of the God\hyp{}knowing men of many races during many ages, and it has truly been one of the greatest powers for good on earth, and therefore no man should lightly regard it, notwithstanding its inherent and acquired defects. Christianity still contrives to move the minds of reflective men with mighty moral emotions.
\vs p195 10:13 But there is no excuse for the involvement of the church in commerce and politics; such unholy alliances are a flagrant betrayal of the Master. And the genuine lovers of truth will be slow to forget that this powerful institutionalized church has often dared to smother newborn faith and persecute truth bearers who chanced to appear in unorthodox raiment.
\vs p195 10:14 It is all too true that such a church would not have survived unless there had been men in the world who preferred such a style of worship. Many spiritually indolent souls crave an ancient and authoritative religion of ritual and sacred traditions. Human evolution and spiritual progress are hardly sufficient to enable all men to dispense with religious authority. And the invisible brotherhood of the kingdom may well include these family groups of various social and temperamental classes if they are only willing to become truly spirit\hyp{}led sons of God. But in this brotherhood of Jesus there is no place for sectarian rivalry, group bitterness, nor assertions of moral superiority and spiritual infallibility.
\vs p195 10:15 These various groupings of Christians may serve to accommodate numerous different types of would\hyp{}be believers among the various peoples of Western civilization, but such division of Christendom presents a grave weakness when it attempts to carry the gospel of Jesus to Oriental peoples. These races do not yet understand that there is a \bibemph{religion of Jesus} separate, and somewhat apart, from Christianity, which has more and more become a \bibemph{religion about Jesus.}
\vs p195 10:16 The great hope of Urantia lies in the possibility of a new revelation of Jesus with a new and enlarged presentation of his saving message which would spiritually unite in loving service the numerous families of his present\hyp{}day professed followers.
\vs p195 10:17 Even secular education could help in this great spiritual renaissance if it would pay more attention to the work of teaching youth how to engage in life planning and character progression. The purpose of all education should be to foster and further the supreme purpose of life, the development of a majestic and well\hyp{}balanced personality. There is great need for the teaching of moral discipline in the place of so much self\hyp{}gratification. Upon such a foundation religion may contribute its spiritual incentive to the enlargement and enrichment of mortal life, even to the security and enhancement of life eternal.
\vs p195 10:18 Christianity is an extemporized religion, and therefore must it operate in low gear. High\hyp{}gear spiritual performances must await the new revelation and the more general acceptance of the real religion of Jesus. But Christianity is a mighty religion, seeing that the commonplace disciples of a crucified carpenter set in motion those teachings which conquered the Roman world in 300 years and then went on to triumph over the barbarians who overthrew Rome. This same Christianity conquered --- absorbed and exalted --- the whole stream of Hebrew theology and Greek philosophy. And then, when this Christian religion became comatose for more than 1,000 years as a result of an overdose of mysteries and paganism, it resurrected itself and virtually reconquered the whole Western world. Christianity contains enough of Jesus’ teachings to immortalize it.
\vs p195 10:19 If Christianity could only grasp more of Jesus’ teachings, it could do so much more in helping modern man to solve his new and increasingly complex problems.
\vs p195 10:20 Christianity suffers under a great handicap because it has become identified in the minds of all the world as a part of the social system, the industrial life, and the moral standards of Western civilization; and thus has Christianity unwittingly seemed to sponsor a society which staggers under the guilt of tolerating science without idealism, politics without principles, wealth without work, pleasure without restraint, knowledge without character, power without conscience, and industry without morality.
\vs p195 10:21 The hope of modern Christianity is that it should cease to sponsor the social systems and industrial policies of Western civilization while it humbly bows itself before the cross it so valiantly extols, there to learn anew from Jesus of Nazareth the greatest truths mortal man can ever hear --- the living gospel of the fatherhood of God and the brotherhood of man.
\quizlink
