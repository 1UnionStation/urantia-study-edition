\upaper{70}{The Evolution of Human Government}
\author{Melchizedek}
\vs p070 0:1 No sooner had man partially solved the problem of making a living than he was confronted with the task of regulating human contacts. The development of industry demanded law, order, and social adjustment; private property necessitated government.
\vs p070 0:2 On an evolutionary world, antagonisms are natural; peace is secured only by some sort of social regulative system. Social regulation is inseparable from social organization; association implies some controlling authority. Government compels the co\hyp{}ordination of the antagonisms of the tribes, clans, families, and individuals.
\vs p070 0:3 Government is an unconscious development; it evolves by trial and error. It does have survival value; therefore it becomes traditional. Anarchy augmented misery; therefore government, comparative law and order, slowly emerged or is emerging. The coercive demands of the struggle for existence literally drove the human race along the progressive road to civilization.
\usection{1.\bibnobreakspace The Genesis of War}
\vs p070 1:1 War is the natural state and heritage of evolving man; peace is the social yardstick measuring civilization’s advancement. Before the partial socialization of the advancing races man was exceedingly individualistic, extremely suspicious, and unbelievably quarrelsome. Violence is the law of nature, hostility the automatic reaction of the children of nature, while war is but these same activities carried on collectively. And wherever and whenever the fabric of civilization becomes stressed by the complications of society’s advancement, there is always an immediate and ruinous reversion to these early methods of violent adjustment of the irritations of human interassociations.
\vs p070 1:2 War is an animalistic reaction to misunderstandings and irritations; peace attends upon the civilized solution of all such problems and difficulties. The Sangik races, together with the later deteriorated Adamites and Nodites, were all belligerent. The Andonites were early taught the golden rule, and, even today, their Eskimo descendants live very much by that code; custom is strong among them, and they are fairly free from violent antagonisms.
\vs p070 1:3 Andon taught his children to settle disputes by each beating a tree with a stick, meanwhile cursing the tree; the one whose stick broke first was the victor. The later Andonites used to settle disputes by holding a public show at which the disputants made fun of and ridiculed each other, while the audience decided the winner by its applause.
\vs p070 1:4 But there could be no such phenomenon as war until society had evolved sufficiently far to actually experience periods of peace and to sanction warlike practices. The very concept of war implies some degree of organization.
\vs p070 1:5 With the emergence of social groupings, individual irritations began to be submerged in the group feelings, and this promoted intratribal tranquillity but at the expense of intertribal peace. Peace was thus first enjoyed by the in\hyp{}group, or tribe, who always disliked and hated the out\hyp{}group, foreigners. Early man regarded it a virtue to shed alien blood.
\vs p070 1:6 But even this did not work at first. When the early chiefs would try to iron out misunderstandings, they often found it necessary, at least once a year, to permit the tribal stone fights. The clan would divide up into two groups and engage in an all\hyp{}day battle. And this for no other reason than just the fun of it; they really enjoyed fighting.
\vs p070 1:7 \pc Warfare persists because man is human, evolved from an animal, and all animals are bellicose. Among the early causes of war were:
\vs p070 1:8 \ublistelem{1.}\bibnobreakspace \bibemph{Hunger,} which led to food raids. Scarcity of land has always brought on war, and during these struggles the early peace tribes were practically exterminated.
\vs p070 1:9 \ublistelem{2.}\bibnobreakspace \bibemph{Woman scarcity ---} an attempt to relieve a shortage of domestic help. Woman stealing has always caused war.
\vs p070 1:10 \ublistelem{3.}\bibnobreakspace \bibemph{Vanity ---} the desire to exhibit tribal prowess. Superior groups would fight to impose their mode of life upon inferior peoples.
\vs p070 1:11 \ublistelem{4.}\bibnobreakspace \bibemph{Slaves ---} need of recruits for the labour ranks.
\vs p070 1:12 \ublistelem{5.}\bibnobreakspace \bibemph{Revenge} was the motive for war when one tribe believed that a neighbouring tribe had caused the death of a fellow tribesman. Mourning was continued until a head was brought home. The war for vengeance was in good standing right on down to comparatively modern times.
\vs p070 1:13 \ublistelem{6.}\bibnobreakspace \bibemph{Recreation ---} war was looked upon as recreation by the young men of these early times. If no good and sufficient pretext for war arose, when peace became oppressive, neighbouring tribes were accustomed to go out in semifriendly combat to engage in a foray as a holiday, to enjoy a sham battle.
\vs p070 1:14 \ublistelem{7.}\bibnobreakspace \bibemph{Religion ---} the desire to make converts to the cult. The primitive religions all sanctioned war. Only in recent times has religion begun to frown upon war. The early priesthoods were, unfortunately, usually allied with the military power. One of the great peace moves of the ages has been the attempt to separate church and state.
\vs p070 1:15 \pc Always these olden tribes made war at the bidding of their gods, at the behest of their chiefs or medicine men. The Hebrews believed in such a “God of battles”; and the narrative of their raid on the Midianites is a typical recital of the atrocious cruelty of the ancient tribal wars; this assault, with its slaughter of all the males and the later killing of all male children and all women who were not virgins, would have done honour to the mores of a tribal chieftain of 200,000 years ago. And all this was executed in the “name of the Lord God of Israel.”
\vs p070 1:16 This is a narrative of the evolution of society --- the natural outworking of the problems of the races --- man working out his own destiny on earth. Such atrocities are not instigated by Deity, notwithstanding the tendency of man to place the responsibility on his gods.
\vs p070 1:17 \pc Military mercy has been slow in coming to mankind. Even when a woman, Deborah, ruled the Hebrews, the same wholesale cruelty persisted. Her general in his victory over the gentiles caused “all the host to fall upon the sword; there was not one left.”
\vs p070 1:18 Very early in the history of the race, poisoned weapons were used. All sorts of mutilations were practised. Saul did not hesitate to require 100 Philistine foreskins as the dowry David should pay for his daughter Michal.
\vs p070 1:19 Early wars were fought between tribes as a whole, but in later times, when two individuals in different tribes had a dispute, instead of both tribes fighting, the two disputants engaged in a duel. It also became a custom for two armies to stake all on the outcome of a contest between a representative chosen from each side, as in the instance of David and Goliath.
\vs p070 1:20 The first refinement of war was the taking of prisoners. Next, women were exempted from hostilities, and then came the recognition of noncombatants. Military castes and standing armies soon developed to keep pace with the increasing complexity of combat. Such warriors were early prohibited from associating with women, and women long ago ceased to fight, though they have always fed and nursed the soldiers and urged them on to battle.
\vs p070 1:21 The practice of declaring war represented great progress. Such declarations of intention to fight betokened the arrival of a sense of fairness, and this was followed by the gradual development of the rules of “civilized” warfare. Very early it became the custom not to fight near religious sites and, still later, not to fight on certain holy days. Next came the general recognition of the right of asylum; political fugitives received protection.
\vs p070 1:22 Thus did warfare gradually evolve from the primitive man hunt to the somewhat more orderly system of the later\hyp{}day “civilized” nations. But only slowly does the social attitude of amity displace that of enmity.
\usection{2.\bibnobreakspace The Social Value of War}
\vs p070 2:1 In past ages a fierce war would institute social changes and facilitate the adoption of new ideas such as would not have occurred naturally in 10,000 years. The terrible price paid for these certain war advantages was that society was temporarily thrown back into savagery; civilized reason had to abdicate. War is strong medicine, very costly and most dangerous; while often curative of certain social disorders, it sometimes kills the patient, destroys the society.
\vs p070 2:2 The constant necessity for national defence creates many new and advanced social adjustments. Society, today, enjoys the benefit of a long list of useful innovations which were at first wholly military and is even indebted to war for the dance, one of the early forms of which was a military drill.
\vs p070 2:3 \pc War has had a social value to past civilizations because it:
\vs p070 2:4 \ublistelem{1.}\bibnobreakspace Imposed discipline, enforced co\hyp{}operation.
\vs p070 2:5 \ublistelem{2.}\bibnobreakspace Put a premium on fortitude and courage.
\vs p070 2:6 \ublistelem{3.}\bibnobreakspace Fostered and solidified nationalism.
\vs p070 2:7 \ublistelem{4.}\bibnobreakspace Destroyed weak and unfit peoples.
\vs p070 2:8 \ublistelem{5.}\bibnobreakspace Dissolved the illusion of primitive equality and selectively stratified society.
\vs p070 2:9 \pc War has had a certain evolutionary and selective value, but like slavery, it must sometime be abandoned as civilization slowly advances. Olden wars promoted travel and cultural intercourse; these ends are now better served by modern methods of transport and communication. Olden wars strengthened nations, but modern struggles disrupt civilized culture. Ancient warfare resulted in the decimation of inferior peoples; the net result of modern conflict is the selective destruction of the best human stocks. Early wars promoted organization and efficiency, but these have now become the aims of modern industry. During past ages war was a social ferment which pushed civilization forward; this result is now better attained by ambition and invention. Ancient warfare supported the concept of a God of battles, but modern man has been told that God is love. War has served many valuable purposes in the past, it has been an indispensable scaffolding in the building of civilization, but it is rapidly becoming culturally bankrupt --- incapable of producing dividends of social gain in any way commensurate with the terrible losses attendant upon its invocation.
\vs p070 2:10 At one time physicians believed in bloodletting as a cure for many diseases, but they have since discovered better remedies for most of these disorders. And so must the international bloodletting of war certainly give place to the discovery of better methods for curing the ills of nations.
\vs p070 2:11 The nations of Urantia have already entered upon the gigantic struggle between nationalistic militarism and industrialism, and in many ways this conflict is analogous to the agelong struggle between the herder\hyp{}hunter and the farmer. But if industrialism is to triumph over militarism, it must avoid the dangers which beset it. The perils of budding industry on Urantia are:
\vs p070 2:12 \ublistelem{1.}\bibnobreakspace The strong drift toward materialism, spiritual blindness.
\vs p070 2:13 \ublistelem{2.}\bibnobreakspace The worship of wealth\hyp{}power, value distortion.
\vs p070 2:14 \ublistelem{3.}\bibnobreakspace The vices of luxury, cultural immaturity.
\vs p070 2:15 \ublistelem{4.}\bibnobreakspace The increasing dangers of indolence, service insensitivity.
\vs p070 2:16 \ublistelem{5.}\bibnobreakspace The growth of undesirable racial softness, biologic deterioration.
\vs p070 2:17 \ublistelem{6.}\bibnobreakspace The threat of standardized industrial slavery, personality stagnation. Labour is ennobling but drudgery is benumbing.
\vs p070 2:18 \pc Militarism is autocratic and cruel --- savage. It promotes social organization among the conquerors but disintegrates the vanquished. Industrialism is more civilized and should be so carried on as to promote initiative and to encourage individualism. Society should in every way possible foster originality.
\vs p070 2:19 Do not make the mistake of glorifying war; rather discern what it has done for society so that you may the more accurately visualize what its substitutes must provide in order to continue the advancement of civilization. And if such adequate substitutes are not provided, then you may be sure that war will long continue.
\vs p070 2:20 Man will never accept peace as a normal mode of living until he has been thoroughly and repeatedly convinced that peace is best for his material welfare, and until society has wisely provided peaceful substitutes for the gratification of that inherent tendency periodically to let loose a collective drive designed to liberate those ever\hyp{}accumulating emotions and energies belonging to the self\hyp{}preservation reactions of the human species.
\vs p070 2:21 But even in passing, war should be honoured as the school of experience which compelled a race of arrogant individualists to submit themselves to highly concentrated authority --- a chief executive. Old\hyp{}fashioned war did select the innately great men for leadership, but modern war no longer does this. To discover leaders society must now turn to the conquests of peace: industry, science, and social achievement.
\usection{3.\bibnobreakspace Early Human Associations}
\vs p070 3:1 In the most primitive society the \bibemph{horde} is everything; even children are its common property. The evolving family displaced the horde in child rearing, while the emerging clans and tribes took its place as the social unit.
\vs p070 3:2 Sex hunger and mother love establish the family. But real government does not appear until superfamily groups have begun to form. In the prefamily days of the horde, leadership was provided by informally chosen individuals. The African Bushmen have never progressed beyond this primitive stage; they do not have chiefs in the horde.
\vs p070 3:3 \pc Families became united by blood ties in clans, aggregations of kinsmen; and these subsequently evolved into tribes, territorial communities. Warfare and external pressure forced the tribal organization upon the kinship clans, but it was commerce and trade that held these early and primitive groups together with some degree of internal peace.
\vs p070 3:4 The peace of Urantia will be promoted far more by international trade organizations than by all the sentimental sophistry of visionary peace planning. Trade relations have been facilitated by development of language and by improved methods of communication as well as by better transportation.
\vs p070 3:5 The absence of a common language has always impeded the growth of peace groups, but money has become the universal language of modern trade. Modern society is largely held together by the industrial market. The gain motive is a mighty civilizer when augmented by the desire to serve.
\vs p070 3:6 \pc In the early ages each tribe was surrounded by concentric circles of increasing fear and suspicion; hence it was once the custom to kill all strangers, later on, to enslave them. The old idea of friendship meant adoption into the clan; and clan membership was believed to survive death --- one of the earliest concepts of eternal life.
\vs p070 3:7 The ceremony of adoption consisted in drinking each other’s blood. In some groups saliva was exchanged in the place of blood drinking, this being the ancient origin of the practice of social kissing. And all ceremonies of association, whether marriage or adoption, were always terminated by feasting.
\vs p070 3:8 In later times, blood diluted with red wine was used, and eventually wine alone was drunk to seal the adoption ceremony, which was signified in the touching of the wine cups and consummated by the swallowing of the beverage. The Hebrews employed a modified form of this adoption ceremony. Their Arab ancestors made use of the oath taken while the hand of the candidate rested upon the generative organ of the tribal native. The Hebrews treated adopted aliens kindly and fraternally. “The stranger that dwells with you shall be as one born among you, and you shall love him as yourself.”
\vs p070 3:9 “Guest friendship” was a relation of temporary hospitality. When visiting guests departed, a dish would be broken in half, one piece being given the departing friend so that it would serve as a suitable introduction for a third party who might arrive on a later visit. It was customary for guests to pay their way by telling tales of their travels and adventures. The storytellers of olden times became so popular that the mores eventually forbade their functioning during either the hunting or harvest seasons.
\vs p070 3:10 The first treaties of peace were the “blood bonds.” The peace ambassadors of two warring tribes would meet, pay their respects, and then proceed to prick the skin until it bled; whereupon they would suck each other’s blood and declare peace.
\vs p070 3:11 The earliest peace missions consisted of delegations of men bringing their choice maidens for the sex gratification of their onetime enemies, the sex appetite being utilized in combating the war urge. The tribe so honoured would pay a return visit, with its offering of maidens; whereupon peace would be firmly established. And soon intermarriages between the families of the chiefs were sanctioned.
\usection{4.\bibnobreakspace Clans and Tribes}
\vs p070 4:1 The first peace group was the family, then the clan, the tribe, and later on the nation, which eventually became the modern territorial state. The fact that the present\hyp{}day peace groups have long since expanded beyond blood ties to embrace nations is most encouraging, despite the fact that Urantia nations are still spending vast sums on war preparations.
\vs p070 4:2 The clans were blood\hyp{}tie groups within the tribe, and they owed their existence to certain common interests, such as:
\vs p070 4:3 \ublistelem{1.}\bibnobreakspace Tracing origin back to a common ancestor.
\vs p070 4:4 \ublistelem{2.}\bibnobreakspace Allegiance to a common religious totem.
\vs p070 4:5 \ublistelem{3.}\bibnobreakspace Speaking the same dialect.
\vs p070 4:6 \ublistelem{4.}\bibnobreakspace Sharing a common dwelling place.
\vs p070 4:7 \ublistelem{5.}\bibnobreakspace Fearing the same enemies.
\vs p070 4:8 \ublistelem{6.}\bibnobreakspace Having had a common military experience.
\vs p070 4:9 \pc The clan headmen were always subordinate to the tribal chief, the early tribal governments being a loose confederation of clans. The native Australians never developed a tribal form of government.
\vs p070 4:10 The clan peace chiefs usually ruled through the mother line; the tribal war chiefs established the father line. The courts of the tribal chiefs and early kings consisted of the headmen of the clans, whom it was customary to invite into the king’s presence several times a year. This enabled him to watch them and the better secure their co\hyp{}operation. The clans served a valuable purpose in local self\hyp{}government, but they greatly delayed the growth of large and strong nations.
\usection{5.\bibnobreakspace The Beginnings of Government}
\vs p070 5:1 Every human institution had a beginning, and civil government is a product of progressive evolution just as much as are marriage, industry, and religion. From the early clans and primitive tribes there gradually developed the successive orders of human government which have come and gone right on down to those forms of social and civil regulation that characterize the second third of the XX century.
\vs p070 5:2 With the gradual emergence of the family units the foundations of government were established in the clan organization, the grouping of consanguineous families. The first real governmental body was the \bibemph{council of the elders.} This regulative group was composed of old men who had distinguished themselves in some efficient manner. Wisdom and experience were early appreciated even by barbaric man, and there ensued a long age of the domination of the elders. This reign of the oligarchy of age gradually grew into the patriarchal idea.
\vs p070 5:3 In the early council of the elders there resided the potential of all governmental functions: executive, legislative, and judicial. When the council interpreted the current mores, it was a court; when establishing new modes of social usage, it was a legislature; to the extent that such decrees and enactments were enforced, it was the executive. The chairman of the council was one of the forerunners of the later tribal chief.
\vs p070 5:4 Some tribes had female councils, and from time to time many tribes had women rulers. Certain tribes of the red man preserved the teaching of Onamonalonton in following the unanimous rule of the “council of seven.”
\vs p070 5:5 \pc It has been hard for mankind to learn that neither peace nor war can be run by a debating society. The primitive “palavers” were seldom useful. The race early learned that an army commanded by a group of clan heads had no chance against a strong one\hyp{}man army. War has always been a kingmaker.
\vs p070 5:6 \pc At first the war chiefs were chosen only for military service, and they would relinquish some of their authority during peacetimes, when their duties were of a more social nature. But gradually they began to encroach upon the peace intervals, tending to continue to rule from one war on through to the next. They often saw to it that one war was not too long in following another. These early war lords were not fond of peace.
\vs p070 5:7 In later times some chiefs were chosen for other than military service, being selected because of unusual physique or outstanding personal abilities. The red men often had two sets of chiefs --- the sachems, or peace chiefs, and the hereditary war chiefs. The peace rulers were also judges and teachers.
\vs p070 5:8 Some early communities were ruled by medicine men, who often acted as chiefs. One man would act as priest, physician, and chief executive. Quite often the early royal insignias had originally been the symbols or emblems of priestly dress.
\vs p070 5:9 And it was by these steps that the executive branch of government gradually came into existence. The clan and tribal councils continued in an advisory capacity and as forerunners of the later appearing legislative and judicial branches. In Africa, today, all these forms of primitive government are in actual existence among the various tribes.
\usection{6.\bibnobreakspace Monarchial Government}
\vs p070 6:1 Effective state rule only came with the arrival of a chief with full executive authority. Man found that effective government could be had only by conferring power on a personality, not by endowing an idea.
\vs p070 6:2 Rulership grew out of the idea of family authority or wealth. When a patriarchal kinglet became a real king, he was sometimes called “father of his people.” Later on, kings were thought to have sprung from heroes. And still further on, rulership became hereditary, due to belief in the divine origin of kings.
\vs p070 6:3 Hereditary kingship avoided the anarchy which had previously wrought such havoc between the death of a king and the election of a successor. The family had a biologic head; the clan, a selected natural leader; the tribe and later state had no natural leader, and this was an additional reason for making the chief\hyp{}kings hereditary. The idea of royal families and aristocracy was also based on the mores of “name ownership” in the clans.
\vs p070 6:4 The succession of kings was eventually regarded as supernatural, the royal blood being thought to extend back to the times of the materialized staff of Prince Caligastia. Thus kings became fetish personalities and were inordinately feared, a special form of speech being adopted for court usage. Even in recent times it was believed that the touch of kings would cure disease, and some Urantia peoples still regard their rulers as having had a divine origin.
\vs p070 6:5 The early fetish king was often kept in seclusion; he was regarded as too sacred to be viewed except on feast days and holy days. Ordinarily a representative was chosen to impersonate him, and this is the origin of prime ministers. The first cabinet officer was a food administrator; others shortly followed. Rulers soon appointed representatives to be in charge of commerce and religion; and the development of a cabinet was a direct step toward depersonalization of executive authority. These assistants of the early kings became the accepted nobility, and the king’s wife gradually rose to the dignity of queen as women came to be held in higher esteem.
\vs p070 6:6 \pc Unscrupulous rulers gained great power by the discovery of poison. Early court magic was diabolical; the king’s enemies soon died. But even the most despotic tyrant was subject to some restrictions; he was at least restrained by the ever\hyp{}present fear of assassination. The medicine men, witch doctors, and priests have always been a powerful check on the kings. Subsequently, the landowners, the aristocracy, exerted a restraining influence. And ever and anon the clans and tribes would simply rise up and overthrow their despots and tyrants. Deposed rulers, when sentenced to death, were often given the option of committing suicide, which gave origin to the ancient social vogue of suicide in certain circumstances.
\usection{7.\bibnobreakspace Primitive Clubs and Secret Societies}
\vs p070 7:1 Blood kinship determined the first social groups; association enlarged the kinship clan. Intermarriage was the next step in group enlargement, and the resultant complex tribe was the first true political body. The next advance in social development was the evolution of religious cults and the political clubs. These first appeared as secret societies and originally were wholly religious; subsequently they became regulative. At first they were men’s clubs; later women’s groups appeared. Presently they became divided into two classes: sociopolitical and religio\hyp{}mystical.
\vs p070 7:2 \pc There were many reasons for the secrecy of these societies, such as:
\vs p070 7:3 \ublistelem{1.}\bibnobreakspace Fear of incurring the displeasure of the rulers because of the violation of some taboo.
\vs p070 7:4 \ublistelem{2.}\bibnobreakspace In order to practice minority religious rites.
\vs p070 7:5 \ublistelem{3.}\bibnobreakspace For the purpose of preserving valuable “spirit” or trade secrets.
\vs p070 7:6 \ublistelem{4.}\bibnobreakspace For the enjoyment of some special charm or magic.
\vs p070 7:7 \pc The very secrecy of these societies conferred on all members the power of mystery over the rest of the tribe. Secrecy also appeals to vanity; the initiates were the social aristocracy of their day. After initiation the boys hunted with the men; whereas before they had gathered vegetables with the women. And it was the supreme humiliation, a tribal disgrace, to fail to pass the puberty tests and thus be compelled to remain outside the men’s abode with the women and children, to be considered effeminate. Besides, noninitiates were not allowed to marry.
\vs p070 7:8 \pc Primitive people very early taught their adolescent youths sex control. It became the custom to take boys away from parents from puberty to marriage, their education and training being intrusted to the men’s secret societies. And one of the chief functions of these clubs was to keep control of adolescent young men, thus preventing illegitimate children.
\vs p070 7:9 Commercialized prostitution began when these men’s clubs paid money for the use of women from other tribes. But the earlier groups were remarkably free from sex laxity.
\vs p070 7:10 The puberty initiation ceremony usually extended over a period of five years. Much self\hyp{}torture and painful cutting entered into these ceremonies. Circumcision was first practised as a rite of initiation into one of these secret fraternities. The tribal marks were cut on the body as a part of the puberty initiation; the tattoo originated as such a badge of membership. Such torture, together with much privation, was designed to harden these youths, to impress them with the reality of life and its inevitable hardships. This purpose is better accomplished by the later appearing athletic games and physical contests.
\vs p070 7:11 But the secret societies did aim at the improvement of adolescent morals; one of the chief purposes of the puberty ceremonies was to impress upon the boy that he must leave other men’s wives alone.
\vs p070 7:12 Following these years of rigorous discipline and training and just before marriage, the young men were usually released for a short period of leisure and freedom, after which they returned to marry and to submit to lifelong subjection to the tribal taboos. And this ancient custom has continued down to modern times as the foolish notion of “sowing wild oats.”
\vs p070 7:13 \pc Many later tribes sanctioned the formation of women’s secret clubs, the purpose of which was to prepare adolescent girls for wifehood and motherhood. After initiation girls were eligible for marriage and were permitted to attend the “bride show,” the coming\hyp{}out party of those days. Women’s orders pledged against marriage early came into existence.
\vs p070 7:14 Presently nonsecret clubs made their appearance when groups of unmarried men and groups of unattached women formed their separate organizations. These associations were really the first schools. And while men’s and women’s clubs were often given to persecuting each other, some advanced tribes, after contact with the Dalamatia teachers, experimented with coeducation, having boarding schools for both sexes.
\vs p070 7:15 \pc Secret societies contributed to the building up of social castes chiefly by the mysterious character of their initiations. The members of these societies first wore masks to frighten the curious away from their mourning rites --- ancestor worship. Later this ritual developed into a pseudo seance at which ghosts were reputed to have appeared. The ancient societies of the “new birth” used signs and employed a special secret language; they also forswore certain foods and drinks. They acted as night police and otherwise functioned in a wide range of social activities.
\vs p070 7:16 All secret associations imposed an oath, enjoined confidence, and taught the keeping of secrets. These orders awed and controlled the mobs; they also acted as vigilance societies, thus practising lynch law. They were the first spies when the tribes were at war and the first secret police during times of peace. Best of all they kept unscrupulous kings on the anxious seat. To offset them, the kings fostered their own secret police.
\vs p070 7:17 These societies gave rise to the first political parties. The first party government was “the strong” \bibemph{vs.} “the weak.” In ancient times a change of administration only followed civil war, abundant proof that the weak had become strong.
\vs p070 7:18 These clubs were employed by merchants to collect debts and by rulers to collect taxes. Taxation has been a long struggle, one of the earliest forms being the tithe, 10\%\ of the hunt or spoils. Taxes were originally levied to keep up the king’s house, but it was found that they were easier to collect when disguised as an offering for the support of the temple service.
\vs p070 7:19 By and by these secret associations grew into the first charitable organizations and later evolved into the earlier religious societies --- the forerunners of churches. Finally some of these societies became intertribal, the first international fraternities.
\usection{8.\bibnobreakspace Social Classes}
\vs p070 8:1 The mental and physical inequality of human beings ensures that social classes will appear. The only worlds without social strata are the most primitive and the most advanced. A dawning civilization has not yet begun the differentiation of social levels, while a world settled in light and life has largely effaced these divisions of mankind, which are so characteristic of all intermediate evolutionary stages.
\vs p070 8:2 As society emerged from savagery to barbarism, its human components tended to become grouped in classes for the following general reasons:
\vs p070 8:3 \ublistelem{1.}\bibnobreakspace \bibemph{Natural ---} contact, kinship, and marriage; the first social distinctions were based on sex, age, and blood --- kinship to the chief.
\vs p070 8:4 \ublistelem{2.}\bibnobreakspace \bibemph{Personal ---} the recognition of ability, endurance, skill, and fortitude; soon followed by the recognition of language mastery, knowledge, and general intelligence.
\vs p070 8:5 \ublistelem{3.}\bibnobreakspace \bibemph{Chance ---} war and emigration resulted in the separating of human groups. Class evolution was powerfully influenced by conquest, the relation of the victor to the vanquished, while slavery brought about the first general division of society into free and bond.
\vs p070 8:6 \ublistelem{4.}\bibnobreakspace \bibemph{Economic ---} rich and poor. Wealth and the possession of slaves was a genetic basis for one class of society.
\vs p070 8:7 \ublistelem{5.}\bibnobreakspace \bibemph{Geographic ---} classes arose consequent upon urban or rural settlement. City and country have respectively contributed to the differentiation of the herder\hyp{}agriculturist and the trader\hyp{}industrialist, with their divergent viewpoints and reactions.
\vs p070 8:8 \ublistelem{6.}\bibnobreakspace \bibemph{Social ---} classes have gradually formed according to popular estimate of the social worth of different groups. Among the earliest divisions of this sort were the demarcations between priest\hyp{}teachers, ruler\hyp{}warriors, capitalist\hyp{}traders, common labourers, and slaves. The slave could never become a capitalist, though sometimes the wage earner could elect to join the capitalistic ranks.
\vs p070 8:9 \ublistelem{7.}\bibnobreakspace \bibemph{Vocational ---} as vocations multiplied, they tended to establish castes and guilds. Workers divided into three groups: the professional classes, including the medicine men, then the skilled workers, followed by the unskilled labourers.
\vs p070 8:10 \ublistelem{8.}\bibnobreakspace \bibemph{Religious ---} the early cult clubs produced their own classes within the clans and tribes, and the piety and mysticism of the priests have long perpetuated them as a separate social group.
\vs p070 8:11 \ublistelem{9.}\bibnobreakspace \bibemph{Racial ---} the presence of two or more races within a given nation or territorial unit usually produces colour castes. The original caste system of India was based on colour, as was that of early Egypt.
\vs p070 8:12 \ublistelem{10.}\bibnobreakspace \bibemph{Age ---} youth and maturity. Among the tribes the boy remained under the watchcare of his father as long as the father lived, while the girl was left in the care of her mother until married.
\vs p070 8:13 \pc Flexible and shifting social classes are indispensable to an evolving civilization, but when \bibemph{class} becomes \bibemph{caste,} when social levels petrify, the enhancement of social stability is purchased by diminishment of personal initiative. Social caste solves the problem of finding one’s place in industry, but it also sharply curtails individual development and virtually prevents social co\hyp{}operation.
\vs p070 8:14 Classes in society, having naturally formed, will persist until man gradually achieves their evolutionary obliteration through intelligent manipulation of the biologic, intellectual, and spiritual resources of a progressing civilization, such as:
\vs p070 8:15 \ublistelem{1.}\bibnobreakspace Biologic renovation of the racial stocks --- the selective elimination of inferior human strains. This will tend to eradicate many mortal inequalities.
\vs p070 8:16 \ublistelem{2.}\bibnobreakspace Educational training of the increased brain power which will arise out of such biologic improvement.
\vs p070 8:17 \ublistelem{3.}\bibnobreakspace Religious quickening of the feelings of mortal kinship and brotherhood.
\vs p070 8:18 \pc But these measures can bear their true fruits only in the distant millenniums of the future, although much social improvement will immediately result from the intelligent, wise, and \bibemph{patient} manipulation of these acceleration factors of cultural progress. Religion is the mighty lever that lifts civilization from chaos, but it is powerless apart from the fulcrum of sound and normal mind resting securely on sound and normal heredity.
\usection{9.\bibnobreakspace Human Rights}
\vs p070 9:1 Nature confers no rights on man, only life and a world in which to live it. Nature does not even confer the right to live, as might be deduced by considering what would likely happen if an unarmed man met a hungry tiger face to face in the primitive forest. Society’s prime gift to man is security.
\vs p070 9:2 \pc Gradually society asserted its rights and, at the present time, they are:
\vs p070 9:3 \ublistelem{1.}\bibnobreakspace Assurance of food supply.
\vs p070 9:4 \ublistelem{2.}\bibnobreakspace Military defence --- security through preparedness.
\vs p070 9:5 \ublistelem{3.}\bibnobreakspace Internal peace preservation --- prevention of personal violence and social disorder.
\vs p070 9:6 \ublistelem{4.}\bibnobreakspace Sex control --- marriage, the family institution.
\vs p070 9:7 \ublistelem{5.}\bibnobreakspace Property --- the right to own.
\vs p070 9:8 \ublistelem{6.}\bibnobreakspace Fostering of individual and group competition.
\vs p070 9:9 \ublistelem{7.}\bibnobreakspace Provision for educating and training youth.
\vs p070 9:10 \ublistelem{8.}\bibnobreakspace Promotion of trade and commerce --- industrial development.
\vs p070 9:11 \ublistelem{9.}\bibnobreakspace Improvement of labour conditions and rewards.
\vs p070 9:12 \ublistelem{10.}\bibnobreakspace The guarantee of the freedom of religious practices to the end that all of these other social activities may be exalted by becoming spiritually motivated.
\vs p070 9:13 \pc When rights are old beyond knowledge of origin, they are often called \bibemph{natural rights.} But human rights are not really natural; they are entirely social. They are relative and ever changing, being no more than the rules of the game --- recognized adjustments of relations governing the ever\hyp{}changing phenomena of human competition.
\vs p070 9:14 What may be regarded as right in one age may not be so regarded in another. The survival of large numbers of defectives and degenerates is not because they have any natural right thus to encumber XX century civilization, but simply because the society of the age, the mores, thus decrees.
\vs p070 9:15 Few human rights were recognized in the European Middle Ages; then every man belonged to someone else, and rights were only privileges or favours granted by state or church. And the revolt from this error was equally erroneous in that it led to the belief that all men are born equal.
\vs p070 9:16 The weak and the inferior have always contended for equal rights; they have always insisted that the state compel the strong and superior to supply their wants and otherwise make good those deficiencies which all too often are the natural result of their own indifference and indolence.
\vs p070 9:17 But this equality ideal is the child of civilization; it is not found in nature. Even culture itself demonstrates conclusively the inherent inequality of men by their very unequal capacity therefor. The sudden and nonevolutionary realization of supposed natural equality would quickly throw civilized man back to the crude usages of primitive ages. Society cannot offer equal rights to all, but it can promise to administer the varying rights of each with fairness and equity. It is the business and duty of society to provide the child of nature with a fair and peaceful opportunity to pursue self\hyp{}maintenance, participate in self\hyp{}perpetuation, while at the same time enjoying some measure of self\hyp{}gratification, the sum of all three constituting human happiness.
\usection{10.\bibnobreakspace Evolution of Justice}
\vs p070 10:1 Natural justice is a man\hyp{}made theory; it is not a reality. In nature, justice is purely theoretic, wholly a fiction. Nature provides but one kind of justice --- inevitable conformity of results to causes.
\vs p070 10:2 Justice, as conceived by man, means getting one’s rights and has, therefore, been a matter of progressive evolution. The concept of justice may well be constitutive in a spirit\hyp{}endowed mind, but it does not spring full\hyp{}fledgedly into existence on the worlds of space.
\vs p070 10:3 Primitive man assigned all phenomena to a person. In case of death the savage asked, not \bibemph{what} killed him, but \bibemph{who?} Accidental murder was not therefore recognized, and in the punishment of crime the motive of the criminal was wholly disregarded; judgment was rendered in accordance with the injury done.
\vs p070 10:4 \pc In the earliest primitive society public opinion operated directly; officers of law were not needed. There was no privacy in primitive life. A man’s neighbours were responsible for his conduct; therefore their right to pry into his personal affairs. Society was regulated on the theory that the group membership should have an interest in, and some degree of control over, the behaviour of each individual.
\vs p070 10:5 It was very early believed that ghosts administered justice through the medicine men and priests; this constituted these orders the first crime detectors and officers of the law. Their early methods of detecting crime consisted in conducting ordeals of poison, fire, and pain. These savage ordeals were nothing more than crude techniques of arbitration; they did not necessarily settle a dispute justly. For example: When poison was administered, if the accused vomited, he was innocent.
\vs p070 10:6 The Old Testament records one of these ordeals, a marital guilt test: If a man suspected his wife of being untrue to him, he took her to the priest and stated his suspicions, after which the priest would prepare a concoction consisting of holy water and sweepings from the temple floor. After due ceremony, including threatening curses, the accused wife was made to drink the nasty potion. If she was guilty, “the water that causes the curse shall enter into her and become bitter, and her belly shall swell, and her thighs shall rot, and the woman shall be accursed among her people.” If, by any chance, any woman could quaff this filthy draught and not show symptoms of physical illness, she was acquitted of the charges made by her jealous husband.
\vs p070 10:7 These atrocious methods of crime detection were practised by almost all the evolving tribes at one time or another. Duelling is a modern survival of the trial by ordeal.
\vs p070 10:8 It is not to be wondered that the Hebrews and other semicivilized tribes practised such primitive techniques of justice administration 3,000 years ago, but it is most amazing that thinking men would subsequently retain such a relic of barbarism within the pages of a collection of sacred writings. Reflective thinking should make it clear that no divine being ever gave mortal man such unfair instructions regarding the detection and adjudication of suspected marital unfaithfulness.
\vs p070 10:9 \pc Society early adopted the paying\hyp{}back attitude of retaliation: an eye for an eye, a life for a life. The evolving tribes all recognized this right of blood vengeance. Vengeance became the aim of primitive life, but religion has since greatly modified these early tribal practices. The teachers of revealed religion have always proclaimed, “‘Vengeance is mine,’ says the Lord.” Vengeance killing in early times was not altogether unlike present\hyp{}day murders under the pretense of the unwritten law.
\vs p070 10:10 Suicide was a common mode of retaliation. If one were unable to avenge himself in life, he died entertaining the belief that, as a ghost, he could return and visit wrath upon his enemy. And since this belief was very general, the threat of suicide on an enemy’s doorstep was usually sufficient to bring him to terms. Primitive man did not hold life very dear; suicide over trifles was common, but the teachings of the Dalamatians greatly lessened this custom, while in more recent times leisure, comforts, religion, and philosophy have united to make life sweeter and more desirable. Hunger strikes are, however, a modern analogue of this old\hyp{}time method of retaliation.
\vs p070 10:11 One of the earliest formulations of advanced tribal law had to do with the taking over of the blood feud as a tribal affair. But strange to relate, even then a man could kill his wife without punishment provided he had fully paid for her. The Eskimos of today, however, still leave the penalty for a crime, even for murder, to be decreed and administered by the family wronged.
\vs p070 10:12 Another advance was the imposition of fines for taboo violations, the provision of penalties. These fines constituted the first public revenue. The practice of paying “blood money” also came into vogue as a substitute for blood vengeance. Such damages were usually paid in women or cattle; it was a long time before actual fines, monetary compensation, were assessed as punishment for crime. And since the idea of punishment was essentially compensation, everything, including human life, eventually came to have a price which could be paid as damages. The Hebrews were the first to abolish the practice of paying blood money. Moses taught that they should “take no satisfaction for the life of a murderer, who is guilty of death; he shall surely be put to death.”
\vs p070 10:13 \pc Justice was thus first meted out by the family, then by the clan, and later on by the tribe. The administration of true justice dates from the taking of revenge from private and kin groups and lodging it in the hands of the social group, the state.
\vs p070 10:14 \pc Punishment by burning alive was once a common practice. It was recognized by many ancient rulers, including Hammurabi and Moses, the latter directing that many crimes, particularly those of a grave sex nature, should be punished by burning at the stake. If “the daughter of a priest” or other leading citizen turned to public prostitution, it was the Hebrew custom to “burn her with fire.”
\vs p070 10:15 Treason --- the “selling out” or betrayal of one’s tribal associates --- was the first capital crime. Cattle stealing was universally punished by summary death, and even recently horse stealing has been similarly punished. But as time passed, it was learned that the severity of the punishment was not so valuable a deterrent to crime as was its certainty and swiftness.
\vs p070 10:16 When society fails to punish crimes, group resentment usually asserts itself as lynch law; the provision of sanctuary was a means of escaping this sudden group anger. Lynching and duelling represent the unwillingness of the individual to surrender private redress to the state.
\usection{11.\bibnobreakspace Laws and Courts}
\vs p070 11:1 It is just as difficult to draw sharp distinctions between mores and laws as to indicate exactly when, at the dawning, night is succeeded by day. Mores are laws and police regulations in the making. When long established, the undefined mores tend to crystallize into precise laws, concrete regulations, and well\hyp{}defined social conventions.
\vs p070 11:2 Law is always at first negative and prohibitive; in advancing civilizations it becomes increasingly positive and directive. Early society operated negatively, granting the individual the right to live by imposing upon all others the command, “you shall not kill.” Every grant of rights or liberty to the individual involves curtailment of the liberties of all others, and this is effected by the taboo, primitive law. The whole idea of the taboo is inherently negative, for primitive society was wholly negative in its organization, and the early administration of justice consisted in the enforcement of the taboos. But originally these laws applied only to fellow tribesmen, as is illustrated by the later\hyp{}day Hebrews, who had a different code of ethics for dealing with the gentiles.
\vs p070 11:3 The oath originated in the days of Dalamatia in an effort to render testimony more truthful. Such oaths consisted in pronouncing a curse upon oneself. Formerly no individual would testify against his native group.
\vs p070 11:4 \pc Crime was an assault upon the tribal mores, sin was the transgression of those taboos which enjoyed ghost sanction, and there was long confusion due to the failure to segregate crime and sin.
\vs p070 11:5 Self\hyp{}interest established the taboo on killing, society sanctified it as traditional mores, while religion consecrated the custom as moral law, and thus did all three conspire in rendering human life more safe and sacred. Society could not have held together during early times had not rights had the sanction of religion; superstition was the moral and social police force of the long evolutionary ages. The ancients all claimed that their olden laws, the taboos, had been given to their ancestors by the gods.
\vs p070 11:6 Law is a codified record of long human experience, public opinion crystallized and legalized. The mores were the raw material of accumulated experience out of which later ruling minds formulated the written laws. The ancient judge had no laws. When he handed down a decision, he simply said, “It is the custom.”
\vs p070 11:7 Reference to precedent in court decisions represents the effort of judges to adapt written laws to the changing conditions of society. This provides for progressive adaptation to altering social conditions combined with the impressiveness of traditional continuity.
\vs p070 11:8 \pc Property disputes were handled in many ways, such as:
\vs p070 11:9 \ublistelem{1.}\bibnobreakspace By destroying the disputed property.
\vs p070 11:10 \ublistelem{2.}\bibnobreakspace By force --- the contestants fought it out.
\vs p070 11:11 \ublistelem{3.}\bibnobreakspace By arbitration --- a third party decided.
\vs p070 11:12 \ublistelem{4.}\bibnobreakspace By appeal to the elders --- later to the courts.
\vs p070 11:13 \pc The first courts were regulated fistic encounters; the judges were merely umpires or referees. They saw to it that the fight was carried on according to approved rules. On entering a court combat, each party made a deposit with the judge to pay the costs and fine after one had been defeated by the other. “Might was still right.” Later on, verbal arguments were substituted for physical blows.
\vs p070 11:14 The whole idea of primitive justice was not so much to be fair as to dispose of the contest and thus prevent public disorder and private violence. But primitive man did not so much resent what would now be regarded as an injustice; it was taken for granted that those who had power would use it selfishly. Nevertheless, the status of any civilization may be very accurately determined by the thoroughness and equity of its courts and by the integrity of its judges.
\usection{12.\bibnobreakspace Allocation of Civil Authority}
\vs p070 12:1 The great struggle in the evolution of government has concerned the concentration of power. The universe administrators have learned from experience that the evolutionary peoples on the inhabited worlds are best regulated by the representative type of civil government when there is maintained proper balance of power between the well\hyp{}co\hyp{}ordinated executive, legislative, and judicial branches.
\vs p070 12:2 \pc While primitive authority was based on strength, physical power, the ideal government is the representative system wherein leadership is based on ability, but in the days of barbarism there was entirely too much war to permit representative government to function effectively. In the long struggle between division of authority and unity of command, the dictator won. The early and diffuse powers of the primitive council of elders were gradually concentrated in the person of the absolute monarch. After the arrival of real kings the groups of elders persisted as quasi\hyp{}legislative\hyp{}judicial advisory bodies; later on, legislatures of co\hyp{}ordinate status made their appearance, and eventually supreme courts of adjudication were established separate from the legislatures.
\vs p070 12:3 The king was the executor of the mores, the original or unwritten law. Later he enforced the legislative enactments, the crystallization of public opinion. A popular assembly as an expression of public opinion, though slow in appearing, marked a great social advance.
\vs p070 12:4 The early kings were greatly restricted by the mores --- by tradition or public opinion. In recent times some Urantia nations have codified these mores into documentary bases for government.
\vs p070 12:5 \pc Urantia mortals are entitled to liberty; they should create their systems of government; they should adopt their constitutions or other charters of civil authority and administrative procedure. And having done this, they should select their most competent and worthy fellows as chief executives. For representatives in the legislative branch they should elect only those who are qualified intellectually and morally to fulfil such sacred responsibilities. As judges of their high and supreme tribunals only those who are endowed with natural ability and who have been made wise by replete experience should be chosen.
\vs p070 12:6 If men would maintain their freedom, they must, after having chosen their charter of liberty, provide for its wise, intelligent, and fearless interpretation to the end that there may be prevented:
\vs p070 12:7 \ublistelem{1.}\bibnobreakspace Usurpation of unwarranted power by either the executive or legislative branches.
\vs p070 12:8 \ublistelem{2.}\bibnobreakspace Machinations of ignorant and superstitious agitators.
\vs p070 12:9 \ublistelem{3.}\bibnobreakspace Retardation of scientific progress.
\vs p070 12:10 \ublistelem{4.}\bibnobreakspace Stalemate of the dominance of mediocrity.
\vs p070 12:11 \ublistelem{5.}\bibnobreakspace Domination by vicious minorities.
\vs p070 12:12 \ublistelem{6.}\bibnobreakspace Control by ambitious and clever would\hyp{}be dictators.
\vs p070 12:13 \ublistelem{7.}\bibnobreakspace Disastrous disruption of panics.
\vs p070 12:14 \ublistelem{8.}\bibnobreakspace Exploitation by the unscrupulous.
\vs p070 12:15 \ublistelem{9.}\bibnobreakspace Taxation enslavement of the citizenry by the state.
\vs p070 12:16 \ublistelem{10.}\bibnobreakspace Failure of social and economic fairness.
\vs p070 12:17 \ublistelem{11.}\bibnobreakspace Union of church and state.
\vs p070 12:18 \ublistelem{12.}\bibnobreakspace Loss of personal liberty.
\vs p070 12:19 \pc These are the purposes and aims of constitutional tribunals acting as governors upon the engines of representative government on an evolutionary world.
\vs p070 12:20 Mankind’s struggle to perfect government on Urantia has to do with perfecting channels of administration, with adapting them to ever\hyp{}changing current needs, with improving power distribution within government, and then with selecting such administrative leaders as are truly wise. While there is a divine and ideal form of government, such cannot be revealed but must be slowly and laboriously discovered by the men and women of each planet throughout the universes of time and space.
\vsetoff
\vs p070 12:21 [Presented by a Melchizedek of Nebadon.]
