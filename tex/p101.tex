\upaper{101}{The Real Nature of Religion}
\uminitoc{True Religion}
\uminitoc{The Fact of Religion}
\uminitoc{The Characteristics of Religion}
\uminitoc{The Limitations of Revelation}
\uminitoc{Religion Expanded by Revelation}
\uminitoc{Progressive Religious Experience}
\uminitoc{A Personal Philosophy of Religion}
\uminitoc{Faith and Belief}
\uminitoc{Religion and Morality}
\uminitoc{Religion as Man’s Liberator}
\author{Melchizedek}
\vs p101 0:1 Religion, as a human experience, ranges from the primitive fear slavery of the evolving savage up to the sublime and magnificent faith liberty of those civilized mortals who are superbly conscious of sonship with the eternal God.
\vs p101 0:2 Religion is the ancestor of the advanced ethics and morals of progressive social evolution. But religion, as such, is not merely a moral movement, albeit the outward and social manifestations of religion are mightily influenced by the ethical and moral momentum of human society. Always is religion the inspiration of man’s evolving nature, but it is not the secret of that evolution.
\vs p101 0:3 Religion, the conviction\hyp{}faith of the personality, can always triumph over the superficially contradictory logic of despair born in the unbelieving material mind. There really is a true and genuine inner voice, that “true light which lights every man who comes into the world.” And this spirit leading is distinct from the ethical prompting of human conscience. The feeling of religious assurance is more than an emotional feeling. The assurance of religion transcends the reason of the mind, even the logic of philosophy. Religion \bibemph{is} faith, trust, and assurance.
\usection{True Religion}
\vs p101 1:1 True religion is not a system of philosophic belief which can be reasoned out and substantiated by natural proofs, neither is it a fantastic and mystic experience of indescribable feelings of ecstasy which can be enjoyed only by the romantic devotees of mysticism. Religion is not the product of reason, but viewed from within, it is altogether reasonable. Religion is not derived from the logic of human philosophy, but as a mortal experience it is altogether logical. Religion is the experiencing of divinity in the consciousness of a moral being of evolutionary origin; it represents true experience with eternal realities in time, the realization of spiritual satisfactions while yet in the flesh.
\vs p101 1:2 \pc The Thought Adjuster has no special mechanism through which to gain self\hyp{}expression; there is no mystic religious faculty for the reception or expression of religious emotions. These experiences are made available through the naturally ordained mechanism of mortal mind. And therein lies one explanation of the Adjuster’s difficulty in engaging in direct communication with the material mind of its constant indwelling.
\vs p101 1:3 The divine spirit makes contact with mortal man, not by feelings or emotions, but in the realm of the highest and most spiritualized thinking. It is your \bibemph{thoughts,} not your feelings, that lead you Godward. The divine nature may be perceived only with the eyes of the mind. But the mind that really discerns God, hears the indwelling Adjuster, is the pure mind. “Without holiness no man may see the Lord.” All such inner and spiritual communion is termed spiritual insight. Such religious experiences result from the impress made upon the mind of man by the combined operations of the Adjuster and the Spirit of Truth as they function amid and upon the ideas, ideals, insights, and spirit strivings of the evolving sons of God.
\vs p101 1:4 Religion lives and prospers, then, not by sight and feeling, but rather by faith and insight. It consists not in the discovery of new facts or in the finding of a unique experience, but rather in the discovery of new and spiritual \bibemph{meanings} in facts already well known to mankind. The highest religious experience is not dependent on prior acts of belief, tradition, and authority; neither is religion the offspring of sublime feelings and purely mystical emotions. It is, rather, a profoundly deep and actual experience of spiritual communion with the spirit influences resident within the human mind, and as far as such an experience is definable in terms of psychology, it is simply the experience of experiencing the reality of believing in God as the reality of such a purely personal experience.
\vs p101 1:5 \pc While religion is not the product of the rationalistic speculations of a material cosmology, it is, nonetheless, the creation of a wholly rational insight which originates in man’s mind\hyp{}experience. Religion is born neither of mystic meditations nor of isolated contemplations, albeit it is ever more or less mysterious and always indefinable and inexplicable in terms of purely intellectual reason and philosophic logic. The germs of true religion originate in the domain of man’s moral consciousness, and they are revealed in the growth of man’s spiritual insight, that faculty of human personality which accrues as a consequence of the presence of the God\hyp{}revealing Thought Adjuster in the God\hyp{}hungry mortal mind.
\vs p101 1:6 Faith unites moral insight with conscientious discriminations of values, and the pre\hyp{}existent evolutionary sense of duty completes the ancestry of true religion. The experience of religion eventually results in the certain consciousness of God and in the undoubted assurance of the survival of the believing personality.
\vs p101 1:7 Thus it may be seen that religious longings and spiritual urges are not of such a nature as would merely lead men to \bibemph{want} to believe in God, but rather are they of such nature and power that men are profoundly impressed with the conviction that they \bibemph{ought} to believe in God. The sense of evolutionary duty and the obligations consequent upon the illumination of revelation make such a profound impression upon man’s moral nature that he finally reaches that position of mind and that attitude of soul where he concludes that he \bibemph{has no right not to believe in God.} The higher and superphilosophic wisdom of such enlightened and disciplined individuals ultimately instructs them that to doubt God or distrust his goodness would be to prove untrue to the \bibemph{realest} and \bibemph{deepest} thing within the human mind and soul --- the divine Adjuster.
\usection{The Fact of Religion}
\vs p101 2:1 The fact of religion consists wholly in the religious experience of rational and average human beings. And this is the only sense in which religion can ever be regarded as scientific or even psychological. The proof that revelation is revelation is this same fact of human experience: the fact that revelation does synthesize the apparently divergent sciences of nature and the theology of religion into a consistent and logical universe philosophy, a co\hyp{}ordinated and unbroken explanation of both science and religion, thus creating a harmony of mind and satisfaction of spirit which answers in human experience those questionings of the mortal mind which craves to know \bibemph{how} the Infinite works out his will and plans in matter, with minds, and on spirit.
\vs p101 2:2 Reason is the method of science; faith is the method of religion; logic is the attempted technique of philosophy. Revelation compensates for the absence of the morontia viewpoint by providing a technique for achieving unity in the comprehension of the reality and relationships of matter and spirit by the mediation of mind. And true revelation never renders science unnatural, religion unreasonable, or philosophy illogical.
\vs p101 2:3 Reason, through the study of science, may lead back through nature to a First Cause, but it requires religious faith to transform the First Cause of science into a God of salvation; and revelation is further required for the validation of such a faith, such spiritual insight.
\vs p101 2:4 There are two basic reasons for believing in a God who fosters human survival:
\vs p101 2:5 \ublistelem{1.}\bibnobreakspace Human experience, personal assurance, the somehow registered hope and trust initiated by the indwelling Thought Adjuster.
\vs p101 2:6 \ublistelem{2.}\bibnobreakspace The revelation of truth, whether by direct personal ministry of the Spirit of Truth, by the world bestowal of divine Sons, or through the revelations of the written word.
\vs p101 2:7 \pc Science ends its reason\hyp{}search in the hypothesis of a First Cause. Religion does not stop in its flight of faith until it is sure of a God of salvation. The discriminating study of science logically suggests the reality and existence of an Absolute. Religion believes unreservedly in the existence and reality of a God who fosters personality survival. What metaphysics fails utterly in doing, and what even philosophy fails partially in doing, revelation does; that is, affirms that this First Cause of science and religion’s God of salvation are \bibemph{one and the same Deity.}
\vs p101 2:8 \pc Reason is the proof of science, faith the proof of religion, logic the proof of philosophy, but revelation is validated only by human \bibemph{experience.} Science yields knowledge; religion yields happiness; philosophy yields unity; revelation confirms the experiential harmony of this triune approach to universal reality.
\vs p101 2:9 The contemplation of nature can only reveal a God of nature, a God of motion. Nature exhibits only matter, motion, and animation --- life. Matter plus energy, under certain conditions, is manifested in living forms, but while natural life is thus relatively continuous as a phenomenon, it is wholly transient as to individualities. Nature does not afford ground for logical belief in human\hyp{}personality survival. The religious man who finds God in nature has already and first found this same personal God in his own soul.
\vs p101 2:10 \pc Faith reveals God in the soul. Revelation, the substitute for morontia insight on an evolutionary world, enables man to see the same God in nature that faith exhibits in his soul. Thus does revelation successfully bridge the gulf between the material and the spiritual, even between the creature and the Creator, between man and God.
\vs p101 2:11 The contemplation of nature does logically point in the direction of intelligent guidance, even living supervision, but it does not in any satisfactory manner reveal a personal God. On the other hand, nature discloses nothing which would preclude the universe from being looked upon as the handiwork of the God of religion. God cannot be found through nature alone, but man having otherwise found him, the study of nature becomes wholly consistent with a higher and more spiritual interpretation of the universe.
\vs p101 2:12 \pc Revelation as an epochal phenomenon is periodic; as a personal human experience it is continuous. Divinity functions in mortal personality as the Adjuster gift of the Father, as the Spirit of Truth of the Son, and as the Holy Spirit of the Universe Spirit, while these three supermortal endowments are unified in human experiential evolution as the ministry of the Supreme.
\vs p101 2:13 True religion is an insight into reality, the faith\hyp{}child of the moral consciousness, and not a mere intellectual assent to any body of dogmatic doctrines. True religion consists in the experience that “the Spirit itself bears witness with our spirit that we are the children of God.” Religion consists not in theologic propositions but in spiritual insight and the sublimity of the soul’s trust.
\vs p101 2:14 Your deepest nature --- the divine Adjuster --- creates within you a hunger and thirst for righteousness, a certain craving for divine perfection. Religion is the faith act of the recognition of this inner urge to divine attainment; and thus is brought about that soul trust and assurance of which you become conscious as the way of salvation, the technique of the survival of personality and all those values which you have come to look upon as being true and good.
\vs p101 2:15 \pc The realization of religion never has been, and never will be, dependent on great learning or clever logic. It is spiritual insight, and that is just the reason why some of the world’s greatest religious teachers, even the prophets, have sometimes possessed so little of the wisdom of the world. Religious faith is available alike to the learned and the unlearned.
\vs p101 2:16 Religion must ever be its own critic and judge; it can never be observed, much less understood, from the outside. Your only assurance of a personal God consists in your own insight as to your belief in, and experience with, things spiritual. To all of your fellows who have had a similar experience, no argument about the personality or reality of God is necessary, while to all other men who are not thus sure of God no possible argument could ever be truly convincing.
\vs p101 2:17 Psychology may indeed attempt to study the phenomena of religious reactions to the social environment, but never can it hope to penetrate to the real and inner motives and workings of religion. Only theology, the province of faith and the technique of revelation, can afford any sort of intelligent account of the nature and content of religious experience.
\usection{The Characteristics of Religion}
\vs p101 3:1 Religion is so vital that it persists in the absence of learning. It lives in spite of its contamination with erroneous cosmologies and false philosophies; it survives even the confusion of metaphysics. In and through all the historic vicissitudes of religion there ever persists that which is indispensable to human progress and survival: the ethical conscience and the moral consciousness.
\vs p101 3:2 Faith\hyp{}insight, or spiritual intuition, is the endowment of the cosmic mind in association with the Thought Adjuster, which is the Father’s gift to man. Spiritual reason, soul intelligence, is the endowment of the Holy Spirit, the Creative Spirit’s gift to man. Spiritual philosophy, the wisdom of spirit realities, is the endowment of the Spirit of Truth, the combined gift of the bestowal Sons to the children of men. And the co\hyp{}ordination and interassociation of these spirit endowments constitute man a spirit personality in potential destiny.
\vs p101 3:3 It is this same spirit personality, in primitive and embryonic form, the Adjuster possession of which survives the natural death in the flesh. This composite entity of spirit origin in association with human experience is enabled, by means of the living way provided by the divine Sons, to survive (in Adjuster custody) the dissolution of the material self of mind and matter when such a transient partnership of the material and the spiritual is divorced by the cessation of vital motion.
\vs p101 3:4 Through religious faith the soul of man reveals itself and demonstrates the potential divinity of its emerging nature by the characteristic manner in which it induces the mortal personality to react to certain trying intellectual and testing social situations. Genuine spiritual faith (true moral consciousness) is revealed in that it:
\vs p101 3:5 \ublistelem{1.}\bibnobreakspace Causes ethics and morals to progress despite inherent and adverse animalistic tendencies.
\vs p101 3:6 \ublistelem{2.}\bibnobreakspace Produces a sublime trust in the goodness of God even in the face of bitter disappointment and crushing defeat.
\vs p101 3:7 \ublistelem{3.}\bibnobreakspace Generates profound courage and confidence despite natural adversity and physical calamity.
\vs p101 3:8 \ublistelem{4.}\bibnobreakspace Exhibits inexplicable poise and sustaining tranquillity notwithstanding baffling diseases and even acute physical suffering.
\vs p101 3:9 \ublistelem{5.}\bibnobreakspace Maintains a mysterious poise and composure of personality in the face of maltreatment and the rankest injustice.
\vs p101 3:10 \ublistelem{6.}\bibnobreakspace Maintains a divine trust in ultimate victory in spite of the cruelties of seemingly blind fate and the apparent utter indifference of natural forces to human welfare.
\vs p101 3:11 \ublistelem{7.}\bibnobreakspace Persists in the unswerving belief in God despite all contrary demonstrations of logic and successfully withstands all other intellectual sophistries.
\vs p101 3:12 \ublistelem{8.}\bibnobreakspace Continues to exhibit undaunted faith in the soul’s survival regardless of the deceptive teachings of false science and the persuasive delusions of unsound philosophy.
\vs p101 3:13 \ublistelem{9.}\bibnobreakspace Lives and triumphs irrespective of the crushing overload of the complex and partial civilizations of modern times.
\vs p101 3:14 \ublistelem{10.}\bibnobreakspace Contributes to the continued survival of altruism in spite of human selfishness, social antagonisms, industrial greeds, and political maladjustments.
\vs p101 3:15 \ublistelem{11.}\bibnobreakspace Steadfastly adheres to a sublime belief in universe unity and divine guidance regardless of the perplexing presence of evil and sin.
\vs p101 3:16 \ublistelem{12.}\bibnobreakspace Goes right on worshipping God in spite of anything and everything. Dares to declare, “Even though he slay me, yet will I serve him.”
\vs p101 3:17 \pc We know, then, by three phenomena, that man has a divine spirit or spirits dwelling within him: first, by personal experience --- religious faith; second, by revelation --- personal and racial; and third, by the amazing exhibition of such extraordinary and unnatural reactions to his material environment as are illustrated by the foregoing recital of 12 spiritlike performances in the presence of the actual and trying situations of real human existence. And there are still others.
\vs p101 3:18 And it is just such a vital and vigorous performance of faith in the domain of religion that entitles mortal man to affirm the personal possession and spiritual reality of that crowning endowment of human nature, religious experience.
\usection{The Limitations of Revelation}
\vs p101 4:1 Because your world is generally ignorant of origins, even of physical origins, it has appeared to be wise from time to time to provide instruction in cosmology. And always has this made trouble for the future. The laws of revelation hamper us greatly by their proscription of the impartation of unearned or premature knowledge. Any cosmology presented as a part of revealed religion is destined to be outgrown in a very short time. Accordingly, future students of such a revelation are tempted to discard any element of genuine religious truth it may contain because they discover errors on the face of the associated cosmologies therein presented.
\vs p101 4:2 Mankind should understand that we who participate in the revelation of truth are very rigorously limited by the instructions of our superiors. We are not at liberty to anticipate the scientific discoveries of 1,000 years. Revelators must act in accordance with the instructions which form a part of the revelation mandate. We see no way of overcoming this difficulty, either now or at any future time. We full well know that, while the historic facts and religious truths of this series of revelatory presentations will stand on the records of the ages to come, within a few short years many of our statements regarding the physical sciences will stand in need of revision in consequence of additional scientific developments and new discoveries. These new developments we even now foresee, but we are forbidden to include such humanly undiscovered facts in the revelatory records. Let it be made clear that revelations are not necessarily inspired. The cosmology of these revelations is \bibemph{not inspired.} It is limited by our permission for the co\hyp{}ordination and sorting of present\hyp{}day knowledge. While divine or spiritual insight is a gift, \bibemph{human wisdom must evolve.}
\vs p101 4:3 \pc Truth is always a revelation: autorevelation when it emerges as a result of the work of the indwelling Adjuster; epochal revelation when it is presented by the function of some other celestial agency, group, or personality.
\vs p101 4:4 In the last analysis, religion is to be judged by its fruits, according to the manner and the extent to which it exhibits its own inherent and divine excellence.
\vs p101 4:5 \pc Truth may be but relatively inspired, even though revelation is invariably a spiritual phenomenon. While statements with reference to cosmology are never inspired, such revelations are of immense value in that they at least transiently clarify knowledge by:
\vs p101 4:6 \ublistelem{1.}\bibnobreakspace The reduction of confusion by the authoritative elimination of error.
\vs p101 4:7 \ublistelem{2.}\bibnobreakspace The co\hyp{}ordination of known or about\hyp{}to\hyp{}be\hyp{}known facts and observations.
\vs p101 4:8 \ublistelem{3.}\bibnobreakspace The restoration of important bits of lost knowledge concerning epochal transactions in the distant past.
\vs p101 4:9 \ublistelem{4.}\bibnobreakspace The supplying of information which will fill in vital missing gaps in otherwise earned knowledge.
\vs p101 4:10 \ublistelem{5.}\bibnobreakspace Presenting cosmic data in such a manner as to illuminate the spiritual teachings contained in the accompanying revelation.
\usection{Religion Expanded by Revelation}
\vs p101 5:1 Revelation is a technique whereby ages upon ages of time are saved in the necessary work of sorting and sifting the errors of evolution from the truths of spirit acquirement.
\vs p101 5:2 Science deals with \bibemph{facts;} religion is concerned only with \bibemph{values.} Through enlightened philosophy the mind endeavours to unite the meanings of both facts and values, thereby arriving at a concept of complete \bibemph{reality.} Remember that science is the domain of knowledge, philosophy the realm of wisdom, and religion the sphere of the faith experience. But religion, nonetheless, presents two phases of manifestation:
\vs p101 5:3 \ublistelem{1.}\bibnobreakspace Evolutionary religion. The experience of primitive worship, the religion which is a mind derivative.
\vs p101 5:4 \ublistelem{2.}\bibnobreakspace Revealed religion. The universe attitude which is a spirit derivative; the assurance of, and belief in, the conservation of eternal realities, the survival of personality, and the eventual attainment of the cosmic Deity, whose purpose has made all this possible. It is a part of the plan of the universe that, sooner or later, evolutionary religion is destined to receive the spiritual expansion of revelation.
\vs p101 5:5 \pc Both science and religion start out with the assumption of certain generally accepted bases for logical deductions. So, also, must philosophy start its career upon the assumption of the reality of three things:
\vs p101 5:6 \ublistelem{1.}\bibnobreakspace The material body.
\vs p101 5:7 \ublistelem{2.}\bibnobreakspace The supermaterial phase of the human being, the soul or even the indwelling spirit.
\vs p101 5:8 \ublistelem{3.}\bibnobreakspace The human mind, the mechanism for intercommunication and interassociation between spirit and matter, between the material and the spiritual.
\vs p101 5:9 \pc Scientists assemble facts, philosophers co\hyp{}ordinate ideas, while prophets exalt ideals. Feeling and emotion are invariable concomitants of religion, but they are not religion. Religion may be the feeling of experience, but it is hardly the experience of feeling. Neither logic (rationalization) nor emotion (feeling) is essentially a part of religious experience, although both may variously be associated with the exercise of faith in the furtherance of spiritual insight into reality, all according to the status and temperamental tendency of the individual mind.
\vs p101 5:10 Evolutionary religion is the outworking of the endowment of the local universe mind adjutant charged with the creation and fostering of the worship trait in evolving man. Such primitive religions are directly concerned with ethics and morals, the sense of human \bibemph{duty.} Such religions are predicated on the assurance of conscience and result in the stabilization of relatively ethical civilizations.
\vs p101 5:11 Personally revealed religions are sponsored by the bestowal spirits representing the three persons of the Paradise Trinity and are especially concerned with the expansion of \bibemph{truth.} Evolutionary religion drives home to the individual the idea of personal duty; revealed religion lays increasing emphasis on loving, the golden rule.
\vs p101 5:12 Evolved religion rests wholly on faith. Revelation has the additional assurance of its expanded presentation of the truths of divinity and reality and the still more valuable testimony of the actual experience which accumulates in consequence of the practical working union of the faith of evolution and the truth of revelation. Such a working union of human faith and divine truth constitutes the possession of a character well on the road to the actual acquirement of a morontial personality.
\vs p101 5:13 \pc Evolutionary religion provides only the assurance of faith and the confirmation of conscience; revelatory religion provides the assurance of faith plus the truth of a living experience in the realities of revelation. The third step in religion, or the third phase of the experience of religion, has to do with the morontia state, the firmer grasp of mota. Increasingly in the morontia progression the truths of revealed religion are expanded; more and more you will know the truth of supreme values, divine goodnesses, universal relationships, eternal realities, and ultimate destinies.
\vs p101 5:14 Increasingly throughout the morontia progression the assurance of truth replaces the assurance of faith. When you are finally mustered into the actual spirit world, then will the assurances of pure spirit insight operate in the place of faith and truth or, rather, in conjunction with, and superimposed upon, these former techniques of personality assurance.
\usection{Progressive Religious Experience}
\vs p101 6:1 The morontia phase of revealed religion has to do with the \bibemph{experience of survival,} and its great urge is the attainment of spirit perfection. There also is present the higher urge of worship, associated with an impelling call to increased ethical service. Morontia insight entails an ever\hyp{}expanding consciousness of the Sevenfold, the Supreme, and even the Ultimate.
\vs p101 6:2 Throughout all religious experience, from its earliest inception on the material level up to the time of the attainment of full spirit status, the Adjuster is the secret of the personal realization of the reality of the existence of the Supreme; and this same Adjuster also holds the secrets of your faith in the transcendental attainment of the Ultimate. The experiential personality of evolving man, united to the Adjuster essence of the existential God, constitutes the potential completion of supreme existence and is inherently the basis for the superfinite eventuation of transcendental personality.
\vs p101 6:3 \pc Moral will embraces decisions based on reasoned knowledge, augmented by wisdom, and sanctioned by religious faith. Such choices are acts of moral nature and evidence the existence of moral personality, the forerunner of morontia personality and eventually of true spirit status.
\vs p101 6:4 The evolutionary type of knowledge is but the accumulation of protoplasmic memory material; this is the most primitive form of creature consciousness. Wisdom embraces the ideas formulated from protoplasmic memory in process of association and recombination, and such phenomena differentiate human mind from mere animal mind. Animals have knowledge, but only man possesses wisdom capacity. Truth is made accessible to the wisdom\hyp{}endowed individual by the bestowal on such a mind of the spirits of the Father and the Sons, the Thought Adjuster and the Spirit of Truth.
\vs p101 6:5 \pc Christ Michael, when bestowed on Urantia, lived under the reign of evolutionary religion up to the time of his baptism. From that moment up to and including the event of his crucifixion he carried forward his work by the combined guidance of evolutionary and revealed religion. From the morning of his resurrection until his ascension he traversed the manifold phases of the morontia life of mortal transition from the world of matter to that of spirit. After his ascension Michael became master of the experience of Supremacy, the realization of the Supreme; and being the one person in Nebadon possessed of unlimited capacity to experience the reality of the Supreme, he forthwith attained to the status of the sovereignty of supremacy in and to his local universe.
\vs p101 6:6 With man, the eventual fusion and resultant oneness with the indwelling Adjuster --- the personality synthesis of man and the essence of God --- constitute him, in potential, a living part of the Supreme and ensure for such a onetime mortal being the eternal birthright of the endless pursuit of finality of universe service for and with the Supreme.
\vs p101 6:7 \pc Revelation teaches mortal man that, to start such a magnificent and intriguing adventure through space by means of the progression of time, he should begin by the organization of knowledge into idea\hyp{}decisions; next, mandate wisdom to labour unremittingly at its noble task of transforming self\hyp{}possessed ideas into increasingly practical but nonetheless supernal ideals, even those concepts which are so reasonable as ideas and so logical as ideals that the Adjuster dares so to combine and spiritize them as to render them available for such association in the finite mind as will constitute them the actual human complement thus made ready for the action of the Truth Spirit of the Sons, the time\hyp{}space manifestations of Paradise truth --- universal truth. The co\hyp{}ordination of idea\hyp{}decisions, logical ideals, and divine truth constitutes the possession of a righteous character, the prerequisite for mortal admission to the ever\hyp{}expanding and increasingly spiritual realities of the morontia worlds.
\vs p101 6:8 The teachings of Jesus constituted the first Urantian religion which so fully embraced a harmonious co\hyp{}ordination of knowledge, wisdom, faith, truth, and love as completely and simultaneously to provide temporal tranquillity, intellectual certainty, moral enlightenment, philosophic stability, ethical sensitivity, God\hyp{}consciousness, and the positive assurance of personal survival. The faith of Jesus pointed the way to finality of human salvation, to the ultimate of mortal universe attainment, since it provided for:
\vs p101 6:9 \ublistelem{1.}\bibnobreakspace Salvation from material fetters in the personal realization of sonship with God, who is spirit.
\vs p101 6:10 \ublistelem{2.}\bibnobreakspace Salvation from intellectual bondage: man shall know the truth, and the truth shall set him free.
\vs p101 6:11 \ublistelem{3.}\bibnobreakspace Salvation from spiritual blindness, the human realization of the fraternity of mortal beings and the morontian awareness of the brotherhood of all universe creatures; the service\hyp{}discovery of spiritual reality and the ministry\hyp{}revelation of the goodness of spirit values.
\vs p101 6:12 \ublistelem{4.}\bibnobreakspace Salvation from incompleteness of self through the attainment of the spirit levels of the universe and through the eventual realization of the harmony of Havona and the perfection of Paradise.
\vs p101 6:13 \ublistelem{5.}\bibnobreakspace Salvation from self, deliverance from the limitations of self\hyp{}consciousness through the attainment of the cosmic levels of the Supreme mind and by co\hyp{}ordination with the attainments of all other self\hyp{}conscious beings.
\vs p101 6:14 \ublistelem{6.}\bibnobreakspace Salvation from time, the achievement of an eternal life of unending progression in God\hyp{}recognition and God\hyp{}service.
\vs p101 6:15 \ublistelem{7.}\bibnobreakspace Salvation from the finite, the perfected oneness with Deity in and through the Supreme by which the creature attempts the transcendental discovery of the Ultimate on the postfinaliter levels of the absonite.
\vs p101 6:16 \pc Such a sevenfold salvation is the equivalent of the completeness and perfection of the realization of the ultimate experience of the Universal Father. And all this, in potential, is contained within the reality of the faith of the human experience of religion. And it can be so contained since the faith of Jesus was nourished by, and was revelatory of, even realities beyond the ultimate; the faith of Jesus approached the status of a universe absolute in so far as such is possible of manifestation in the evolving cosmos of time and space.
\vs p101 6:17 Through the appropriation of the faith of Jesus, mortal man can foretaste in time the realities of eternity. Jesus made the discovery, in human experience, of the Final Father, and his brothers in the flesh of mortal life can follow him along this same experience of Father discovery. They can even attain, as they are, the same satisfaction in this experience with the Father as did Jesus as he was. New potentials were actualized in the universe of Nebadon consequent upon the terminal bestowal of Michael, and one of these was the new illumination of the path of eternity that leads to the Father of all, and which can be traversed even by the mortals of material flesh and blood in the initial life on the planets of space. Jesus was and is the new and living way whereby man can come into the divine inheritance which the Father has decreed shall be his for but the asking. In Jesus there is abundantly demonstrated both the beginnings and endings of the faith experience of humanity, even of divine humanity.
\usection{A Personal Philosophy of Religion}
\vs p101 7:1 An idea is only a theoretical plan for action, while a positive decision is a validated plan of action. A stereotype is a plan of action accepted without validation. The materials out of which to build a personal philosophy of religion are derived from both the inner and the environmental experience of the individual. The social status, economic conditions, educational opportunities, moral trends, institutional influences, political developments, racial tendencies, and the religious teachings of one’s time and place all become factors in the formulation of a personal philosophy of religion. Even the inherent temperament and intellectual bent markedly determine the pattern of religious philosophy. Vocation, marriage, and kindred all influence the evolution of one’s personal standards of life.
\vs p101 7:2 A philosophy of religion evolves out of a basic growth of ideas plus experimental living as both are modified by the tendency to imitate associates. The soundness of philosophic conclusions depends on keen, honest, and discriminating thinking in connection with sensitivity to meanings and accuracy of evaluation. Moral cowards never achieve high planes of philosophic thinking; it requires courage to invade new levels of experience and to attempt the exploration of unknown realms of intellectual living.
\vs p101 7:3 Presently new systems of values come into existence; new formulations of principles and standards are achieved; habits and ideals are reshaped; some idea of a personal God is attained, followed by enlarging concepts of relationship thereto.
\vs p101 7:4 \pc The great difference between a religious and a nonreligious philosophy of living consists in the nature and level of recognized values and in the object of loyalties. There are four phases in the evolution of religious philosophy: Such an experience may become merely conformative, resigned to submission to tradition and authority. Or it may be satisfied with slight attainments, just enough to stabilize the daily living, and therefore becomes early arrested on such an adventitious level. Such mortals believe in letting well enough alone. A third group progress to the level of logical intellectuality but there stagnate in consequence of cultural slavery. It is indeed pitiful to behold giant intellects held so securely within the cruel grasp of cultural bondage. It is equally pathetic to observe those who trade their cultural bondage for the materialistic fetters of a science, falsely so called. The fourth level of philosophy attains freedom from all conventional and traditional handicaps and dares to think, act, and live honestly, loyally, fearlessly, and truthfully.
\vs p101 7:5 The acid test for any religious philosophy consists in whether or not it distinguishes between the realities of the material and the spiritual worlds while at the same moment recognizing their unification in intellectual striving and in social serving. A sound religious philosophy does not confound the things of God with the things of Caesar. Neither does it recognize the aesthetic cult of pure wonder as a substitute for religion.
\vs p101 7:6 Philosophy transforms that primitive religion which was largely a fairy tale of conscience into a living experience in the ascending values of cosmic reality.
\usection{Faith and Belief}
\vs p101 8:1 Belief has attained the level of faith when it motivates life and shapes the mode of living. The acceptance of a teaching as true is not faith; that is mere belief. Neither is certainty nor conviction faith. A state of mind attains to faith levels only when it actually dominates the mode of living. Faith is a living attribute of genuine personal religious experience. One believes truth, admires beauty, and reverences goodness, but does not worship them; such an attitude of saving faith is centred on God alone, who is all of these personified and infinitely more.
\vs p101 8:2 Belief is always limiting and binding; faith is expanding and releasing. Belief fixates, faith liberates. But living religious faith is more than the association of noble beliefs; it is more than an exalted system of philosophy; it is a living experience concerned with spiritual meanings, divine ideals, and supreme values; it is God\hyp{}knowing and man\hyp{}serving. Beliefs may become group possessions, but faith must be personal. Theologic beliefs can be suggested to a group, but faith can rise up only in the heart of the individual religionist.
\vs p101 8:3 Faith has falsified its trust when it presumes to deny realities and to confer upon its devotees assumed knowledge. Faith is a traitor when it fosters betrayal of intellectual integrity and belittles loyalty to supreme values and divine ideals. Faith never shuns the problem\hyp{}solving duty of mortal living. Living faith does not foster bigotry, persecution, or intolerance.
\vs p101 8:4 Faith does not shackle the creative imagination, neither does it maintain an unreasoning prejudice toward the discoveries of scientific investigation. Faith vitalizes religion and constrains the religionist heroically to live the golden rule. The zeal of faith is according to knowledge, and its strivings are the preludes to sublime peace.
\usection{Religion and Morality}
\vs p101 9:1 No professed revelation of religion could be regarded as authentic if it failed to recognize the duty demands of ethical obligation which had been created and fostered by preceding evolutionary religion. Revelation unfailingly enlarges the ethical horizon of evolved religion while it simultaneously and unfailingly expands the moral obligations of all prior revelations.
\vs p101 9:2 When you presume to sit in critical judgment on the primitive religion of man (or on the religion of primitive man), you should remember to judge such savages and to evaluate their religious experience in accordance with their enlightenment and status of conscience. Do not make the mistake of judging another’s religion by your own standards of knowledge and truth.
\vs p101 9:3 True religion is that sublime and profound conviction within the soul which compellingly admonishes man that it would be wrong for him not to believe in those morontial realities which constitute his highest ethical and moral concepts, his highest interpretation of life’s greatest values and the universe’s deepest realities. And such a religion is simply the experience of yielding intellectual loyalty to the highest dictates of spiritual consciousness.
\vs p101 9:4 The search for beauty is a part of religion only in so far as it is ethical and to the extent that it enriches the concept of the moral. Art is only religious when it becomes diffused with purpose which has been derived from high spiritual motivation.
\vs p101 9:5 The enlightened spiritual consciousness of civilized man is not concerned so much with some specific intellectual belief or with any one particular mode of living as with discovering the truth of living, the good and right technique of reacting to the ever\hyp{}recurring situations of mortal existence. Moral consciousness is just a name applied to the human recognition and awareness of those ethical and emerging morontial values which duty demands that man shall abide by in the day\hyp{}by\hyp{}day control and guidance of conduct.
\vs p101 9:6 \pc Though recognizing that religion is imperfect, there are at least two practical manifestations of its nature and function:
\vs p101 9:7 \ublistelem{1.}\bibnobreakspace The spiritual urge and philosophic pressure of religion tend to cause man to project his estimation of moral values directly outward into the affairs of his fellows --- the ethical reaction of religion.
\vs p101 9:8 \ublistelem{2.}\bibnobreakspace Religion creates for the human mind a spiritualized consciousness of divine reality based on, and by faith derived from, antecedent concepts of moral values and co\hyp{}ordinated with superimposed concepts of spiritual values. Religion thereby becomes a censor of mortal affairs, a form of glorified moral trust and confidence in reality, the enhanced realities of time and the more enduring realities of eternity.
\vs p101 9:9 \pc Faith becomes the connection between moral consciousness and the spiritual concept of enduring reality. Religion becomes the avenue of man’s escape from the material limitations of the temporal and natural world to the supernal realities of the eternal and spiritual world by and through the technique of salvation, the progressive morontia transformation.
\usection{Religion as Man’s Liberator}
\vs p101 10:1 Intelligent man knows that he is a child of nature, a part of the material universe; he likewise discerns no survival of individual personality in the motions and tensions of the mathematical level of the energy universe. Nor can man ever discern spiritual reality through the examination of physical causes and effects.
\vs p101 10:2 A human being is also aware that he is a part of the ideational cosmos, but though concept may endure beyond a mortal life span, there is nothing inherent in concept which indicates the personal survival of the conceiving personality. Nor will the exhaustion of the possibilities of logic and reason ever reveal to the logician or to the reasoner the eternal truth of the survival of personality.
\vs p101 10:3 The material level of law provides for causality continuity, the unending response of effect to antecedent action; the mind level suggests the perpetuation of ideational continuity, the unceasing flow of conceptual potentiality from pre\hyp{}existent conceptions. But neither of these levels of the universe discloses to the inquiring mortal an avenue of escape from partiality of status and from the intolerable suspense of being a transient reality in the universe, a temporal personality doomed to be extinguished upon the exhaustion of the limited life energies.
\vs p101 10:4 It is only through the morontial avenue leading to spiritual insight that man can ever break the fetters inherent in his mortal status in the universe. Energy and mind do lead back to Paradise and Deity, but neither the energy endowment nor the mind endowment of man proceeds directly from such Paradise Deity. Only in the spiritual sense is man a child of God. And this is true because it is only in the spiritual sense that man is at present endowed and indwelt by the Paradise Father. Mankind can never discover divinity except through the avenue of religious experience and by the exercise of true faith. The faith acceptance of the truth of God enables man to escape from the circumscribed confines of material limitations and affords him a rational hope of achieving safe conduct from the material realm, whereon is death, to the spiritual realm, wherein is life eternal.
\vs p101 10:5 \pc The purpose of religion is not to satisfy curiosity about God but rather to afford intellectual constancy and philosophic security, to stabilize and enrich human living by blending the mortal with the divine, the partial with the perfect, man and God. It is through religious experience that man’s concepts of ideality are endowed with reality.
\vs p101 10:6 \pc Never can there be either scientific or logical proofs of divinity. Reason alone can never validate the values and goodnesses of religious experience. But it will always remain true: Whosoever wills to do the will of God shall comprehend the validity of spiritual values. This is the nearest approach that can be made on the mortal level to offering proofs of the reality of religious experience. Such faith affords the only escape from the mechanical clutch of the material world and from the error distortion of the incompleteness of the intellectual world; it is the only discovered solution to the impasse in mortal thinking regarding the continuing survival of the individual personality. It is the only passport to completion of reality and to eternity of life in a universal creation of love, law, unity, and progressive Deity attainment.
\vs p101 10:7 Religion effectually cures man’s sense of idealistic isolation or spiritual loneliness; it enfranchises the believer as a son of God, a citizen of a new and meaningful universe. Religion assures man that, in following the gleam of righteousness discernible in his soul, he is thereby identifying himself with the plan of the Infinite and the purpose of the Eternal. Such a liberated soul immediately begins to feel at home in this new universe, his universe.
\vs p101 10:8 When you experience such a transformation of faith, you are no longer a slavish part of the mathematical cosmos but rather a liberated volitional son of the Universal Father. No longer is such a liberated son fighting alone against the inexorable doom of the termination of temporal existence; no longer does he combat all nature, with the odds hopelessly against him; no longer is he staggered by the paralysing fear that, perchance, he has put his trust in a hopeless phantasm or pinned his faith to a fanciful error.
\vs p101 10:9 Now, rather, are the sons of God enlisted together in fighting the battle of reality’s triumph over the partial shadows of existence. At last all creatures become conscious of the fact that God and all the divine hosts of a well\hyp{}nigh limitless universe are on their side in the supernal struggle to attain eternity of life and divinity of status. Such faith\hyp{}liberated sons have certainly enlisted in the struggles of time on the side of the supreme forces and divine personalities of eternity; even the stars in their courses are now doing battle for them; at last they gaze upon the universe from within, from God’s viewpoint, and all is transformed from the uncertainties of material isolation to the sureties of eternal spiritual progression. Even time itself becomes but the shadow of eternity cast by Paradise realities upon the moving panoply of space.
\vsetoff
\vs p101 10:10 [Presented by a Melchizedek of Nebadon.]
\quizlink
