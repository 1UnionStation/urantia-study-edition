\upaper{168}{The Resurrection of Lazarus}
\author{Midwayer Commission}
\vs p168 0:1 It was shortly after noon when Martha started out to meet Jesus as he came over the brow of the hill near Bethany. Her brother, Lazarus, had been dead four days and had been laid away in their private tomb at the far end of the garden late on Sunday afternoon. The stone at the entrance of the tomb had been rolled in place on the morning of this day, Thursday.
\vs p168 0:2 When Martha and Mary sent word to Jesus concerning Lazarus’s illness, they were confident the Master would do something about it. They knew that their brother was desperately sick, and though they hardly dared hope that Jesus would leave his work of teaching and preaching to come to their assistance, they had such confidence in his power to heal disease that they thought he would just speak the curative words, and Lazarus would immediately be made whole. And when Lazarus died a few hours after the messenger left Bethany for Philadelphia, they reasoned that it was because the Master did not learn of their brother’s illness until it was too late, until he had already been dead for several hours.
\vs p168 0:3 But they, with all of their believing friends, were greatly puzzled by the message which the runner brought back Tuesday forenoon when he reached Bethany. The messenger insisted that he heard Jesus say, \textcolour{ubdarkred}{“\ldots this sickness is really not to the death.”} Neither could they understand why he sent no word to them nor otherwise proffered assistance.
\vs p168 0:4 Many friends from near\hyp{}by hamlets and others from Jerusalem came over to comfort the sorrow\hyp{}stricken sisters. Lazarus and his sisters were the children of a well\hyp{}to\hyp{}do and honourable Jew, one who had been the leading resident of the little village of Bethany. And notwithstanding that all three had long been ardent followers of Jesus, they were highly respected by all who knew them. They had inherited extensive vineyards and olive orchards in this vicinity, and that they were wealthy was further attested by the fact that they could afford a private burial tomb on their own premises. Both of their parents had already been laid away in this tomb.
\vs p168 0:5 Mary had given up the thought of Jesus’ coming and was abandoned to her grief, but Martha clung to the hope that Jesus would come, even up to the time on that very morning when they rolled the stone in front of the tomb and sealed the entrance. Even then she instructed a neighbour lad to keep watch down the Jericho road from the brow of the hill to the east of Bethany; and it was this lad who brought tidings to Martha that Jesus and his friends were approaching.
\vs p168 0:6 When Martha met Jesus, she fell at his feet, exclaiming, “Master, if you had been here, my brother would not have died!” Many fears were passing through Martha’s mind, but she gave expression to no doubt, nor did she venture to criticize or question the Master’s conduct as related to Lazarus’s death. When she had spoken, Jesus reached down and, lifting her upon her feet, said, \textcolour{ubdarkred}{“Only have faith, Martha, and your brother shall rise again.”} Then answered Martha: “I know that he will rise again in the resurrection of the last day; and even now I believe that whatever you shall ask of God, our Father will give you.”
\vs p168 0:7 Then said Jesus, looking straight into the eyes of Martha: \textcolour{ubdarkred}{“I am the resurrection and the life; he who believes in me, though he dies, yet shall he live. In truth, whosoever lives and believes in me shall never really die. Martha, do you believe this?”} And Martha answered the Master: “Yes, I have long believed that you are the Deliverer, the Son of the living God, even he who should come to this world.”
\vs p168 0:8 Jesus having inquired for Mary, Martha went at once into the house and, whispering to her sister, said, “The Master is here and has asked for you.” And when Mary heard this, she rose up quickly and hastened out to meet Jesus, who still tarried at the place, some distance from the house, where Martha had first met him. The friends who were with Mary, seeking to comfort her, when they saw that she rose up quickly and went out, followed her, supposing that she was going to the tomb to weep.
\vs p168 0:9 Many of those present were Jesus’ bitter enemies. That is why Martha had come out to meet him alone, and also why she went in secretly to inform Mary that he had asked for her. Martha, while craving to see Jesus, desired to avoid any possible unpleasantness which might be caused by his coming suddenly into the midst of a large group of his Jerusalem enemies. It had been Martha’s intention to remain in the house with their friends while Mary went to greet Jesus, but in this she failed, for they all followed Mary and so found themselves unexpectedly in the presence of the Master.
\vs p168 0:10 Martha led Mary to Jesus, and when she saw him, she fell at his feet, exclaiming, “If you had only been here, my brother would not have died!” And when Jesus saw how they all grieved over the death of Lazarus, his soul was moved with compassion.
\vs p168 0:11 When the mourners saw that Mary had gone to greet Jesus, they withdrew for a short distance while both Martha and Mary talked with the Master and received further words of comfort and exhortation to maintain strong faith in the Father and complete resignation to the divine will.
\vs p168 0:12 The human mind of Jesus was mightily moved by the contention between his love for Lazarus and the bereaved sisters and his disdain and contempt for the outward show of affection manifested by some of these unbelieving and murderously intentioned Jews. Jesus indignantly resented the show of forced and outward mourning for Lazarus by some of these professed friends inasmuch as such false sorrow was associated in their hearts with so much bitter enmity toward himself. Some of these Jews, however, were sincere in their mourning, for they were real friends of the family.
\usection{1.\bibnobreakspace At the Tomb of Lazarus}
\vs p168 1:1 After Jesus had spent a few moments in comforting Martha and Mary, apart from the mourners, he asked them, \textcolour{ubdarkred}{“Where have you laid him?”} Then Martha said, “Come and see.” And as the Master followed on in silence with the two sorrowing sisters, he wept. When the friendly Jews who followed after them saw his tears, one of them said: “Behold how he loved him. Could not he who opened the eyes of the blind have kept this man from dying?” By this time they were standing before the family tomb, a small natural cave, or declivity, in the ledge of rock which rose up some 9\,m at the far end of the garden plot.
\vs p168 1:2 \pc It is difficult to explain to human minds just why Jesus wept. While we have access to the registration of the combined human emotions and divine thoughts, as of record in the mind of the Personalized Adjuster, we are not altogether certain about the real cause of these emotional manifestations. We are inclined to believe that Jesus wept because of a number of thoughts and feelings which were going through his mind at this time, such as:
\vs p168 1:3 \ublistelem{1.}\bibnobreakspace He felt a genuine and sorrowful sympathy for Martha and Mary; he had a real and deep human affection for these sisters who had lost their brother.
\vs p168 1:4 \ublistelem{2.}\bibnobreakspace He was perturbed in his mind by the presence of the crowd of mourners, some sincere and some merely pretenders. He always resented these outward exhibitions of mourning. He knew the sisters loved their brother and had faith in the survival of believers. These conflicting emotions may possibly explain why he groaned as they came near the tomb.
\vs p168 1:5 \ublistelem{3.}\bibnobreakspace He truly hesitated about bringing Lazarus back to the mortal life. His sisters really needed him, but Jesus regretted having to summon his friend back to experience the bitter persecution which he well knew Lazarus would have to endure as a result of being the subject of the greatest of all demonstrations of the divine power of the Son of Man.
\vs p168 1:6 \pc And now we may relate an interesting and instructive fact: Although this narrative unfolds as an apparently natural and normal event in human affairs, it has some very interesting side lights. While the messenger went to Jesus on Sunday, telling him of Lazarus’s illness, and while Jesus sent word that it was \textcolour{ubdarkred}{“not to the death,”} at the same time he went in person up to Bethany and even asked the sisters, \textcolour{ubdarkred}{“Where have you laid him?”} Even though all of this seems to indicate that the Master was proceeding after the manner of this life and in accordance with the limited knowledge of the human mind, nevertheless, the records of the universe reveal that Jesus’ Personalized Adjuster issued orders for the indefinite detention of Lazarus’s Thought Adjuster on the planet subsequent to Lazarus’s death, and that this order was made of record just 15 minutes before Lazarus breathed his last.
\vs p168 1:7 Did the divine mind of Jesus know, even before Lazarus died, that he would raise him from the dead? We do not know. We know only what we are herewith placing on record.
\vs p168 1:8 \pc Many of Jesus’ enemies were inclined to sneer at his manifestations of affection, and they said among themselves: “If he thought so much of this man, why did he tarry so long before coming to Bethany? If he is what they claim, why did he not save his dear friend? What is the good of healing strangers in Galilee if he cannot save those whom he loves?” And in many other ways they mocked and made light of the teachings and works of Jesus.
\vs p168 1:9 And so, on this Thursday afternoon at about 14:30, was the stage all set in this little hamlet of Bethany for the enactment of the greatest of all works connected with the earth ministry of Michael of Nebadon, the greatest manifestation of divine power during his incarnation in the flesh, since his own resurrection occurred after he had been liberated from the bonds of mortal habitation.
\vs p168 1:10 The small group assembled before Lazarus’s tomb little realized the presence near at hand of a vast concourse of all orders of celestial beings assembled under the leadership of Gabriel and now in waiting, by direction of the Personalized Adjuster of Jesus, vibrating with expectancy and ready to execute the bidding of their beloved Sovereign.
\vs p168 1:11 When Jesus spoke those words of command, \textcolour{ubdarkred}{“Take away the stone,”} the assembled celestial hosts made ready to enact the drama of the resurrection of Lazarus in the likeness of his mortal flesh. Such a form of resurrection involves difficulties of execution which far transcend the usual technique of the resurrection of mortal creatures in morontia form and requires far more celestial personalities and a far greater organization of universe facilities.
\vs p168 1:12 When Martha and Mary heard this command of Jesus directing that the stone in front of the tomb be rolled away, they were filled with conflicting emotions. Mary hoped that Lazarus was to be raised from the dead, but Martha, while to some extent sharing her sister’s faith, was more exercised by the fear that Lazarus would not be presentable, in his appearance, to Jesus, the apostles, and their friends. Said Martha: “Must we roll away the stone? My brother has now been dead four days, so that by this time decay of the body has begun.” Martha also said this because she was not certain as to why the Master had requested that the stone be removed; she thought maybe Jesus wanted only to take one last look at Lazarus. She was not settled and constant in her attitude. As they hesitated to roll away the stone, Jesus said: \textcolour{ubdarkred}{“Did I not tell you at the first that this sickness was not to the death? Have I not come to fulfil my promise? And after I came to you, did I not say that, if you would only believe, you should see the glory of God? Wherefore do you doubt? How long before you will believe and obey?”}
\vs p168 1:13 When Jesus had finished speaking, his apostles, with the assistance of willing neighbours, laid hold upon the stone and rolled it away from the entrance to the tomb.
\vs p168 1:14 \pc It was the common belief of the Jews that the drop of gall on the point of the sword of the angel of death began to work by the end of the third day, so that it was taking full effect on the fourth day. They allowed that the soul of man might linger about the tomb until the end of the third day, seeking to reanimate the dead body; but they firmly believed that such a soul had gone on to the abode of departed spirits ere the fourth day had dawned.
\vs p168 1:15 These beliefs and opinions regarding the dead and the departure of the spirits of the dead served to make sure, in the minds of all who were now present at Lazarus’s tomb and subsequently to all who might hear of what was about to occur, that this was really and truly a case of the raising of the dead by the personal working of one who declared he was \textcolour{ubdarkred}{“the resurrection and the life.”}
\usection{2.\bibnobreakspace The Resurrection of Lazarus}
\vs p168 2:1 As this company of some 45 mortals stood before the tomb, they could dimly see the form of Lazarus, wrapped in linen bandages, resting on the right lower niche of the burial cave. While these earth creatures stood there in almost breathless silence, a vast host of celestial beings had swung into their places preparatory to answering the signal for action when it should be given by Gabriel, their commander.
\vs p168 2:2 Jesus lifted up his eyes and said: \textcolour{ubdarkred}{“Father, I am thankful that you heard and granted my request. I know that you always hear me, but because of those who stand here with me, I thus speak with you, that they may believe that you have sent me into the world, and that they may know that you are working with me in that which we are about to do.”} And when he had prayed, he cried with a loud voice, “Lazarus, come forth!”
\vs p168 2:3 Though these human observers remained motionless, the vast celestial host was all astir in unified action in obedience to the Creator’s word. In just 12 seconds of earth time the hitherto lifeless form of Lazarus began to move and presently sat up on the edge of the stone shelf whereon it had rested. His body was bound about with grave cloths, and his face was covered with a napkin. And as he stood up before them --- alive --- Jesus said, \textcolour{ubdarkred}{“Loose him and let him go.”}
\vs p168 2:4 All, save the apostles, with Martha and Mary, fled to the house. They were pale with fright and overcome with astonishment. While some tarried, many hastened to their homes.
\vs p168 2:5 Lazarus greeted Jesus and the apostles and asked the meaning of the grave cloths and why he had awakened in the garden. Jesus and the apostles drew to one side while Martha told Lazarus of his death, burial, and resurrection. She had to explain to him that he had died on Sunday and was now brought back to life on Thursday, inasmuch as he had had no consciousness of time since falling asleep in death.
\vs p168 2:6 \pc As Lazarus came out of the tomb, the Personalized Adjuster of Jesus, now chief of his kind in this local universe, gave command to the former Adjuster of Lazarus, now in waiting, to resume abode in the mind and soul of the resurrected man.
\vs p168 2:7 \pc Then went Lazarus over to Jesus and, with his sisters, knelt at the Master’s feet to give thanks and offer praise to God. Jesus, taking Lazarus by the hand, lifted him up, saying: \textcolour{ubdarkred}{“My son, what has happened to you will also be experienced by all who believe this gospel except that they shall be resurrected in a more glorious form. You shall be a living witness of the truth which I spoke --- I am the resurrection and the life. But let us all now go into the house and partake of nourishment for these physical bodies.”}
\vs p168 2:8 \pc As they walked toward the house, Gabriel dismissed the extra groups of the assembled heavenly host while he made record of the first instance on Urantia, and the last, where a mortal creature had been resurrected in the likeness of the physical body of death.
\vs p168 2:9 \pc Lazarus could hardly comprehend what had occurred. He knew he had been very sick, but he could recall only that he had fallen asleep and been awakened. He was never able to tell anything about these four days in the tomb because he was wholly unconscious. Time is nonexistent to those who sleep the sleep of death.
\vs p168 2:10 Though many believed in Jesus as a result of this mighty work, others only hardened their hearts the more to reject him. By noon the next day this story had spread over all Jerusalem. Scores of men and women went to Bethany to look upon Lazarus and talk with him, and the alarmed and disconcerted Pharisees hastily called a meeting of the Sanhedrin that they might determine what should be done about these new developments.
\usection{3.\bibnobreakspace Meeting of the Sanhedrin}
\vs p168 3:1 Even though the testimony of this man raised from the dead did much to consolidate the faith of the mass of believers in the gospel of the kingdom, it had little or no influence on the attitude of the religious leaders and rulers at Jerusalem except to hasten their decision to destroy Jesus and stop his work.
\vs p168 3:2 \pc At 13:00 the next day, Friday, the Sanhedrin met to deliberate further on the question, “What shall we do with Jesus of Nazareth?” After more than two hours of discussion and acrimonious debate, a certain Pharisee presented a resolution calling for Jesus’ immediate death, proclaiming that he was a menace to all Israel and formally committing the Sanhedrin to the decision of death, without trial and in defiance of all precedent.
\vs p168 3:3 Time and again had this august body of Jewish leaders decreed that Jesus be apprehended and brought to trial on charges of blasphemy and numerous other accusations of flouting the Jewish sacred law. They had once before even gone so far as to declare he should die, but this was the first time the Sanhedrin had gone on record as desiring to decree his death in advance of a trial. But this resolution did not come to a vote since 14 members of the Sanhedrin resigned in a body when such an unheard\hyp{}of action was proposed. While these resignations were not formally acted upon for almost two weeks, this group of 14 withdrew from the Sanhedrin on that day, never again to sit in the council. When these resignations were subsequently acted upon, five other members were thrown out because their associates believed they entertained friendly feelings toward Jesus. With the ejection of these 19 men the Sanhedrin was in a position to try and to condemn Jesus with a solidarity bordering on unanimity.
\vs p168 3:4 The following week Lazarus and his sisters were summoned to appear before the Sanhedrin. When their testimony had been heard, no doubt could be entertained that Lazarus had been raised from the dead. Though the transactions of the Sanhedrin virtually admitted the resurrection of Lazarus, the record carried a resolution attributing this and all other wonders worked by Jesus to the power of the prince of devils, with whom Jesus was declared to be in league.
\vs p168 3:5 No matter what the source of his wonder\hyp{}working power, these Jewish leaders were persuaded that, if he were not immediately stopped, very soon all the common people would believe in him; and further, that serious complications with the Roman authorities would arise since so many of his believers regarded him as the Messiah, Israel’s deliverer.
\vs p168 3:6 It was at this same meeting of the Sanhedrin that Caiaphas the high priest first gave expression to that old Jewish adage, which he so many times repeated: “It is better that one man die, than that the community perish.”
\vs p168 3:7 Although Jesus had received warning of the doings of the Sanhedrin on this dark Friday afternoon, he was not in the least perturbed and continued resting over the Sabbath with friends in Bethpage, a hamlet near Bethany. Early Sunday morning Jesus and the apostles assembled, by prearrangement, at the home of Lazarus, and taking leave of the Bethany family, they started on their journey back to the Pella encampment.\fnc{ \bibexpl{Informational: first printing \ldots{}Bethphage\ldots{}; --- The 1955 text uses Bethpage in all 13 occurrences of this word. In the 4\ts{th} printing, the original was changed to Bethphage here, and at ten other locations; the remaining two were changed in the 9\ts{th} printing. These changes were presumably made because Bethphage is the spelling found in English Bibles since the Authorized Version (King James) of 1611. While the apparent misspelling in The Urantia Book is not theologically or historically significant, it seems unlikely to the committee that so many identical typographical errors could have occurred, so the spelling “Bethpage” must have been used in the original manuscript. The committee made its decision to retain the original form based on three factors: 1) It is the only form found in the text of the UB itself; 2)The form is a reasonably accurate transliteration of the sound of the original; and 3) Though the form found in the UB is uncommon, it is not unique --- the spelling having been found in numerous texts pre-dating the UB and in various references down to the present day including a number on the Web.}}
\usection{4.\bibnobreakspace The Answer to Prayer}
\vs p168 4:1 On the way from Bethany to Pella the apostles asked Jesus many questions, all of which the Master freely answered except those involving the details of the resurrection of the dead. Such problems were beyond the comprehension capacity of his apostles; therefore did the Master decline to discuss these questions with them. Since they had departed from Bethany in secret, they were alone. Jesus therefore embraced the opportunity to say many things to the ten which he thought would prepare them for the trying days just ahead.
\vs p168 4:2 The apostles were much stirred up in their minds and spent considerable time discussing their recent experiences as they were related to prayer and its answering. They all recalled Jesus’ statement to the Bethany messenger at Philadelphia, when he said plainly, \textcolour{ubdarkred}{“This sickness is not really to the death.”} And yet, in spite of this promise, Lazarus actually died. All that day, again and again, they reverted to the discussion of this question of the answer to prayer.
\vs p168 4:3 Jesus’ answers to their many questions may be summarized as follows:
\vs p168 4:4 \ublistelem{1.}\bibnobreakspace Prayer is an expression of the finite mind in an effort to approach the Infinite. The making of a prayer must, therefore, be limited by the knowledge, wisdom, and attributes of the finite; likewise must the answer be conditioned by the vision, aims, ideals, and prerogatives of the Infinite. There never can be observed an unbroken continuity of material phenomena between the making of a prayer and the reception of the full spiritual answer thereto.
\vs p168 4:5 \ublistelem{2.}\bibnobreakspace When a prayer is apparently unanswered, the delay often betokens a better answer, although one which is for some good reason greatly delayed. When Jesus said that Lazarus’s sickness was really not to the death, he had already been dead 11 hours. No sincere prayer is denied an answer except when the superior viewpoint of the spiritual world has devised a better answer, an answer which meets the petition of the spirit of man as contrasted with the prayer of the mere mind of man.
\vs p168 4:6 \ublistelem{3.}\bibnobreakspace The prayers of time, when indited by the spirit and expressed in faith, are often so vast and all\hyp{}encompassing that they can be answered only in eternity; the finite petition is sometimes so fraught with the grasp of the Infinite that the answer must long be postponed to await the creation of adequate capacity for receptivity; the prayer of faith may be so all\hyp{}embracing that the answer can be received only on Paradise.
\vs p168 4:7 \ublistelem{4.}\bibnobreakspace The answers to the prayer of the mortal mind are often of such a nature that they can be received and recognized only after that same praying mind has attained the immortal state. The prayer of the material being can many times be answered only when such an individual has progressed to the spirit level.
\vs p168 4:8 \ublistelem{5.}\bibnobreakspace The prayer of a God\hyp{}knowing person may be so distorted by ignorance and so deformed by superstition that the answer thereto would be highly undesirable. Then must the intervening spirit beings so translate such a prayer that, when the answer arrives, the petitioner wholly fails to recognize it as the answer to his prayer.
\vs p168 4:9 \ublistelem{6.}\bibnobreakspace All true prayers are addressed to spiritual beings, and all such petitions must be answered in spiritual terms, and all such answers must consist in spiritual realities. Spirit beings cannot bestow material answers to the spirit petitions of even material beings. Material beings can pray effectively only when they “pray in the spirit.”
\vs p168 4:10 \ublistelem{7.}\bibnobreakspace No prayer can hope for an answer unless it is born of the spirit and nurtured by faith. Your sincere faith implies that you have in advance virtually granted your prayer hearers the full right to answer your petitions in accordance with that supreme wisdom and that divine love which your faith depicts as always actuating those beings to whom you pray.
\vs p168 4:11 \ublistelem{8.}\bibnobreakspace The child is always within his rights when he presumes to petition the parent; and the parent is always within his parental obligations to the immature child when his superior wisdom dictates that the answer to the child’s prayer be delayed, modified, segregated, transcended, or postponed to another stage of spiritual ascension.
\vs p168 4:12 \ublistelem{9.}\bibnobreakspace Do not hesitate to pray the prayers of spirit longing; doubt not that you shall receive the answer to your petitions. These answers will be on deposit, awaiting your achievement of those future spiritual levels of actual cosmic attainment, on this world or on others, whereon it will become possible for you to recognize and appropriate the long\hyp{}waiting answers to your earlier but ill\hyp{}timed petitions.
\vs p168 4:13 \ublistelem{10.}\bibnobreakspace All genuine spirit\hyp{}born petitions are certain of an answer. Ask and you shall receive. But you should remember that you are progressive creatures of time and space; therefore must you constantly reckon with the time\hyp{}space factor in the experience of your personal reception of the full answers to your manifold prayers and petitions.
\usection{5.\bibnobreakspace What Became of Lazarus}
\vs p168 5:1 Lazarus remained at the Bethany home, being the centre of great interest to many sincere believers and to numerous curious individuals, until the days of the crucifixion of Jesus, when he received warning that the Sanhedrin had decreed his death. The rulers of the Jews were determined to put a stop to the further spread of the teachings of Jesus, and they well judged that it would be useless to put Jesus to death if they permitted Lazarus, who represented the very peak of his wonder\hyp{}working, to live and bear testimony to the fact that Jesus had raised him from the dead. Already had Lazarus suffered bitter persecution from them.\fnc{Lazarus remained at the Bethany home, being the centre of great interest to many sincere believers and to numerous curious individuals, until the \bibtextul{day} of the crucifixion of Jesus, when he received warning that the Sanhedrin had decreed his death. \bibexpl{The change from “day” to “days” here is required because the former is inconsistent with the ensuing narrative (at \bibref[174:0.1]{p174 0:1}, \bibref[175:3.1]{p175 3:1}, and \bibref[177:5.3]{p177 5:3} which would place the time of Lazarus’s flight between Tuesday at midnight (when his death was decreed by the Sanhedrin) and Wednesday evening (when “certain ones” at the camp “knew that Lazarus had taken hasty flight from Bethany”) --- two days before the crucifixion of Jesus. Because of the near impossibility of a typographical error leading from “week” in the manuscript to the “day” found in the 1955 text, the committee rejected the “week” resolution (found in numerous printings) and adopted “days.” If the original manuscript read “days,” the loss of only a single character in typesetting would create the problematic “day.” This is a very common type of error and well within the realm of possibility. Though “days” is a new resolution to this problem and therefore unfamiliar to readers --- perhaps some will see it as a “stretch” --- it bears repeating (as with West/west at \bibref[79:5.6]{p079 5:6}) that if the 1955 text had originally read “days,” there would have been no contradiction in the text and the issue would never have been raised in the first place.}}
\vs p168 5:2 And so Lazarus took hasty leave of his sisters at Bethany, fleeing down through Jericho and across the Jordan, never permitting himself to rest long until he had reached Philadelphia. Lazarus knew Abner well, and here he felt safe from the murderous intrigues of the wicked Sanhedrin.
\vs p168 5:3 Soon after this Martha and Mary disposed of their lands at Bethany and joined their brother in Perea. Meantime, Lazarus had become the treasurer of the church at Philadelphia. He became a strong supporter of Abner in his controversy with Paul and the Jerusalem church and ultimately died, when 67 years old, of the same sickness that carried him off when he was a younger man at Bethany.
