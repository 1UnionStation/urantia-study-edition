\upaper{78}{The Violet Race after the Days of Adam}
\uminitoc{Racial and Cultural Distribution}
\uminitoc{The Adamites in the Second Garden}
\uminitoc{Early Expansions of the Adamites}
\uminitoc{The Andites}
\uminitoc{The Andite Migrations}
\uminitoc{The Last Andite Dispersions}
\uminitoc{The Floods in Mesopotamia}
\uminitoc{The Sumerians --- Last of the Andites}
\author{Archangel}
\vs p078 0:1 The second Eden was the cradle of civilization for almost 30,000 years. Here in Mesopotamia the Adamic peoples held forth, sending out their progeny to the ends of the earth, and latterly, as amalgamated with the Nodite and Sangik tribes, were known as the Andites. From this region went those men and women who initiated the doings of historic times, and who have so enormously accelerated cultural progress on Urantia.
\vs p078 0:2 This paper depicts the planetary history of the violet race, beginning soon after the default of Adam, about 35,000\,B.C., and extending down through its amalgamation with the Nodite and Sangik races, about 15,000\,B.C., to form the Andite peoples and on to its final disappearance from the Mesopotamian homelands, about 2000\,B.C.
\usection{Racial and Cultural Distribution}
\vs p078 1:1 Although the minds and morals of the races were at a low level at the time of Adam’s arrival, physical evolution had gone on quite unaffected by the exigencies of the Caligastia rebellion. Adam’s contribution to the biologic status of the races, notwithstanding the partial failure of the undertaking, enormously upstepped the people of Urantia.
\vs p078 1:2 Adam and Eve also contributed much that was of value to the social, moral, and intellectual progress of mankind; civilization was immensely quickened by the presence of their offspring. But 35,000 years ago the world at large possessed little culture. Certain centres of civilization existed here and there, but most of Urantia languished in savagery. Racial and cultural distribution was as follows:
\vs p078 1:3 \ublistelem{1.}\bibnobreakspace \bibemph{The violet race --- Adamites and Adamsonites.} The chief centre of Adamite culture was in the second garden, located in the triangle of the Tigris and Euphrates rivers; this was indeed the cradle of Occidental and Indian civilizations. The secondary or northern centre of the violet race was the Adamsonite headquarters, situated east of the southern shore of the Caspian Sea near the Kopet mountains. From these two centres there went forth to the surrounding lands the culture and life plasm which so immediately quickened all the races.
\vs p078 1:4 \ublistelem{2.}\bibnobreakspace \bibemph{Pre\hyp{}Sumerians and other Nodites.} There were also present in Mesopotamia, near the mouth of the rivers, remnants of the ancient culture of the days of Dalamatia. With the passing millenniums, this group became thoroughly admixed with the Adamites to the north, but they never entirely lost their Nodite traditions. Various other Nodite groups that had settled in the Levant were, in general, absorbed by the later expanding violet race.
\vs p078 1:5 \ublistelem{3.}\bibnobreakspace \bibemph{The Andonites} maintained five or six fairly representative settlements to the north and east of the Adamson headquarters. They were also scattered throughout Turkestan, while isolated islands of them persisted throughout Eurasia, especially in mountainous regions. These aborigines still held the northlands of the Eurasian continent, together with Iceland and Greenland, but they had long since been driven from the plains of Europe by the blue man and from the river valleys of farther Asia by the expanding yellow race.
\vs p078 1:6 \ublistelem{4.}\bibnobreakspace \bibemph{The red man} occupied the Americas, having been driven out of Asia over 50,000 years before the arrival of Adam.
\vs p078 1:7 \ublistelem{5.}\bibnobreakspace \bibemph{The yellow race.} The Chinese peoples were well established in control of eastern Asia. Their most advanced settlements were situated to the north\hyp{}west of modern China in regions bordering on Tibet.
\vs p078 1:8 \ublistelem{6.}\bibnobreakspace \bibemph{The blue race.} The blue men were scattered all over Europe, but their better centres of culture were situated in the then fertile valleys of the Mediterranean basin and in north\hyp{}western Europe. Neanderthal absorption had greatly retarded the culture of the blue man, but he was otherwise the most aggressive, adventurous, and exploratory of all the evolutionary peoples of Eurasia.
\vs p078 1:9 \ublistelem{7.}\bibnobreakspace \bibemph{Pre\hyp{}Dravidian India.} The complex mixture of races in India --- embracing every race on earth, but especially the green, orange, and black --- maintained a culture slightly above that of the outlying regions.
\vs p078 1:10 \ublistelem{8.}\bibnobreakspace \bibemph{The Sahara civilization.} The superior elements of the indigo race had their most progressive settlements in what is now the great Sahara desert. This indigo\hyp{}black group carried extensive strains of the submerged orange and green races.
\vs p078 1:11 \ublistelem{9.}\bibnobreakspace \bibemph{The Mediterranean basin.} The most highly blended race outside of India occupied what is now the Mediterranean basin. Here blue men from the north and Saharans from the south met and mingled with Nodites and Adamites from the east.
\vs p078 1:12 \pc This was the picture of the world prior to the beginnings of the great expansions of the violet race, about 25,000 years ago. The hope of future civilization lay in the second garden between the rivers of Mesopotamia. Here in south\hyp{}western Asia there existed the potential of a great civilization, the possibility of the spread to the world of the ideas and ideals which had been salvaged from the days of Dalamatia and the times of Eden.
\vs p078 1:13 Adam and Eve had left behind a limited but potent progeny, and the celestial observers on Urantia waited anxiously to find out how these descendants of the erring Material Son and Daughter would acquit themselves.
\usection{The Adamites in the Second Garden}
\vs p078 2:1 For thousands of years the sons of Adam laboured along the rivers of Mesopotamia, working out their irrigation and flood\hyp{}control problems to the south, perfecting their defences to the north, and attempting to preserve their traditions of the glory of the first Eden.
\vs p078 2:2 The heroism displayed in the leadership of the second garden constitutes one of the amazing and inspiring epics of Urantia’s history. These splendid souls never wholly lost sight of the purpose of the Adamic mission, and therefore did they valiantly fight off the influences of the surrounding and inferior tribes while they willingly sent forth their choicest sons and daughters in a steady stream as emissaries to the races of earth. Sometimes this expansion was depleting to the home culture, but always these superior peoples would rehabilitate themselves.
\vs p078 2:3 The civilization, society, and cultural status of the Adamites were far above the general level of the evolutionary races of Urantia. Only among the old settlements of Van and Amadon and the Adamsonites was there a civilization in any way comparable. But the civilization of the second Eden was an artificial structure --- \bibemph{it had not been evolved} ---  and was therefore doomed to deteriorate until it reached a natural evolutionary level.
\vs p078 2:4 Adam left a great intellectual and spiritual culture behind him, but it was not advanced in mechanical appliances since every civilization is limited by available natural resources, inherent genius, and sufficient leisure to ensure inventive fruition. The civilization of the violet race was predicated on the presence of Adam and on the traditions of the first Eden. After Adam’s death and as these traditions grew dim through the passing millenniums, the cultural level of the Adamites steadily deteriorated until it reached a state of reciprocal balance with the status of the surrounding peoples and the naturally evolving cultural capacities of the violet race.
\vs p078 2:5 But the Adamites were a real nation around 19,000\,B.C., numbering 4,500,000, and already they had poured forth millions of their progeny into the surrounding peoples.
\usection{Early Expansions of the Adamites}
\vs p078 3:1 The violet race retained the Edenic traditions of peacefulness for many millenniums, which explains their long delay in making territorial conquests. When they suffered from population pressure, instead of making war to secure more territory, they sent forth their excess inhabitants as teachers to the other races. The cultural effect of these earlier migrations was not enduring, but the absorption of the Adamite teachers, traders, and explorers was biologically invigorating to the surrounding peoples.
\vs p078 3:2 Some of the Adamites early journeyed westward to the valley of the Nile; others penetrated eastward into Asia, but these were a minority. The mass movement of the later days was extensively northward and thence westward. It was, in the main, a gradual but unremitting northward push, the greater number making their way north and then circling westward around the Caspian Sea into Europe.
\vs p078 3:3 About 25,000 years ago many of the purer elements of the Adamites were well on their northern trek. And as they penetrated northward, they became less and less Adamic until, by the times of their occupation of Turkestan, they had become thoroughly admixed with the other races, particularly the Nodites. Very few of the pure\hyp{}line violet peoples ever penetrated far into Europe or Asia.
\vs p078 3:4 From about 30,000 to 10,000\,B.C. epoch\hyp{}making racial mixtures were taking place throughout south\hyp{}western Asia. The highland inhabitants of Turkestan were a virile and vigorous people. To the north\hyp{}west of India much of the culture of the days of Van persisted. Still to the north of these settlements the best of the early Andonites had been preserved. And both of these superior races of culture and character were absorbed by the northward\hyp{}moving Adamites. This amalgamation led to the adoption of many new ideas; it facilitated the progress of civilization and greatly advanced all phases of art, science, and social culture.
\vs p078 3:5 \pc As the period of the early Adamic migrations ended, about 15,000\,B.C., there were already more descendants of Adam in Europe and central Asia than anywhere else in the world, even than in Mesopotamia. The European blue races had been largely infiltrated. The lands now called Russia and Turkestan were occupied throughout their southern stretches by a great reservoir of the Adamites mixed with Nodites, Andonites, and red and yellow Sangiks. Southern Europe and the Mediterranean fringe were occupied by a mixed race of Andonite and Sangik peoples --- orange, green, and indigo --- with a sprinkling of the Adamite stock. Asia Minor and the central\hyp{}eastern European lands were held by tribes that were predominantly Andonite.
\vs p078 3:6 A blended coloured race, about this time greatly reinforced by arrivals from Mesopotamia, held forth in Egypt and prepared to take over the disappearing culture of the Euphrates valley. The black peoples were moving farther south in Africa and, like the red race, were virtually isolated.
\vs p078 3:7 The Saharan civilization had been disrupted by drought and that of the Mediterranean basin by flood. The blue races had, as yet, failed to develop an advanced culture. The Andonites were still scattered over the Arctic and central Asian regions. The green and orange races had been exterminated as such. The indigo race was moving south in Africa, there to begin its slow but long\hyp{}continued racial deterioration.
\vs p078 3:8 The peoples of India lay stagnant, with a civilization that was unprogressing; the yellow man was consolidating his holdings in central Asia; the brown man had not yet begun his civilization on the near\hyp{}by islands of the Pacific.
\vs p078 3:9 \pc These racial distributions, associated with extensive climatic changes, set the world stage for the inauguration of the Andite era of Urantia civilization. These early migrations extended over a period of 10,000 years, from 25,000 to 15,000\,B.C. The later or Andite migrations extended from about 15,000 to 6000\,B.C.
\vs p078 3:10 It took so long for the earlier waves of Adamites to pass over Eurasia that their culture was largely lost in transit. Only the later Andites moved with sufficient speed to retain the Edenic culture at any great distance from Mesopotamia.
\usection{The Andites}
\vs p078 4:1 The Andite races were the primary blends of the pure\hyp{}line violet race and the Nodites plus the evolutionary peoples. In general, Andites should be thought of as having a far greater percentage of Adamic blood than the modern races. In the main, the term Andite is used to designate those peoples whose racial inheritance was from \bibfrac{1}{8}\ts{th} to \bibfrac{1}{6}\ts{th} violet. Modern Urantians, even the northern white races, contain much less than this percentage of the blood of Adam.
\vs p078 4:2 The earliest Andite peoples took origin in the regions adjacent to Mesopotamia more than 25,000 years ago and consisted of a blend of the Adamites and Nodites. The second garden was surrounded by concentric circles of diminishing violet blood, and it was on the periphery of this racial melting pot that the Andite race was born. Later on, when the migrating Adamites and Nodites entered the then fertile regions of Turkestan, they soon blended with the superior inhabitants, and the resultant race mixture extended the Andite type northward.
\vs p078 4:3 The Andites were the best all\hyp{}round human stock to appear on Urantia since the days of the pure\hyp{}line violet peoples. They embraced most of the highest types of the surviving remnants of the Adamite and Nodite races and, later, some of the best strains of the yellow, blue, and green men.
\vs p078 4:4 \pc These early Andites were not Aryan; they were pre\hyp{}Aryan. They were not white; they were pre\hyp{}white. They were neither an Occidental nor an Oriental people. But it is Andite inheritance that gives to the polyglot mixture of the so\hyp{}called white races that generalized homogeneity which has been called Caucasoid.
\vs p078 4:5 \pc The purer strains of the violet race had retained the Adamic tradition of peace\hyp{}seeking, which explains why the earlier race movements had been more in the nature of peaceful migrations. But as the Adamites united with the Nodite stocks, who were by this time a belligerent race, their Andite descendants became, for their day and age, the most skillful and sagacious militarists ever to live on Urantia. Thenceforth the movements of the Mesopotamians grew increasingly military in character and became more akin to actual conquests.
\vs p078 4:6 These Andites were adventurous; they had roving dispositions. An increase of either Sangik or Andonite stock tended to stabilize them. But even so, their later descendants never stopped until they had circumnavigated the globe and discovered the last remote continent.
\usection{The Andite Migrations}
\vs p078 5:1 For 20,000 years the culture of the second garden persisted, but it experienced a steady decline until about 15,000\,B.C., when the regeneration of the Sethite priesthood and the leadership of Amosad inaugurated a brilliant era. The massive waves of civilization which later spread over Eurasia immediately followed the great renaissance of the Garden consequent upon the extensive union of the Adamites with the surrounding mixed Nodites to form the Andites.
\vs p078 5:2 These Andites inaugurated new advances throughout Eurasia and North Africa. From Mesopotamia through Sinkiang the Andite culture was dominant, and the steady migration toward Europe was continuously offset by new arrivals from Mesopotamia. But it is hardly correct to speak of the Andites as a race in Mesopotamia proper until near the beginning of the terminal migrations of the mixed descendants of Adam. By this time even the races in the second garden had become so blended that they could no longer be considered Adamites.
\vs p078 5:3 The civilization of Turkestan was constantly being revived and refreshed by the newcomers from Mesopotamia, especially by the later Andite cavalrymen. The so\hyp{}called Aryan mother tongue was in process of formation in the highlands of Turkestan; it was a blend of the Andonic dialect of that region with the language of the Adamsonites and later Andites. Many modern languages are derived from this early speech of these central Asian tribes who conquered Europe, India, and the upper stretches of the Mesopotamian plains. This ancient language gave the Occidental tongues all of that similarity which is called Aryan.
\vs p078 5:4 \pc By 12,000\,B.C. \bibfrac{3}{4}\ts{ers} of the Andite stock of the world was resident in northern and eastern Europe, and when the later and final exodus from Mesopotamia took place, 65\% of these last waves of emigration entered Europe.
\vs p078 5:5 \pc The Andites not only migrated to Europe but to northern China and India, while many groups penetrated to the ends of the earth as missionaries, teachers, and traders. They contributed considerably to the northern groups of the Saharan Sangik peoples. But only a few teachers and traders ever penetrated farther south in Africa than the headwaters of the Nile. Later on, mixed Andites and Egyptians followed down both the east and west coasts of Africa well below the equator, but they did not reach Madagascar.
\vs p078 5:6 These Andites were the so\hyp{}called Dravidian and later Aryan conquerors of India; and their presence in central Asia greatly upstepped the ancestors of the Turanians. Many of this race journeyed to China by way of both Sinkiang and Tibet and added desirable qualities to the later Chinese stocks. From time to time small groups made their way into Japan, Formosa, the East Indies, and southern China, though very few entered southern China by the coastal route.
\vs p078 5:7 132 of this race, embarking in a fleet of small boats from Japan, eventually reached South America and by intermarriage with the natives of the Andes established the ancestry of the later rulers of the Incas. They crossed the Pacific by easy stages, tarrying on the many islands they found along the way. The islands of the Polynesian group were both more numerous and larger then than now, and these Andite sailors, together with some who followed them, biologically modified the native groups in transit. Many flourishing centres of civilization grew up on these now submerged lands as a result of Andite penetration. Easter Island was long a religious and administrative centre of one of these lost groups. But of the Andites who navigated the Pacific of long ago none but the 132 ever reached the mainland of the Americas.
\vs p078 5:8 \pc The migratory conquests of the Andites continued on down to their final dispersions, from 8000 to 6000\,B.C. As they poured out of Mesopotamia, they continuously depleted the biologic reserves of their homelands while markedly strengthening the surrounding peoples. And to every nation to which they journeyed, they contributed humour, art, adventure, music, and manufacture. They were skillful domesticators of animals and expert agriculturists. For the time being, at least, their presence usually improved the religious beliefs and moral practices of the older races. And so the culture of Mesopotamia quietly spread out over Europe, India, China, northern Africa, and the Pacific Islands.
\usection{The Last Andite Dispersions}
\vs p078 6:1 The last three waves of Andites poured out of Mesopotamia between 8000 and 6000\,B.C. These three great waves of culture were forced out of Mesopotamia by the pressure of the hill tribes to the east and the harassment of the plainsmen of the west. The inhabitants of the Euphrates valley and adjacent territory went forth in their final exodus in several directions:
\vs p078 6:2 65\% entered Europe by the Caspian Sea route to conquer and amalgamate with the newly appearing white races --- the blend of the blue men and the earlier Andites.
\vs p078 6:3 10\%, including a large group of the Sethite priests, moved eastward through the Elamite highlands to the Iranian plateau and Turkestan. Many of their descendants were later driven into India with their Aryan brethren from the regions to the north.
\vs p078 6:4 10\% of the Mesopotamians turned eastward in their northern trek, entering Sinkiang, where they blended with the Andite\hyp{}yellow inhabitants. The majority of the able offspring of this racial union later entered China and contributed much to the immediate improvement of the northern division of the yellow race.
\vs p078 6:5 10\% of these fleeing Andites made their way across Arabia and entered Egypt.
\vs p078 6:6 \pc 5\% of the Andites, the very superior culture of the coastal district about the mouths of the Tigris and Euphrates who had kept themselves free from intermarriage with the inferior neighbouring tribesmen, refused to leave their homes. This group represented the survival of many superior Nodite and Adamite strains.
\vs p078 6:7 \pc The Andites had almost entirely evacuated this region by 6000\,B.C., though their descendants, largely mixed with the surrounding Sangik races and the Andonites of Asia Minor, were there to give battle to the northern and eastern invaders at a much later date.
\vs p078 6:8 The cultural age of the second garden was terminated by the increasing infiltration of the surrounding inferior stocks. Civilization moved westward to the Nile and the Mediterranean islands, where it continued to thrive and advance long after its fountainhead in Mesopotamia had deteriorated. And this unchecked influx of inferior peoples prepared the way for the later conquest of all Mesopotamia by the northern barbarians who drove out the residual strains of ability. Even in later years the cultured residue still resented the presence of these ignorant and uncouth invaders.
\usection{The Floods in Mesopotamia}
\vs p078 7:1 The river dwellers were accustomed to rivers overflowing their banks at certain seasons; these periodic floods were annual events in their lives. But new perils threatened the valley of Mesopotamia as a result of progressive geologic changes to the north.
\vs p078 7:2 For thousands of years after the submergence of the first Eden the mountains about the eastern coast of the Mediterranean and those to the north\hyp{}west and north\hyp{}east of Mesopotamia continued to rise. This elevation of the highlands was greatly accelerated about 5000\,B.C., and this, together with greatly increased snowfall on the northern mountains, caused unprecedented floods each spring throughout the Euphrates valley. These spring floods grew increasingly worse so that eventually the inhabitants of the river regions were driven to the eastern highlands. For almost 1,000 years scores of cities were practically deserted because of these extensive deluges.
\vs p078 7:3 \pc Almost 5,000 years later, as the Hebrew priests in Babylonian captivity sought to trace the Jewish people back to Adam, they found great difficulty in piecing the story together; and it occurred to one of them to abandon the effort, to let the whole world drown in its wickedness at the time of Noah’s flood, and thus to be in a better position to trace Abraham right back to one of the three surviving sons of Noah.
\vs p078 7:4 The traditions of a time when water covered the whole of the earth’s surface are universal. Many races harbour the story of a world\hyp{}wide flood some time during past ages. The Biblical story of Noah, the ark, and the flood is an invention of the Hebrew priesthood during the Babylonian captivity. There has never been a universal flood since life was established on Urantia. The only time the surface of the earth was completely covered by water was during those Archeozoic ages before the land had begun to appear.
\vs p078 7:5 But Noah really lived; he was a wine maker of Aram, a river settlement near Erech. He kept a written record of the days of the river’s rise from year to year. He brought much ridicule upon himself by going up and down the river valley advocating that all houses be built of wood, boat fashion, and that the family animals be put on board each night as the flood season approached. He would go to the neighbouring river settlements every year and warn them that in so many days the floods would come. Finally a year came in which the annual floods were greatly augmented by unusually heavy rainfall so that the sudden rise of the waters wiped out the entire village; only Noah and his immediate family were saved in their houseboat.
\vs p078 7:6 \pc These floods completed the disruption of Andite civilization. With the ending of this period of deluge, the second garden was no more. Only in the south and among the Sumerians did any trace of the former glory remain.
\vs p078 7:7 The remnants of this, one of the oldest civilizations, are to be found in these regions of Mesopotamia and to the north\hyp{}east and north\hyp{}west. But still older vestiges of the days of Dalamatia exist under the waters of the Persian Gulf, and the first Eden lies submerged under the eastern end of the Mediterranean Sea.
\usection{The Sumerians --- Last of the Andites}
\vs p078 8:1 When the last Andite dispersion broke the biologic backbone of Mesopotamian civilization, a small minority of this superior race remained in their homeland near the mouths of the rivers. These were the Sumerians, and by 6000\,B.C. they had become largely Andite in extraction, though their culture was more exclusively Nodite in character, and they clung to the ancient traditions of Dalamatia. Nonetheless, these Sumerians of the coastal regions were the last of the Andites in Mesopotamia. But the races of Mesopotamia were already thoroughly blended by this late date, as is evidenced by the skull types found in the graves of this era.
\vs p078 8:2 It was during the floodtimes that Susa so greatly prospered. The first and lower city was inundated so that the second or higher town succeeded the lower as the headquarters for the peculiar artcrafts of that day. With the later diminution of these floods, Ur became the centre of the pottery industry. About 7,000 years ago Ur was on the Persian Gulf, the river deposits having since built up the land to its present limits. These settlements suffered less from the floods because of better controlling works and the widening mouths of the rivers.
\vs p078 8:3 \pc The peaceful grain growers of the Euphrates and Tigris valleys had long been harassed by the raids of the barbarians of Turkestan and the Iranian plateau. But now a concerted invasion of the Euphrates valley was brought about by the increasing drought of the highland pastures. And this invasion was all the more serious because these surrounding herdsmen and hunters possessed large numbers of tamed horses. It was the possession of horses which gave them a tremendous military advantage over their rich neighbours to the south. In a short time they overran all Mesopotamia, driving forth the last waves of culture which spread out over all of Europe, western Asia, and northern Africa.
\vs p078 8:4 These conquerors of Mesopotamia carried in their ranks many of the better Andite strains of the mixed northern races of Turkestan, including some of the Adamson stock. These less advanced but more vigorous tribes from the north quickly and willingly assimilated the residue of the civilization of Mesopotamia and presently developed into those mixed peoples found in the Euphrates valley at the beginning of historic annals. They quickly revived many phases of the passing civilization of Mesopotamia, adopting the arts of the valley tribes and much of the culture of the Sumerians. They even sought to build a third tower of Babel and later adopted the term as their national name.
\vs p078 8:5 When these barbarian cavalrymen from the north\hyp{}east overran the whole Euphrates valley, they did not conquer the remnants of the Andites who dwelt about the mouth of the river on the Persian Gulf. These Sumerians were able to defend themselves because of superior intelligence, better weapons, and their extensive system of military canals, which were an adjunct to their irrigation scheme of interconnecting pools. They were a united people because they had a uniform group religion. They were thus able to maintain their racial and national integrity long after their neighbours to the north\hyp{}west were broken up into isolated city\hyp{}states. No one of these city groups was able to overcome the united Sumerians.
\vs p078 8:6 And the invaders from the north soon learned to trust and prize these peace\hyp{}loving Sumerians as able teachers and administrators. They were greatly respected and sought after as teachers of art and industry, as directors of commerce, and as civil rulers by all peoples to the north and from Egypt in the west to India in the east.
\vs p078 8:7 After the breakup of the early Sumerian confederation the later city\hyp{}states were ruled by the apostate descendants of the Sethite priests. Only when these priests made conquests of the neighbouring cities did they call themselves kings. The later city kings failed to form powerful confederations before the days of Sargon because of deity jealousy. Each city believed its municipal god to be superior to all other gods, and therefore they refused to subordinate themselves to a common leader.
\vs p078 8:8 The end of this long period of the weak rule of the city priests was terminated by Sargon, the priest of Kish, who proclaimed himself king and started out on the conquest of the whole of Mesopotamia and adjoining lands. And for the time, this ended the city\hyp{}states, priest\hyp{}ruled and priest\hyp{}ridden, each city having its own municipal god and its own ceremonial practices.
\vs p078 8:9 After the breakup of this Kish confederation there ensued a long period of constant warfare between these valley cities for supremacy. And the rulership variously shifted between Sumer, Akkad, Kish, Erech, Ur, and Susa.
\vs p078 8:10 About 2500\,B.C. the Sumerians suffered severe reverses at the hands of the northern Suites and Guites. Lagash, the Sumerian capital built on flood mounds, fell. Erech held out for 30 years after the fall of Akkad. By the time of the establishment of the rule of Hammurabi the Sumerians had become absorbed into the ranks of the northern Semites, and the Mesopotamian Andites passed from the pages of history.
\vs p078 8:11 From 2500 to 2000\,B.C. the nomads were on a rampage from the Atlantic to the Pacific. The Nerites\fnst{\textbf{Nerites}, Could these be the same as ``the people of Nairi'' (arm. \textarm{Նաիրի}), identified with the ancient kingdom of Urartu, also known as the ``Kingdom of Van''? The names ``Nairi'' and ``Urartu'' are used interchangeably in the Assyrian inscriptions of king Ashurnasirpal~II (883--859\,B.C.) and Shalmaneser~III (860--825\,B.C.)} constituted the final eruption of the Caspian group of the Mesopotamian descendants of the blended Andonite and Andite races. What the barbarians failed to do to effect the ruination of Mesopotamia, subsequent climatic changes succeeded in accomplishing.
\vs p078 8:12 \pc And this is the story of the violet race after the days of Adam and of the fate of their homeland between the Tigris and Euphrates. Their ancient civilization finally fell due to the emigration of superior peoples and the immigration of their inferior neighbours. But long before the barbarian cavalrymen conquered the valley, much of the Garden culture had spread to Asia, Africa, and Europe, there to produce the ferments which have resulted in the XX century civilization of Urantia.
\vsetoff
\vs p078 8:13 [Presented by an Archangel of Nebadon.]
\quizlink
