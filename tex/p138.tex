\upaper{138}{Training the Kingdom’s Messengers}
\author{Midwayer Commission}
\vs p138 0:1 After preaching the sermon on “The Kingdom,” Jesus called the six apostles together that afternoon and began to disclose his plans for visiting the cities around and about the Sea of Galilee. His brothers James and Jude were very much hurt because they were not called to this conference. Up to this time they had regarded themselves as belonging to Jesus’ inner circle of associates. But Jesus planned to have no close relatives as members of this corps of apostolic directors of the kingdom. This failure to include James and Jude among the chosen few, together with his apparent aloofness from his mother ever since the experience at Cana, was the starting point of an ever\hyp{}widening gulf between Jesus and his family. This situation continued throughout his public ministry --- they very nearly rejected him --- and these differences were not fully removed until after his death and resurrection. His mother constantly wavered between attitudes of fluctuating faith and hope, and increasing emotions of disappointment, humiliation, and despair. Only Ruth, the youngest, remained unswervingly loyal to her father\hyp{}brother.
\vs p138 0:2 Until after the resurrection, Jesus’ entire family had very little to do with his ministry. If a prophet is not without honour save in his own country, he is not without understanding appreciation save in his own family.
\usection{1.\bibnobreakspace Final Instructions}
\vs p138 1:1 The next day, Sunday, June 23, A.D.\,26, Jesus imparted his final instructions to the six. He directed them to go forth, two and two, to teach the glad tidings of the kingdom. He forbade them to baptize and advised against public preaching. He went on to explain that later he would permit them to preach in public, but that for a season, and for many reasons, he desired them to acquire practical experience in dealing personally with their fellow men. Jesus purposed to make their first tour entirely one of \bibemph{personal work.} Although this announcement was something of a disappointment to the apostles, still they saw, at least in part, Jesus’ reason for thus beginning the proclamation of the kingdom, and they started out in good heart and with confident enthusiasm. He sent them forth by twos, James and John going to Kheresa, Andrew and Peter to Capernaum, while Philip and Nathaniel went to Tarichea.
\vs p138 1:2 Before they began this first two weeks of service, Jesus announced to them that he desired to ordain 12 apostles to continue the work of the kingdom after his departure and authorized each of them to choose one man from among his early converts for membership in the projected corps of apostles. John spoke up, asking: “But, Master, will these six men come into our midst and share all things equally with us who have been with you since the Jordan and have heard all your teaching in preparation for this, our first labour for the kingdom?” And Jesus replied: \textcolour{ubdarkred}{“Yes, John, the men you choose shall become one with us, and you will teach them all that pertains to the kingdom, even as I have taught you.”} After thus speaking, Jesus left them.
\vs p138 1:3 The six did not separate to go to their work until they had exchanged many words in discussion of Jesus’ instruction that each of them should choose a new apostle. Andrew’s counsel finally prevailed, and they went forth to their labours. In substance Andrew said: “The Master is right; we are too few to encompass this work. There is need for more teachers, and the Master has manifested great confidence in us inasmuch as he has entrusted us with the choosing of these six new apostles.” This morning, as they separated to go to their work, there was a bit of concealed depression in each heart. They knew they were going to miss Jesus, and besides their fear and timidity, this was not the way they had pictured the kingdom of heaven being inaugurated.
\vs p138 1:4 It had been arranged that the six were to labour for two weeks, after which they were to return to the home of Zebedee for a conference. Meantime Jesus went over to Nazareth to visit with Joseph and Simon and other members of his family living in that vicinity. Jesus did everything humanly possible, consistent with his dedication to the doing of his Father’s will, to retain the confidence and affection of his family. In this matter he did his full duty and more.
\vs p138 1:5 While the apostles were out on this mission, Jesus thought much about John, now in prison. It was a great temptation to use his potential powers to release him, but once more he resigned himself to \textcolour{ubdarkred}{“wait upon the Father’s will.”}
\usection{2.\bibnobreakspace Choosing the Six}
\vs p138 2:1 This first missionary tour of the six was eminently successful. They all discovered the great value of direct and personal contact with men. They returned to Jesus more fully realizing that, after all, religion is purely and wholly a matter of \bibemph{personal experience.} They began to sense how hungry were the common people to hear words of religious comfort and spiritual good cheer. When they assembled about Jesus, they all wanted to talk at once, but Andrew assumed charge, and as he called upon them one by one, they made their formal reports to the Master and presented their nominations for the six new apostles.
\vs p138 2:2 Jesus, after each man had presented his selection for the new apostleships, asked all the others to vote upon the nomination; thus all six of the new apostles were formally accepted by all of the older six. Then Jesus announced that they would all visit these candidates and give them the call to service.
\vs p138 2:3 The newly selected apostles were:
\vs p138 2:4 \ublistelem{1.}\bibnobreakspace \bibemph{Matthew Levi,} the customs collector of Capernaum, who had his office just to the east of the city, near the borders of Batanea. He was selected by Andrew.
\vs p138 2:5 \ublistelem{2.}\bibnobreakspace \bibemph{Thomas Didymus,} a fisherman of Tarichea and onetime carpenter and stone mason of Gadara. He was selected by Philip.
\vs p138 2:6 \ublistelem{3.}\bibnobreakspace \bibemph{James Alpheus,} a fisherman and farmer of Kheresa, was selected by James Zebedee.
\vs p138 2:7 \ublistelem{4.}\bibnobreakspace \bibemph{Judas Alpheus,} the twin brother of James Alpheus, also a fisherman, was selected by John Zebedee.
\vs p138 2:8 \ublistelem{5.}\bibnobreakspace \bibemph{Simon Zelotes} was a high officer in the patriotic organization of the Zealots, a position which he gave up to join Jesus’ apostles. Before joining the Zealots, Simon had been a merchant. He was selected by Peter.
\vs p138 2:9 \ublistelem{6.}\bibnobreakspace \bibemph{Judas Iscariot} was an only son of wealthy Jewish parents living in Jericho. He had become attached to John the Baptist, and his Sadducee parents had disowned him. He was looking for employment in these regions when Jesus’ apostles found him, and chiefly because of his experience with finances, Nathaniel invited him to join their ranks. Judas Iscariot was the only Judean among the 12 apostles.
\vs p138 2:10 \pc Jesus spent a full day with the six, answering their questions and listening to the details of their reports, for they had many interesting and profitable experiences to relate. They now saw the wisdom of the Master’s plan of sending them out to labour in a quiet and personal manner before the launching of their more pretentious public efforts.
\usection{3.\bibnobreakspace The Call of Matthew and Simon}
\vs p138 3:1 The next day Jesus and the six went to call upon Matthew, the customs collector. Matthew was awaiting them, having balanced his books and made ready to turn the affairs of his office over to his brother. As they approached the toll house, Andrew stepped forward with Jesus, who, looking into Matthew’s face, said, \textcolour{ubdarkred}{“Follow me.”} And he arose and went to his house with Jesus and the apostles.
\vs p138 3:2 Matthew told Jesus of the banquet he had arranged for that evening, at least that he wished to give such a dinner to his family and friends if Jesus would approve and consent to be the guest of honour. And Jesus nodded his consent. Peter then took Matthew aside and explained that he had invited one Simon to join the apostles and secured his consent that Simon be also bidden to this feast.
\vs p138 3:3 \pc After a noontide luncheon at Matthew’s house they all went with Peter to call upon Simon the Zealot, whom they found at his old place of business, which was now being conducted by his nephew. When Peter led Jesus up to Simon, the Master greeted the fiery patriot and only said, \textcolour{ubdarkred}{“Follow me.”}
\vs p138 3:4 \pc They all returned to Matthew’s home, where they talked much about politics and religion until the hour of the evening meal. The Levi family had long been engaged in business and tax gathering; therefore many of the guests bidden to this banquet by Matthew would have been denominated “publicans and sinners” by the Pharisees.
\vs p138 3:5 In those days, when a reception\hyp{}banquet of this sort was tendered a prominent individual, it was the custom for all interested persons to linger about the banquet room to observe the guests at meat and to listen to the conversation and speeches of the men of honour. Accordingly, most of the Capernaum Pharisees were present on this occasion to observe Jesus’ conduct at this unusual social gathering.
\vs p138 3:6 As the dinner progressed, the joy of the diners mounted to heights of good cheer, and everybody was having such a splendid time that the onlooking Pharisees began, in their hearts, to criticize Jesus for his participation in such a light\hyp{}hearted and carefree affair. Later in the evening, when they were making speeches, one of the more malignant of the Pharisees went so far as to criticize Jesus’ conduct to Peter, saying: “How dare you to teach that this man is righteous when he eats with publicans and sinners and thus lends his presence to such scenes of careless pleasure making.” Peter whispered this criticism to Jesus before he spoke the parting blessing upon those assembled. When Jesus began to speak, he said: \textcolour{ubdarkred}{“In coming here tonight to welcome Matthew and Simon to our fellowship, I am glad to witness your light\hyp{}heartedness and social good cheer, but you should rejoice still more because many of you will find entrance into the coming kingdom of the spirit, wherein you shall more abundantly enjoy the good things of the kingdom of heaven. And to you who stand about criticizing me in your hearts because I have come here to make merry with these friends, let me say that I have come to proclaim joy to the socially downtrodden and spiritual liberty to the moral captives. Need I remind you that they who are whole need not a physician, but rather those who are sick? I have come, not to call the righteous, but sinners.”}
\vs p138 3:7 And truly this was a strange sight in all Jewry: to see a man of righteous character and noble sentiments mingling freely and joyously with the common people, even with an irreligious and pleasure\hyp{}seeking throng of publicans and reputed sinners. Simon Zelotes desired to make a speech at this gathering in Matthew’s house, but Andrew, knowing that Jesus did not want the coming kingdom to become confused with the Zealots’ movement, prevailed upon him to refrain from making any public remarks.
\vs p138 3:8 Jesus and the apostles remained that night in Matthew’s house, and as the people went to their homes, they spoke of but one thing: the goodness and friendliness of Jesus.
\usection{4.\bibnobreakspace The Call of the Twins}
\vs p138 4:1 On the morrow all nine of them went by boat over to Kheresa to execute the formal calling of the next two apostles, James and Judas the twin sons of Alpheus, the nominees of James and John Zebedee. The fisherman twins were expecting Jesus and his apostles and were therefore awaiting them on the shore. James Zebedee presented the Master to the Kheresa fishermen, and Jesus, gazing on them, nodded and said, \textcolour{ubdarkred}{“Follow me.”}
\vs p138 4:2 \pc That afternoon, which they spent together, Jesus fully instructed them concerning attendance upon festive gatherings, concluding his remarks by saying: \textcolour{ubdarkred}{“All men are my brothers. My Father in heaven does not despise any creature of our making. The kingdom of heaven is open to all men and women. No man may close the door of mercy in the face of any hungry soul who may seek to gain an entrance thereto. We will sit at meat with all who desire to hear of the kingdom. As our Father in heaven looks down upon men, they are all alike. Refuse not therefore to break bread with Pharisee or sinner, Sadducee or publican, Roman or Jew, rich or poor, free or bond. The door of the kingdom is wide open for all who desire to know the truth and to find God.”}
\vs p138 4:3 \pc That night at a simple supper at the Alpheus home, the twin brothers were received into the apostolic family. Later in the evening Jesus gave his apostles their first lesson dealing with the origin, nature, and destiny of unclean spirits, but they could not comprehend the import of what he told them. They found it very easy to love and admire Jesus but very difficult to understand many of his teachings.
\vs p138 4:4 After a night of rest the entire party, now numbering 11, went by boat over to Tarichea.
\usection{5.\bibnobreakspace The Call of Thomas and Judas}
\vs p138 5:1 Thomas the fisherman and Judas the wanderer met Jesus and the apostles at the fisher\hyp{}boat landing at Tarichea, and Thomas led the party to his near\hyp{}by home. Philip now presented Thomas as his nominee for apostleship and Nathaniel presented Judas Iscariot, the Judean, for similar honours. Jesus looked upon Thomas and said: \textcolour{ubdarkred}{“Thomas, you lack faith; nevertheless, I receive you. Follow me.”} To Judas Iscariot the Master said: \textcolour{ubdarkred}{“Judas, we are all of one flesh, and as I receive you into our midst, I pray that you will always be loyal to your Galilean brethren. Follow me.”}
\vs p138 5:2 \pc When they had refreshed themselves, Jesus took the 12 apart for a season to pray with them and to instruct them in the nature and work of the Holy Spirit, but again did they largely fail to comprehend the meaning of those wonderful truths which he endeavoured to teach them. One would grasp one point and one would comprehend another, but none of them could encompass the whole of his teaching. Always would they make the mistake of trying to fit Jesus’ new gospel into their old forms of religious belief. They could not grasp the idea that Jesus had come to proclaim a new gospel of salvation and to establish a new way of finding God; they did not perceive that he \bibemph{was} a new revelation of the Father in heaven.
\vs p138 5:3 The next day Jesus left his 12 apostles quite alone; he wanted them to become acquainted and desired that they be alone to talk over what he had taught them. The Master returned for the evening meal, and during the after\hyp{}supper hours he talked to them about the ministry of seraphim, and some of the apostles comprehended his teaching. They rested for a night and the next day departed by boat for Capernaum.
\vs p138 5:4 Zebedee and Salome had gone to live with their son David so that their large home could be turned over to Jesus and his 12 apostles. Here Jesus spent a quiet Sabbath with his chosen messengers; he carefully outlined the plans for proclaiming the kingdom and fully explained the importance of avoiding any clash with the civil authorities, saying: \textcolour{ubdarkred}{“If the civil rulers are to be rebuked, leave that task to me. See that you make no denunciations of Caesar or his servants.”} It was this same evening that Judas Iscariot took Jesus aside to inquire why nothing was done to get John out of prison. And Judas was not wholly satisfied with Jesus’ attitude.
\usection{6.\bibnobreakspace The Week of Intensive Training}
\vs p138 6:1 The next week was devoted to a program of intense training. Each day the six new apostles were put in the hands of their respective nominators for a thoroughgoing review of all they had learned and experienced in preparation for the work of the kingdom. The older apostles carefully reviewed, for the benefit of the younger six, Jesus’ teachings up to that hour. Evenings they all assembled in Zebedee’s garden to receive Jesus’ instruction.
\vs p138 6:2 It was at this time that Jesus established the mid\hyp{}week holiday for rest and recreation. And they pursued this plan of relaxation for one day each week throughout the remainder of his material life. As a general rule, they never prosecuted their regular activities on Wednesday. On this weekly holiday Jesus would usually take himself away from them, saying: \textcolour{ubdarkred}{“My children, go for a day of play. Rest yourselves from the arduous labours of the kingdom and enjoy the refreshment that comes from reverting to your former vocations or from discovering new sorts of recreational activity.”} While Jesus, at this period of his earth life, did not actually require this day of rest, he conformed to this plan because he knew it was best for his human associates. Jesus was the teacher --- the Master; his associates were his pupils --- disciples.
\vs p138 6:3 \pc Jesus endeavoured to make clear to his apostles the difference between his teachings and his \bibemph{life among them} and the teachings which might subsequently spring up \bibemph{about} him. Said Jesus: \textcolour{ubdarkred}{“My kingdom and the gospel related thereto shall be the burden of your message. Be not sidetracked into preaching \bibemph{about} me and \bibemph{about} my teachings. Proclaim the gospel of the kingdom and portray my revelation of the Father in heaven but do not be misled into the bypaths of creating legends and building up a cult having to do with beliefs and teachings \bibemph{about} my beliefs and teachings.”} But again they did not understand why he thus spoke, and no man dared to ask why he so taught them.
\vs p138 6:4 In these early teachings Jesus sought to avoid controversies with his apostles as far as possible excepting those involving wrong concepts of his Father in heaven. In all such matters he never hesitated to correct erroneous beliefs. There was just \bibemph{one} motive in Jesus’ postbaptismal life on Urantia, and that was a better and truer revelation of his Paradise Father; he was the pioneer of the new and better way to God, the way of faith and love. Ever his exhortation to the apostles was: \textcolour{ubdarkred}{“Go seek for the sinners; find the downhearted and comfort the anxious.”}
\vs p138 6:5 Jesus had a perfect grasp of the situation; he possessed unlimited power, which might have been utilized in the furtherance of his mission, but he was wholly content with means and personalities which most people would have regarded as inadequate and would have looked upon as insignificant. He was engaged in a mission of enormous dramatic possibilities, but he insisted on going about his Father’s business in the most quiet and undramatic manner; he studiously avoided all display of power. And he now planned to work quietly, at least for several months, with his 12 apostles around about the Sea of Galilee.
\usection{7.\bibnobreakspace Another Disappointment}
\vs p138 7:1 Jesus had planned for a quiet missionary campaign of five months’ personal work. He did not tell the apostles how long this was to last; they worked from week to week. And early on this first day of the week, just as he was about to announce this to his 12 apostles, Simon Peter, James Zebedee, and Judas Iscariot came to have private converse with him. Taking Jesus aside, Peter made bold to say: “Master, we come at the behest of our associates to inquire whether the time is not now ripe to enter into the kingdom. And will you proclaim the kingdom at Capernaum, or are we to move on to Jerusalem? And when shall we learn, each of us, the positions we are to occupy with you in the establishment of the kingdom --- ” and Peter would have gone on asking further questions, but Jesus raised an admonitory hand and stopped him. And beckoning the other apostles standing near by to join them, Jesus said: \textcolour{ubdarkred}{“My little children, how long shall I bear with you! Have I not made it plain to you that my kingdom is not of this world? I have told you many times that I have not come to sit on David’s throne, and now how is it that you are inquiring which place each of you will occupy in the Father’s kingdom? Can you not perceive that I have called you as ambassadors of a spiritual kingdom? Do you not understand that soon, very soon, you are to represent me in the world and in the proclamation of the kingdom, even as I now represent my Father who is in heaven? Can it be that I have chosen you and instructed you as messengers of the kingdom, and yet you do not comprehend the nature and significance of this coming kingdom of divine pre\hyp{}eminence in the hearts of men? My friends, hear me once more. Banish from your minds this idea that my kingdom is a rule of power or a reign of glory. Indeed, all power in heaven and on earth will presently be given into my hands, but it is not the Father’s will that we use this divine endowment to glorify ourselves during this age. In another age you shall indeed sit with me in power and glory, but it behoves us now to submit to the will of the Father and to go forth in humble obedience to execute his bidding on earth.”}
\vs p138 7:2 Once more were his associates shocked, stunned. Jesus sent them away two and two to pray, asking them to return to him at noontime. On this crucial forenoon they each sought to find God, and each endeavoured to cheer and strengthen the other, and they returned to Jesus as he had bidden them.
\vs p138 7:3 Jesus now recounted for them the coming of John, the baptism in the Jordan, the marriage feast at Cana, the recent choosing of the six, and the withdrawal from them of his own brothers in the flesh, and warned them that the enemy of the kingdom would seek also to draw them away. After this short but earnest talk the apostles all arose, under Peter’s leadership, to declare their undying devotion to their Master and to pledge their unswerving loyalty to the kingdom, as Thomas expressed it, “To this coming kingdom, no matter what it is and even if I do not fully understand it.” They all truly \bibemph{believed in Jesus,} even though they did not fully comprehend his teaching.
\vs p138 7:4 Jesus now asked them how much money they had among them; he also inquired as to what provision had been made for their families. When it developed that they had hardly sufficient funds to maintain themselves for two weeks, he said: \textcolour{ubdarkred}{“It is not the will of my Father that we begin our work in this way. We will remain here by the sea two weeks and fish or do whatever our hands find to do; and in the meantime, under the guidance of Andrew, the first chosen apostle, you shall so organize yourselves as to provide for everything needful in your future work, both for the present personal ministry and also when I shall subsequently ordain you to preach the gospel and instruct believers.”} They were all greatly cheered by these words; this was their first clear\hyp{}cut and positive intimation that Jesus designed later on to enter upon more aggressive and pretentious public efforts.\fnc{\ldots{}this was their first \bibtextul{clearcut} and positive intimation\ldots{} \bibexpl{This word is found eight additional times; all are hyphenated.}}
\vs p138 7:5 The apostles spent the remainder of the day perfecting their organization and completing arrangements for boats and nets for embarking on the morrow’s fishing as they had all decided to devote themselves to fishing; most of them had been fishermen, even Jesus was an experienced boatman and fisherman. Many of the boats which they used the next few years had been built by Jesus’ own hands. And they were good and trustworthy boats.
\vs p138 7:6 Jesus enjoined them to devote themselves to fishing for two weeks, adding, \textcolour{ubdarkred}{“And then will you go forth to become fishers of men.”} They fished in three groups, Jesus going out with a different group each night. And they all so much enjoyed Jesus! He was a good fisherman, a cheerful companion, and an inspiring friend; the more they worked with him, the more they loved him. Said Matthew one day: “The more you understand some people, the less you admire them, but of this man, even the less I comprehend him, the more I love him.”
\vs p138 7:7 This plan of fishing two weeks and going out to do personal work in behalf of the kingdom for two weeks was followed for more than five months, even to the end of this year of A.D.\,26, until after the cessation of those special persecutions which had been directed against John’s disciples subsequent to his imprisonment.
\usection{8.\bibnobreakspace First Work of the Twelve}
\vs p138 8:1 After disposing of the fish catches of two weeks, Judas Iscariot, the one chosen to act as treasurer of the 12, divided the apostolic funds into six equal portions, funds for the care of dependent families having been already provided. And then near the middle of August, in the year A.D.\,26, they went forth two and two to the fields of work assigned by Andrew. The first two weeks Jesus went out with Andrew and Peter, the second two weeks with James and John, and so on with the other couples in the order of their choosing. In this way he was able to go out at least once with each couple before he called them together for the beginning of their public ministry.
\vs p138 8:2 Jesus taught them to preach the forgiveness of sin through \bibemph{faith in God} without penance or sacrifice, and that the Father in heaven loves all his children with the same eternal love. He enjoined his apostles to refrain from discussing:
\vs p138 8:3 \ublistelem{1.}\bibnobreakspace The work and imprisonment of John the Baptist.
\vs p138 8:4 \ublistelem{2.}\bibnobreakspace The voice at the baptism. Said Jesus: \textcolour{ubdarkred}{“Only those who heard the voice may refer to it. Speak only that which you have heard from me; speak not hearsay.”}
\vs p138 8:5 \ublistelem{3.}\bibnobreakspace The turning of the water into wine at Cana. Jesus seriously charged them, saying, \textcolour{ubdarkred}{“Tell no man about the water and the wine.”}
\vs p138 8:6 \pc They had wonderful times throughout these five or six months during which they worked as fishermen every alternate two weeks, thereby earning enough money to support themselves in the field for each succeeding two weeks of missionary work for the kingdom.
\vs p138 8:7 The common people marveled at the teaching and ministry of Jesus and his apostles. The rabbis had long taught the Jews that the ignorant could not be pious or righteous. But Jesus’ apostles were both pious and righteous; yet they were cheerfully ignorant of much of the learning of the rabbis and the wisdom of the world.
\vs p138 8:8 \pc Jesus made plain to his apostles the difference between the repentance of so\hyp{}called good works as taught by the Jews and the change of mind by faith --- the new birth --- which he required as the price of admission to the kingdom. He taught his apostles that \bibemph{faith} was the only requisite to entering the Father’s kingdom. John had taught them “repentance --- to flee from the wrath to come.” Jesus taught, \textcolour{ubdarkred}{“Faith is the open door for entering into the present, perfect, and eternal love of God.”} Jesus did not speak like a prophet, one who comes to declare the word of God. He seemed to speak of himself as one having authority. Jesus sought to divert their minds from miracle seeking to the finding of a real and personal experience in the satisfaction and assurance of the indwelling of God’s spirit of love and saving grace.
\vs p138 8:9 The disciples early learned that the Master had a profound respect and sympathetic regard for \bibemph{every} human being he met, and they were tremendously impressed by this uniform and unvarying consideration which he so consistently gave to all sorts of men, women, and children. He would pause in the midst of a profound discourse that he might go out in the road to speak good cheer to a passing woman laden with her burden of body and soul. He would interrupt a serious conference with his apostles to fraternize with an intruding child. Nothing ever seemed so important to Jesus as the \bibemph{individual human} who chanced to be in his immediate presence. He was master and teacher, but he was more --- he was also a friend and neighbour, an understanding comrade.
\vs p138 8:10 \pc Though Jesus’ public teaching mainly consisted in parables and short discourses, he invariably taught his apostles by questions and answers. He would always pause to answer sincere questions during his later public discourses.
\vs p138 8:11 The apostles were at first shocked by, but early became accustomed to, Jesus’ treatment of women; he made it very clear to them that women were to be accorded equal rights with men in the kingdom.
\usection{9.\bibnobreakspace Five Months of Testing}
\vs p138 9:1 This somewhat monotonous period of alternate fishing and personal work proved to be a gruelling experience for the 12 apostles, but they endured the test. With all of their grumblings, doubts, and transient dissatisfactions they remained true to their vows of devotion and loyalty to the Master. It was their personal association with Jesus during these months of testing that so endeared him to them that they all (save Judas Iscariot) remained loyal and true to him even in the dark hours of the trial and crucifixion. Real men simply could not actually desert a revered teacher who had lived so close to them and had been so devoted to them as had Jesus. Through the dark hours of the Master’s death, in the hearts of these apostles all reason, judgment, and logic were set aside in deference to just one extraordinary human emotion --- the supreme sentiment of friendship\hyp{}loyalty. These five months of work with Jesus led these apostles, each one of them, to regard him as the best \bibemph{friend} he had in all the world. And it was this human sentiment, and not his superb teachings or marvellous doings, that held them together until after the resurrection and the renewal of the proclamation of the gospel of the kingdom.
\vs p138 9:2 Not only were these months of quiet work a great test to the apostles, a test which they survived, but this season of public inactivity was a great trial to Jesus’ family. By the time Jesus was prepared to launch forth on his public work, his entire family (except Ruth) had practically deserted him. On only a few occasions did they attempt to make subsequent contact with him, and then it was to persuade him to return home with them, for they came near to believing that he was beside himself. They simply could not fathom his philosophy nor grasp his teaching; it was all too much for those of his own flesh and blood.
\vs p138 9:3 \pc The apostles carried on their personal work in Capernaum, Bethsaida\hyp{}Julias, Chorazin, Gerasa, Hippos, Magdala, Cana, Bethlehem of Galilee, Jotapata, Ramah, Safed, Gischala, Gadara, and Abila. Besides these towns they laboured in many villages as well as in the countryside. By the end of this period the 12 had worked out fairly satisfactory plans for the care of their respective families. Most of the apostles were married, some had several children, but they had made such arrangements for the support of their home folks that, with some little assistance from the apostolic funds, they could devote their entire energies to the Master’s work without having to worry about the financial welfare of their families.
\usection{10.\bibnobreakspace Organization of the Twelve}
\vs p138 10:1 The apostles early organized themselves in the following manner:
\vs p138 10:2 \ublistelem{1.}\bibnobreakspace Andrew, the first chosen apostle, was designated chairman and director general of the 12.
\vs p138 10:3 \ublistelem{2.}\bibnobreakspace Peter, James, and John were appointed personal companions of Jesus. They were to attend him day and night, to minister to his physical and sundry needs, and to accompany him on those night vigils of prayer and mysterious communion with the Father in heaven.
\vs p138 10:4 \ublistelem{3.}\bibnobreakspace Philip was made steward of the group. It was his duty to provide food and to see that visitors, and even the multitude of listeners at times, had something to eat.
\vs p138 10:5 \ublistelem{4.}\bibnobreakspace Nathaniel watched over the needs of the families of the 12. He received regular reports as to the requirements of each apostle’s family and, making requisition on Judas, the treasurer, would send funds each week to those in need.
\vs p138 10:6 \ublistelem{5.}\bibnobreakspace Matthew was the fiscal agent of the apostolic corps. It was his duty to see that the budget was balanced, the treasury replenished. If the funds for mutual support were not forthcoming, if donations sufficient to maintain the party were not received, Matthew was empowered to order the 12 back to their nets for a season. But this was never necessary after they began their public work; he always had sufficient funds in the treasurer’s hands to finance their activities.
\vs p138 10:7 \ublistelem{6.}\bibnobreakspace Thomas was manager of the itinerary. It devolved upon him to arrange lodgings and in a general way select places for teaching and preaching, thereby ensuring a smooth and expeditious travel schedule.
\vs p138 10:8 \ublistelem{7.}\bibnobreakspace James and Judas the twin sons of Alpheus were assigned to the management of the multitudes. It was their task to deputize a sufficient number of assistant ushers to enable them to maintain order among the crowds during the preaching.
\vs p138 10:9 \ublistelem{8.}\bibnobreakspace Simon Zelotes was given charge of recreation and play. He managed the Wednesday programs and also sought to provide for a few hours of relaxation and diversion each day.
\vs p138 10:10 \ublistelem{9.}\bibnobreakspace Judas Iscariot was appointed treasurer. He carried the bag. He paid all expenses and kept the books. He made budget estimates for Matthew from week to week and also made weekly reports to Andrew. Judas paid out funds on Andrew’s authorization.
\vs p138 10:11 \pc In this way the 12 functioned from their early organization up to the time of the reorganization made necessary by the desertion of Judas, the betrayer. The Master and his disciple\hyp{}apostles went on in this simple manner until Sunday, January 12, A.D.\,27, when he called them together and formally ordained them as ambassadors of the kingdom and preachers of its glad tidings. And soon thereafter they prepared to start for Jerusalem and Judea on their first public preaching tour.
\quizlink
